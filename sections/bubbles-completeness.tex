We are now going to prove the \emph{completeness} of the \kl{bi-intuitionistic} (and
propositional) fragment $\sysBHB$ of system $\sysB$, by simulating the nested
sequent system \kl{DBiInt} of Postniece. In \sidecite{postniece_deep_2009} she
shows that this calculus is sound and complete with respect to another calculus
\kl{LBiInt}, and in Chapter 4 of her thesis \sidecite{postniece_proof_2010} she
proves that \kl{LBiInt} is sound and complete with respect to the Kripke
semantics of \kl{bi-intuitionistic} logic. Importantly, the cut rule is shown to be
\emph{admissible} in both systems, through a syntactic process of
cut-elimination in \kl{LBiInt}. We will rely on this result to obtain
admissibility of the cut rule \rsf{i{\ua}} in $\sysBHB$, and by extension
in $\sysB$, $\sysBH$ and $\sysBB$. It might be interesting to have our own
internal cut-elimination procedure for system $\sysB$, notably to unveil its
computational content in the spirit of the \kl{Curry-Howard correspondence}. But this
would lead us astray from the purpose of this thesis, and thus we leave this
task for future work.

\begin{definition}[Structure]
  The \emph{structures} of \kl{DBiInt} are generated by the following grammar:
  $$X, Y \Coloneq \emptyset \mid A \mid (X,Y) \mid X \dseq Y$$ The
  structural connective ``,'' (comma) is associative and commutative and
  $\emptyset$ is its unit. We always consider structures modulo these
  equivalences.
\end{definition}

\begin{definition}[Structure translation]
  The \emph{translation} $\dtrans{X}$ of a structure $X$ as a multiset of items
  $\Gamma$ is defined recursively as follows:
  \begin{align*}
    \dtrans{\emptyset} &= \emptyset &
    \dtrans{(X, Y)} &= \dtrans{X}, \dtrans{Y} \\
    \dtrans{A} &= A &
    \dtrans{(X \dseq Y)} &= \dtrans{X} \seq \dtrans{Y}
  \end{align*}
\end{definition}

Note that the translation $\dtrans{(-)}$ is clearly \emph{injective}: in fact
structures are isomorphic to multisets of items that contain only \emph{open}
subsolutions. Thus from now on, we will always apply the translation implicitly,
and rely on meta-variables $X, Y$ to distinguish structures from arbitrary
solutions when necessary.

The rules of \kl{DBiInt} are given in \reffig{rules-dbiint}. Note that like
bubble calculi, \kl{DBiInt} is truly a \emph{\kl{deep inference}} system, in the
sense that rules can be applied on sequents nested arbitrarily deep inside
structures\sidenote{Our presentation of rules is slightly different from
\cite{postniece_deep_2009}: the contexts in which rules apply are left implicit,
and thus we do not rely on their polarity. The counterpart is that rules always
apply on sequents and never on formulas, which makes them more verbose. Also we
do not rely on the notion of ``top-level formulas'' of a structure, making the
propagation rules yet more verbose.}. The main difference lies in the fact that
proofs in \kl{DBiInt} are \emph{trees} built up by composing traditional
\kl{inference rules} with multiple premisses, while we use closed solutions (neutral
bubbles) to internalize the tree structure of proofs inside solutions. This
gives a lot of expressive power since closed solutions can themselves be nested
in open solutions and thus \emph{polarized}, a phenomemon which cannot be
simulated in \kl{DBiInt}. This is why we did not prove soundness in
\refsec{bubbles-soundness} by simulating directly system $\sysB$ in
\kl{DBiInt}, and conversely this will explain the ease with which \sys{DBiInt}
can be simulated inside system $\sysB$.

\begin{figure*}
  % \renewcommand{\seq}{\dseq}
\begin{framed}
\renewcommand{\arraystretch}{2}
\begin{mathpar}
\begin{array}{c}
\text{\textsc{Identity}} \\[1em]
\R[\intro(dbiint){id}]
  {}
  {X, A \seq A, Y}
\end{array}
\and
\begin{array}{c@{\quad}c}
\multicolumn{2}{c}{\textsc{Propagation}} \\[1em]
\R[\rsf{\seq_{L1}}]
  {X, A, (X', A \seq Y') \seq Y}
  {X, (X', A \seq Y') \seq Y}
&
\R[\rsf{\seq_{R1}}]
  {X \seq (X' \seq A, Y'), A, Y}
  {X \seq (X' \seq A, Y'), Y}
\\
\R[\rsf{\seq_{L2}}]
  {X, A \seq (X', A \seq Y'), Y}
  {X, A \seq (X' \seq Y'), Y}
&
\R[\rsf{\seq_{R2}}]
  {X, (X' \seq A, Y') \seq A, Y}
  {X, (X' \seq Y') \seq A, Y}
\end{array}
\and
\begin{array}{c@{\quad}c}
\multicolumn{2}{c}{\text{\textsc{Logic}}} \\[1em]
\R[\rsf{\bot_L}]
  {}
  {X, \bot \seq Y}
&
\R[\rsf{\top_R}]
  {}
  {X \seq \top, Y}
\\
\R[\rsf{\land_L}]
  {X, A \land B, A, B \seq Y}
  {X, A \land B \seq Y}
&
\R[\rsf{\land_R}]
  {X \seq A, A \land B, Y}
  {X \seq B, A \land B, Y}
  {X \seq A \land B, Y}
\\
\R[\rsf{\lor_L}]
  {X, A \lor B, A \seq Y}
  {X, A \lor B, B \seq Y}
  {X, A \lor B \seq Y}
&
\R[\rsf{\lor_R}]
  {X \seq A, B, A \lor B, Y}
  {X \seq A \lor B, Y}
\\
\R[\rsf{\limp_L}]
  {X, A \limp B \seq A, Y}
  {X, A \limp B, B \seq Y}
  {X, A \limp B \seq Y}
&
\R[\rsf{\limp_R}]
  {X \seq (A \seq B), A \limp B, Y}
  {X \seq A \limp B, Y}
\\
\R[\rsf{\lsub_L}]
  {X, A \lsub B, (A \seq B) \seq Y}
  {X, A \lsub B \seq Y}
&
\R[\rsf{\lsub_R}]
  {X \seq A, A \lsub B, Y}
  {X, B \seq A \lsub B, Y}
  {X \seq A \lsub B, Y}
\end{array}
\end{mathpar}
\end{framed}

  \caption{Rules of the deep nested sequent system \kl{DBiInt}}
  \labfig{rules-dbiint}
\end{figure*}

\begin{definition}[Syntactic entailment]
  We say that $\Gamma$ \emph{entails} $\Delta$ in a fragment $\mathsf{F}$ of
  rules of system $\sysB$, written $\Gamma \prov{\mathsf{F}} \Delta$, if and
  only if $\Gamma \seq \Delta \step{\mathsf{F}} \piq{}$. Similarly, we say
  that $X$ entails $Y$ in a fragment $\mathsf{F}$ of rules of \kl{DBiInt},
  written $X \prov{\mathsf{F}} Y$, if and only if $X \seq Y$ has a proof in
  \kl{DBiInt} using only rules in $\mathsf{F}$.
\end{definition}

\begin{lemma}[Simulation of \kl{DBiInt}]\lablemma{simulation-dbiint}
  
  If $X \prov{\kl{DBiInt}} Y$ then $X \prov{\sysBHB \setminus
  \{\rsf{i{\ua}}\}} Y$.
\end{lemma}
\begin{proof}
  By induction on the derivation of $X \prov{\kl{DBiInt}} Y$. The detailed
  proof is available in appendix (\refsec{app:bubbles-completeness}). 
\end{proof}

Assuming that the consequence relation of the Kripke semantics used by Postniece
to prove the completeness of \kl{DBiInt} coincides with the order relation of
Heyting-Brouwer algebras, we get the following fact:

\begin{fact}[Completeness of \kl{DBiInt}]\labfact{completeness-dbiint}
  If $A \sementHB B$ then $A \prov{\kl{DBiInt}} B$.
\end{fact}

Combined with the simulation of \kl{DBiInt} from \reflemma{simulation-dbiint},
this gives us the \emph{cut-free} completeness of $\sysBHB$:

\begin{theorem}[Cut-free completeness]\labthm{bubbles-completeness}
  If $A \sementHB B$ then $A \prov{\sysBHB \setminus \{\rsf{i{\ua}}\}} B$.
\end{theorem}
% \begin{proof}
%   This follows immediately from \reffact{completeness-dbiint} and
%   \reflemma{simulation-dbiint}.
% \end{proof}

In fact there are other rules of $\sysBHB$ that were not used in the simulation,
namely the $\mathbb{F}$-rule \rsf{f{\ua}}, and all $\mathbb{M}$-rules other
than \rsf{p}. Combined with the soundness of $\sysBHB$ (Corollary
\ref{cor:soundness}), this gives us the following \emph{admissibility} theorem:

\begin{theorem}[Admissibility]\labthm{bubbles:cut-admissibility}

  If $\prov{\sysBHB} A$ then $\prov{\sysBHB \setminus
  \{\rsf{i{\ua}},\rsf{f{\ua}},\rsf{p{-}},\rsf{p{+}},\rsf{a},\rsf{a{-}},\rsf{a{+}}\}}
  A$.
\end{theorem}

Although these rules are admissible, they do not seem to be derivable from other
rules. We believe that they might help in making proofs more \emph{compact} by
improving \emph{factorizability}, just like the cut rule does.
% We also conjecture that they become necessary (except \rsf{i{\ua}}) if one
% wants to show the completeness of $\sysBHB$ without any $\mathbb{H}$-rule ---
% and thus the admissibility of \emph{logical connectives}. But this has yet to be
% demonstrated.

As in \kl{sequent calculus}, every rule of system $\sysB$ other than
\rsf{i{\ua}} satisfies the \emph{subformula property}:

\begin{fact}[Subformula property]\label{cor:subformula-property} If $S
  \step{\sysB \setminus \{\rsf{i{\ua}}\}} T$ and $A \subsol T$, then there is a
  formula $B$ such that $A$ is a subformula of $B$ and $B \subsol S$.
\end{fact}

Thanks to cut admissibility, we thus get that $\sysBHB$ is \emph{analytic}. This
has many nice consequences, a well-known one being that when searching for a
proof of a given solution $S$, one does not need to come up with or ``invent'' a
formula that does not appear in $S$. This is crucial when designing
\emph{automated} decision procedures because it reduces drastically the search
space, but is also desirable in the setting of \emph{interactive} proof
building. Indeed with our \kl{Proof-by-Action} interpretation of bubble calculi
(\refsec{bubbles-pba}), this means that all logical reasoning can be performed
by \kl{direct manipulation} of \emph{what is already there}. Then the cut rule is
indispensable, but confined to a role of \emph{theory building}: it allows the
creation of \emph{lemmas}, in order to make proofs shorter and more tractable by
humans.

As noted in \cite{postniece_deep_2009}, one can simply ignore rules related to
the exclusion connective $\lsub$ to get a sound and complete system for
\kl{intuitionistic} logic. In \kl{DBiInt}, these rules are the \kl{introduction rules}
\rsf{{\lsub}_R} and \rsf{{\lsub}_L}, as well as the propagation rules
\rsf{{\seq}_{L1}} and \rsf{{\seq}_{R2}}. Indeed the latter are only useful in
combination with the former, since \rsf{{\lsub}_L} is the only rule of
\kl{DBiInt} that can introduce nested sequents in negative contexts. The
situation is similar in system $\sysB$, and in fact the proof of
\reflemma{simulation-dbiint} shows that the \kl{intuitionistic} fragment $\sysBH$ is
sufficient to simulate \kl{DBiInt} without the aforementioned rules. The dual
argument can be made for \kl{dual-intuitionistic} logic, and thus we obtain
(cut-free) \kl{intuitionistic} (resp. \kl{dual-intuitionistic}) completeness of $\sysBH$
(resp. $\sysBB$):

\begin{corollary}[Intuitionistic completeness]
  \sbr
  \begin{itemize}
    \item If $A \sementH B$ then $A \prov{\sysBH \setminus
    \{\rsf{i{\ua}}\}} B$.
    \item If $A \sementB B$ then $A \prov{\sysBB \setminus
    \{\rsf{i{\ua}}\}} B$.
  \end{itemize}
\end{corollary}

% Note that $\{\rsf{f{+}{-}{\da}}, \rsf{f{-}{-}{\ua}},
% \rsf{{\lsub}{-}}, \rsf{{\lsub}{+}}\}$ are the only rules of $\sysBHB$ that
% involve negative solutions and/or exclusions.

\begin{marginfigure}
  $$
  \R[{\limp}{-}]
  {\R[{\limp}{+}]
  {\R[{\bot}{+}]
  {\R[{\bot}{-}]
  {\R[\rsf{p}]
  {\R[\rsf{f{+}{\da}}]
  {\R[\rsf{f{+}{+}{\da}}]
  {\R[\rsf{i{\da}}]
  {\R[\rsf{p}{+}]
  {\R[\rsf{p}]
  {{} \piq{}}
  {{} \piq{\piq{}}}}
  {{} \piq{\seq (\piq{})}}}
  {{} \piq{\seq (A \seq A)}}}
  {{} \piq{\seq (A \seq), A}}}
  {{} \piq{\seq (A \seq)} A}}
  {{} \piq{\seq (A \seq) \sep \piq{}} A}}
  {{} \piq{\seq (A \seq) \sep \bot \seq} A}}
  {{} \piq{\seq (A \seq \bot) \sep \bot \seq} A}}
  {{} \piq{\seq \neg A \sep \bot \seq} A}}
  {\neg \neg A \seq A}
  $$
  \caption{Proof of DNE in system $\sysB$}
  \labfig{dne-bubbles}
\end{marginfigure}

\reffig{dne-bubbles} shows a proof of the double-negation elimination law
$\mathrm{DNE} \defeq \neg \neg A \seq A$ in system $\sysB$. Since $\sysBH$ is
\kl{intuitionistically} complete, the well-known double-negation embedding of
\kl{classical} logic into \kl{intuitionistic} logic tells us that $\neg \neg A$ is
provable in $\sysBH$ (and a fortiori in $\sysB$) if $A$ is a theorem of
\kl{classical} logic. Combining the two previous facts, we obtain the \kl{classical}
completeness of system $\sysB$. In fact the proof of DNE only relies on the use
of the \rsf{f{+}{+}{\da}} rule, so we can make the following stronger
statement:

\begin{corollary}[\kl{Classical} completeness]\label{cor:bubbles-completeness-classical}
  If $A$ is a theorem of \kl{classical} logic, then $\prov{\sysBH \cup
  \{\rsf{f{+}{+}{\da}}\}} A$.
\end{corollary}
\begin{proof}
  By the double-negation embedding, we have $\prov{\sysBH} \neg \neg A$. Then we
  can build the following derivation:
  $$
  \R[\rsf{i{\ua}}]
  {\prftree[dotted]
  {\R[\rsf{p}]
  {\R[\rsf{f{+}{\da}}]
  {\prftree[r][d]{DNE}  
  {\R[\rsf{p}]
  {{} \piq{}}
  {{} \piq{\piq{}}}}
  {{} \piq{\neg \neg A \seq A}}}
  {{} \piq{\neg \neg A \seq} A}}
  {{} \piq{\piq{} \sep \neg \neg A \seq} A}}
  {{} \piq{\seq \neg \neg A \sep \neg \neg A \seq} A}}
  {\seq A}
  $$
\end{proof}

Alas this argument makes use of the \rsf{i{\ua}} rule. Note however that
the reason we chose to prove completeness of $\sysBHB$ by simulating a rather
exotic system like \kl{DBiInt}, was that standard \kl{sequent calculi} for
\kl{bi-intuitionistic} logic like the one of Rauszer
\sidecite{rauszer_formalization_1974} are not \emph{cut-free} complete; and in
our literature review, \kl{DBiInt} was the cut-free system closest in its
syntax and rules to system $\sysB$. But for \kl{classical} logic we do not have this
limitation, and thus it is straightforward to simulate directly a cut-free
\kl{sequent calculus} such as \kl{G3cp} \sidecite{negri_structural_2001}:

\begin{lemma}[Simulation of \kl{G3cp}]
  If $\Gamma \prov{\kl{G3cp}} \Delta$, then $\Gamma \prov{\sysBH \cup
  \{\rsf{f{+}{+}}\} \setminus \{\rsf{i{\ua}}\}} \Delta$.
\end{lemma}
\begin{proof}
  By induction on the \kl{G3cp} derivation, see
  \refsec{app:bubbles-completeness} for the detailed proof.
\end{proof}

Lastly, let us mention a recent result of Goré and Shillito
\cite{gore_bi-intuitionistic_2020}, where they uncover a distinction between a
\emph{weak} and a \emph{strong} consequence relation in the semantics of
\kl{bi-intuitionistic} logic. Although they define the same set of theorems, these
two relations have different properties at the meta-level, and thus the authors
argue that they define two distinct logics, called respectively \kl{wBIL} and
\kl{sBIL}. At the end of the article, they conjecture that the various existing
calculi in the literature are sound and complete for \kl{wBIL}, including a
calculus designed by Postniece. Since our completeness proof is by simulation of
the system \kl{DBiInt} by the same author, we follow this conjecture regarding
the completeness of the \kl{bi-intuitionistic} fragment $\sysBHB$ of system $\sysB$.
For soundness, we would need to clarify the relationship between Heyting-Brouwer
algebras and these consequence relations, which stem instead from an analysis of
the Kripke semantics of \kl{bi-intuitionistic} logic. Since system $\sysB$ offers a
very expressive syntax, it would be interesting to investigate its ability to
capture both \kl{wBIL} and \sys{sBIL}, maybe by using distinct sets of flow
rules. Goré and Shillito suggest that a framework that captures both
\emph{provability} and \emph{refutability} ``in one shot'' would be needed, and
we believe system $\sysB$ might just provide this: indeed a derivation $S \steps{}
\piq{}$ can be read both as a \emph{proof} of $\psint{S}$, and a
\emph{refutation} of $\nsint{S}$.