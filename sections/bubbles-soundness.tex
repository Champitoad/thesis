\subsection{Heyting and Brouwer algebras}

\begin{figure*}
  \tikzfig{1}{0.83}{venn-algebras}
  \caption{Relationship between the various algebras interpreting \kl{system~B}}
  \labfig{venn-algebras}
\end{figure*}

We are now going to prove the soundness of \kl{system~B} with respect to
various classes of \emph{algebras}. While the full system is \kl{classical} and thus
sound only in \emph{Boolean} algebras, most rules are sound in larger classes of
algebras, namely: \emph{Heyting} algebras for \kl{intuitionistic} logic,
\emph{Brouwer} algebras for \kl{dual-intuitionistic} logic, and
\emph{Heyting-Brouwer} algebras for \kl{bi-intuitionistic} logic. These 4 classes are
all instances of \emph{\kl{bounded lattices}}, and their relationship is summarized
in the Venn \kl{diagram} of \reffig{venn-algebras}.

First we recall the definitions of the various algebras:

\begin{definition}[Bounded lattice]\labdef{bounded-lattice} A \intro{bounded
  lattice} is a structure $(\mathcal{A}, \sement{}, \ltop, \lbot, \lmeet,
  \ljoin)$ such that:
  \begin{itemize}
    \item $(\mathcal{A}, \sement{})$ is a partial order, i.e. for every $a, b, c
    \in \mathcal{A}$ we have:
      \begin{itemize}
        \item $a \sement{} a$;
        \item if $a \sement{} b$ and $b \sement{} a$ then $a = b$;
        \item if $a \sement{} b$ and $b \sement{} c$ then $a \sement{} c$.
      \end{itemize}
    \item $\lbot$ and $\ltop$ are respectively the smallest and greatest
    elements of $(\mathcal{A}, \sement{})$, i.e. for every $a \in \mathcal{A}$ we
    have $\bot \sement{} a$ and $a \sement{} \ltop$;
    \item For every pair of elements $a, b \in \mathcal{A}$, $a \ljoin b$ is
    their join (least upper bound) and $a \lmeet b$ their meet (greatest lower
    bound), that is:
    \begin{itemize}
      \item $a \sement{} a \ljoin b$, $b \sement{} a \ljoin b$ and $a \ljoin b \sement{} c$ for all $c \in \mathcal{A}$ s.t. $a \sement{} c$ and $b \sement{} c$;
      \item $a \lmeet b \sement{} a$, $a \lmeet b \sement{} b$ and $c \sement{} a \lmeet b$ for all $c \in \mathcal{A}$ s.t. $c \sement{} a$ and $c \sement{} b$.
    \end{itemize}
  \end{itemize}
\end{definition}

\begin{remark}
  As mentioned in the introduction, we only conjecture the soundness of rules
  for quantifiers: this would require considering \emph{complete} lattices, i.e.
  with meets and joins for arbitrary sets rather than just pairs\sidenote{See
  for instance \cite[Section~4]{forster_completeness_2021} for a concise
  treatment of the soundness and completeness of \kl{intuitionistic} and
  \kl{classical} \kl{natural deduction} for \kl{first-order logic} with respect
  to algebraic semantics.}.
\end{remark}

As the notation strongly suggests, the greatest and smallest elements $\top$ and
$\bot$ will model respectively truth and absurdity, while the meet $\lmeet$ and
join $\ljoin$ will model conjunction and disjunction. In fact the conditions of
\refdef{bounded-lattice} are very close to the rules of \kl{natural deduction}
for these connectives, by replacing the \kl{sequent} operator $\seq$ with the
partial order relation $\sement{}$. The same idea can be applied to the
implication connective, and adding a corresponding \kl{exponential} operation
$\lexp$ indeed gives the definition of a \kl{Heyting algebra}:

\begin{definition}[Heyting algebra]
  A \intro{Heyting algebra} is a structure $(\mathcal{A}, \sement{}, \ltop, \lbot,
  \lmeet, \ljoin, \lexp)$ such that $(\mathcal{A}, \sement{}, \ltop, \lbot,
  \lmeet, \ljoin)$ is a \kl{bounded lattice} and for every pair $a, b \in
  \mathcal{A}$, the \intro{exponential} $a \lexp b$ is the greatest element of
  the set $\compr{c \in \mathcal{A}}{c \lmeet a \sement{} b}$. That is, $(a \lexp
  b) \lmeet a \sement{} b$ and $c \sement{} a \lexp b$ for all $c \in \mathcal{A}$
  s.t. $c \lmeet a \sement{} b$.
\end{definition}

By dualizing this definition, we get a \kl{co-exponential} operation $\lcoexp$
that models the \kl{exclusion} connective, and thus \kl{dual-intuitionistic} logic in
so-called \kl{Brouwer algebras}:

\begin{definition}[Brouwer algebra]
  A \intro{Brouwer algebra} is a structure $(\mathcal{A}, \sement{}, \ltop,
  \lbot, \lmeet, \ljoin, \lcoexp)$ such that $(\mathcal{A}, \sement{}, \ltop,
  \lbot, \lmeet, \ljoin)$ is a \kl{bounded lattice} and for every pair $a, b \in
  \mathcal{A}$, the \intro{co-exponential} $a \lcoexp b$ is the smallest element
  of the set $\compr{c \in \mathcal{A}}{b \sement{} a \ljoin c}$. That is, $b
  \sement{} a \ljoin (b \lcoexp a)$ and $b \lcoexp a \sement{} c$ for all $c \in
  \mathcal{A}$ s.t. $b \sement{} a \ljoin c$.
\end{definition}

Then we can model \kl{bi-intuitionistic} logic, which comprises both implication
and \kl{exclusion}, by just taking pairs of a \kl{Heyting algebra} and a
\kl{Brouwer algebra} on the same \kl{bounded lattice}:

\begin{definition}[Heyting-Brouwer algebra]
  A \intro{Heyting-Brouwer algebra} is a structure $(\mathcal{A}, \sement{},
  \ltop, \lbot, \lmeet, \ljoin, \lexp, \lcoexp)$ such that $(\mathcal{A},
  \sement{}, \ltop, \lbot, \lmeet, \ljoin, \lexp)$ is a \kl{Heyting algebra} and
  $(\mathcal{A}, \sement{}, \ltop, \lbot, \lmeet, \ljoin, \lcoexp)$ is a
  \kl{Brouwer algebra}.
\end{definition}

Finally, we recover \kl{classical} logic by collapsing \kl{exponentials} and
\kl{co-exponentials} to their \kl{classical} definitions, giving a
characterization of \kl{Boolean algebras}:

\begin{definition}\labdef{boolean-algebra}
  A \intro{Boolean algebra} is a \kl{Heyting-Brouwer algebra} $(\mathcal{A}, \sement{},
  \ltop, \lbot, \lmeet, \ljoin, \lexp, \lcoexp)$ such that for every $a, b \in
  \mathcal{A}$, $a \lexp b = (\ltop \lcoexp a) \ljoin b$ and $a \lcoexp b = a
  \lmeet (b \lexp \bot)$.
\end{definition}

\begin{remark}
\refdef{boolean-algebra} can be shown equivalent to more
usual definitions of \kl{Boolean algebras}, that are based only on lattice operations
and a primitive complement operation modelling negation; but including the proof
here would lead us out of the scope of this chapter.
% \todo{Proof in appendix?}
\end{remark}

In the rest of this chapter, we will freely assimilate formulas with their
interpretation in the various algebras. Indeed, since we only consider the
abstract classes of all algebras and never deal with particular instances, they
will stand in perfect bijection.

\begin{definition}[Semantic entailment]
  We write $A \intro*\sement{\mathcall{X}} B$ (resp. $A
  \intro*\semequiv{\mathcall{X}} B$) to express that $A \sement{} B$ (resp. $A
  \sement{} B$ and $B \sement{} A$) in every algebra of the class
  $\mathcall{X}$. More precisely, $\mathcall{X}$ can be one of
  $\intro*\Lattice$, $\intro*\Heyting$, $\intro*\Brouwer$,
  $\intro*\HeytingBrouwer$ or $\intro*\Boolean$, which stand respectively for
  \kl{bounded lattices}, \kl[Heyting algebra]{Heyting}, \kl[Brouwer
  algebra]{Brouwer}, \kl[Heyting-Brouwer algebra]{Heyting-Brouwer} and
  \kl[Boolean algebra]{Boolean} algebras. We write $A \reintro*\sement{} B$
  (resp. $A \reintro*\semequiv{} B$) as a shorthand for $A \sementH B$ (resp. $A
  \reintro*\semequiv{\Heyting} B$).
\end{definition}

\subsection{Duality}

We now prove a number of lemmas that characterize \emph{duality} both
semantically, typically between \kl{Heyting algebras} and \kl{Brouwer algebras}, and syntactically
in the rules of \kl{system~B}. This will be useful later on to shorten some
proofs.

\begin{definition}[Dual formula]\labdef{dual-formula}
  The \emph{dual formula} $\reintro*\soldual{A}$ of a formula $A$ is defined recursively
  as follows:
  \begin{align*}
    \soldual{a} &= a &\\
    \soldual{\top} &= \bot &
    \soldual{\bot} &= \top \\
    \soldual{(A \land B)} &= \soldual{A} \lor \soldual{B} &
    \soldual{(A \lor B)} &= \soldual{A} \land \soldual{B} \\
    \soldual{(A \limp B)} &= \soldual{B} \lsub \soldual{A} &
    \soldual{(A \lsub B)} &= \soldual{B} \limp \soldual{A}
  \end{align*}
\end{definition}

\begin{fact}[Duality]\labfact{duality}
  \sbr
  \begin{itemize}
    \item $A \sementH B$ if and only if $\soldual{B} \sementB \soldual{A}$
    \item $A \sementB B$ if and only if $\soldual{B} \sementH \soldual{A}$
    \item $A \sementX B$ if and only if $\soldual{B} \sementX \soldual{A}$ when $\ACVar
    \in \{\HeytingBrouwer, \Boolean\}$.
  \end{itemize}
\end{fact}

We omit the proof of \reffact{duality}, but this can easily be obtained from the
soundness and completeness of a symmetric \kl{sequent calculus} for
\kl{bi-intuitionistic} logic; see for instance
\sidecite[][Lemma~2]{restall_extending_1997}.

\begin{definition}[Dual solution]\labdef{dual-solution}
  The \emph{dual solution} $\intro*\soldual{S}$ of a \kl{solution} $S$ is defined
  mutually recursively as follows:
  \begin{align*}
    \soldual{(\Gamma \J \Delta)} &= \soldual{\Delta} \mathbin{\soldual{\J}} \soldual{\Gamma} &
    \soldual{(S_1 \sep \ldots \sep S_n)} &= \soldual{S_1} \sep \ldots \sep \soldual{S_n} \\
    \soldual{A} &= \soldual{A} &
    \soldual{(I_1, \ldots, I_n)} &= \soldual{I_1}, \ldots, \soldual{I_n} \\
    \soldual{\seq} &= {\seq} &
    \soldual{\piq{\cS}} &= {\piq{\soldual{\cS}}}
  \end{align*}
  For \kl{contexts}, the \kl{hole} is self-dual: $\soldual{\hole} = \hole$. This
  entails in particular that $\soldual{S}\select{\soldual{T}} =
  \soldual{S\select{T}}$.
\end{definition}

Graphically, the dual of a \kl{solution} $S$ is $S$ where the colors of \kl{items} have
been swapped --- i.e. blue \kl{items} become red and red \kl{items} become blue --- and
formulas have been dualized (\refdef{dual-formula}).

\begin{definition}\labdef{item-depth}
  The \emph{depth} $\sdepth{I}$ of an \kl{item} $I$ is defined recursively as
  follows:
  \begin{align*}
    \sdepth{A} &= 0 \\
    \sdepth{\Gamma \seq \Delta} &= 1 + \max_{J \in \Gamma \cup \Delta}{\sdepth{J}} \\
    \sdepth{\Gamma \piq{\cS} \Delta} &= 1 + \max_{J \in \Gamma \cup \cS \cup \Delta}{\sdepth{J}}
  \end{align*}
\end{definition}

\begin{lemma}[Involutivity]\lablemma{involutivity}
  $\soldual{\soldual{I}} = I$.
\end{lemma}
\begin{proof}
  By induction on $\sdepth{I}$.
  \begin{itemize}
    \item[\textbf{Formula}] Suppose $I = A$. Then we conclude by a
    straightforward induction on $A$.
    \item[\textbf{Unsaturated solution}] Suppose $I = \Gamma \seq \Delta$. Then by
    definition we have $\soldual{(\soldual{\Gamma \seq \Delta})} =
    \soldual{(\soldual{\Delta} \seq \soldual{\Gamma})} =
    \soldual{\soldual{\Gamma}} \seq \soldual{\soldual{\Delta}}$, and we conclude
    by IH.
    \item[\textbf{Saturated solution}] Suppose $I = \Gamma \piq{\cS}
    \Delta$. Then by definition we have $\soldual{(\soldual{\Gamma
    \piq{\cS} \Delta})} = \soldual{(\soldual{\Delta}
    \piq{\soldual{\cS}} \soldual{\Gamma})} =
    \soldual{\soldual{\Gamma}} \piq{\soldual{\soldual{\cS}}}
    \soldual{\soldual{\Delta}}$, and we conclude by IH.
  \end{itemize}
\end{proof}

\begin{lemma}[Shallow rule duality]\lablemma{local-rule-duality}
  If $S \lstep{} T$ then $\soldual{S} \lstep{} \soldual{T}$.
\end{lemma}
\begin{proof}
  There is a bijection among the rules of \kl{system~B}, that matches each rule
  $\irule{r}{S}{T}$ to its dual $\irule{\soldual{r}}{\soldual{S}}{\soldual{T}}$.
  By involutivity (\reflemma{involutivity}), this bijection is self-inverse:
  $\soldual{\soldual{r}} = r$. It is most easily observed in
  the graphical presentation of the rules (\reffig{graphical-B}), where looking
  for the dual rule boils down to swapping red and blue (and mirroring logical
  connectives). The mapping goes as follows:
  \begin{mathpar}
  \begin{array}{r@{\quad\leftrightarrow\quad}l}
    \mathsf{i{\da}} & \mathsf{i{\da}} \\
    \mathsf{i{\ua}} & \mathsf{i{\ua}} \\
  \end{array}
  \and
  \begin{array}{r@{\quad\leftrightarrow\quad}l}
    \mathsf{w{-}} & \mathsf{w{+}} \\
    \mathsf{c{-}} & \mathsf{c{+}} \\
  \end{array}
  \\
  \begin{array}{r@{\quad\leftrightarrow\quad}l}
    \mathsf{f{-}} & \mathsf{f{+}} \\
    \mathsf{f{-}{+}{\da}} & \mathsf{f{+}{-}{\da}} \\
    \mathsf{f{-}{-}{\ua}} & \mathsf{f{+}{+}{\ua}} \\
    \mathsf{f{-}{+}{\ua}} & \mathsf{f{+}{-}{\ua}} \\
    \mathsf{f{-}{-}{\da}} & \mathsf{f{+}{+}{\da}} \\
  \end{array}
  \and
  \begin{array}{r@{\quad\leftrightarrow\quad}l}
    \mathsf{p} & \mathsf{p} \\
    \mathsf{p{-}} & \mathsf{p{+}} \\
    \mathsf{a} & \mathsf{a} \\
    \mathsf{a{-}} & \mathsf{a{+}} \\
  \end{array}
  \\
  \begin{array}{r@{\quad\leftrightarrow\quad}l}
    \top{-} & \bot{+} \\
    \bot{-} & \top{+} \\
    \land{-} & \lor{+} \\
    \lor{-} & \land{+} \\
    {\limp}{-} & {\lsub}{+} \\
    {\lsub}{-} & {\limp}{+} \\
    \forall{-} & \exists{+} \\
    \exists{-} & \forall{+} \\
  \end{array}
  \end{mathpar}
  Notice that some rules are self-dual, namely the identity rules
  {\kl{i{\da}}} and {\kl{i{\ua}}}, and the membrane rules
  {\kl{p}} and {\kl{a}}.
\end{proof}

\begin{lemma}[Rule duality]\lablemma{rule-duality}
  If $S \step{} T$ then $\soldual{S} \step{} \soldual{T}$.
\end{lemma}
\begin{proof}
  Let $U\hole$, $S_0$ and $T_0$ such that $S = U\select{S_0}$, $T =
  U\select{T_0}$ and $S_0 \lstep{} T_0$. By \reflemma{local-rule-duality} we have
  $\soldual{S_0} \lstep{} \soldual{T_0}$, and thus
  $\soldual{U}\select{\soldual{S_0}} \step{} \soldual{U}\select{\soldual{T_0}}$,
  or equivalently $\soldual{U\select{S_0}} \step{} \soldual{U\select{T_0}}$.
\end{proof}

\begin{lemma}[Interpretation duality]\lablemma{int-duality}
  $\soldual{\psint{I}} = \nsint{\soldual{I}}$ and $\soldual{\nsint{I}} =
  \psint{\soldual{I}}$.
\end{lemma}
\begin{proof}
  By a straightforward induction on $\sdepth{I}$.
\end{proof}

\begin{lemma}\lablemma{int-invert}
  $\psint{\soldual{S}} \sementX \psint{\soldual{T}}$ if and only if $\nsint{T} \sementX
  \nsint{S}$ when $\ACVar \in \{\HeytingBrouwer, \Boolean\}$.
\end{lemma}
\begin{proof}
  By duality (\reffact{duality}) we have $\soldual{\psint{\soldual{T}}} \sementX
  \soldual{\psint{\soldual{S}}}$, and then by \reflemma{int-duality}
  $\nsint{\soldual{\soldual{T}}} \sementX \nsint{\soldual{\soldual{S}}}$. We
  conclude by involutivity (\reflemma{involutivity}).
\end{proof}

\subsection{Shallow soundness}

In the following we give a number of (in)equalities that hold in the various
classes of algebras. They can easily be checked by building derivations in an
adequate \kl{sequent calculus}.

\begin{fact}[Commutativity]\labfact{lattice-commutativity}
  $A \lor B \semequiv{\Lattice} B \lor A$ and $A \land B \semequiv{\Lattice} B \land A$.
\end{fact}

\begin{fact}[Idempotency]\labfact{idempotency}
  $A \lor A \semequiv{\Lattice} A$ and $A \land A \semequiv{\Lattice} A$.
\end{fact}

\begin{fact}[Currying]\labfact{currying}
  \begin{align*}
    A \limp (B \limp C) &\semequiv{} (A \land B) \limp C \\
    (A \lsub B) \lsub C &\semequiv{\Brouwer} A \lsub (B \lor C)
  \end{align*}
\end{fact}

\begin{fact}[Distributivity]\labfact{distributivity}
  \begin{align*}
    A \land (B \lor C) &\semequiv{\Lattice} (A \land B) \lor (A \land C) \\
    A \lor (B \land C) &\semequiv{\Lattice} (A \lor B) \land (A \lor C) \\
    A \limp B \land C &\semequiv{} (A \limp B) \land (A \limp C) \\
    A \lor B \limp C &\semequiv{} (A \limp B) \land (A \limp C) \\
    A \lor B \lsub C &\semequiv{\Brouwer} (A \lsub B) \lor (A \lsub C) \\
    A \lsub B \land C &\semequiv{\Brouwer} (A \lsub B) \lor (A \lsub C)
  \end{align*}
\end{fact}

\begin{fact}[Weak distributivity]\labfact{weakdistrib}
  \begin{align*}
    (A \limp B) \lor C &\sement{} A \limp (B \lor C) \\
    A \limp (B \lor C) &\sementC (A \limp B) \lor C \\
    (A \land B) \lsub C &\sementB A \land (B \lsub C) \\
    A \land (B \lsub C) &\sementC (A \land B) \lsub C
  \end{align*}
\end{fact}

\begin{fact}\labfact{gencut}
  \begin{align*}
  (A \lor B) \land (C \limp D) &\sement{} (A \limp C) \limp (B \lor D) \\
  (A \lor B) \land (C \limp D) &\sementHB (A \lsub C) \lor (B \lor D) \\
  (A \lor B) \land (A \limp B) &\semequiv{} B
  \end{align*}
\end{fact}

\begin{fact}\labfact{impsub}
  $(A \lsub B) \limp C \sementHB A \limp B \lor C$.
\end{fact}

The following definition will be used pervasively to reason by induction on the
tree structure induced by \kl{branching operators}:

\begin{definition}
  The \intro{depth} $\intro*\sdepth{\J}$ of a \kl{branching operator} $\J$ is defined
  recursively as follows:
  \begin{align*}
    \sdepth{\seq} &= 0 \\
    \sdepth{\piq{\cS}} &= 1 + \max_{S \in \cS}{\sdepth{S}}
  \end{align*}
\end{definition}

% \begin{lemma}\lablemma{bubbles-multiweak}
%   $\psint{\Gamma \seq \Delta} \sement{} \psint{\Gamma', \Gamma \seq \Delta, \Delta'}$.
% \end{lemma}
% \begin{proof}
%   $$
%   \begin{array}{rcll}
%     \psint{\Gamma \seq \Delta}
%     &=& \nsint{\Gamma} \limp \psint{\Delta} & \\
%     &\sement{}& \nsint{\Gamma'} \wedge \nsint{\Gamma} \limp \psint{\Delta}} &\text{(Weakening)} \\
%     &=& \psint{\Gamma', \Gamma \seq \Delta, \Delta'} & \\
%   \end{array}
%   $$
% \end{proof}

% \begin{lemma}[Sharing]
%   $\psint{\Gamma \seq \Delta} \sement{} \psint{\Gamma \piq{\cS} \Delta}$.
% \end{lemma}
% \begin{proof}
%   Let $\cS = S_1 \sep \ldots \sep S_n$. We proceed by induction on
%   $\sdepth{\piq{\cS}}$.
%   \begin{itemize}proof
%     \item[\bcase] Suppose $\sdepth{\piq{\cS}} = 1$, and
%     let $1 \leq i \leq n$. Then we know that $S_i = \Gamma_i \seq \Delta_i$, and
%     by \reflemma{bubbles-multiweak} we get $\psint{\Gamma \seq \Delta} \sement{}
%     \psint{S_i \mix (\Gamma \seq \Delta)}$. Thus we have
%     $$
%     \begin{array}{rcll}
%       \psint{\Gamma \seq \Delta}
%       &\sement{}& \bigwedge_{1 \leq i \leq n}{\psint{S_i \mix (\Gamma \seq \Delta)}} & \\
%       &=& \psint{\Gamma \piq{\cS} \Delta}
%     \end{array}
%     $$
%     \item[\rcase] Suppose $\sdepth{\piq{\cS}} > 1$.

%   \end{itemize}
% \end{proof}

Now we can prove a few lemmas that generalize some semantic (in)equalities to the
interpretation of \kl{solutions} with arbitrary \kl{branching operators}. All detailed
proofs are available in appendix (\refsec{app:bubbles-soundness}).

\begin{lemma}[Generalized weakening]\lablemma{bubbles-weakening}
  $\psint{S} \sement{} \psint{S \mix (\Gamma \seq \Delta)}$.
\end{lemma}
\begin{proof}
  By induction on $\sdepth{\J}$, with $S = \Gamma' \J \Delta'$.
\end{proof}

\begin{lemma}[Generalized contraction]\lablemma{bubbles-contraction}
  $\psint{S \mix (\seq I, I)} \semequiv{} \psint{S \mix (\seq I)}$ and
  $\psint{S \mix (I, I \seq)} \semequiv{} \psint{S \mix (I \seq)}$.
\end{lemma}
\begin{proof}
  By induction on $\sdepth{\J}$, with $S = \Gamma \J \Delta$.
\end{proof}

\begin{lemma}[Generalized weak distributivity]\lablemma{bubbles-weakdistrib}
  \begin{align}
    \psint{\Gamma \J \Delta} \lor \psint{I} &\sement{} \psint{\Gamma \J I, \Delta} \label{eqn:weakdistrib-one} \\
    \psint{\Gamma \J I, \Delta} &\sementC \psint{\Gamma \J \Delta} \lor \psint{I} \label{eqn:weakdistrib-two} \\
    \nsint{\Gamma, I \J \Delta} &\sementB \nsint{I} \land \nsint{\Gamma \J \Delta} \label{eqn:weakdistrib-three} \\
    \nsint{I} \land \nsint{\Gamma \J \Delta} &\sementC \nsint{\Gamma, I \J \Delta} \label{eqn:weakdistrib-four}
  \end{align}
\end{lemma}
\begin{proof}
  (\ref{eqn:weakdistrib-one}) holds by induction on $\sdepth{\J}$, using the
  corresponding inequality from \reffact{weakdistrib}. The proof of
  (\ref{eqn:weakdistrib-two}) is the same, except that we use the converse
  inequality of \reffact{weakdistrib} that holds in \kl{Boolean algebras}.
  (\ref{eqn:weakdistrib-three}) and (\ref{eqn:weakdistrib-four}) hold by duality
  from (\ref{eqn:weakdistrib-one}) and (\ref{eqn:weakdistrib-two}).
\end{proof}

\begin{lemma}[Generalized currying]\lablemma{bubbles-currying}
  \begin{align}
    \psint{\Gamma, I \J \Delta} &\semequiv{} \nsint{I} \limp \psint{\Gamma \J \Delta} \label{eqn:currying-one} \\
    \nsint{\Gamma \J I, \Delta} &\semequiv{\Brouwer} \nsint{\Gamma \J \Delta} \lsub \psint{I} \label{eqn:currying-two}
  \end{align}
\end{lemma}
\begin{proof}
  (\ref{eqn:currying-one}) holds by induction on $\sdepth{\J}$, and
  (\ref{eqn:currying-two}) by duality.
\end{proof}

% \begin{lemma}[Mix]\lablemma{bubbles-mix}
%   $\psint{S} \land \psint{T} \sement{} \psint{S \mix T}$.
% \end{lemma}

% \begin{lemma}\lablemma{bubbles-piq}
%   $\psint{S \mix T} \sement{} \psint{S \mix \piq{T}}$.
% \end{lemma}

Lastly, we mention a technical property of the rules that will be necessary for the
final proof of soundness to go through:

\begin{fact}[Top-level genericity]\labfact{bubbles-top-level}
  If $S \lstep{} T$, then $S \mix (\Gamma \seq \Delta) \lstep{} T \mix (\Gamma \seq \Delta)$.
\end{fact}

All the previous facts and lemmas can now be used to prove \emph{shallow
soundness}, i.e. that the interpretation of each rule of \kl{system~B} maps to
an (in)equality in some class of algebras:

\begin{lemma}[Shallow soundness]\lablemma{bubbles-local-soundness}
  
  If $S \lstep{} T$ then $\psint{T \mix (\Gamma \seq \Delta)} \sementC \psint{S
  \mix (\Gamma \seq \Delta)}$.
  % and $\nsint{S \mix (\Gamma \seq \Delta)} \sement{}
  % \nsint{T \mix (\Gamma \seq \Delta)}$.
\end{lemma}
\begin{proof}
  $S \lstep{} T$ implies $\psint{T} \sementC \psint{S}$, which is shown by
  inspection of each rule of \kl{system~B} (see \refsec{app:bubbles-soundness}).
  That we can mix an arbitrary top-level \kl(sequent){context} $\Gamma \seq \Delta$ into $S$
  and $T$ follows from \reffact{bubbles-top-level}.
\end{proof}

Since some rules only hold \kl{classically}, the statement for the full system
is relative to \kl{Boolean algebras}. But from the detailed proof in
\refsec{app:bubbles-soundness}, we can identify two fragments $\kl{\sysBH}$ and
$\kl{\sysBHB}$ of \kl{system~B} that are sound respectively for \kl{Heyting
algebras} and \kl{Heyting-Brouwer algebras}:

\begin{corollary}\labcor{lsoundness}
  Let
  \begin{align*}
    \intro{\sysBHB} &\defeq \sysB \setminus \{\kl{f{-}{+}{\ua}}, \kl{f{+}{+}{\da}}, \kl{f{+}{-}{\ua}}, \kl{f{-}{-}{\da}}\} \\
    \intro{\sysBH} &\defeq \kl{\sysBHB} \setminus \{\kl{f{+}{-}{\da}}, \kl{f{-}{-}{\ua}}, \kl{{\lsub}{-}}, \kl{{\lsub}{+}}\}
  \end{align*}
  Then we have:
  \begin{itemize}
    \item $S \lstep{\kl{\sysBH}} T$ implies $\psint{T} \sement{} \psint{S}$
    \item $S \lstep{\kl{\sysBHB}} T$ implies $\psint{T} \sementHB \psint{S}$
    % \item $S \lstep{\text{\sysB}} T$ implies $\psint{T} \sementC \psint{S}$
  \end{itemize}
\end{corollary}

In order to get the last missing fragment $\kl{\sysBB}$ sound with respect to \kl{Brouwer
algebras}, we need dual lemmas that are relative to the \kl{negative interpretation}
$\nsint{-}$ instead of the \kl{positive interpretation} $\psint{-}$, since
implication is replaced by \kl{exclusion}. To avoid verbosity, we only formulate the
main lemma, and assume that its proof will go through mechanically:

\begin{lemma}[Shallow co-soundness]\lablemma{bubbles-local-cosoundness}
  If $S \lstep{} T$ then $\nsint{S \mix (\Gamma \seq \Delta)} \sementC \nsint{T
  \mix (\Gamma \seq \Delta)}$.
\end{lemma}

Then from the (assumed) proof of \reflemma{bubbles-local-cosoundness} we get:
\begin{corollary}\labcor{lcosoundness}
  Let $\intro{\sysBB} \defeq \kl{\sysBHB} \setminus \{\kl{f{-}{+}{\da}},
  \kl{f{+}{+}{\ua}}, \kl{{\limp}{-}}, \kl{{\limp}{+}}\}$. Then $S
  \lstep{\kl{\sysBB}} T$ implies $\nsint{S} \sementB \nsint{T}$.
\end{corollary}

The full situation is summarized in \reffig{venn-algebras}.

\subsection{Contextual soundness}

\begin{lemma}[Functoriality]\lablemma{bubbles-functoriality}
  Let $\ACVar \in \{\Heyting, \HeytingBrouwer, \Boolean\}$.
  \sbr
  \begin{itemize}
    \item $\psint{I} \sementX \psint{J}$ implies $\psint{(\seq I) \mix S} \sementX
    \psint{(\seq J) \mix S}$
    % \item $\nsint{I} \sementX \nsint{J}$ implies $\nsint{(\seq I) \mix S} \sementX
    % \nsint{(\seq J) \mix S}$
    \item $\nsint{J} \sementX \nsint{I}$ implies $\psint{(I \seq) \mix S}
    \sementX \psint{(J \seq) \mix S}$
    % \item $\psint{J} \sementX \psint{I}$ implies $\nsint{(I
    % \seq) \mix S} \sementX \nsint{(J \seq) \mix S}$
  \end{itemize}
\end{lemma}
\begin{proof}
  Let $S = \Gamma \J \Delta$. We proceed by induction on $\sdepth{\J}$.
  \begin{itemize}
    \item[\bcase] Suppose $\sdepth{\J} = 0$. Then $\J = {\seq}$,
    and we have
    $$
    \begin{array}{rcll}
      \psint{(\seq I) \mix S}
      &=& \psint{\Gamma \seq I, \Delta} &\\
      &=& \nsint{\Gamma} \limp \psint{I} \lor \psint{\Delta} &\\
      &\sementX& \nsint{\Gamma} \limp \psint{J} \lor \psint{\Delta} &\text{(Hypothesis)}\\
      &=& \psint{\Gamma \seq J, \Delta} &\\
      &=& \psint{(\seq J) \mix S} &
    \end{array}
    $$
    $$
    \begin{array}{rcll}
      \psint{(I \seq) \mix S}
      &=& \psint{\Gamma, I \seq \Delta} &\\
      &=& \nsint{\Gamma} \land \nsint{I} \limp \psint{\Delta} &\\
      &\sementX& \nsint{\Gamma} \land \nsint{J} \limp \psint{\Delta} &\text{(Hypothesis)} \\
        % &\begin{array}{rl}
        %    &\text{Contravariant functoriality of $\_{\limp}$} \\
        %   +&\text{Functoriality of $\_{\lor}$} \\
        %   +&\text{Hypothesis}
        % \end{array}\\
      &=& \psint{\Gamma, J \seq \Delta} &\\
      &=& \psint{(J \seq) \mix S} &
    \end{array}
    $$
    \item[\rcase] Suppose $\sdepth{\J} > 0$. Then $\J =
    {\piq{\cS}}$, and for all $S_0 = \Gamma_0 \JB \Delta_0 \in
    \cS$ we have that $\sdepth{\JB} < \sdepth{\J}$. Thus we have
    $$
    \begin{array}{rcll}
      \psint{(\seq I) \mix S}
      &=& \psint{\Gamma \piq{\cS} I, \Delta} & \\
      &=& \bigwedge_{S_0 \in \cS}{\psint{(\Gamma \seq I, \Delta) \mix S_0}} & \\
      &=& \bigwedge_{S_0 \in \cS}{\psint{(\seq I) \mix ((\Gamma \seq \Delta) \mix S_0)}} & \\
      &\sementX& \bigwedge_{S_0 \in \cS}{\psint{(\seq J) \mix ((\Gamma \seq \Delta) \mix S_0)}} &\text{(IH)} \\
      &=& \bigwedge_{S_0 \in \cS}{\psint{(\Gamma \seq J, \Delta) \mix S_0}} & \\
      &=& \psint{\Gamma \piq{\cS} J, \Delta} & \\
      &=& \psint{(\seq J) \mix S} &
    \end{array}
    $$
    $$
    \begin{array}{rcll}
      \psint{(I \seq) \mix S}
      &=& \psint{\Gamma, I \piq{\cS} \Delta} & \\
      &=& \bigwedge_{S_0 \in \cS}{\psint{(\Gamma, I \seq \Delta) \mix S_0}} & \\
      &=& \bigwedge_{S_0 \in \cS}{\psint{(I \seq) \mix ((\Gamma \seq \Delta) \mix S_0)}} & \\
      &\sementX& \bigwedge_{S_0 \in \cS}{\psint{(J \seq) \mix ((\Gamma \seq \Delta) \mix S_0)}} &\text{(IH)} \\
      &=& \bigwedge_{S_0 \in \cS}{\psint{(\Gamma, J \seq \Delta) \mix S_0}} & \\
      &=& \psint{\Gamma, J \piq{\cS} \Delta} & \\
      &=& \psint{(J \seq) \mix S} &
    \end{array}
    $$
  \end{itemize}
\end{proof}

In order to ease reasoning by induction on \kl{contexts}, we give a formulation
equivalent to \refdef{solution-context} as a context-free grammar:
\begin{fact}
  \reintro{Contexts} $S\hole$ are generated by the following grammar:
  $$
    S\hole \Coloneq \hole \mid \Gamma \J S\hole, \Delta
                          \mid \Gamma, S\hole \J \Delta
                          \mid \Gamma \piq{\cS \sep S\hole} \Delta
  $$
\end{fact}

\begin{definition}\labdef{solctx-depth}
  The \reintro{depth} $\intro*\cdepth{S\hole}$ of a \kl{context} $S\hole$ is
defined recursively as follows:
\begin{align*}
  \cdepth{\hole} &= 0 \\
  \cdepth{\Gamma \J S\hole, \Delta} = \cdepth{\Gamma, S\hole \J \Delta} =
  \cdepth{\Gamma \piq{\cS \sep S\hole} \Delta} &= 1 + \cdepth{S\hole}
\end{align*}
\end{definition}

\begin{lemma}[Contextual soundness]\lablemma{bubbles-ctx-soundness}

  If $S \lstep{} T$ then $\psint{U\select{T} \mix (\Gamma \seq \Delta)} \sementC
  \psint{U\select{S} \mix (\Gamma \seq \Delta)}$.
  % and $\nsint{S \mix (\Gamma \seq \Delta)} \sement{}
  % \nsint{T \mix (\Gamma \seq \Delta)}$.
\end{lemma}
\begin{proof}
  By induction on $\cdepth{U\hole}$.
  \begin{itemize}
    \item[\bcase] Suppose $\cdepth{U\hole}$ = 0. Then $U\hole =
    \hole$, and we conclude by shallow soundness
    (\reflemma{bubbles-local-soundness}).
    \item[\textbf{Positive case}] Suppose $\cdepth{U\hole} > 0$ and $U\hole =
    \Gamma' \J U_0\hole, \Delta'$. Then by IH we have $\psint{U_0\select{T}}
    \sementC \psint{U_0\select{S}}$, and thus
    $$
    \begin{array}{rcll}
      \psint{(\Gamma' \J U_0\select{T}, \Delta') \mix (\Gamma \seq \Delta)}
      &=& \psint{(\seq U_0\select{T}) \mix (\Gamma, \Gamma' \J \Delta', \Delta)} &\\
      &\sementC& \psint{(\seq U_0\select{S}) \mix (\Gamma, \Gamma' \J \Delta', \Delta)} &\text{(\reflemma{bubbles-functoriality})}\\
      &=& \psint{(\Gamma' \J U_0\select{S}, \Delta') \mix (\Gamma \seq \Delta)} &
    \end{array}
    $$

    \item[\textbf{Negative case}] Suppose $\cdepth{U\hole} > 0$ and $U\hole =
    \Gamma', U_0\hole \J \Delta'$. Then by \reflemma{local-rule-duality} we have
    $\soldual{S} \lstep{} \soldual{T}$, and thus by IH
    $\psint{\soldual{U_0}\select{\soldual{T}}} \sementC
    \psint{\soldual{U_0}\select{\soldual{S}}}$, or equivalently
    $\psint{\soldual{U_0\select{T}}} \sementC \psint{\soldual{U_0\select{S}}}$.
    Then by \reflemma{int-invert} we get $\nsint{U_0\select{S}} \sementC
    \nsint{U_0\select{T}}$, and thus
    $$
    \begin{array}{rcll}
      \psint{(\Gamma', U_0\select{T} \J \Delta') \mix (\Gamma \seq \Delta)}
      &=& \psint{(U_0\select{T} \seq) \mix (\Gamma, \Gamma' \J \Delta', \Delta)} &\\
      &\sementC& \psint{(U_0\select{S} \seq) \mix (\Gamma, \Gamma' \J \Delta', \Delta)} &\text{(\reflemma{bubbles-functoriality})}\\
      &=& \psint{(\Gamma', U_0\select{S} \J \Delta') \mix (\Gamma \seq \Delta)} &
    \end{array}
    $$

    \item[\textbf{Neutral case}] Suppose $\cdepth{U\hole} > 0$ and $U\hole =
    \Gamma \piq{\cS \sep U_0\hole} \Delta$. Then by IH we have
    $\psint{U_0\select{T} \mix (\Gamma \seq \Delta)} \sementC
    \psint{U_0\select{S} \mix (\Gamma \seq \Delta)}$, and thus
    $$
    \begin{array}{rcll}
      \psint{\Gamma \piq{\cS \sep U_0\select{T}} \Delta}
      &=& \psint{\Gamma \piq{\cS} \Delta} \land \psint{U_0\select{T} \mix (\Gamma \seq \Delta)} &\\
      &\sementC& \psint{\Gamma \piq{\cS} \Delta} \land \psint{U_0\select{S} \mix (\Gamma \seq \Delta)} &\\
      &=& \psint{\Gamma \piq{\cS \sep U_0\select{S}} \Delta} &
    \end{array}
    $$
  \end{itemize}
\end{proof}

\begin{theorem}[Soundness]\labthm{bubbles-soundness}
  If $S \step{} T$ then $\psint{T} \sementC \psint{S}$.
\end{theorem}
\begin{proof}
  By definition of ${\step{}}$ and \reflemma{bubbles-ctx-soundness} with $\Gamma
  = \Delta = \emptyset$.
\end{proof}

We also get for free soundness with respect to the \kl{negative interpretation},
which we call \emph{co-soundness}:

\begin{theorem}[Co-soundness]\labthm{bubbles-cosoundness}
  If $S \step{} T$ then $\nsint{S} \sementC \nsint{T}$.
\end{theorem}
\begin{proof}
  By \reflemma{rule-duality} we have $\soldual{S} \step{} \soldual{T}$,
  and thus by soundness $\psint{\soldual{T}} \sementC \psint{\soldual{S}}$. Then
  we can conclude by \reflemma{int-invert}.
\end{proof}

As for shallow soundness (\refcor{lsoundness} and \refcor{lcosoundness}), we can
easily generalize the proof of \reflemma{bubbles-ctx-soundness} to \kl{Heyting
algebras} and \kl{Heyting-Brouwer algebras}, and thus extend our soundness
result to \kl{intuitionistic} and \kl{bi-intuitionistic} logic:

\begin{corollary}\labcor{soundness}
  \sbr
  \begin{itemize}
    \item $S \step{\kl{\sysBH}} T$ implies $\psint{T} \sement{} \psint{S}$
    \item $S \step{\kl{\sysBHB}} T$ implies $\psint{T} \sementHB \psint{S}$
  \end{itemize}
\end{corollary}
\begin{proof}
  \reflemma{bubbles-local-soundness} is the only lemma used in the proof of
  \reflemma{bubbles-ctx-soundness} that relies on \kl{Boolean algebras}. Thus we
  can easily replace it by \refcor{lsoundness} to get soundness in
  \kl{Heyting-Brouwer algebras}.

  For soundness in \kl{Heyting algebras}, we know that the negative case in the
  proof of \reflemma{bubbles-ctx-soundness} will never happen because
  formulas cannot contain \kl{exclusions}. The other cases only depend on
  \reflemma{bubbles-local-soundness}, thus we can again replace it by
  \refcor{lsoundness}.
\end{proof}

Once again in order to extend contextual soundness to \kl{dual-intuitionistic} logic,
we need to dualize lemmas to the \kl{negative interpretation}:

\begin{lemma}[Co-functoriality]\lablemma{bubbles-cofunctoriality}
  Let $\ACVar \in \{\Brouwer, \HeytingBrouwer, \Boolean\}$.
  \sbr
  \begin{itemize}
    \item $\nsint{I} \sementX \nsint{J}$ implies $\nsint{(\seq I) \mix S}
    \sementX \nsint{(\seq J) \mix S}$
    \item $\psint{J} \sementX \psint{I}$ implies $\nsint{(I \seq) \mix S}
    \sementX \nsint{(J \seq) \mix S}$
  \end{itemize}
\end{lemma}

\begin{lemma}[Contextual co-soundness]\lablemma{bubbles-ctx-cosoundness}
  If $S \lstep{} T$ then $\nsint{U\select{S} \mix (\Gamma \seq \Delta)} \sementC
  \nsint{U\select{T} \mix (\Gamma \seq \Delta)}$.
\end{lemma}

From the assumed proof of \reflemma{bubbles-ctx-cosoundness}, we finally get:

\begin{corollary}\labcor{cosoundness}
  $S \step{\kl{\sysBB}} T$ implies $\nsint{S} \sementB \nsint{T}$.
\end{corollary}

Combined with the completeness proof of the next section, this will give us our
main result that $\kl{\sysBH}$, $\kl{\sysBB}$, $\kl{\sysBHB}$ and $\sysB$
capture exactly provability in \kl{intuitionistic}, \kl{dual-intuitionistic},
\kl{bi-intuitionistic} and \kl{classical} logic.