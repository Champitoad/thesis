\subsection{Modifying rules}

\begin{figure*}
  \fontsize{10}{10.5}\selectfont
\begin{framed}
\renewcommand{\arraystretch}{2}
\begin{mathpar}
\begin{array}{c}
\identity \\[1em]

\R[\intro{i{\da}}]
    {\piq{}}
    {\Gamma, A \J A, \Delta}
\end{array}
\\
\begin{array}{c@{\quad}c}
\multicolumn{2}{c}{\flow} \\[1em]

% \R[\intro{s{-}}]
%     {\piq{S \seq} \mix \Gamma \J \Delta}
%     {\Gamma, (\piq{S}) \J \Delta}
% &
% \R[\intro{s{+}}]
%     {\Gamma \J \Delta \mix \piq{\seq S}}
%     {\Gamma \J (\piq{S}), \Delta}
% \\

\R[\intro{f{-}{\da}}]
    {\Gamma, I \piq{\Gamma', I \JB \Delta' \sep \cS} \Delta}
    {\Gamma, I \piq{\Gamma' \JB \Delta' \sep \cS} \Delta}
&
\R[\intro{f{+}{\da}}]
    {\Gamma \piq{\cS \sep \Gamma' \JB I, \Delta'} I, \Delta}
    {\Gamma \piq{\cS \sep \Gamma' \JB \Delta'} I, \Delta}
\\
\R[\intro{f{-}{+}{\da}}]
    {\Gamma, I \J (\Gamma', I \JB \Delta'), \Delta}
    {\Gamma, I \J (\Gamma' \JB \Delta'), \Delta}
&
\R[\intro{f{+}{-}{\da}}]
    {\Gamma, (\Gamma' \JB I, \Delta') \J I, \Delta}
    {\Gamma, (\Gamma' \JB \Delta') \J I, \Delta}
\\
\R[\intro{f{-}{-}{\ua}}]
    {\Gamma, I, (\Gamma', I \JB \Delta') \J \Delta}
    {\Gamma, (\Gamma', I \JB \Delta') \J \Delta}
&
\R[\intro{f{+}{+}{\ua}}]
    {\Gamma \J (\Gamma' \JB I, \Delta'), I, \Delta}
    {\Gamma \J (\Gamma' \JB I, \Delta'), \Delta}
\\
\R[\intro{f{-}{+}{\ua}}]
    {\Gamma, I \J (\Gamma', I \JB \Delta'), \Delta}
    {\Gamma \J (\Gamma', I \JB \Delta'), \Delta}
&
\R[\intro{f{+}{-}{\ua}}]
    {\Gamma, (\Gamma' \JB I, \Delta') \J I, \Delta}
    {\Gamma, (\Gamma' \JB I, \Delta') \J \Delta}
\\
\R[\intro{f{-}{-}{\da}}]
    {\Gamma, I, (\Gamma', I \JB \Delta') \J \Delta}
    {\Gamma, I, (\Gamma' \JB \Delta') \J \Delta}
&
\R[\intro{f{+}{+}{\da}}]
    {\Gamma \J (\Gamma' \JB I, \Delta'), I, \Delta}
    {\Gamma \J (\Gamma' \JB \Delta'), I, \Delta}
\end{array}
\and
\begin{array}{cc}
\multicolumn{2}{c}{\membrane} \\[1em]

\multicolumn{2}{c}{
\R[\intro{p}]
    {\Gamma \piq{\cS} \Delta}
    {\Gamma \piq{\cS \sep {\piq{}}} \Delta}
} \\
\R[\intro{p{-}}]
    {\piq{}}
    {\Gamma, (\piq{}) \J \Delta}
&
\R[\intro{p{+}}]
    {\piq{}}
    {\Gamma \J (\piq{}), \Delta}
\\

\multicolumn{2}{c}{
\R[\intro{a}]
    {\Gamma \piq{S} \Delta}
    {\Gamma \piq{{\piq{S}}} \Delta}
} \\
\R[\intro{a{-}}]
    {\Gamma, S \J \Delta}
    {\Gamma, (\piq{S}) \J \Delta}
&
\R[\intro{a{+}}]
    {\Gamma \J S, \Delta}
    {\Gamma \J (\piq{S}), \Delta}
\\
\end{array}
\\
\begin{array}{c@{\quad}c}
\multicolumn{2}{c}{\heating} \\[1em]

\R[\intro{\top{-}}]
    {\Gamma \J \Delta}
    {\Gamma, \top \J \Delta}
&
\R[\intro{\top{+}}]
    {\Gamma \J (\piq{}), \Delta}
    {\Gamma \J \top, \Delta}
\\
\R[\intro{\bot{-}}]
    {\Gamma, (\piq{}) \J \Delta}
    {\Gamma, \bot \J \Delta}
&
\R[\intro{\bot{+}}]
    {\Gamma \J \Delta}
    {\Gamma \J \bot, \Delta}
\\
\R[\intro{\land{-}}]
    {\Gamma, A, B \J \Delta}
    {\Gamma, A \land B \J \Delta}
&
\R[\intro{\land{+}}]
    {\Gamma \piq{\seq A \sep \seq B\ } \Delta}
    {\Gamma \seq A \land B, \Delta}
\\
\R[\intro{\lor{-}}]
    {\Gamma \piq{A \seq \sep B\seq} \Delta}
    {\Gamma, A \lor B \seq \Delta}
&
\R[\intro{\lor{+}}]
    {\Gamma \J A, B, \Delta}
    {\Gamma \J A \lor B, \Delta}
\\
\R[\intro{{\limp}{-}}]
    {\Gamma, A \limp B \piq{\seq A \sep B\seq} \Delta}
    {\Gamma, A \limp B \seq \Delta}
&
\R[\intro{{\limp}{+}}]
    {\Gamma \J (A \seq B), \Delta}
    {\Gamma \J A \limp B, \Delta}
\\
\R[\intro{{\lsub}{-}}]
    {\Gamma, (A \seq B) \J \Delta}
    {\Gamma, A \lsub B \J \Delta}
&
\R[\intro{{\lsub}{+}}]
    {\Gamma \piq{\seq A \sep B\seq} A \lsub B, \Delta}
    {\Gamma \seq A \lsub B, \Delta}
\\
\R[\intro{\forall{-}}]
    {\Gamma, \forall x. A, \subst{A}{t}{x} \J \Delta}
    {\Gamma, \forall x. A \J \Delta}
&
\R[\intro{\forall{+}}]
    {\Gamma \J A, \Delta}
    {\Gamma \J \forall x. A, \Delta}
\\
\R[\intro{\exists{-}}]
    {\Gamma, A \J \Delta}
    {\Gamma, \exists x. A \J \Delta}
&
\R[\intro{\exists{+}}]
    {\Gamma \J \subst{A}{t}{x}, \exists x. A, \Delta}
    {\Gamma \J \exists x. A, \Delta}
\end{array}
\end{mathpar}

In the {\kl{\forall{+}}} and {\kl{\exists{-}}} rules, $x$ is not free in
$\Gamma$, $\Delta$ and $\J$.
\end{framed}

  \caption{Rules for the \kl{invertible} \kl{bubble calculus} \kl{B_{inv}}}
  \labfig{sequent-B-inv}
\end{figure*}

An important thing to note, is that all the rules of \kl{DBiInt} are
\emph{\kl{invertible}}\sidenote{Lemma 5.2.4 in Postniece's thesis
  \cite{postniece_proof_2010}.}. Thus it follows immediately from
\reflemma{simulation-dbiint} that one can just take the translation of the rules
of \kl{DBiInt} in system $\sysB$, and get a complete, fully \kl{invertible}
calculus. But this would be a waste of the expressive power and nice properties
of system $\sysB$, like linearity and locality.

Instead, we will target precisely the non-\kl{invertible} rules of system $\sysB$,
and modify only those. From the proof of \reflemma{bubbles-local-soundness}, we
can identify which rules of \kl{system~B} are \kl{invertible}, and which are
probably not. Indeed if the soundness of a rule only relies on a chain of
equivalences, then it is necessarily \kl{invertible}. On the contrary if it relies on
an inequality, then it is probably not \kl{invertible}\sidenote{To ensure that it is
not \kl{invertible}, we would need additionally to find a counter-model that
invalidates the converse inequality.}.

\begin{fact}[Invertibility of \kl{system~B}]
  All rules in the fragment $\mathbb{I} \cup \{\rsf{c{-}},\rsf{c{+}}\} \cup
    \mathbb{M} \cup \mathbb{H} \setminus \{\rsf{{\limp}{-}},\rsf{{\lsub}{+}}\}$ of
  \kl{system~B} are \kl{invertible}.
\end{fact}

Thus the only remaining rules of \kl{system~B} that are (most probably) not
\kl{invertible} are the \kl{weakening} rules $\{\rsf{w{-}},\rsf{w{+}}\}$, all the
$\mathbb{F}$-rules, and the $\mathbb{H}$-rules $\{\rsf{{\limp}{-}},
\rsf{{\lsub}{+}}\}$ that \emph{apply} an implication/exclusion\sidenote{For
quantifier rules, we conjecture that as in \kl{sequent calculus}, the rules
$\{\rsf{\forall{+}}, \rsf{\exists{-}}\}$ are \kl{invertible}, while the rules
$\{\rsf{\forall{-}}, \rsf{\exists{+}}\}$ are not. And as in \kl{sequent calculus},
this can be remedied by systematically duplicating the instantiated formula.}.
In \reffig{sequent-B-inv} we define the \kl{B_{inv}} calculus, which results
from the following modifications to the previous rules:
\begin{description}
  \item[\textbf{Weakening}]
    Here we follow a standard technique in \kl{sequent calculus}, that merges the
    \kl{weakening} rule in all \emph{terminal} rules of the calculus (i.e. rules with
    no premisses). In \kl{bubble calculi}, the notion of premiss is captured by
    \kl{neutral} bubbles; thus we incorporate \kl{weakenings} in all rules that solve
    \kl{subgoals} by closing solutions with no \kl{neutral} bubbles. Those are the rules
    $\{\rsf{i{\da}},\rsf{p{-}},\rsf{p{+}}\}$, which for the occasion have
    also been generalized to arbitrary solutions. Indeed in system $\sysB$ we
    restricted them to open solutions to make them \emph{local}, but here the
    \kl{weakenings} break locality anyway, and the general version improves
    \emph{factorizability} by solving instantly all \kl{subgoals} inside $\J$.

  \item[\textbf{Flow}]
    As shown by the simulation of \reflemma{simulation-dbiint}, the
    \emph{propagation} rules of \kl{DBiInt} combine an instance of
    \emph{\kl{contraction}} followed by the application of a flow rule on the
    duplicated formula. Thus we can make all $\mathbb{F}$-rules of system
    $\sysB$ \kl{invertible} by systematically duplicating the moved formula, although
    this breaks \emph{linearity}.

    A downside of propagation rules in the style of \kl{DBiInt}, is that they
    create a lot of unnecessary copies of the moved formula $A$. Often, one will
    want to move $A$ in a \kl{subgoal}/supergoal at a distance $n$ in the proof tree,
    with $n > 1$. Usually this would be performed by $n$ applications of
    $\mathbb{F}$-rules, which by linearity indeed just move the formula. But with
    propagation rules, $n$ copies of $A$ will be created, with one copy in each
    \kl{subgoal} met on the path to the destination.

    To prevent this, one would need a way to copy formulas at an arbitrary
    distance. This can be done with \kl{inference rules} that are \emph{doubly} deep,
    by encoding the path to the destination as a second solution \kl{context} inside
    the \kl{context} where the rule is applied\sidenote{Such rules are sometimes
    called \emph{super-switch} rules in the \kl{deep inference} literature, see for
    instance Chapter 8, Section 2.1 of \cite{guenot_nested_2013}.}. It turns out
    to be hard to express in bubbles, because this requires a syntactic way to
    describe \kl{contexts} that correspond to valid flow paths of arbitrary
    length\sidenote{This problem is solved trivially in the flower calculus
    (\refch{flowers}), by using so-called \emph{pollination} rules.}. But in
    principle it should be feasible, and would enable a more comfortable use in
    a \kl{Proof-by-Action} setting.

    Note also that we removed the \rsf{f{\ua}} rule of \reffig{sequent-B}.
    Indeed even after turning it into a propagation rule, the moved copy of the
    duplicated \kl{subgoal} $S$ cannot be weakened because it lives in a \kl{neutral}
    bubble. Thus the rule stays non-\kl{invertible}, and cannot be included in
    \kl{B_{inv}}. Fortunately, we showed that it is \kl{admissible} in
    \refthm{bubbles:cut-admissibility}, so this is not problematic.

  \item[\textbf{Implication/Exclusion}]
    The last source of non-invertibility is the $\mathbb{H}$-rules
    \rsf{{\limp}{-}} and \rsf{{\lsub}{+}}, that respectively allow to use an
    implication hypothesis, and prove an exclusion conclusion.
    % This problem is has been known in the \kl{sequent calculus} literature at least
    % since \sidecite{nikolai1952vorob}, and a solution is proposed in
    % \sidecite{dyckhoff_contraction-free_1992} by splitting the rule
    % \rsf{{\limp}{-}} into  
    Here we can just duplicate the implication/exclusion formula, as in the
    \kl{introduction rules} of \kl{DBiInt}. Also like in \sys{DBiInt}, we removed
    the \kl{contraction} rules \rsf{c{-}} and \rsf{c{+}}, which are now merged with
    these two rules as well as the $\mathbb{F}$-rules. Although \kl{contraction}
    rules are \kl{invertible}, they induce a lot of complexity in proof search,
    because it is hard to predict the (occurrences of) formulas that need to be
    duplicated, and one can duplicate \emph{ad infinitum}. Thus it is preferable
    to design a calculus where they are \kl{admissible}. But unlike what is done in
    \kl{DBiInt}, we did not incorporate \kl{contraction} in other
    $\mathbb{H}$-rules. Thus we cannot simulate exactly all the \kl{introduction
    rules} of \kl{DBiInt} in \sys{B_{inv}}.

\end{description}

\begin{remark}
  We also changed the \rsf{\bot{-}} and \rsf{\top{+}} rules, so that they create
  \kl{polarized}, closed empty solutions. This makes them both local, and generic
  with respect to the status of the ambient solution. The previous version can
  then be simulated by combination with the popping rules \rsf{p{-}} and
  \rsf{p{+}}.
\end{remark}

These modifications only change superficially the proof of soundness, and thus
we do not redo it. As for completeness, we would need to prove that the
\kl{contraction} rules are \kl{admissible}, in order to solve the aforementioned problem
of simulating \kl{DBiInt}'s \kl{introduction rules}:

\begin{lemma}[Admissibility of contraction]\lablemma{admissibility-contraction}
  % If $\prov{\kl{B_{inv}} \cup \{\rsf{c{-}},\rsf{c{+}}\}} S$, then
  % $\prov{\kl{B_{inv}}} S$.
  \sbr
  \begin{itemize}
    \item If $\prov{\kl{B_{inv}}} S\select{\Gamma, A, A \J \Delta}$, then
          $\prov{\kl{B_{inv}}} S\select{\Gamma, A \J \Delta}$.
    \item If $\prov{\kl{B_{inv}}} S\select{\Gamma \J A, A, \Delta}$, then
          $\prov{\kl{B_{inv}}} S\select{\Gamma \J A, \Delta}$.
  \end{itemize}
\end{lemma}

Note that it is sufficient to prove admissibility of \kl{contraction} on
formulas, rather than on arbitrary \kl{items}. Indeed we only need it to simulate the
\kl{introduction rules} of \kl{DBiInt}, which always duplicate formulas. For now
we only conjecture completeness of \kl{B_{inv}}, since it not clear what method
should be used to prove \reflemma{admissibility-contraction}. In her thesis
\cite{postniece_proof_2010} (Lemma 5.2.3), Postniece does a proof by recurrence
on the depth of the derivation, relying on the fact that all \kl{introduction
rules} of \kl{DBiInt} preserve the principal formula; but this is precisely what
we are trying to avoid with our version of the rules. Of course, if we either
give up on this constraint or include \kl{contraction} rules in \kl{B_{inv}}, then we
immediately get our desired result: \kl{B_{inv}} is a fully \kl{invertible} calculus,
where the same fragments as system $\sysB$ capture \kl{intuitionistic},
\kl{dual-intuitionistic}, \kl{bi-intuitionistic} and \kl{classical} logic.

\subsection{Semi-automated proof search}\labsubsec{bubbles-search}

In the \kl{intuitionistic} (propositional) fragment of \kl{B_{inv}}, a canonical way
to search for a proof of a formula $A$ consists in the following $5$
\emph{phases}, applied successively in a loop until the empty closed solution is
reached:
\begin{description}
  \item[\textbf{Decomposition}] Decompose $A$ by applying recursively
    $\mathbb{H}$-rules, until either atoms, \kl{negative} implications
    $\hypo{\limp}$, \kl{negative} disjunctions $\hypo{\lor}$, or \kl{positive}
    conjunctions $\conc{\land}$ are reached. Indeed since the \rsf{{\limp}{-}}
    rule duplicates the implication, it cannot be used to decompose it.
    
    As for the \rsf{\lor{-}} and \rsf{\land{+}} rules, they can only be applied
    when the formula is in an \emph{open} solution. A first option is to let the
    system automatically distribute them in all open subsolutions that are
    reachable, so that it can keep decomposing them. But this might create an
    explosion in the number of created \kl{subgoals}. Another option is to let the
    user manually decompose them. We believe this second option is preferable,
    if one wants to keep control over the proof search process. Indeed, it is
    only natural that the user should be able to choose which \emph{cases} to
    consider when building a proof.

  \item[\textbf{Absorption}] Apply the absorption rules
    $\{\rsf{a},\rsf{a{-}},\rsf{a{+}}\}$ wherever possible. This will prevent
    atoms from being unnecessarily stuck on \kl{neutral} membranes in the next phase.
    This phase can also be trivially automated.

  \item[\textbf{Linking}] Try to bring together every pair of dual atoms, so
    that they annihilate each other in an instance of the \rsf{i{\da}}
    rule. This is reminiscent of our \emph{drag-and-drop} actions of
    \refch{pba}. In a touch-based \kl{GUI}, rather than dragging a complex formula
    onto another complex formula, one could \emph{pinch} together the two atoms:
    if there is no $\mathbb{F}$-law stucking one of the atoms on some membrane
    (orange arrows in \reffig{bubbles-porosity}), then the pinch succeeds, by
    closing the subsolution at the location where it ends with
    \rsf{i{\da}}. Thus the user can choose the \kl{subgoal} to solve by
    controlling the destination of the pinch, which can be seen as a more
    symmetric and powerful version of \kl{DnD} actions. Generally though, one will
    want to apply the following \emph{rule of thumb} (pun intended):
    \begin{fact}[Rule of thumb]
      When linking a pair of dual atoms, follow these steps:
      \begin{enumerate}
        \item put your \emph{thumb} on the \emph{outermost} atom, and your
              \emph{index} on the \emph{innermost} atom;
        \item try to bring your index to your thumb;
        \item if you get stuck on a membrane, try to bring your thumb to your
              index;
        \item if you again get stuck, then give up on this pair.
      \end{enumerate}
    \end{fact}
    The point of this heuristic, is that it should maximize the
    \emph{factorization} of the proof: when it succeeds, it will solve the
    \kl{subgoal} that is located closest to the root of the \kl{goal}, maximizing the size
    of the pruned branch, and thus the number of \kl{subgoals} solved in one go. It
    can also be used to completely automate this phase.

  \item[\textbf{Popping}] Pop every closed empty \kl{bubble} in the \kl{goal} with the
  rules $\{\rsf{p},\rsf{p{-}},\rsf{p{+}}\}$. This phase can also be trivially
  automated, and corresponds to the unit elimination phase in \kl{subformula linking}
  (\refsec{action}).

  \item[\textbf{Application}] When there are no more pairs of dual atoms, or all
    the remaining pairs have been given up (last step of the rule of thumb), let
    $S$ be the current \kl{goal}, and
    $$\imps{S} \defeq \compr{(S_0\hole, A, B, \J, \Delta)}{S = S_0\select{\Gamma,
          A \limp B \J \Delta} \text{ for some $\Gamma$}}$$

    If $\imps{S} = \emptyset$, then $S$ should not be provable, and we can stop
    the proof search procedure. Otherwise for each $(S_0\hole, A, B, \J, \Delta)
    \in \imps{S}$, we might need to apply the \rsf{{\limp}{-}} rule on $\hypo{A
    \limp B}$, either directly in $S_0\hole$ if $\J = {\seq}$ and $\max_{I \in
    \Delta}{\soldepth{I}} = 0$ (\refdef{item-depth}), otherwise in some subgoal
    $T\select{U} \in \J \cup \Delta$. This is where the proof needs
    \emph{insight}, because it is not clear if the antecedant $A$ will be
    provable with the \kl{context} available in $S_0\hole$ or in one of the $T\hole$,
    or if the hypothesis $B$ is even needed at all.

    A first possibility is to let the user rely on her intuition, by choosing
    manually a specific subsolution in $\imps{S}$ to apply the \rsf{{\limp}{-}}
    rule upon. Additionally, she might need to determine a \kl{subgoal} $T\select{U}$
    in which $A$ is provable, and first duplicate $\hypo{A\limp B}$ in $T\hole$
    before applying \rsf{{\limp}{-}}. This will always be possible with the
    $\mathbb{F}$-rules \rsf{f{-}{\da}} and \rsf{f{-}{+}{\da}}.
    Ideally, she would also pick the most general $T\hole$ to factorize the
    proof, by minimizing its depth $\soldepth{T\hole}$ (\refdef{solctx-depth}).

    A second possibility is to duplicate eagerly every $\hypo{A \limp B}$ of
    $\imps{S}$ in every open subsolutions of $S$ where it can be so, and then
    apply \rsf{{\limp}{-}} on all the newly created copies. To avoid an
    explosion of the size of $S$, the system should mark all the copies as
    \emph{used}, so that during the next \textbf{Application} phases, all the
    already \emph{used} copies are ignored, and only the original occurrence of
    $\hypo{A \limp B}$ is considered.

    Then we can restart the procedure, by applying the \textbf{Decomposition}
    phase to every copy of $A$ and $B$.
\end{description}

\begin{remark}
  By adopting $\mathbb{H}$-rules in the style of \kl{DBiInt}'s \kl{introduction
  rules}, we would make the \textbf{Decomposition} phase, and thus the whole
  procedure inoperable, since the \textbf{Linking} phase depends crucially on
  it. Allowing \kl{contraction} rules would also jeopardize the potential
  completeness of the procedure, because \kl{contraction} might be needed at
  unpredictable moments, and on unpredictable formulas.
\end{remark}

A strength of our proof search procedure, compared to the state-of-the-art in
other formalisms, is that most of its automation preserves the \emph{size}
(number of atoms) and the \emph{structure} of the \kl{goal}:
\begin{itemize}
  \item In the \textbf{Decomposition} phase, if we opt out of the automatic
  distribution of \kl{negative} $\hypo{\lor}$ and \kl{positive} $\conc{\land}$, then the
  system will only apply $\mathbb{H}$-rules that split logical connectives, and
  create a \emph{partition} of the atoms of the \kl{goal} by enclosing them in
  bubbles. Thus the size of the \kl{goal} is kept intact, and the structure modified
  but in a controlled, local way.

  \item In the \textbf{Absorption} phase, we simply merge some membranes
  together, preserving both the size and the structure of the \kl{goal}.

  \item In the \textbf{Linking} phase, the particular way in which we use
  $\mathbb{F}$-rules ensures that we only decrease the number of atoms. Indeed,
  if we assume as discussed earlier that we have ``super-flow'' rules that copy
  at a distance, then either:
  \begin{enumerate}
    \item the link is successful, and the two created copies of atoms are
  immediately destroyed by the \rsf{i{\da}} rule. Then the solved
  subsolution is entirely pruned out, decreasing the size of the \kl{goal}; or
    \item the link fails, but then we can instantly ``undo'' it. Or rather, one
  should consider that rules are applied only when the link is successful.
  \end{enumerate}

  \item In the \textbf{Popping} phase, entire branches of the \kl{goal} are pruned
  out, decreasing the size of the \kl{goal}.
\end{itemize}

Then only the automation of the \textbf{Application} phase (and part of the
\textbf{Decomposition} phase) is susceptible of both significantly increasing
the size of the \kl{goal}, and altering its global structure. But as is the case for
every phase, the user can easily opt out of this automation, and do the
reasoning manually when it is necessary to keep the \kl{goal} understandable by
humans. Typically in an educational setting, it should be quite instructive to
have the ability to perform \textbf{Decomposition} and \textbf{Linking} by hand
(literally).

\subsection{Failure of full iconicity}

Because of the implicit \kl{contraction} in the rules \rsf{{\limp}{-}} and
\rsf{{\lsub}{+}}, one cannot fully decompose a formula into an equivalent
solution by deterministically applying a sequence of $\mathbb{H}$-rules (and
possibly $\mathbb{F}$-rules, to distribute \kl{positive} conjunctions and \kl{negative}
disjunctions). Thus \kl{B_{inv}} fails to be \emph{fully \kl{iconic}}, because it
relies on the \emph{\kl{symbolic}} connectives $\limp$ and $\lsub$ to represent
logical statements.

\begin{marginfigure}
  $$
\begin{array}{r@{\quad\mapsto\quad}l@{\vspace{0.75em}}}
  \hypo{\top} & \hbubble{\phantom{\top}} \\
  \hypo{\bot} & \hbbubble{\phantom{\bot}} \\
  \hypo{A \land B} & \hbubble{\hypo{A}~~~\hypo{B}} \\
  \hypo{A \lor B} & \hbbubble{\bubble{\hypo{A}}~~~\bubble{\hypo{B}}} \\
  \hypo{A \limp B} & ? \\
  \hypo{A \lsub B} & \hbubble{\hypo{A}~~~\conc{B}} \\
\end{array}
$$
$$
\begin{array}{r@{\quad\mapsto\quad}l@{\vspace{0.75em}}}
  \conc{\top} & \cbbubble{\phantom{\top}} \\
  \conc{\bot} & \cbubble{\phantom{\bot}} \\
  \conc{A \land B} & \cbbubble{\bubble{\conc{A}}~~~\bubble{\conc{B}}} \\
  \conc{A \lor B} & \cbubble{\conc{A}~~~\conc{B}} \\
  \conc{A \limp B} & \cbubble{\hypo{A}~~~\conc{B}} \\
  \conc{A \lsub B} & ?
\end{array}
$$
  \caption{Mapping of formulas to equivalent solutions}
  \labfig{bubbles-native}
\end{marginfigure}

This can be understood as resulting from the inability of solutions to represent
natively \emph{\kl{negative} implications} and \emph{\kl{positive} subtractions}, although
they can represent natively all other polarizations of connectives. This is
illustrated by the mapping of \reffig{bubbles-native} from \kl{polarized} formulas to
equivalent solutions, which is really just the $\mathbb{H}$-rules of system
$\sysB$ (\reffig{graphical-B}) where the right-hand solution is enclosed in a
\kl{bubble} of the corresponding \kl{polarity}. The reader can easily check that if $A$ is
mapped to $S$, then $\psint{A} \semequiv \psint{S}$.

\begin{marginfigure}
  $$
  \R[\rsf{e}]
    {\Gamma, (\Delta, A \seq B) \seq C}
    {\Gamma, (\Delta \seq A \limp B) \seq C}
  $$
  \caption{\kl{Left introduction rule} for $\limp$ in \kl{JN}}
  \labfig{jn-rule-e}
\end{marginfigure}

This seems to be a fundamental limitation of system $\sysB$, caused by its
symmetric treatment of implication and subtraction. For instance in the \kl{nested
sequent} calculus \kl{JN} of Guenot for implicative logic \cite[Chapter
3]{guenot_nested_2013}, which is fully decomposable, \kl{nested sequents} that appear
in negative \kl{contexts} are interpreted as implications, as illustrated by the
\kl{left introduction rule} \rsf{e} (\reffig{jn-rule-e}). But we cannot do this
in system $\sysB$, because this would conflict with the subtractive reading of
\kl{negative} solutions, i.e. $\nsint{A \seq B} = \nsint{A} \lsub \nsint{B}$. In
\refch{flowers}, the problem will also be solved through an asymmetric treatment
of \kl{nested sequents}, capturing only \kl{intuitionistic} logic instead of
\kl{bi-intuitionistic} logic. But this is a small price to pay, since
\kl{bi-intuitionistic} logic does not (currently) have any applications in the
realm of interactive theorem proving.