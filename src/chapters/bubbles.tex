\setchapterpreamble[u]{\margintoc}
\chapter{Bubble Calculi}
\labch{bubbles}

We introduce a new kind of nested sequent proof systems dubbed \emph{bubble
calculi}. Inspired by the \emph{membrane} mechanism of the chemical abstract
machine \sidecite[10em]{berry_chemical_1989}, so-called \emph{bubbles}
internalize the notion of \emph{subgoal} inside sequents, rather than through
the tree structure induced by inference rules. This allows for a more
hierarchical representation of the proof state, where contexts can be shared
between different subgoals. In addition to the usual textual syntax, bubble
calculi can be expressed in a graphical syntax, where logical meaning is
captured by \emph{physical} constraints on diagrammatic manipulations, instead
of \emph{virtual} restrictions on available inference rules. In the chemical
metaphor, \emph{intuitionism} is then characterized as the phenomenon of
\emph{repulsion} between objects that have the same polarity.

The chapter is organized mostly chronologically, following the evolution of our
idea of a bubble calculus through the addition of new features. We start in
\refsec{chemical} with the genesis of the idea, coming from the observation that
our Proof-by-Action paradigm (\refch{pba}) lends itself quite naturally to a
metaphorical interpretation, where actions are seen as \emph{chemical}
reactions. In \refsec{first-calculus} we introduce a first proof system for
intuitionistic logic, based on multiset rewriting rules for so-called
\emph{solutions} made of bubbles and formulas. Then in \refsec{non-determinism}
we motivate our quest for a system where all introduction rules for logical
connectives are \emph{invertible}, in order to reduce non-determinism in proof
search. To that effect, we extend in \refsec{colors} the syntax of solutions so
that they can hold multiple parallel conclusions that themselves contain
solutions, corresponding to a \emph{polarization} of bubbles. This gives rise in
\refsec{polarized-calculus} to the full \emph{polarized bubble calculus}, based
on a novel distinction between \emph{branching} and \emph{non-branching}
solutions, corresponding respectively to the branchings of
\sidecite{guenot_nested_2013} and more traditional nested sequents. In
\refsec{bubbles-soundness} we show that by adding or removing 4 inference rules
that define the \emph{porosity} of bubbles, one gets 4 subsystems that capture
respectively intuitionistic, dual-intuitionistic, bi-intuitionistic and
classical logic. Finally in \refsec{bubbles-completeness} we support this claim
by showing that our system is not only sound, but also complete with respect to
standard sequent calculi.


\section{The chemical metaphor}\labsec{chemical}

\section{A first calculus}\labsec{first-calculus}

\section{Reducing non-determinism}\labsec{non-determinism}

\section{Coloring bubbles}\labsec{colors}

\section{Polarized bubble calculus}\labsec{polarized-calculus}

\section{Soundness}\labsec{bubbles-soundness}

\section{Completeness}\labsec{bubbles-completeness}