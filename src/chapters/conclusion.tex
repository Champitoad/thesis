\setchapterpreamble[u]{\margintoc}
\chapter{Conclusion}
\labch{concl}

\todo{ Future work: adequation between textual and graphical versions of various
calculi}

\todo{ Explore the question of the logics associated to various fragments of the
$\mathbb{F}$-rules of system \sys{B}.}

\section{Conclusion and Perspectives}

This work started as a very practical effort. Discovering and
understanding the links with more theoretically grounded approaches,
and especially deep inference, made us aware that there may be more
proof theoretical depth to this idea than we first thought. But, most
importantly, adapting the logical rules and tools of deep inference to
the practical question we encountered, allowed us to structure our
proposal and to define the ``right'' behavior for the system. We were
able to extend the deep inference approach to the use of
equalities~\refsec{equality}, which may be an originality of this
work. It seems imaginable to proceed similarly with other mathematical
relations. 

More generally, we hope that our treatment of equality can be the
start for providing graphical or gestural tools to perform algebraic
transformations of expressions (be there in the conclusion or in
hypotheses). As mentioned above, Window Inference could serve as an
inspiration here. This seems promising to us, since describing such a
transformation is notoriously tedious when using textual commands.

Even a small prototype allowed us to experiment on some non-trivial
examples and to make some first encouraging experiences. In various
cases, like the one described in section~\refsec{edukera}, we have
observed shorter or more straightforward proofs than in textual
provers. Another nice point is that some syntactical details, like the
name of proof tactics become irrelevant in the gestural setting. More
generally, we feel that using such a system, one may indeed develop a
good intuition for the behavior of the logical items. But this is
obviously a user interface or user experience question which is too
early to quantify. Also, some novel questions appear when implementing
such a graphical system: what are the good user interface choices, how
to obtain a good look-and-feel, what visual feedback the system should
provide\dots

On the other hand, we should acknowledge that certain styles of proofs,
where a large number of subcases can be immediately solved through the
same short textual tactic sequence, may be less well suited for the
gestural approach (the \ssreflect~\cite{SSR} dialect for Coq is very
well suited for such cases).


Among future lines of work, it will be interesting to explore how some
automation fits into this framework. One example is the \emph{point-and-shoot}
paradigm of \cite{PbP}. But the DnD feature could open up new possibilities,
like having the system perform some automated deduction to prove equivalences or
implications between the two squared formulas (which would thus no longer be
required to be strictly equal or unifiable).

Another obvious and important point to be tackled next is to provide a
smooth way to invoke a library of lemmas in a graphical proof. We
believe this could raise some interesting questions.

An also promising line of work is to extend our approach to classical
logic. A point being that the graphical setting could smoothly handle
multiple conclusions with less spurious overhead than text commands.


An important difference with the days of the
pioneering work on proof-by-pointing is that developers can now rely on
powerful and standardized libraries, which make the construction of
user interfaces much faster and easier, giving new room for
experimentation and proposals. But bringing everything together in
simple commands remains a complicated theoretical and development task.
