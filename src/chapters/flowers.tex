\setchapterpreamble[u]{\margintoc}
\chapter{Flower Calculus}
\labch{flowers}

\todo{Make this into a citation}

In a certain flower garden, each flower was either red, yellow,
or blue, and all three colors were represented. A statistician
once visited the garden and made the observation that what-
ever three flowers you picked, at least one of them was bound
to be red. A second statistician visited the garden and made
the observation that whatever three flowers you picked, at
least one was bound to be yellow.
Two logic students heard about this and got into an ar-
gument. The first student said: "It therefore follows that what-
ever three flowers you pick, at least one is bound to be blue,
doesn't it?" The second student said: "Of course not!"
% Which student was right, and why?

--- Raymond Smullyan, "The Flower Garden", in "To Mock a Mockingbird"

\todo{IDEA: \emph{superposition of polarities} in existential graphs. While a
cut in bubble calculi is the insertion of \emph{two} occurrences of the same
formula with \emph{distinct} polarities, the insertion rule only introduces
\emph{one} occurrence of the formula, which plays either a negative role when
it is iterated, or a positive role when it is deiterated}.

\section{Syntax}

\section{Calculus}

\section{Semantics}

\section{Soundness}

\section{Completeness}

\section{Automated proof search}

\section{Towards a Curry-Howard correspondence}