\setchapterpreamble[u]{\margintoc}
\chapter{Existential Graphs}
\labch{eg}

C. S. Peirce is famous for his contributions to symbolic logic, including among
others his eponymous law for classical logic, and his pioneering work on the
algebra of relations and quantification \cite{peirce_algebra_1885}. But far less
widespread are his achievements in the realm of graphical logic, or \emph{iconic
logic} as Shin calls it \cite{10.7551/mitpress/3633.001.0001}. He dedicated a
large chunk of his life investigating diagrammatic systems, starting in 1882
with the \emph{entitative graphs} and culminating with the \emph{existential
graphs}, which he developed from 1896 until his death in 1914
\cite{Roberts+1973}. Interestingly, Peirce perceived existential graphs
(thereafter ``EG'') as his \textit{``chef d'oeuvre''}, and that they
\textit{``ought to be the logic of the future''}\sidenote[][3em]{Both citations
are sourced in page 11 of \cite{Roberts+1973}.}.

Recent works have started to realize this vision: for example Sowa based his
conceptual graphs for computerized knowledge representation on EG
\cite{sowa_conceptual_1976}, Brady, Trimble
\cite{brady_categorical_2000}\cite{brady_string_nodate} and Haydon, Sobociński
\cite{pietarinen_compositional_2020} proposed various reconstructions of EG
through the lens of \emph{topology} and \emph{category theory}, and Melliès and
Zeilberger \cite{mellies_bifibrational_2016} refined the interpretation of Brady
and Trimble by making further connections with \emph{linear
logic}\cite{girard-linear-1987}. The full story has yet to be told, but we hope
that our work will constitute one more step towards the vision Peirce had in
mind.

In this chapter, we propose a self-contained exposition of EG, that tries at the
same time to be faithful to the original presentation of the systems by Peirce,
and more modern in some aspects of their formalization. The goal will be to
familiarize the reader with the unique approach to proofs inherent to EG, which
can be difficult to relate to more standard frameworks like Hilbert and Gentzen
proof systems, and even deep inference proof systems like the calculus of
structures. This shall prove useful to get a good understanding of the
historical and technical foundations behind our \emph{flower calculus}, to be
introduced in \refch{flowers}.

The chapter is organized as follows: we start in \refsec{alpha} by presenting
the diagrammatic syntax of the system \sys{Alpha} of EG for classical
propositional logic. In \refsec{illative}, we introduce the inference rules of
\sys{Alpha} for manipulating existential graphs, called \emph{illative
transformations} by Peirce. In \refsec{multisets}, we give an equivalent
formulation of the syntax and rules of \sys{Alpha} as a \emph{multiset}
rewriting system. In \refsec{eg-soundness}, we take advantage of our
reformulation to give a simpler proof of soundness for \sys{Alpha} that what can
be found in the literature. In \refsec{eg-completeness} we give a novel
syntactic proof of completeness for \sys{Alpha} by simulation of a calculus of
structures. In \refsec{beta}, we illustrate the original mechanism of
\emph{lines of identity} used by Peirce to handle quantifiers and equality in
his \sys{Beta} system. We end in \refsec{gardens} by showing how to recast lines
of identity in a more traditional binder-based syntax.

% \todo{IDEA (where should we put it?): \emph{superposition of polarities} in
% existential graphs. While a cut in bubble calculi is the insertion of \emph{two}
% occurrences of the same formula with \emph{distinct} polarities, the insertion
% rule only introduces \emph{one} occurrence of the formula, which plays either a
% negative role when it is iterated, or a positive role when it is deiterated}


\section{\sys{Alpha} graphs}\labsec{alpha}

Peirce designed in total three systems of EG, which he called respectively
\sys{Alpha}, \sys{Beta} and \sys{Gamma}. They were invented chronologically in
that order, which also captures their relationship in terms of complexity:
\sys{Alpha} is the foundation on which the other systems are built, and can
today be understood as a diagrammatic calculus for classical
\emph{propositional} logic. As we will see in \refsec{Quantification},
\sys{Beta} corresponds to a variable-free representation of \emph{predicate}
logic without function symbols, and with primitive support for \emph{equality}.
The last system \sys{Gamma} is more experimental, with various unfinished
features that have been interpreted as attempts to capture \emph{modal}
\sidecite{zeman_graphical_1964} and \emph{higher-order} logics\sidenote{A less
known fact is that at the end of his life, Peirce pushed his experimentations
beyond the scope of \emph{logic} in the contemporary sense of the word (it was
already the case for modal logic in his time), with so-called \emph{tinctured
existential graphs} (Chapter 6 in \cite{Roberts+1973}). Roughly, the idea was to
represent a variety of \emph{modes of expression} with different background
shades on the canvas where sentences are scribed, not unlike our graphical
depiction of the \emph{status} of solutions in \reffig{graphical-B}. In addition
to the usual act of asserting the truth of a proposition, one could for instance
express a \emph{subjective} or \emph{objective} possibility, or signify an
interrogative or imperative mood, all by using different colors. For print in
publications, he would in fact use \emph{heraldic tinctures} instead of colors,
hence the ``tinctured'' qualificative. The precise meaning and purpose of
tinctured EG remains elusive to this day, and might constitute the most esoteric
part of his work. We see a possible connection with the type-theoretical concept
of \emph{judgment}, which was rehabilitated by Martin-Löf from the philosophy of
Kant \cite{Martin-Lof1996-MAROTM-7}, who was of great influence on Peirce's own
philosophy \cite{sep-peirce}.}.

The most fundamental concept of \sys{Alpha} is the \emph{sheet of assertion},
denoted by $\SA$ thereafter. It is the space where statements are scribed by the
reasoner, typically a sheet of paper or a blackboard. In a proof assistant, this
would either be the buffer of a text editor holding the theory the user is
developing, or the proof view displaying goals to be proved, depending on who
the reasoner is (the user or the computer, respectively). This last analogy
suggests an important property of $\SA$: it must offer a \emph{virtually
infinite} amount of space, so that one can perform as much reasoning as needed.
Just like a Turing machine has an infinite tape, so that one can perform as much
computation as needed.

As its name indicates, scribing a statement on $\SA$ amounts to \emph{asserting
its truth}. Thus very naturally, the empty $\SA$ where nothing is scribed will
denote vacuous truth, traditionally symbolized by the formula
$\top$\sidenote{Peirce had an \emph{externalist} conception of truth, where the
assertions made by the \process{Graphist}, i.e. the person scribing on $\SA$,
refer to the world \emph{outside} of the sheet (the universe of discourse). By
not scribing anything, the \process{Graphist} thus refrains from asserting
anything about the world, only assuming implicit truths. This is illustrated by
the following quote from \cite{Roberts+1973}, reminiscent of an interaction
between a proof assistant (the \process{Graphist}) and her user (\process{the
Interpreter}): \textit{``One of the parties, called the Graphist, is responsible
for scribing the original graphs at the beginning of the investigation or
discussion; the other, called the Interpreter, draws inferences from these
graphs by changing them in accordance with the permissions of the system. The
[...] sheet, before anything is scribed on it, represents whatever is taken for
granted at the outset by the Graphist and Interpreter.''}.}.

As we know from natural deduction, asserting the truth of the conjunction $P
\land Q$ of two propositions $P$ and $Q$, amounts to asserting the truth of $P$
and the truth of $Q$. In \sys{Alpha}, there is no need to introduce the symbolic
connective $\land$, since one can just write both $P$ and $Q$ at distinct
locations on $\SA$:
$$P~~~Q$$
More generally, one might consider any two portions $G$ and $H$ of $\SA$, and
interpret their \emph{juxtaposition} $G~H$ as signifying that we assert the
truth of their conjunction. This leads us to formulate the first fundamental
principle of \sys{Alpha}:
$$\text{\emph{Any portion of $\SA$ is a graph.}}$$
In particular, the entire $\SA$ itself is a graph.

Asserting the truth of the negation $\neg P$ of a proposition $P$, amounts to
\emph{denying} the truth of $P$. Using the original notation of Peirce, this is
done in \sys{Alpha} by \emph{enclosing} $P$ in a closed curve like so:
$$\pcut{P}$$
Peirce called such curves \emph{cuts}\sidenote{Not to be confused with the name
given to instances of the \emph{cut rule} in sequent calculus. Although there is
a connection, since in \sys{LK}, one occurrence of the cut formula in the
premisses of the rule is negated.}, because they ought to be seen as literal
cuts in the paper sheet that embodies $\SA$. Note that they do not need to be
circles: all that matters is that $P$ is in a separate area from the rest of
$\SA$. This is precisely the content of the \emph{Jordan curve theorem} in
topology, and thus we can take cuts to be arbitrary Jordan curves. This entails
in particular that cuts cannot intersect each other, but can be freely nested
inside each other. Then as for juxtaposition, one can replace $P$ by any graph
$G$, i.e. any portion of $\SA$, as long as the cut does not intersect other cuts
in $G$.

With just these two \emph{icons}, juxtaposition and cuts, one can therefore
assert the truth of any proposition made up of conjunctions and negations, and
built from atomic propositions. Importantly, the only symbols needed for doing
so are the letters $P, Q, R\ldots$ denoting atomic propositions.

Now, it is well-known that $\{\land,\neg\}$ is \emph{functionally complete},
meaning that any boolean truth function can be expressed as the composition of
boolean conjunctions and negations. In particular, the symbolic definitions of
falsehood $\bot \defeq \neg\top$, classical disjunction $A \lor B \defeq
\neg(\neg A \land \neg B)$ and classical implication $A \limp B \defeq \neg (A
\land \neg B)$ can be expressed by the following three graphs\sidenote{Note the
resemblance with the translation of formulas as solutions in
\reffig{bubbles-native}, in particular for negative disjunctions.}:
\begin{mathpar}
  \pcut{\phantom{A}}
  \and
  \pcut{\pcut{A}~~~\pcut{B}}
  \and
  \pcut{A~~~\pcut{B}}
\end{mathpar}
Thus one can easily encode any propositional formula into a classically
equivalent graph. Conversely, one can translate a graph into a classically
equivalent formula, as has been shown for instance in
\sidecite{10.7551/mitpress/3633.001.0001}. In fact, there are usually many
possible formula readings of a given graph: this is because juxtaposition of
graphs is a \emph{variadic} operation, as opposed to conjunction of formulas
which is \emph{binary}. Also, because of the topological nature of $\SA$,
juxtaposition is naturally \emph{associative} and \emph{commutative}: the
locations of two juxtaposed graphs do not matter, as long as they live in the
same area delimited by cuts. This property is called the \emph{isotropy} of
$\SA$ in \cite{minghui_graphical_2019}. Hence, graphs can be seen as an
associative-commutative normal form for propositional formulas built from atoms
with $\{\land,\neg\}$\sidenote{In a first version of EG called \emph{entitative
graphs}, Peirce used juxtaposition to denote \emph{disjunction} instead of
conjunction. Although $\{\lor,\neg\}$ is also functionally complete, Peirce
quickly grew unsatisfied with these entitative graphs, stating that EG formed
``a far preferable system on the whole'' (Ms 280, pp. 21-22). I find it
interesting that more contemporary works in logic have also made the choice to
take conjunction and negation as their primitive operations, like the tensorial
logic of Melliès \cite{mellies_micrological_2017}, or the realizability
constructions for linear logic in Girard's transcendental syntax
\cite{eng_stellar_2020}.}.

\section{Illative transformations}\labsec{illative}

In order to have a proof system, one needs a collection of \emph{inference
rules} for deducing true statements from other true statements. In \sys{Alpha},
inference rules are implemented by what Peirce called \emph{illative
transformations} on graphs. In modern terminology, they correspond to
\emph{rewriting} rules that can be applied to any subgraph/portion of $\SA$. By
measuring the depth of a subgraph as the number of cuts in which it is enclosed,
we thus have that the rules of \sys{Alpha} are applicable on subgraphs of
arbitrary depth. This makes \sys{Alpha} deserving of the title of \emph{deep}
inference system.

\begin{marginfigure}
  $$\ncut{P~~~\pcut{\ncut{Q}~~~R}}$$
  \caption{Peirce's notation for emphasizing negative areas}
  \labfig{peirce-neg-areas}
\end{marginfigure}

\begin{marginfigure}
  $$\oncut{P~~~\opcut{\oncut{Q}~~~R}}$$
  \caption{Drawing negative areas literally in negative}
  \labfig{our-neg-areas}
\end{marginfigure}

Before introducing the rules, let us make a small change in the way we depict
the graphs. The idea is that we want to visualize more clearly the
\emph{polarity} of any subgraph $G$, understood as the \emph{parity} of the
number of cuts (negations) enclosing $G$. In one of his unpublished manuscripts
(Ms 514), Peirce did this by \emph{shading} negative areas --- those enclosed in
an odd number of cuts --- in gray, as illustrated in \reffig{peirce-neg-areas}
\sidecite{sowa_peirces_2011}. Unconstrained by hand-drawing, one could adopt an
even more iconic notation, where negative areas are \emph{literally} drawn like
a negative in photography, by inverting white and black. The example of
\reffig{peirce-neg-areas} would then be drawn as in \reffig{our-neg-areas}.
However in this thesis, we will stick to Peirce's notation, which is both less
straining for the eyes by being less contrasted, and more economical in ink for
print. A nice advantage of these notations is that they remove the need to count
manually the number of cuts starting from the top-level of $\SA$: the
information is immediately apparent in the subgraph, and thus completely
\emph{local}\sidenote{A similar device is used in the deep inference system
\sys{ISp} of Tiu \cite{tiu_local_2006}, where the polarities of substructures
are attached to them as explicit labels.}.

\begin{remark}\labremark{eg-polarity}
  Whereas in bubble calculi the concept of polarity was understood as a property
  of \emph{objects} --- i.e. utterances of propositions --- by assigning them
  opposite colors (blue and red), the previous notations for graphs suggest that
  it is instead a property of the \emph{space} in which objects reside. This is
  more natural from the point of view of \emph{game semantics}: for instance in
  a game of chess, the two players can easily exchange their roles by switching
  places or rotating the board by 180°, rather than by repainting laboriously
  each piece in the opposite color.
\end{remark}

Quite surprisingly, Peirce showed that one only needs $5$ inference rules to get
a \emph{strongly complete} system, in the sense that if the truth of a graph $G$
entails the truth of another graph $H$, then $G$ can always be rewritten into
$H$ by applying exclusively instances of these $5$ rules\sidenote{Of course
Peirce did not show completeness formally in the sense of modern model theory,
although Sowa argues in \cite{sowa_peirces_2011} (Section 4) that he had started
to develop his own model theory equivalent to Tarski's (but closer to the
\emph{game-theoretical semantics} of Hintikka \cite{Hintikka1973-HINLLA}).
% One can find a modern categorical treatment of completeness with respect to
% Boolean algebras, based on a rigorous formalization of the geometry and algebra
% of \sys{Alpha}, in \cite{brady_categorical_2000}.
}. A nice way to understand the rules of \sys{Alpha} is as \emph{edition
principles}, like the most basic actions one executes pervasively when editing
text on a computer\sidenote{Even though computers did not exist yet in Peirce's
time! In fact, Martin Irvine argues in \cite{irvine_semiotics_2022} that Peirce
anticipated many developments in computer science and information technologies,
such as the use of electrical switches to compute boolean functions, whose
invention is usually attributed to Claude Shannon.}. The first two rules are the
most powerful and mysterious in all systems of EG, and can be applied in areas
of any polarity:
\begin{itemize}
  \item[\textbf{Iteration} \textit{(Copy \& Paste)}]
    A graph $G$ may be duplicated at any depth inside of a juxtaposed graph $H$.
    Using our notation for holed contexts from previous chapters, this can be
    represented schematically like so:
    \begin{mathpar}
      G~~~H\select{\phantom{G}} ~~~\step~~~ G~~~H\select{G}
      \and
      \nsheet{G~~~H\select{\phantom{G}} ~~~\step~~~ G~~~H\select{G}}
    \end{mathpar}
    In particular, $H$ can be taken to be any empty portion of $\SA$ in the same
    area as $G$, giving that $G ~\step~ G~G$.
  \item[\textbf{Deiteration} \textit{(Factorization)}]
    Formally, this is the converse of \rsf{Iteration}:
    \begin{mathpar}
      G~~~H\select{G} ~~~\step~~~ G~~~H\select{\phantom{G}}
      \and
      \nsheet{G~~~H\select{G} ~~~\step~~~ G~~~H\select{\phantom{G}}}
    \end{mathpar}
    Its interpretation as an edition principle is a bit trickier, but it can be
    understood as a form of \emph{sharing} of information. Indeed, it roughly
    says that a subgraph $G$ can be erased if it already occurs ``higher'' on
    $\SA$. Also this does precisely the opposite of copy-pasting, which is known
    in software engineering as \emph{factorization}\sidenote{This is closely
    related to the kind of factorization at work in bubble calculi. In
    particular, the fact that the factorizing occurence is higher and usually
    outside of a cut is very reminiscent of the \emph{outward} flow rules of
    system $\sysB$ (those whose name ends with $\uparrow$ in
    \reffig{sequent-B}); and the deduplicating effect makes \rsf{Deiteration}
    even closer to the variant of the same rules in \sys{B_{inv}}
    (\reffig{sequent-B-inv}).}.
\end{itemize}
The applicability of the next two rules depends on the polarity of the
subgraph's area:
\begin{itemize}
  \item[\textbf{Insertion}]
    Any graph $G$ may be inserted in a negative area:
    $$\nsheet{~~~\step~~~ G}$$
    This is akin to a \emph{weakening} rule, stating that if a proposition is
    known to be true, one might add (useless) hypotheses at will. The closest
    equivalent we found in the deep inference literature is indeed the weakening
    rule \rsf{wl{\downarrow}} of \sys{ISp} in \sidecite{tiu_local_2006}.
  \item[\textbf{Erasure} \textit{(Deletion)}]
    Any graph $G$ occurring in a positive area may be erased:
    $$G ~~~\step~~~$$
    This is exactly the dual of \rsf{Insertion}, stating that if a proposition
    is known to be true, then one might as well refrain from asserting it.
    % It is strongly related to the cut rule of sequent calculus, as will be
    % demonstrated shortly.
\end{itemize}
The last rule is more of a \emph{space management} principle that works as an
\emph{isotopy}, i.e. a bidirectional topological deformation:
\begin{itemize}
  \item[\textbf{Double-cut}]
    A \emph{double-cut} may be inserted or erased around any graph $G$:
    \begin{mathpar}
      \ncut{\pcut{G}} ~~~\leftrightarrow~~~ G
      \and
      \nsheet{\pcut{\ncut{G}} ~~~\leftrightarrow~~~ G}
    \end{mathpar}
    The bidirectional arrow $\leftrightarrow$ expresses that the rule can be
    applied in both directions\sidenote{We could have merged \rsf{Iteration} and
    \rsf{Deiteration} in this way, but chose to follow the original presentation
    of the rules by Peirce instead, as exposed in \cite{Roberts+1973}.}.
    Logically, this corresponds to the classical equivalence $\neg\neg A
    \semequiv A$, where in particular the erasure direction $\neg\neg A \limp A$
    is not true intuitionistically. Topollogically, the double-cut forms a
    \emph{ring}, that separates $G$ from the rest of $\SA$ while preserving its
    polarity. Then the two directions of the rules can be understood as the
    following dual \emph{homotopies}:
    \begin{itemize}
      \item[\textbf{Contraction}] The ring is created by cutting $\SA$ around
      $G$, and then \emph{contracting} the inner area where $G$ resides on
      itself. This effectively ``pulls apart'' $G$ from the rest of the sheet,
      leaving apparent in the empty space of the ring whatever lies behind
      $\SA$. Peirce thought of positive and negative areas as being the
      \emph{recto} and \emph{verso} of $\SA$, respectively. Thus in the positive
      version of the rule (on the left), the ring would represent negative empty
      space on the verso of $\SA$.
      \item[\textbf{Expansion}] The ring is erased by \emph{expanding} the inner
      area where $G$ resides towards the outer border of the ring. Unfolding the
      metaphor to its conclusion, the inner area is then ``glued back'' to the
      rest of $\SA$\sidenote{This is reminiscent of the \emph{absorption} rules
      $\{\rsf{a},\rsf{a{-},\rsf{a{+}}}\}$ of system $\sysB$, as is very clear in
      their graphical presentation (\reffig{graphical-B}).}.
    \end{itemize}
\end{itemize}
A remarkable feat of Peirce's rules, on which he insisted very much, is that
they are only expressed in terms of \emph{insertions} and \emph{omissions} of
graphs on $\SA$. He thought that those were the \emph{smallest} steps in which
reasoning could be dissected, making his system extremely appropriate for
\emph{analytical} purposes\sidenote{See Section 7.11 of \cite{Roberts+1973}.}.
We already considered this question of decomposing logical inferences into their
most elementary operations, when reflecting on the graphical presentation of
\sys{BJ} at the end of \refsec{bubbles-graphical-rules}. In this setting, the
most basic insertions and omissions we could find were not logically
\emph{sound}, whereas in \sys{Alpha} they are. This is quite promising, and
prompts us to reevalute our conception of \emph{atomicity} in logical reasoning.

\begin{marginfigure}
  \setlength{\fboxsep}{2pt}
\setlength{\arraycolsep}{0pt}
\newcommand{\vsp}{\vspace{-0.5em}}
\newcommand{\stkf}{\tikzfig{0.8}{0.45}}
$$
\begin{array}{r@{\quad}c@{\vsp}}
                                  &\stkf{eg-peirce-law-0} \\
       \xstep{\mathsf{Insertion}}       &\stkf{eg-peirce-law-1-a} \\
       \xstep{\mathsf{Iteration}}      &\stkf{eg-peirce-law-2-a} \\
       \xstep{\mathsf{Iteration}}      &\stkf{eg-peirce-law-3-a} \\
       \xstep{\mathsf{Insertion}}       &\stkf{eg-peirce-law-4-a} \\
       \xstep{\mathsf{Double{-}cut}} &\stkf{eg-peirce-law-5-a} \\
\end{array}
$$

  \caption{A derivation of Peirce's law in \sys{Alpha}}
  \labfig{eg-peirce-law}
\end{marginfigure}

\reffig{eg-peirce-law} shows a derivation of Peirce's law with the rules of
\sys{Alpha}. Note that the direction of arrows has been reversed compared to the
above presentation: as usual, we prefer to read rules from conclusion to
premiss, starting from the goal to prove --- here the graph associated to the
formula $((P \limp Q) \limp P) \limp P$ --- that we reduce to the empty goal,
represented by the empty $\SA$. Also, the reader unfamiliar with EG might find
it hard to convince herself that all the steps followed in the derivation are
\emph{sound} logically. We suggest her to either build a \emph{syntactic}
intuition for the rules by praticing them on various tautologies of
propositional logic, or to wait until we give a formal \emph{semantic} proof of
soundness in \refsec{eg-soundness}.

% For lack of space, we will not provide a general proof of soundness for
% \sys{Alpha}, which requires a bit of work (especially in the case of
% \rsf{Iteration} and \rsf{Deiteration}). We suggest the reader to either:
% \begin{enumerate*}
%   \item build a \emph{syntactic} intuition by practicing the rules with various
%   tautologies of propositional logic;
%   \item read up the existing literature on the subject, for instance in Appendix
%   4 of \cite{Roberts+1973};
%   \item wait for the proof of soundness of our flower calculus with respect to
%   Kripke semantics in \refsec{Soundness}.
% \end{enumerate*}

% \todo{ Then compare sequent and EG derivations of the example on p. 111 of
% \cite{Roberts+1973}, where the latter exposes a lot less steps. Thus, which
% should be considered more atomic? We will not fret too much over this question,
% our goal in the end being to hide trivial reasoning steps from the user. We will
% see in \refsec{flowers-prover} that this is not incompatible with the analytic
% aspect of EG. }

\section{Graphs as multisets}\labsec{multisets}

As noted by various authors\sidenote{See for instance the Tree Existential
Graphs of Roberts and Pronovost \cite{roberts_existential_1992}, or Section 2.2
in \cite{brady_categorical_2000}.}, the nesting of cuts on $\SA$ induces a
\emph{tree} structure on graphs: each cut constitutes a node, whose children are
either leaves corresponding to atomic propositions residing in the area of the
cut, or nodes corresponding to nested cuts. Empty cuts have no children, and
thus also form leaves of the tree. Then $\SA$ may be seen either as a forest of
atoms and cuts, or as a rooted tree whose root represents $\SA$, and is
distinguished from cut nodes. This can be captured by the following grammar:
\begin{mathpar}
  \SA \Coloneq G 
  \and
  G, H, K \Coloneq g_1, \ldots, g_n
  \and
  g, h, k \Coloneq P \mid [G]
\end{mathpar}

\begin{example}
The graph of \reffig{peirce-neg-areas} may be written as either one of the
following expressions:
\begin{mathpar}

  [P, [[Q], R]] \and
  [P, [R, [Q]]] \and
  [[[Q], R], P] \and
  [[R, [Q]], P]
\end{mathpar}
\end{example}

To abstract from the specific order in which nodes are sequenced in this
notation, and thus represent faithfully the isotropy of $\SA$, we define
$\alpha$-graphs as (recursive) \emph{finite multisets}:

\begin{definition}[$\alpha$-graph]\labdef{alpha-graph} 
  
  The sets of \emph{$\alpha$-nodes} $\anodes$ and \emph{$\alpha$-graphs}
  $\agraphs$ are defined mutually inductively as follows:
  \begin{itemize}
    \item \textbf{(Atom)} If $P$ is an atomic proposition, then $P \in \anodes$;
    \item \textbf{(Graph)} If $G \subset \anodes$ is a finite multiset, then $G
    \in \agraphs$;
    \item \textbf{(Cut)} If $G \in \agraphs$, then $[G] \in \anodes$.
  \end{itemize}
  % An \emph{$\alpha$-graph} is a multiset of atomic propositions and
  % $\alpha$-graphs.
\end{definition}
% Note that with this definition, the distinction between $\SA$ and cuts is
% implicitly captured by whether a multiset belongs to another multiset (cut), or
% is at the ``top-level'' ($\SA$). We could also have followed the approach used
% for the solutions of bubble calculi (\refdef{solution}), by identifying cuts
% (bubbles) as a separate construct defined mutually recursively with
% $\alpha$-graphs (solutions).

Note the similarity with the definitions of bubbles (cuts) and solutions
(graphs) (\refdef{solution}). The main difference between $\alpha$-graphs and
solutions, is that in the former formulas are restricted to atoms, and they are
not \emph{polarized} (see \refremark{eg-polarity}).

The $5$ rules of \sys{Alpha} can now be formalized as multiset rewriting rules
on $\alpha$-graphs. But first, we need a notion of \emph{context} in which rules
apply:

\begin{definition}[$\alpha$-graph context]
  An \emph{$\alpha$-graph context} $G\hole$ is an $\alpha$-graph which contains
  exactly one occurrence of the special graph $\hole$ called the \emph{hole}.
  The hole can always be \emph{filled} (substituted) with any other graph $H$ or
  context $K\hole$, producing a new graph $\cfill{G}{H}$ or context
  $\cfill{G}{K\hole}$. In particular, filling with the empty graph $\emptyset$
  will yield a graph $\cfill{G}{\phantom{G}}$, which is just $G\hole$ with its
  hole removed.
\end{definition}

Then we can determine the polarity of a context by counting recursively the
number of cuts enclosing its hole:

\begin{definition}[Parity]
  The \emph{parity} $\parity{G\hole}$ of a context $G\hole$ is defined
  inductively by:
  \begin{align*}
    \parity{H, \hole} &= 0 \\
    \parity{H, [G\hole]} &= \parity{G\hole} + 1
  \end{align*}
\end{definition}

\begin{definition}[Polarity]
  We say that a context $G\hole$ is \emph{positive} if $\parity{G\hole}$ is
  even, and \emph{negative} otherwise. We denote positive and negative contexts
  respectively by $G^+\hole$ and $G^-\hole$.
\end{definition}

The inductive version of the rules of \sys{Alpha} is given in \reffig{alpha}, as
a set of unary inference rules on $\alpha$-graphs: when read \emph{top-down},
they correspond to usual inferences from premiss to conclusion, as we first
introduced them in \refsec{illative}.
% This gives them a \emph{static}, \emph{a posteriori} meaning, since this is
% typically how you would check the validity of an established inference
% \emph{after the fact}.
But as already mentioned there, we will rather emphasize their \emph{bottom-up}
reading: then they express the different ways in which one may choose to
simplify a goal.

\begin{definition}[Derivation]
  We write $G \step H$ to indicate a rewrite \emph{step} in \sys{Alpha}, that is
  an instance of some rule from \reffig{alpha} with $H$ as premiss and $G$ as
  conclusion. A \emph{derivation} $G \nsteps{n} H$ is a sequence of rewrite
  steps $G_0 \step G_1 \ldots \step G_n$ with $G_0 = G$, $G_n = H$ and $n \geq
  0$. Generally the length $n$ of the derivation does not matter, and we just
  write $G \steps H$.
\end{definition}

\begin{definition}[Proof]\labdef{eg-proof}
  A \emph{proof} of an $\alpha$-graph $G$ is a derivation $G \steps \emptyset$.
\end{definition}

\begin{figure}
  \begin{framed}
{\textsc{Alpha}}
\vspace{1.5em}
\begin{mathpar}
  \R[\intro{Iter}]
    {K\select{G, H\select{\phantom{G}}}}
    {K\select{G, H\select{G}}}
  \and
  \R[\intro{Deit}]
    {K\select{G, H\select{G}}}
    {K\select{G, H\select{\phantom{G}}}}
  \\
  \R[\intro{Ins}]
    {K^-\select{\phantom{G}}}
    {K^-\select{G}}
  \and
  \R[\intro{Del}]
    {K^+\select{G}}
    {K^+\select{\phantom{G}}}
  \\
  \R[\intro{Dcut{\da}}]
    {K\select{G}}
    {K\select{[[G]]}}
  \and
  \R[\intro{Dcut{\ua}}]
    {K\select{[[G]]}}
    {K\select{G}}
\end{mathpar}
\end{framed}
  \caption{Inductive presentation of the rules of \sys{Alpha}}
  \labfig{alpha}
\end{figure}


\section{Soundness}\labsec{eg-soundness}


\section{Completeness}\labsec{eg-completeness}


\section{\sys{Beta} graphs}\labsec{beta}


\section{Gardens}\labsec{gardens}