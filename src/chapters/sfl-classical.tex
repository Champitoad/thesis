\setchapterpreamble[u]{\margintoc}
\chapter{Classical Logic and Parallel Conclusions}
\labch{sfl-classical}

In virtually every proof assistant, the goals the user is faced with are
sequents of the form $\Gamma \seq C$, with a \emph{single} conclusion $C$ to be
proved under many hypotheses in $\Gamma$. Historically, this form of sequent was
introduced by Gentzen to formalize the rules of intuitionistic logic in his
sequent calculus \sys{LJ}. But his main interest was in classical logic, as
intuitionistic logic was still in its infancy and almost all of mathematics had
been developed in a classical setting. Interestingly, he found that the right
syntax to develop a rich metatheory of his classical sequent calculus \sys{LK}
consisted in \emph{multi-conclusion} sequents of the form $\Gamma \seq \Delta$,
where $\Delta$ is a list of conclusions that should be read
\emph{disjunctively}. That is, a sequent
$$A_1, \ldots, A_n \seq C_1, \ldots, C_m$$
has the same meaning as the formula
$$\bigwedge_{i=1}^{n}{A_i} \limp \bigvee_{j=1}^{m}{C_j}$$

\begin{marginfigure}
  \begin{mathpar}
    \R[R\lor]
      {\Gamma \seq A, B, \Delta}
      {\Gamma \seq A \lor B, \Delta}
  \end{mathpar}
  \caption{Multiplicative right introduction rule for disjunction}
  \labfig{multintro}
\end{marginfigure}

Later logicians have proposed various multi-conclusion sequent calculi for
intuitionistic logic, the most famous being \sys{GHCP} from Dragalin
\sidecite{dragalin_mathematical_1990}. Dyckhoff noted that such calculi are more
suited to implementation of automated proof search procedures
\sidecite{dyckhoff_contraction-free_1992} because they allow sharing code with
classical calculi, but more importantly because one can delay choices of
disjuncts to prove with the ``multiplicative''\sidenote{Terminology borrowed
from linear logic, where {\rnmsf{R\lor}} is exactly the right introduction rule
for multiplicative disjunction $\parr$.} introduction rule {\rnmsf{R\lor}}
(\reffig{multintro}). This property is also desirable in the setting of
interactive proof search as noted by Ayers\sidenote{Section 3.1.3 of his thesis
\cite{ayers_thesis}.}, because this reduces the need for the user to guess in
advance or backtrack; and contrary to a computer, it is very easy for a human
to lose track of the proof search history when backtracking multiple times.

That being said, we do not know of any proof assistant, whether classical or
intuitionistic, that exposes goals with multiple conclusions in its user
interface. One reason is that most proof/tactic languages are based on the rules
of natural deduction, which use single-conclusion sequents. Another reason is
that having one conclusion removes the need to designate it with an explicit
name or number, as is the case with hypotheses\sidenote{The current trend is to
have user-chosen or automatically generated strings for names as in Coq and
Lean, but some provers like HOL Light ask for the position in the list as an
integer to designate a particular hypothesis.}. And the explicit handling of
names in tactic invokations is known to be tedious and time-consuming, to the
point that some tactic languages like SSReflect have been designed around this
problem \sidecite{SSR}. Thus having multiple conclusions would only double the
effort for no compelling reason.

However in our graphical paradigm based on direct manipulation, hypotheses and
conclusions are designated by the act of \emph{pointing} at them with a mouse,
finger or any other pointing device\sidenote{With the recent advances in natural
language processing and voice recognition, one could also imagine a system based
on the selection of subterms by spelling their content. Then click and DnD
actions could be triggered by voice commands once the subterms they apply to
have been selected. This could be an important alternative for users with
impaired vision and/or motricity.}. This opens up the possibility of exposing
multiple conclusions in the interface, with associated graphical proof actions.
In the layout of Actema, this corresponds to having zero or multiple red items
instead of a single one. While we did not implement such an extension, we
explore in this section the theoretical foundations that would be necessary for
it to come to fruition.

\subsection{Click Actions}

In \refsec{clicks}


\subsection{DnD Actions}

\todo{TODO}
