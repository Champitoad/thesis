\setchapterpreamble[u]{\margintoc}
\chapter{Classical Logic and Parallel Conclusions}
\labch{sfl-classical}

In virtually every proof assistant, the goals the user is faced with are
sequents of the form $\Gamma \seq C$, with a \emph{single} conclusion $C$ to be
proved under many hypotheses in $\Gamma$. Historically, this form of sequent was
introduced by Gentzen to formalize the rules of intuitionistic logic in his
sequent calculus \sys{LJ}. But his main interest was in classical logic, as
intuitionistic logic was still in its infancy and almost all of mathematics had
been developed in a classical setting. Interestingly, he found that the right
syntax to develop a rich metatheory of his classical sequent calculus \sys{LK}
consisted in \emph{multi-conclusion} sequents of the form $\Gamma \seq \Delta$,
where $\Delta$ is a list of conclusions that should be read
\emph{disjunctively}. That is, a sequent
$$A_1, \ldots, A_n \seq C_1, \ldots, C_m$$
has the same meaning as the formula
$$\bigwedge_{i=1}^{n}{A_i} \limp \bigvee_{j=1}^{m}{C_j}$$

\begin{marginfigure}
  \begin{mathpar}
    \R[\lor R]
      {\Gamma \seq A, B, \Delta}
      {\Gamma \seq A \lor B, \Delta}
  \end{mathpar}
  \caption{Multiplicative right introduction rule for disjunction}
  \labfig{multintro}
\end{marginfigure}

Later logicians have proposed various multi-conclusion sequent calculi for
intuitionistic logic, the most famous being \sys{GHCP} from Dragalin
\sidecite{dragalin_mathematical_1990}. Dyckhoff noted that such calculi are more
suited to implementation of automated proof search procedures
\sidecite{dyckhoff_contraction-free_1992} because they allow sharing code with
classical calculi, but more importantly because one can delay choices of
disjuncts to prove with the ``multiplicative''\sidenote{Terminology borrowed
from linear logic, where {\rnmsf{\lor R}} is exactly the right introduction rule
for multiplicative disjunction $\parr$.} introduction rule {\rnmsf{\lor R}}
(\reffig{multintro}). This property is also desirable in the setting of
interactive proof search as noted by Ayers\sidenote{Section 3.1.3 of his thesis
\cite{ayers_thesis}.}, because this reduces the need for the user to guess in
advance or backtrack; and contrary to a computer, it is very easy for a human to
lose track of the proof search history when backtracking multiple times.

That being said, we do not know of any proof assistant, whether classical or
intuitionistic, that exposes goals with multiple conclusions in its user
interface. One reason is that most proof/tactic languages are based on the rules
of natural deduction, which use single-conclusion sequents. Another reason is
that having one conclusion removes the need to designate it with an explicit
name or number, as is the case with hypotheses\sidenote{The current trend is to
have user-chosen or automatically generated strings for names as in Coq and
Lean, but some provers like HOL Light ask for the position in the list as an
integer to designate a particular hypothesis.}. And the explicit handling of
names in tactic invokations is known to be tedious and time-consuming, to the
point that some tactic languages like SSReflect have been designed around this
problem \sidecite{SSR}. Thus having multiple conclusions would only double the
effort for no compelling reason.

However in our graphical paradigm based on direct manipulation, hypotheses and
conclusions are designated by the act of \emph{pointing} at them with a mouse,
finger or any other pointing device\sidenote{With the recent advances in natural
language processing and voice recognition, one could also imagine a system based
on the selection of subterms by spelling their content. Then click and DnD
actions could be triggered by voice commands once the subterms they apply to
have been selected. This could be an important alternative for users with
impaired vision and/or motricity.}. This opens up the possibility of exposing
multiple conclusions in the interface, with associated graphical proof actions.
In the layout of Actema, this corresponds to having zero or multiple red items
instead of a single one. While we did not implement such an extension, we
explore in this chapter its design, and the theoretical foundations that would
be necessary for it to come to fruition.

\subsection{Click Actions}

In \reftab{click-rules}, we showed how click actions in Actema are in direct
correspondance with the rules of the single-conclusion sequent calculus \sys{LJ}
for intuitionistic logic. Following the literature mentioned earlier, we just
need to replace two actions/introduction rules to get a multi-conclusion system
capturing either intuitionistic or classical first-order logic:

\begin{marginfigure}
  \begin{mathpar}
    \R[{\limp}R*c]
      {\Gamma, A \seq B, \Delta}
      {\Gamma \seq A \limp B, \Delta}
  \end{mathpar}
  \caption{Classical multi-conclusion right introduction rule for implication}
  \labfig{multi-imp-intro-class}
\end{marginfigure}

\begin{marginfigure}
  \begin{mathpar}
    \R[{\limp}R*i]
      {\Gamma, A \seq B}
      {\Gamma \seq A \limp B, \Delta}
  \end{mathpar}
  \caption{Intuitionistic multi-conclusion right introduction rule for implication}
  \labfig{multi-imp-intro-intui}
\end{marginfigure}

\begin{itemize}
  \item clicking on a red disjunction $A \lor B$ breaks it into two conclusions
  $A$ and $B$. This is the dual behavior to click actions on blue conjunctions,
  and corresponds to the {\rnmsf{\lor R}} rule of \reffig{multintro}, which is
  common to both the intuitionistic and classical variants;
  \item as before, clicking on a red implication $A \limp B$ breaks it into an
  hypothesis $A$ and a conclusion $B$. Without further changes, this corresponds
  to the right introduction rule from the classical sequent calculus \sys{LK} of
  Gentzen (named {\rnmsf{{\limp}R*c}} in \reffig{multi-imp-intro-class}), and
  our set of actions becomes a proof system for classical logic. To go back to
  intuitionistic logic, one needs the additional behavior that all the other
  conclusions of the goal are removed. This corresponds to the right
  introduction rule from the \sys{GHCP} calculus of Dragalin (named
  {\rnmsf{{\limp}R*i}} in \reffig{multi-imp-intro-intui}).
\end{itemize}

\begin{remark}
  In the special case of intuitionistic sequents with one conclusion, the two
  variants {\rnmsf{{\limp}R*c}} and {\rnmsf{{\limp}R*i}} collapse into the usual
  {\rnmsf{{\limp}R}} rule.
\end{remark}
Note that we only modified the behavior of the disjunction $\lor$ and
implication $\limp$ connectives; and for the latter, only in the case when there
are at least two parallel conclusions, and thus implicitly a disjunction. Then
it is interesting to notice that the classical behavior of the other connectives
($\bot, \land, \forall, \exists$) essentially arises from their interaction with
(positive) disjunctive statements.

If we stick to intuitionistic logic, the benefits of having multiple conclusions
are unclear. Indeed while the {\rnmsf{\lor R}} rule is invertible, the
{\rnmsf{{\limp}R*i}} rule is not, and thus at some point the choice of which
conclusion to prove must be made by the user irreversibly, even if the choice is
delayed. On the other hand the {\rnmsf{{\limp}R*c}} rule \emph{is} invertible:
this is known to allow the formulation of sequent calculi for classical logic
where \emph{all} rules are invertible, like the \sys{G3c} calculus of
\sidecite{negri_structural_2001}. In the propositional case, this gives a
constructive decision procedure for the question of provability: given a sequent
$\Gamma \seq \Delta$, one just has to choose any formula in $\Gamma$ or $\Delta$
and apply the introduction rule associated to its main connective, or the axiom
rule whenever possible. In Actema, this would correspond to having the user
click randomly on blue and red items until all goals are solved. The procedure
ends because all introduction rules destroy the main connective, and none of
them duplicate formulas: thus the total size of the sequent decreases strictly
after each rule application.

\begin{marginfigure}
  \begin{mathpar}
    \R[\forall L*]
      {\Gamma, \forall x. A, \subst{A}{t}{x} \seq \Delta}
      {\Gamma, \forall x. A \seq \Delta}
    \and
    \R[\exists R*]
      {\Gamma \seq \subst{A}{t}{x}, \exists x. A, \Delta}
      {\Gamma \seq \exists x. A, \Delta}
  \end{mathpar}
  \caption{Multi-conclusion instantiation rules for quantifiers}
  \labfig{multi-inst}
\end{marginfigure}

When dealing with quantifiers, the situation is not so simple: if one wants
invertible introduction rules, it is necessary to duplicate the quantified
formula being instantiated, which can seen as the root cause of undecidability
in predicate logic as noted by Girard\sidenote{Section 3.3.2 of The Blind Spot
\cite{girard:hal-01322183}.}. This is already what happens in Actema for the
universal quantifier: dropping a term $t$ on a blue item $\forall x. A$ will
produce a new hypothesis $\subst{A}{t}{x}$, while keeping the original $\forall
x. A$ item. This corresponds to the invertible left introduction rule of
\sys{G3c} ({\rnmsf{\forall L*}} in \reffig{multi-inst}). But in the
single-conclusion framework, dropping a term $t$ on a red item $\exists x. A$
necessarily replaces it by the instantiated conclusion $\subst{A}{t}{x}$.
Allowing multiple conclusions circumvents this problem and restores the symmetry
between $\forall$ and $\exists$, since we can create a new conclusion for
$\subst{A}{t}{x}$ while preserving the old one. This corresponds to the
invertible right introduction rule of \sys{G3c} ({\rnmsf{\exists R*}} in
\reffig{multi-inst}).


\subsection{DnD Actions}

\todo{TODO}
