\setchapterpreamble[u]{\margintoc}
\chapter{Integration in a Proof Assistant}
\labch{plugin}

In the previous chapters, we introduced the Proof-by-Action paradigm
(\refch{pba}), and tried to convince the reader that it is both theoretically
sound with its firm grounding in deep inference proof theory (\refch{sfl} and
\refch{sfl-classical}), and practically useful by analysing proofs of
mathematical problems expressed within it (\refch{advanced}). We also mentioned
multiple times our prototype of interface implementing Proof-by-Action called
Actema, and in particular the fact that it exists as a \emph{standalone} web
application with its own proof engine. This is convenient for distributing it
online as a publicly available website, so that people can immediately try it
out without the hassles of installation procedures. However due to both
historical choices in its design and lack of human resources for development,
Actema's proof engine is quite limited in features:
\begin{itemize}
  \item it can only handle goals expressed in many-sorted intuitionistic
    first-order logic (hereafter iFOL), whereas all state-of-the-art proof
    assistants support higher-order logic in one form or another; and higher-order
    features are crucial when one wants to formalize many mathematical notions in a
    concise way, as witnessed by the example of \refsec{funcs};
  \item it does not implement a certified logical kernel for checking proof
    objects, which makes it hard to trust and interoperate with;
  \item it has no mechanism for adding new mathematical notations, only ad hoc
    support for arithmetical expressions; thus formulas become very quickly
    impossible to read and manipulate;
  \item it has poor support for managing libraries of definitions, lemmas and
    proofs, partly because of the previous items.
\end{itemize}
To address the previous limitations, and thus enable a confrontation of the
Proof-by-Action paradigm to real mathematical developments, we decided to build
\texttt{coq-actema}, a Coq plugin that directly connects Actema to a running
instance of the Coq proof assistant. The idea is that Actema should act as an
enhanced graphical, interactive proof view that integrates in the usual
text-based workflow of proof scripts. In the following, we discuss briefly some
motivations that lead to the development of coq-actema, and describe its over-
all design and architecture.