\setchapterpreamble[u]{\margintoc}
\chapter{Symmetric Bubble Calculi}
\labch{bubbles-symm}

\todo{Introductory paragraph}

The chapter is organized as follows: in \refsec{non-determinism} we motivate our
quest for a system where all introduction rules for logical connectives are
\emph{invertible}, in order to reduce non-determinism in proof search and enable
a fully \emph{iconic} approach to proof building. To that effect, we relax in
\refsec{branching} the restriction to single-conclusion solutions, which
requires a new distinction between \emph{closed} and \emph{open} solutions. This
gives rise in \refsec{colors} to an extension of the syntax of solutions, where
bubbles can themselves be \emph{polarized}. In \refsec{symmetric-calculus} we
introduce a core \emph{symmetric bubble calculus} called ``system \sys{B}''. In
\refsec{bubbles-soundness} we show that by adding or removing 4 inference rules
to system \sys{B} that define the \emph{porosity} of bubbles, one gets 4 systems
that capture respectively intuitionistic, dual-intuitionistic, bi-intuitionistic
and classical logic. In \refsec{bubbles-completeness} we support this claim by
showing that the systems are not only sound, but also complete with respect to
standard sequent calculi. In \refsec{invertible-calculus} we present a fully
invertible variant of system \sys{B}, inspired by ideas that will be developed
in subsequent chapters. Despite the invertibility of introduction rules, it
turns out that this variant does not satisfy the \emph{decomposability}
property. We fix this defect in \refsec{decomposable-calculus} with yet another
variant of the system, finally achieving full iconicity.

\todo{Where should we mention cut-elimination as a future work? Or maybe it will
follow from the completeness theorem.}

\todo{Mention that we only \emph{conjecture} soundness and completeness for
quantifiers, and that when adding exclusion into the mix (rules
{\rnmsf{f{+}{-}}} and {\rnmsf{f{-}{-}}}) we only get a conservative extension of
constant-domain predicate logic, which is \emph{not} a conservative extension of
predicate logic (see \cite{crolard_subtractive_2001} and
\cite{aschieri_natural_2018})}

\todo{Mention the recent result of \cite{gore_bi-intuitionistic_2020} about the
distinction between weak and strong bi-intuitionistic logic, following their
claim that we are probably in wBIL since we are complete with respect to the
system of Postniece.}

\section{Non-determinism and iconicity}\labsec{non-determinism}

In all known sequent calculus formulations of intuitionistic logic, there are at
least two rules which are invariably \emph{non-invertible}:
\begin{enumerate}
  \item a left introduction rule for $\limp$ (there might be many ones, as in
  the calculus \sys{LJT} of \sidecite{dyckhoff_contraction-free_1992});
  \item the right introduction for either:
    \begin{itemize}
      \item $\lor$ when sequents have at most or exactly one conclusion;
      \item $\limp$ when sequents have multiple conclusions, e.g. in the
        multi-succedant variant of \sys{LJT} in
        \cite{dyckhoff_contraction-free_1992}.
    \end{itemize}
\end{enumerate}
In \sys{BJ}, this means that click actions on blue $\hypo{\limp}$ and red
$\conc{\lor}$ need to be performed in a specific order to be able to complete
proofs.

In his thesis \cite{guenot_nested_2013}, Guenot introduced a specific kind of
nested sequent system, where like in \sys{BJ} inference rules can be expressed
as rewriting rules. An interesting feature of these systems is that all
introduction rules for connectives are \emph{invertible}, which means that in
proof search, formulas can be completely decomposed until atoms are reached
before applying other rules. Non-determinism then arises in the choice of atoms
that are to be connected in axioms, as well as the choice of sub-sequents to be
duplicated for reuse.

In our graphical setting, this would translate to an interface where all click
actions are redundant. Although we already considered this possibility in
\refsec{dnd-completeness}, here it goes further by making even \emph{logical
connectives} superfluous, since all other rules work purely on the structure of
sequents. This means that all logical connectives could be replaced by
metaphorical constructs like bubbles, which suggest \emph{physically} the
possible transformations on the proof state. Unfortunately, the systems in
\cite{guenot_nested_2013} only handle classical logic and the implicative
fragment of intuitionistic logic. Thus began our quest for a nested sequent
system in the style of Guenot capturing full intuitionistic logic\sidenote{Other
nested sequent systems for full intuitionistic logic exist
\cite{fitting-nested-2014}, but they are based on tree-shaped proofs, and thus
ignore the whole \emph{raison d'être} of our concept of bubble.}.


\section{Conclusions and branching}\labsec{branching}

The first direction we followed was to relax the constraint that solutions must
be single-conclusion. Indeed as already noted in \refsec{sfl-backtracking}, a
notable property of sequent calculi with multiple conclusions is that their
right introduction rule for $\lor$ is invertible.

The main difficulty lies in the way one should interpret a multi-conclusion
solution $S$ as a formula $\sinterp{S}$. If we just take the asymmetric
interpretation (\refdef{ainterp}) and group conclusions disjunctively instead of
conjunctively, we get
$$
\sinterp{\Gamma \piq{\mathcal{S}} \Delta} =
\bigwedge \Gamma \limp \bigvee \Delta \land \bigwedge_{S \in \mathcal{S}}{\sinterp{S}}
$$
But this interpretation breaks on the 0-ary case when $\Delta$ is empty: instead
of seeing $\Gamma \piq{\mathcal{S}}$ as a node of the proof tree with hypotheses
$\Gamma$ and subgoals $\mathcal{S}$, it trivializes it to $\sinterp{\Gamma
\piq{\mathcal{S}}} = \bigwedge \Gamma \limp \bot$, i.e. a goal where one has to
find a contradiction in $\Gamma$; which is obviously not what we have in mind.

\begin{marginfigure}
  $$
  \R[\land R*]
    {\Gamma \seq A, \Delta}
    {\Gamma \seq B, \Delta}
    {\Gamma \seq A \land B, \Delta}
  $$
  \caption{Multi-conclusion right introduction rule for conjunction}
  \labfig{multi-and-intro}
\end{marginfigure}

A key observation was that in the rules of multi-conclusion sequent calculi, one
usually distributes the context $\Delta$ of conclusions in all premisses: this
restores a perfect symmetry with respect to the context of hypotheses $\Gamma$,
as illustrated by the {\rnm{\land R*}} rule (\reffig{multi-and-intro}). Then our
idea was that instead of implementing distribution/sharing of conclusions inside
inference rules, we could do it implicitly in the interpretation of solutions.
This is already what happens in the asymmetric interpretation for hypotheses
(\refdef{ainterp}); indeed the context $\Gamma$ is shared among subgoals,
because:
\begin{enumerate}
  \item it appears on the left of an implication $\limp$
  \item bubbles are joined conjunctively, and
  \item implication distributes over conjunction thanks to the equivalence $A
  \limp B \land C \semequiv (A \limp B) \land (A \limp C)$.
\end{enumerate}
But what does it mean precisely to share conclusions among subgoals? If we
consider the two following solutions:
$$
\underbrace{\bubble{\hypo{A}~~~\conc{B}}~~~\bubble{\hypo{C}~~~\conc{D}}~~~\conc{E}}_{S} \qquad\qquad
\underbrace{\bubble{\hypo{A}~~~\conc{B}~~~\conc{E}}~~~\bubble{\hypo{C}~~~\conc{D}~~~\conc{E}}}_{T}
$$
we would like to have $\sinterp{S} \semequiv \sinterp{T} \semequiv (A \limp B
\lor E) \land (C \limp D \lor E)$. Since disjunction distributes over
conjunction, a first naive try would give the following interpretation, where we
just replaced $\land$ by $\lor$ compared to the previous attempt:
$$
\sinterp{\Gamma \piq{\mathcal{S}} \Delta} =
\bigwedge{\Gamma} \limp \bigwedge_{S \in \mathcal{S}}{\sinterp{S}} \lor \bigvee \Delta
$$
But this immediately fails whenever $\mathcal{S} = \emptyset$, because it
trivializes to $\bigwedge \Gamma \limp \top \lor \bigvee \Delta \semequiv \top$
instead of $\bigwedge \Gamma \limp \bigvee \Delta$. The only way we found around
this defect was to internalize \emph{syntactically} a distinction between two
kinds of solutions, by assigning them one of two \emph{statuses}\sidenote{In the
terminology of Martin-Löf, we could say that we now have two distinct forms of
\emph{judgment}.}:
\begin{itemize}
  \item \emph{closed} solutions $\Gamma \piq{\mathcal{S}} \Delta$ correspond
  to branching nodes in the proof tree, or to closed leaves when $\mathcal{S} =
  \emptyset$ (i.e. solved subgoals). Thus it becomes sensical to have
  $\sinterp{\Gamma \piq{} \Delta} = \top$. In the asymmetric interpretation,
  closed solutions were encoded by solutions with no conclusions;
  \item \emph{open} solutions $\Gamma \seq \Delta$ correspond to open leaves
  in the proof tree (i.e. unsolved subgoals). In the asymmetric interpretation,
  they were encoded by solutions with one conclusion.
\end{itemize}
Then we keep the last proposed interpretation for closed solutions, and
interpret open solutions like usual sequents:
$$\sinterp{\Gamma \seq \Delta} = \bigwedge{\Gamma} \limp \bigvee{\Delta}$$ To be
able to abstract from the particular kind of solution at hand, we reframe the
syntax of solutions with so-called \emph{branching} operators $\J$:
\begin{align*}
  S, T, U &\Coloneq \Gamma \J \Delta \\
  \J, \JB &\Coloneq {\seq} \mid \piq{\mathcal{S}}
\end{align*}
Graphically, closed solutions with no bubbles can be distinguished from open
solutions by painting their \emph{background} on the proof canvas in green, the
intent being to suggest that they have already been solved. A pathological
example is the distinction between the closed empty bubble
$\bbubble{\phantom{a}}$ and the open empty bubble $\bubble{\phantom{a}}$, who
are interpreted respectively by $\sinterp{\piq{\piq{}}} = \top$ and
$\sinterp{\piq{\seq}} = \bot$.

Now coming back to our target example,
% we must explicitly assign a status to each subsolution:
% $$
% \underbrace{\bsheet{\bubble{\hypo{A}~~~\conc{B}}~~~\bubble{\hypo{C}~~~\conc{D}}~~~\conc{E}}}_{S}
% \qquad\qquad
% \underbrace{\bsheet{\bubble{\hypo{A}~~~\conc{B}~~~\conc{E}}~~~\bubble{\hypo{C}~~~\conc{D}~~~\conc{E}}}}_{T}
% $$
% However
the interpretation still fails, because we associate two non-equivalent
formulas to $S$ and $T$. To show this, let us try to derive the equivalence
through some algebraic developments:
\begin{align}
  \sinterp{S} &= \top \limp ((A \limp B) \land (C \limp D)) \lor E \nonumber\\
              &\semequiv ((A \limp B) \land (C \limp D)) \lor E \nonumber\\
              &\semequiv ((A \limp B) \lor E) \land ((C \limp D) \lor E) \nonumber\\
              &\semequiv (A \limp B \lor E) \land (C \limp D \lor E) \labeq{grishin}\\
              &\semequiv ((A \limp B) \land (C \limp D)) \lor E \nonumber\\
  \sinterp{T} &= \top \limp ((A \limp B \lor E) \land (C \limp D \lor E)) \lor \bot \nonumber
\end{align}
Wait, we did manage to prove it! The trick resides in \refeq{grishin}, which
uses twice the equivalence $(A \limp B) \lor C \equiv A \limp (B \lor C)$. It
turns out that this equivalence is true in classical logic, but \emph{not} in
intuitionistic logic. More precisely, it is the implication $G \triangleq (A
\limp (B \lor C)) \limp ((A \limp B) \lor C)$ which is not provable
intuitionistically, since it can easily be shown equivalent to the law of
excluded middle\sidenote{This was already noticed in
\cite{clouston-annotation-free-2013}, with the linear version $(A \multimap (B
\parr C)) \multimap ((A \multimap B) \parr C)$ of $G$ called Grishin (a) and its
converse Grishin (b). More precisely, it is affirmed that while Grishin (b) is
valid in \sys{FILL}, the restriction of the classical multiplicative linear
logic \sys{MLL} to single-conclusion sequents, adding Grishin (a) makes
\sys{FILL} collapse to \sys{MLL}.}. Thus according to this interpretation, $S$
entails $T$ but $T$ does not entail $S$, which means that it is not able to
account for the \emph{factorization} of common conclusions in distinct subgoals.

To remedy this situation, we opted for a different strategy: instead of finding
a logical formula capturing the distributive semantics of conclusions over
subgoals, we hardcode the latter by defining the interpretation function on
closed solutions through \emph{non-structural} recursion. This gives the
following final definitions:

\begin{definition}[Mix operator]\labdef{mixop}
  The commutative \emph{mix operator} $\mix$ on solutions is defined by:
  \begin{align*}
    \Gamma \J \Delta \mix \Gamma' \seq \Delta' &=
      \Gamma, \Gamma' \J \Delta, \Delta' \\
    \Gamma \piq{\mathcal{S}} \Delta \mix \Gamma' \piq{\mathcal{S'}} \Delta' &=
      \Gamma, \Gamma' \piq{\mathcal{S} \sep \mathcal{S'}} \Delta, \Delta' \\
  \end{align*}
\end{definition}

\begin{definition}[Symmetric interpretation]\labdef{sinterp}
  The \emph{symmetric interpretation} of a solution is defined recursively by:
  \begin{align*}
    \sinterp{\Gamma \piq{\mathcal{S}} \Delta} &=
      \bigwedge_{S \in \mathcal{S}} \sinterp{S \mix \Gamma \seq \Delta} \\
    \sinterp{\Gamma \seq \Delta} &=
      \bigwedge \Gamma \limp \bigvee \Delta
  \end{align*}
\end{definition}

This is the right approach for interpreting solutions with multiple conclusions,
as will be demonstrated formally in \refsec{bubbles-soundness}.

\section{Coloring bubbles}\labsec{colors}

\subsection{Red bubbles}

\begin{marginfigure}
  $$
  \R[\mathsf{{\limp}{+}c}]
    {\Gamma, A \J B, \Delta}
    {\Gamma \J A \limp B, \Delta}
  $$
  \caption{Classical multi-conclusion version of ${\limp}{+}$}
  \label{wrong-imp-pos}
\end{marginfigure}

With our new symmetric interpretation, we can start generalizing the rules of
\sys{BJ} to multiple conclusions. While for most rules one just has to replace
single-conclusion (resp. no-conclusion) solutions with open (resp. closed) ones
(more details will be given in the next section), the ${\limp}{+}$ rule stands
out as particularly problematic. Indeed if we content ourselves with the natural
generalization {\rnmsf{{\limp}{+}c}} of \reffig{wrong-imp-pos}, then we can
easily build a proof of the excluded middle like in \reffig{lk-tnd}, and thus
collapse to classical logic. This fact is well-known in the literature on
multi-conclusion intuitionistic sequent calculi, and the solution is usually to
discard the context of conclusions $\Delta$, as in the {\rnm{{\limp}R*i}} rule
of \reffig{multi-imp-intro}. But this would make our rule both non-local and
non-invertible.

\begin{marginfigure}
  $$
  \begin{array}{rclr}
    \hypo{A}~~~\cbubble{\color{black}S} &\step &\cbubble{\hypo{A}~~~\color{black}S} &\mathsf{f}{-}{+} \vspace{1em}\\
    \conc{A}~~~\cbubble{\color{black}S} &\step &\cbubble{\conc{A}~~~\color{black}S} &\mathsf{f}{+}{+} \\
  \end{array}
  $$
  \caption{$\mathbb{F}$-rules for red bubbles}
  \labfig{flow-red-bubbles}
\end{marginfigure}

A better solution comes from the nested sequent systems of Fitting
\sidecite{fitting-nested-2014} and Clouston et al.
\sidecite{clouston-annotation-free-2013}, where sequents can appear as
\emph{conclusions} of other sequents. In our chemical metaphor, this corresponds
to having \emph{red bubbles}. Then the key idea is to allow hypotheses to flow
into sequents that appear as conclusions\sidenote{This corresponds to the
{\rnm{Lift}} rule of \cite{fitting-nested-2014} and {\rnm{pl_1}} rule of
\cite{clouston-annotation-free-2013}.}, but \emph{not other conclusions}.
Graphically, this means that blue items can enter red bubbles (rule
{\rnmsf{f{-}{+}}} of \reffig{flow-red-bubbles}), but red items cannot: this is
reminiscent of the electromagnetic phenomemon of \emph{repulsion} between
objects charged with the same polarity.

\begin{figure*}
  \setlength{\fboxsep}{2pt}
\setlength{\arraycolsep}{0pt}
\newcommand{\vsp}{\vspace{2em}}
$$
\begin{array}[t]{rcr@{\qquad}|@{\qquad}rcr@{\vsp}}
       &\text{\textbf{Grishin (b)}} &&
       &\text{\textbf{Grishin (a)}} & \\

       &\stkfig{1}{bubbles-grishin-b-0} &{\limp}{+}, {\lor}{+} &
       &\stkfig{1}{bubbles-grishin-a-0} &{\lor}{+}, {\limp}{+} \\

\steps &\stkfig{1}{bubbles-grishin-b-1} &{\mathsf{f{-}{+}{\da}}} &
\steps &\stkfig{1}{bubbles-grishin-a-1} &{\mathsf{f{-}{+}{\da}}} \\

\step  &\stkfig{1}{bubbles-grishin-b-2} &{\lor}{-}, {\limp}{-} &
\step  &\stkfig{1}{bubbles-grishin-a-2} &{\limp}{-}, {\lor}{-} \\

\steps &\stkfig{1}{bubbles-grishin-b-3} &\mathsf{f{-}{\da}}, \mathsf{f{+}{\da}} &
\steps &\stkfig{1}{bubbles-grishin-a-3} &\mathsf{f{-}{\da}}, \mathsf{f{+}{\da}} \\

\steps &\stkfig{1}{bubbles-grishin-b-4} & &
\steps &\stkfig{1}{bubbles-grishin-a-4} &
\end{array}
$$

  \caption{Proof attempts for Grishin (a) and Grishin (b)}
  \labfig{bubbles-grishin}
\end{figure*}

To illustrate why this works, let us consider how one can manipulate with red
bubbles the classical equivalence $ (A \limp B) \lor C \semequiv A \limp (B \lor
C)$, that we already stumbled upon in the previous section. The begginings of
the proofs for both directions of the equivalence are depicted parallely in
\reffig{bubbles-grishin}. Indeed both proofs have a very similar structure:
\begin{enumerate}
  \item the first step is to decompose the conclusion with the new version of
  the rules {\rnm{{\lor}{+}}} and {\rnm{{\limp}{+}}}. While the former simply
  splits disjunctions in two, the latter encapsulates the antecedant and
  consequent of implications in a red bubble: the goal is to forbid the use of
  the antecedant to prove conclusions other than the consequent, as will become
  apparent later;
  \item then in both cases we want to apply the hypothesis $\hypo{A}$ in a
  forward step, either with $\hypo{A \limp B}$ or $\hypo{A \limp (B \lor C)}$.
  To do so, we need to bring the two hypotheses together in the same solution.
  And since items are trapped within bubbles, the only way to go is to move the
  blue $\hypo{A}$ inside the red bubble with the {\rnmsf{f{-}{+}}} rule;
  \item this time we decompose the hypothesis with the new version of the rules
  {\rnm{{\lor}{-}}} and {\rnm{{\limp}{-}}}. They are basically a local variant
  of those of \sys{BJ}: we encapsulate both subformulas in separate bubbles, but
  without touching to the conclusions of the ambient solution;
  \item now that all formulas have been decomposed, it only remains to bring
  together dual atoms for annihilation, and pop all empty bubbles. In Grishin
  (b) this is easy, because all necessary movements (indicated by green arrows)
  are valid: they only cross gray bubbles inward. In Grishin (a) this works for
  $\hypo{A}$ and $\conc{B}$, but not for $\conc{C}$ (orange dotted arrow): it
  would cross the red bubble, which is expressedly forbidden.
\end{enumerate}
Thus in order to prove Grishin (a) and recover classical logic, it suffices
either to add the {\rnmsf{f{+}{+}}} rule allowing red items to enter red bubbles
(\reffig{flow-red-bubbles}), or to use the {\rnmsf{{\limp}{+}c}} rule which
avoids red bubbles altogether. In the following we will settle for the first
option: we find it more elegant, because it explains the distinction between
intuitionistic and classical logic as a kind of \emph{physical law} independent
of logical connectives.

\subsection{Blue bubbles}

Now it is only natural to wonder: since bubbles can be colored in red, or
charged positively, would it also make sense to have \emph{blue} bubbles charged
\emph{negatively}? The answer is \emph{yes}, but we need to broader our logical
view and consider more exotic beasts: the adequately named
\emph{dual-intuitionistic} logic, and \emph{bi-intuitionistic logic}.

\begin{marginfigure}
  $$
  \begin{array}{rclr}
    \conc{A}~~~\hbubble{\color{black}S} &\step &\hbubble{\conc{A}~~~\color{black}S} &\mathsf{f}{+}{-} \vspace{1em}\\
    \hypo{A}~~~\hbubble{\color{black}S} &\step &\hbubble{\hypo{A}~~~\color{black}S} &\mathsf{f}{-}{-} \\
  \end{array}
  $$
  \caption{$\mathbb{F}$-rules for blue bubbles}
  \labfig{flow-blue-bubbles}
\end{marginfigure}

But for now let us stay at a purely syntactic level. The idea is very simple,
and can be summarized in two words: \emph{color swap}. Thus the law that ``blue
items can enter red bubbles, but red items cannot'' becomes a new law that ``red
items can enter blue bubbles, but blue items cannot'', which is enforced by
allowing only the use of the {\rnmsf{f{+}{-}}} rule in
\reffig{flow-blue-bubbles}. Well this is neat, but will not be of much use if
there is no way to spawn blue bubbles. Be it as it may: we can just craft a new
logical connective! Since red bubbles are produced by the implication connective
$A \limp B$, we define a dual \emph{exclusion} connective $A \lsub B$ (read
``$A$ excludes $B$''\sidenote{We ask for the reader's leniency regarding our
choice of symbol and terminology: in set theory this would be total nonsense,
since $A \subset B$ would read ``$A$ is included in $B$''. Even worse, in the
boolean algebra induced by set operations, $A \subset B$ is interpreted as $A$
\emph{implies} $B$\ldots~But all the arrow symbols were already taken, and we
want to emphasize the duality between exclusion and implication by mirroring the
symbol, as it is traditionally done with conjunction $\land$ and disjunction
$\lor$.}), whose heating rules are those of $\limp$ with swapped colors
(\reffig{heating-exclusion}).

\begin{marginfigure}
  $$
  \begin{array}{rclr}
    \hypo{A \lsub B} &\step &\hbubble{\hypo{A}~~~\conc{B}} &{\lsub}{-}\vspace{1em}\\
    \conc{A \lsub B} &\step &{\bubble{\conc{A}}}~~~\bubble{\hypo{B}} &{\lsub}{+}
  \end{array}
  $$
  \caption{$\mathbb{H}$-rules for exclusion $\lsub$}
  \labfig{heating-exclusion}
\end{marginfigure}

Not very surprisingly, the exclusion connective has already been studied in the
literature on intuitionistic logic, starting with the seminal paper of Rauszer
on \emph{Heyting-Brouwer logic}, i.e. intuitionistic logic to which we add
exclusion \sidecite{Rauszer1974-RAUSAA}. In this paper, exclusion was called
\emph{pseudo-difference}, to evoke its close connection with set-theoretical
difference. Indeed given two sets $A$ and $B$, one can define the set $A
\setminus B$ by comprehension as $\{x \mid x \in A \land x \not\in B\}$, which
is the set $A$ from which all elements of $B$ have been \emph{excluded}. With an
interpretation in boolean algebras, this corresponds to the classical connective
defined by the truth table of $A \land \neg B$, which is dual to the truth table
of $\neg A \lor B$ defining material implication.

While the first paper of Rauszer \cite{Rauszer1974-RAUSAA} belongs to the Polish
tradition of algebraic logic, she also explored in later works the
proof-theoretic \sidecite{rauszer_formalization_1974} and model-theoretic
\sidecite{rauszer_applications_1977} sides of the question. Many authors have
then deepened the proof theory of exclusion, whether in isolation from
implication in \emph{dual-intuitionistic} logic
\sidecite{urbas_dual-intuitionistic_1996}\sidecite{gore_dual_2000}, or with both
connectives in \emph{bi-intuitionistic} logic as in Rauszer's original
work\sidenote{Crolard \cite{crolard_subtractive_2001} and Aschieri
\cite{aschieri_natural_2018} have also explored the computational counterpart of
exclusion through the Curry-Howard correspondence, which is claimed by the first
author to be a typing operator for \emph{first-class coroutines}.}
\sidecite{postniece_proof_2010}\sidecite{pinto_relating_2011}. In particular, we
are going to rely in \refsec{bubbles-completeness} on the deep inference
calculus developed by Postniece in \cite{postniece_deep_2009} to get
completeness and cut admissibility of our symmetric bubble calculus introduced
in the next section.

\subsection{Polarized interpretation}

Let us now extend the formal definition of bubbles so that they can be colored:

\begin{definition}[Bubble]\labdef{pol-bubble}
  A \emph{bubble} is a solution enclosed in a membrane, which can be either
  unpolarized (neutral), charged positively, or charged negatively.
\end{definition}

Neutral bubbles are the usual ones depicted in gray, while positive and negative
bubbles correspond respectively to red and blue bubbles. We also update the
definition of solutions, which can now be open or closed:

\begin{definition}[Solution]\labdef{pol-solution}
  
  A \emph{solution} is a multiset of ions and bubbles. Its \emph{status} is
  either \emph{closed} or \emph{open}. Solutions $S$ can be represented
  textually with the following syntax:
  \begin{align*}
    S, T, U &\Coloneq \Gamma \J \Delta &
    \mathcal{S} &\Coloneq S_1 \sep \ldots \sep S_n \\
    \iota &\Coloneq A \mid S &
    \Gamma, \Delta &\Coloneq \iota_1, \ldots, \iota_n \\
    \J, \JB &\Coloneq {\seq} \mid {\piq{\mathcal{S}}} &&
  \end{align*}
\end{definition}

Note that in the textual syntax, bubbles are identified with \emph{subsolutions}
(\refdef{subsolution}), and their polarity is determined by their position
relative to branching operators; that is, a bubble $S \subsol T$ is either:
\begin{itemize}
  \item \emph{neutral} if $T = \Gamma \piq{\mathcal{S}} \Delta$ and $S \in
  \mathcal{S}$;
  \item \emph{positive} if $T = \Gamma \J \Delta$ and $S \in \Delta$;
  \item \emph{negative} if $T = \Gamma \J \Delta$ and $S \in \Gamma$.
\end{itemize}

Then we need to split our symmetric interpretation accordingly, so that positive
bubbles are mapped to implications, and negative bubbles to
exclusions\sidenote{Here we took inspiration from the work of Clouston et al. on
nested sequents for \sys{FILL} \cite{clouston-annotation-free-2013}.}:

\begin{definition}[Polarized symmetric interpretation]\labdef{pol-sinterp}
  The \emph{positive} and \emph{negative symmetric interpretations} of solutions
  $\psinterp{\cdot}$ and $\nsinterp{\cdot}$ are defined by mutual recursion as
  follows:
  \begin{align*}
    \psinterp{A} &= A &
    \nsinterp{A} &= A \\
    \psinterp{\Gamma \piq{\mathcal{S}} \Delta} &=
      \bigwedge_{S \in \mathcal{S}} \psinterp{S \mix \Gamma \seq \Delta} &
    \nsinterp{\Gamma \piq{\mathcal{S}} \Delta} &=
      \bigvee_{S \in \mathcal{S}} \nsinterp{S \mix \Gamma \seq \Delta} \\
    \psinterp{\Gamma \seq \Delta} &=
      \bigwedge_{\iota \in \Gamma} \nsinterp{\iota} \limp \bigvee_{\iota \in \Delta} \psinterp{\iota} &
    \nsinterp{\Gamma \seq \Delta} &=
      \bigwedge_{\iota \in \Gamma} \nsinterp{\iota} \lsub \bigvee_{\iota \in \Delta} \psinterp{\iota}
  \end{align*}
\end{definition}

One can easily check that the interpretation of a solution that does not contain
any negative (resp. positive) bubble will not contain any occurrence of the
exclusion (resp. implication) connective. This will be crucial later to
represent proofs of both intuitionistic, dual-intuitionistic and
bi-intuitionistic logic in the same system.

\section{Symmetric calculus}\labsec{symmetric-calculus}

\begin{figure*}
  \fontsize{10}{10.5}\selectfont
\begin{framed}
\renewcommand{\arraystretch}{2}
\begin{mathpar}
\begin{array}{r@{\quad}l}
\multicolumn{2}{c}{\identity} \\[1em]

\R[\mathsf{i{\da}}]
    {\Gamma \piq{} \Delta}
    {\Gamma, A \seq A, \Delta}
&
\R[\mathsf{i{\ua}}]
    {\Gamma \piq{\seq A \sep A \seq} \Delta}
    {\Gamma \seq \Delta}
\end{array}
\and
\begin{array}{c@{\quad}c}
\multicolumn{2}{c}{\resource} \\[1em]

\R[\mathsf{w{-}}]
    {\Gamma \J \Delta}
    {\Gamma, I \J \Delta}
&
\R[\mathsf{w{+}}]
    {\Gamma \J \Delta}
    {\Gamma \J I, \Delta}
\\
\R[\mathsf{c{-}}]
    {\Gamma, I, I \J \Delta}
    {\Gamma, I \J \Delta}
&
\R[\mathsf{c{+}}]
    {\Gamma \J I, I, \Delta}
    {\Gamma \J I, \Delta}
\end{array}
\\
\begin{array}{c@{\quad}c}
\multicolumn{2}{c}{\flow} \\[1em]

% \R[\mathsf{s{-}}]
%     {\piq{S \seq} \mix \Gamma \J \Delta}
%     {\Gamma, (\piq{S}) \J \Delta}
% &
% \R[\mathsf{s{+}}]
%     {\Gamma \J \Delta \mix \piq{\seq S}}
%     {\Gamma \J (\piq{S}), \Delta}
% \\

\multicolumn{2}{c}{
\R[\mathsf{f{\ua}}]
    {\Gamma \piq{\mathcal{S} \sep \Gamma' \piq{\mathcal{S'}} \Delta' \sep S} \Delta}
    {\Gamma \piq{\mathcal{S} \sep \Gamma' \piq{\mathcal{S'} \sep S} \Delta'} \Delta}
} \\

% \R[\mathsf{n{-}{\ua}}]
%     {(\Gamma, (\piq{\cS}) \J \Delta) \mix \piq{S}}
%     {\Gamma, (\piq{\cS \sep S}) \J \Delta}
% &
% \R[\mathsf{n{+}{\ua}}]
%     {\piq{S} \mix (\Gamma \J (\piq{\cS}), \Delta)}
%     {\Gamma  \J (\piq{\cS \sep S}), \Delta}
% \\
\R[\mathsf{f{-}{\da}}]
    {\Gamma \piq{\Gamma', I \JB \Delta' \sep \cS} \Delta}
    {\Gamma, I \piq{\Gamma' \JB \Delta' \sep \cS} \Delta}
&
\R[\mathsf{f{+}{\da}}]
    {\Gamma \piq{\cS \sep \Gamma' \JB I, \Delta'} \Delta}
    {\Gamma \piq{\cS \sep \Gamma' \JB \Delta'} I, \Delta}
\\
\R[\mathsf{f{-}{+}}{\da}]
    {\Gamma \J (\Gamma', I \JB \Delta'), \Delta}
    {\Gamma, I \J (\Gamma' \JB \Delta'), \Delta}
&
\R[\mathsf{f{+}{-}}{\da}]
    {\Gamma, (\Gamma' \JB I, \Delta') \J \Delta}
    {\Gamma, (\Gamma' \JB \Delta') \J I, \Delta}
\\
\R[\mathsf{f{-}{-}{\ua}}]
    {\Gamma, I, (\Gamma' \JB \Delta') \J \Delta}
    {\Gamma, (\Gamma', I \JB \Delta') \J \Delta}
&
\R[\mathsf{f{+}{+}{\ua}}]
    {\Gamma \J (\Gamma' \JB \Delta'), I, \Delta}
    {\Gamma \J (\Gamma' \JB I, \Delta'), \Delta}
\\
\R[\mathsf{f{-}{+}}{\ua}]
    {\Gamma, I \J (\Gamma' \JB \Delta'), \Delta}
    {\Gamma \J (\Gamma', I \JB \Delta'), \Delta}
&
\R[\mathsf{f{+}{-}}{\ua}]
    {\Gamma, (\Gamma' \JB \Delta') \J I, \Delta}
    {\Gamma, (\Gamma' \JB I, \Delta') \J \Delta}
\\
\R[\mathsf{f{-}{-}{\da}}]
    {\Gamma, (\Gamma', I \JB \Delta') \J \Delta}
    {\Gamma, I, (\Gamma' \JB \Delta') \J \Delta}
&
\R[\mathsf{f{+}{+}{\da}}]
    {\Gamma \J (\Gamma' \JB I, \Delta'), \Delta}
    {\Gamma \J (\Gamma' \JB \Delta'), I, \Delta}
\end{array}
\and
\begin{array}{cc}
\multicolumn{2}{c}{\membrane} \\[1em]

\multicolumn{2}{c}{
\R[\mathsf{p}]
    {\Gamma \piq{\cS} \Delta}
    {\Gamma \piq{\cS \sep \piq{}} \Delta}
} \\
\R[\mathsf{p{-}}]
    {\Gamma \piq{} \Delta}
    {\Gamma, (\piq{}) \seq \Delta}
&
\R[\mathsf{p{+}}]
    {\Gamma \piq{} \Delta}
    {\Gamma \seq (\piq{}), \Delta}
\\

\multicolumn{2}{c}{
\R[\mathsf{a}]
    {\Gamma \piq{S} \Delta}
    {\Gamma \piq{\piq{S}} \Delta}
} \\
\R[\mathsf{a{-}}]
    {\Gamma, S \J \Delta}
    {\Gamma, (\piq{S}) \J \Delta}
&
\R[\mathsf{a{+}}]
    {\Gamma \J S, \Delta}
    {\Gamma \J (\piq{S}), \Delta}
\\
\end{array}
\\
\begin{array}{c@{\quad}c}
\multicolumn{2}{c}{\heating} \\[1em]

\R[\top{-}]
    {\Gamma \J \Delta}
    {\Gamma, \top \J \Delta}
&
\R[\top{+}]
    {\Gamma  \piq{} \Delta}
    {\Gamma \seq \top, \Delta}
\\
\R[\bot{-}]
    {\Gamma \piq{} \Delta}
    {\Gamma, \bot \seq \Delta}
&
\R[\bot{+}]
    {\Gamma \J \Delta}
    {\Gamma \J \bot, \Delta}
\\
\R[\land{-}]
    {\Gamma, A, B \J \Delta}
    {\Gamma, A \land B \J \Delta}
&
\R[\land{+}]
    {\Gamma \piq{\seq A \sep \seq B {}} \Delta}
    {\Gamma \seq A \land B, \Delta}
\\
\R[\lor{-}]
    {\Gamma \piq{A \seq \sep B\seq} \Delta}
    {\Gamma, A \lor B \seq \Delta}
&
\R[\lor{+}]
    {\Gamma \J A, B, \Delta}
    {\Gamma \J A \lor B, \Delta}
\\
\R[{\limp}{-}]
    {\Gamma \piq{\seq A \sep B\seq} \Delta}
    {\Gamma, A \limp B \seq \Delta}
&
\R[{\limp}{+}]
    {\Gamma \J (A \seq B), \Delta}
    {\Gamma \J A \limp B, \Delta}
\\
\R[{\lsub}{-}]
    {\Gamma, (A \seq B) \J \Delta}
    {\Gamma, A \lsub B \J \Delta}
&
\R[{\lsub}{+}]
    {\Gamma \piq{\seq A \sep B\seq} \Delta}
    {\Gamma \seq A \lsub B, \Delta}
\\
\R[\forall{-}]
    {\Gamma, \subst{A}{t}{x} \J \Delta}
    {\Gamma, \forall x. A \J \Delta}
&
\R[\forall{+}]
    {\Gamma \J A, \Delta}
    {\Gamma \J \forall x. A, \Delta}
\\
\R[\exists{-}]
    {\Gamma, A \J \Delta}
    {\Gamma, \exists x. A \J \Delta}
&
\R[\exists{+}]
    {\Gamma \J \subst{A}{t}{x}, \Delta}
    {\Gamma \J \exists x. A, \Delta}
\end{array}
\end{mathpar}

In the {\rnm{\forall{+}}} and {\rnm{\exists{-}}} rules, $x$ is not free in
$\Gamma$, $\Delta$ and $\J$.
\end{framed}

  \caption{Sequent-style presentation of the symmetric bubble calculus \sys{B}}
  \labfig{sequent-B}
\end{figure*}

\section{Soundness}\labsec{bubbles-soundness}

\todo{ Soundness proof with respect to Heyting-Brouwer algebras, and deduce from
  it soundness of intuitionistic, dual-intuitionistic and classical fragments.}

\section{Completeness}\labsec{bubbles-completeness}

\todo{ Completeness proof by simulation of DBiInt from
\cite{postniece_deep_2009}, and deduce from it completeness of intuitionistic,
dual-intuitionistic and classical fragments + cut admissibility}

\section{Invertible calculus}\labsec{invertible-calculus}

\section{Decomposable calculus}\labsec{decomposable-calculus}

\todo{IDEA: add (duplicating variants of) the {\rnm{pl_2}} and {\rnm{pr_2}}
rules of \cite{clouston-annotation-free-2013} into the calculus. Indeed, they
allow taking red items outside of red bubbles: thus if the proof can be made
outside with a smaller context, it is more general and immediately solves all
subgoals, improving \emph{factorizability}.}
