\setchapterpreamble[u]{\margintoc}
\chapter{Symmetric Bubble Calculi}
\labch{bubbles-symm}

\todo{Introductory paragraph}

The chapter is organized as follows: in \refsec{non-determinism} we motivate our
quest for a system where all introduction rules for logical connectives are
\emph{invertible}, in order to reduce non-determinism in proof search and enable
a fully \emph{iconic} approach to proof building. To that effect, we relax in
\refsec{branching} the restriction to single-conclusion solutions, which
requires a new distinction between \emph{closed} and \emph{open} solutions. This
gives rise in \refsec{colors} to an extension of the syntax of solutions, where
bubbles can themselves be \emph{polarized}. In \refsec{symmetric-calculus} we
introduce a core \emph{symmetric bubble calculus} called ``system \sys{B}''. In
\refsec{bubbles-soundness} we show that by adding or removing 4 inference rules
to system \sys{B} that define the \emph{porosity} of bubbles, one gets 4 systems
that capture respectively intuitionistic, dual-intuitionistic, bi-intuitionistic
and classical logic. In \refsec{bubbles-completeness} we support this claim by
showing that the systems are not only sound, but also complete with respect to
standard sequent calculi. In \refsec{invertible-calculus} we present a fully
invertible variant of system \sys{B}, inspired by ideas that will be developed
in subsequent chapters. Despite the invertibility of introduction rules, it
turns out that this variant does not satisfy the \emph{decomposability}
property. We fix this defect in \refsec{decomposable-calculus} with yet another
variant of the system, finally achieving full iconicity.

\section{Non-determinism and iconicity}\labsec{non-determinism}

In all known sequent calculus formulations of intuitionistic logic, there are at
least two rules which are invariably \emph{non-invertible}:
\begin{enumerate}
  \item a left introduction rule for $\limp$ (there might be many ones, as in
  the calculus \sys{LJT} of \sidecite{dyckhoff_contraction-free_1992});
  \item the right introduction for either:
    \begin{itemize}
      \item $\lor$ when sequents have at most or exactly one conclusion;
      \item $\limp$ when sequents have multiple conclusions, e.g. in the
        multi-succedant variant of \sys{LJT} in
        \cite{dyckhoff_contraction-free_1992}.
    \end{itemize}
\end{enumerate}
In \sys{BJ}, this means that click actions on blue $\hypo{\limp}$ and red
$\conc{\lor}$ need to be performed in a specific order to be able to complete
proofs.

In his thesis \cite{guenot_nested_2013}, Guenot introduced a specific kind of
nested sequent system, where like in \sys{BJ} inference rules can be expressed
as rewriting rules. An interesting feature of these systems is that all
introduction rules for connectives are \emph{invertible}, which means that in
proof search, formulas can be completely decomposed until atoms are reached
before applying other rules. Non-determinism then arises in the choice of atoms
that are to be connected in axioms, as well as the choice of sub-sequents to be
duplicated for reuse.

In our graphical setting, this would translate to an interface where all click
actions are redundant. Although we already considered this possibility in
\refsec{dnd-completeness}, here it goes further by making even \emph{logical
connectives} superfluous, since all other rules work purely on the structure of
sequents. This means that all logical connectives could be replaced by
metaphorical constructs like bubbles, which suggest \emph{physically} the
possible transformations on the proof state. Unfortunately, the systems in
\cite{guenot_nested_2013} only handle classical logic and the implicative
fragment of intuitionistic logic. Thus began our quest for a nested sequent
system in the style of Guenot capturing full intuitionistic logic\sidenote{Other
nested sequent systems for full intuitionistic logic exist
\cite{fitting-nested-2014}, but they are based on tree-shaped proofs, and
thus ignore the whole \emph{raison d'être} of our concept of bubble.}.


\section{Conclusions and branching}\labsec{branching}

The first direction we followed was to relax the constraint that solutions must
be single-conclusion. Indeed as already noted in \refsec{sfl-backtracking}, a
notable property of sequent calculi with multiple conclusions is that their
right introduction rule for $\lor$ is invertible.

The main difficulty lies in the way one should interpret a multi-conclusion
solution $S$ as a formula $\rinterp{S}$. If we just take the asymmetric
interpretation (\refdef{ainterp}) and group conclusions disjunctively instead of
conjunctively, we get
$$
\rinterp{\Gamma \piq{\mathcal{S}} \Delta} =
\bigwedge \Gamma \limp \bigvee \Delta \land \bigwedge_{S \in \mathcal{S}}{\rinterp{S}}
$$
But this interpretation breaks on the 0-ary case when $\Delta$ is empty: instead
of seeing $\Gamma \piq{\mathcal{S}}$ as a node of the proof tree with hypotheses
$\Gamma$ and subgoals $\mathcal{S}$, it trivializes it to $\rinterp{\Gamma
\piq{\mathcal{S}}} = \bigwedge \Gamma \limp \bot$, i.e. a goal where one has to
find a contradiction in $\Gamma$; which is obviously not what we have in mind.

\begin{marginfigure}
  $$
  \R[\land R*]
    {\Gamma \seq A, \Delta}
    {\Gamma \seq B, \Delta}
    {\Gamma \seq A \land B, \Delta}
  $$
  \caption{Multi-conclusion right introduction rule for conjunction}
  \labfig{multi-and-intro}
\end{marginfigure}

A key observation was that in the rules of multi-conclusion sequent calculi, one
usually distributes the context $\Delta$ of conclusions in all premisses: this
restores a perfect symmetry with respect to the context of hypotheses $\Gamma$,
as illustrated by the {\rnm{\land R*}} rule (\reffig{multi-and-intro}). Then our
idea was that instead of implementing distribution/sharing of conclusions inside
inference rules, we could do it implicitly in the interpretation of solutions.
This is already what happens in the asymmetric interpretation for hypotheses
(\refdef{ainterp}); indeed the context $\Gamma$ is shared among subgoals,
because:
\begin{enumerate}
  \item it appears on the left of an implication $\limp$
  \item bubbles are joined conjunctively, and
  \item implication distributes over conjunction thanks to the equivalence $A
  \limp B \land C \semequiv (A \limp B) \land (A \limp C)$.
\end{enumerate}
But what does it mean precisely to share conclusions among subgoals? If we
consider the two following solutions:
$$
\underbrace{\bubble{\hypo{A}~~~\conc{B}}~~~\bubble{\hypo{C}~~~\conc{D}}~~~\conc{E}}_{S} \qquad\qquad
\underbrace{\bubble{\hypo{A}~~~\conc{B}~~~\conc{E}}~~~\bubble{\hypo{C}~~~\conc{D}~~~\conc{E}}}_{T}
$$
we would like to have $\rinterp{S} \semequiv \rinterp{T} \semequiv (A \limp B
\lor E) \land (C \limp D \lor E)$. Since disjunction distributes over
conjunction, a first naive try would give the following interpretation, where we
just replaced $\land$ by $\lor$ compared to the previous attempt:
$$
\rinterp{\Gamma \piq{\mathcal{S}} \Delta} =
\bigwedge{\Gamma} \limp \bigwedge_{S \in \mathcal{S}}{\rinterp{S}} \lor \bigvee \Delta
$$
But this immediately fails whenever $\mathcal{S} = \emptyset$, because it
trivializes to $\bigwedge \Gamma \limp \top \lor \bigvee \Delta \semequiv \top$
instead of $\bigwedge \Gamma \limp \bigvee \Delta$. The only way we found around
this defect was to internalize \emph{syntactically} a distinction between two
kinds of solutions\sidenote{In the terminology of Martin-Löf, we could say that
we now have two distinct forms of \emph{judgment}.}:
\begin{itemize}
  \item \emph{closed} solutions $\Gamma \piq{\mathcal{S}} \Delta$ correspond
  to branching nodes in the proof tree, or to closed leaves when $\mathcal{S} =
  \emptyset$ (i.e. solved subgoals). Thus it becomes sensical to have
  $\rinterp{\Gamma \piq{} \Delta} = \top$. In the asymmetric interpretation,
  closed solutions were encoded by solutions with no conclusions;
  \item \emph{open} solutions $\Gamma \seq \Delta$ correspond to open leaves
  in the proof tree (i.e. unsolved subgoals). In the asymmetric interpretation,
  they were encoded by solutions with one conclusion.
\end{itemize}
Then we keep the last proposed interpretation for closed solutions, and
interpret open solutions like usual sequents:
$$\rinterp{\Gamma \seq \Delta} = \bigwedge{\Gamma} \limp \bigvee{\Delta}$$ To be
able to abstract from the particular kind of solution at hand, we reframe the
syntax of solutions with so-called \emph{branching} operators $\J$:
\begin{align*}
  S, T, U &\Coloneq \Gamma \J \Delta \\
  \J, \JB &\Coloneq {\seq} \mid \piq{\mathcal{S}}
\end{align*}
Graphically, closed solutions with no bubbles can be distinguished from
open solutions by painting their \emph{background} on the proof canvas in
green, the intent being to suggest that they have already been solved. A
pathological example is the distinction between the closed empty bubble
$\bbubble{\phantom{a}}$ and the open empty bubble $\bubble{\phantom{a}}$,
who are interpreted respectively by $\rinterp{\piq{\piq{}}} = \top$ and
$\rinterp{\piq{\seq}} = \bot$.

Now if we come back to our target example, it still fails because we assign two
non-equivalent interpretations to $S$ and $T$. To show this, let us try to
derive the equivalence through some algebraic developments:
\begin{align}
  \rinterp{S} &= \top \limp ((A \limp B) \land (C \limp D)) \lor E \nonumber\\
              &\semequiv ((A \limp B) \land (C \limp D)) \lor E \\
              &\semequiv ((A \limp B) \lor E) \land ((C \limp D) \lor E) \\
              &\semequiv (A \limp B \lor E) \land (C \limp D \lor E) \labeq{grishin}\\
              &\semequiv ((A \limp B) \land (C \limp D)) \lor E \\
  \rinterp{T} &= \top \limp ((A \limp B \lor E) \land (C \limp D \lor E)) \lor \bot \nonumber
\end{align}
Wait, we did manage to prove it! The trick resides in \refeq{grishin}, which
uses twice the equivalence $(A \limp B) \lor C \equiv A \limp (B \lor C)$. It
turns out that this equivalence is true in classical logic, but \emph{not} in
intuitionistic logic. More precisely, it is the implication $G \triangleq (A
\limp (B \lor C)) \limp ((A \limp B) \lor C)$ which is not provable
intuitionistically\sidenote{This was already noticed in
\cite{clouston-annotation-free-2013}, with the linear version $(A \multimap (B
\parr C)) \multimap ((A \multimap B) \parr C)$ of $G$ called Grishin (a) and its
converse Grishin (b). More precisely, it is affirmed that while Grishin (b) is
valid in \sys{FILL}, the restriction of the classical multiplicative linear
logic \sys{MLL} to single-conclusion sequents, adding Grishin (a) makes
\sys{FILL} collapse to \sys{MLL}.}. Thus according to this interpretation $T$
entails $S$, but $S$ does not entail $T$.

To remedy this situation, we opted for a different strategy: instead of finding
a logical formula capturing the distributive semantics of conclusions over
subgoals, we hardcode the latter by defining the interpretation function on
closed solutions through \emph{non-structural} recursion. This gives the
following final definition:

\begin{definition}[Symmetric interpretation]\labdef{sinterp}
  The \emph{symmetric interpretation} of a solution is defined recursively by:
  \begin{align*}
    \rinterp{\Gamma \piq{\mathcal{S}} \Delta} &=
      \bigwedge_{S \in \mathcal{S}} \rinterp{S \mix \Gamma \seq \Delta} \\
    \rinterp{\Gamma \seq \Delta} &=
      \bigwedge \Gamma \seq \bigvee \Delta
  \end{align*}
  where the commutative \emph{mix} operator $\mix$ on solutions is defined by:
  \begin{align*}
    \Gamma \J \Delta \mix \Gamma' \seq \Delta' &=
      \Gamma, \Gamma' \J \Delta, \Delta' \\
    \Gamma \piq{\mathcal{S}} \Delta \mix \Gamma' \piq{\mathcal{S'}} \Delta' &=
      \Gamma, \Gamma' \piq{\mathcal{S} \sep \mathcal{S'}} \Delta, \Delta' \\
  \end{align*}
\end{definition}

This is the right approach for interpreting solutions with multiple conclusions,
as will be demonstrated formally in \refsec{bubbles-soundness}.

\section{Coloring bubbles}\labsec{colors}

\begin{marginfigure}
  $$
  \R[\mathsf{{\limp}{+}c}]
    {\Gamma, A \J B, \Delta}
    {\Gamma \J A \limp B, \Delta}
  $$
  \caption{Classical multi-conclusion version of ${\limp}{+}$}
  \label{wrong-imp-pos}
\end{marginfigure}

With our new symmetric interpretation, we can start generalizing the rules of
\sys{BJ} to multiple conclusions. While for most rules one just has to replace
single-conclusion (resp. no-conclusion) solutions with open (resp. closed) ones
(more details will be given in the next section), the ${\limp}{+}$ rule stands
out as particularly problematic. Indeed if we content ourselves with the natural
generalization {\rnmsf{{\limp}{+}c}} of \reffig{wrong-imp-pos}, then we can
easily build a proof of the excluded middle like in \reffig{lk-tnd}, and thus
collapse to classical logic. This fact is well-known in the literature on
multi-conclusion intuitionistic sequent calculi, and the solution is usually to
discard the context of conclusions $\Delta$, as in the {\rnm{{\limp}R*i}} rule
of \reffig{multi-imp-intro}. But this would make our rule both non-local and
non-invertible.

A better solution comes from the nested sequent systems of Fitting
\sidecite{fitting-nested-2014} and \sidecite{clouston-annotation-free-2013},
where sequents can appear as \emph{conclusions} of other sequents. In our
chemical metaphor, this corresponds to having \emph{red bubbles}. Then the key
idea is to allow hypotheses to flow into sequents that appear as
conclusions\sidenote{This corresponds to the {\rnm{Lift}} rule of
\cite{fitting-nested-2014} and {\rnm{pl_1}} rule of
\cite{clouston-annotation-free-2013}.}, but \emph{not other conclusions}.
Graphically, this means that only blue items can enter red bubbles, not red
items: this is reminiscent of the eletromagnetic phenomemon of \emph{repulsion}
between objects charged with the same polarity.

\begin{figure*}
  \setlength{\fboxsep}{2pt}
\setlength{\arraycolsep}{0pt}
\newcommand{\vsp}{\vspace{2em}}
$$
\begin{array}[t]{rcr@{\qquad}|@{\qquad}rcr@{\vsp}}
       &\text{\textbf{Grishin (b)}} &&
       &\text{\textbf{Grishin (a)}} & \\

       &\stkfig{1}{bubbles-grishin-b-0} &{\limp}{+}, {\lor}{+} &
       &\stkfig{1}{bubbles-grishin-a-0} &{\lor}{+}, {\limp}{+} \\

\steps{} &\stkfig{1}{bubbles-grishin-b-1} &{\mathsf{f{-}{+}{\da}}} &
\steps{} &\stkfig{1}{bubbles-grishin-a-1} &{\mathsf{f{-}{+}{\da}}} \\

\step{}  &\stkfig{1}{bubbles-grishin-b-2} &{\lor}{-}, {\limp}{-} &
\step{}  &\stkfig{1}{bubbles-grishin-a-2} &{\limp}{-}, {\lor}{-} \\

\steps{} &\stkfig{1}{bubbles-grishin-b-3} &\mathsf{f{-}{\da}}, \mathsf{f{+}{\da}} &
\steps{} &\stkfig{1}{bubbles-grishin-a-3} &\mathsf{f{-}{\da}}, \mathsf{f{+}{\da}} \\

\steps{} &\stkfig{1}{bubbles-grishin-b-4} & &
\steps{} &\stkfig{1}{bubbles-grishin-a-4} &
\end{array}
$$

  \caption{Proof attempts for Grishin (a) and Grishin (b)}
  \labfig{bubbles-grishin}
\end{figure*}

To illustrate why this works logically, let us consider the proof of the
classical equivalence $ (A \limp B) \lor C \lequiv A \limp (B \lor C)$, that we
already stumbled upon in the previous section. 


\section{Symmetric calculus}\labsec{symmetric-calculus}

\begin{figure*}
  \fontsize{10}{10.5}\selectfont
\begin{framed}
\renewcommand{\arraystretch}{2}
\begin{mathpar}
\begin{array}{r@{\quad}l}
\multicolumn{2}{c}{\intro{\identity}} \\[1em]

\R[\intro{i{\da}}]
    {\Gamma \piq{} \Delta}
    {\Gamma, A \seq A, \Delta}
&
\R[\intro{i{\ua}}]
    {\Gamma \piq{\seq A \sep A \seq} \Delta}
    {\Gamma \seq \Delta}
\end{array}
\and
\begin{array}{c@{\quad}c}
\multicolumn{2}{c}{\intro{\resource}} \\[1em]

\R[\intro{w{-}}]
    {\Gamma \J \Delta}
    {\Gamma, I \J \Delta}
&
\R[\intro{w{+}}]
    {\Gamma \J \Delta}
    {\Gamma \J I, \Delta}
\\
\R[\intro{c{-}}]
    {\Gamma, I, I \J \Delta}
    {\Gamma, I \J \Delta}
&
\R[\intro{c{+}}]
    {\Gamma \J I, I, \Delta}
    {\Gamma \J I, \Delta}
\end{array}
\\
\begin{array}{c@{\quad}c}
\multicolumn{2}{c}{\intro{\flow}} \\[1em]

% \R[\intro{s{-}}]
%     {\piq{S \seq} \mix \Gamma \J \Delta}
%     {\Gamma, (\piq{S}) \J \Delta}
% &
% \R[\intro{s{+}}]
%     {\Gamma \J \Delta \mix \piq{\seq S}}
%     {\Gamma \J (\piq{S}), \Delta}
% \\

\multicolumn{2}{c}{
\R[\intro{f{\ua}}]
    {\Gamma \piq{\mathcal{S} \sep \Gamma' \piq{\mathcal{S'}} \Delta' \sep S} \Delta}
    {\Gamma \piq{\mathcal{S} \sep \Gamma' \piq{\mathcal{S'} \sep S} \Delta'} \Delta}
} \\

% \R[\intro{n{-}{\ua}}]
%     {(\Gamma, (\piq{\cS}) \J \Delta) \mix \piq{S}}
%     {\Gamma, (\piq{\cS \sep S}) \J \Delta}
% &
% \R[\intro{n{+}{\ua}}]
%     {\piq{S} \mix (\Gamma \J (\piq{\cS}), \Delta)}
%     {\Gamma  \J (\piq{\cS \sep S}), \Delta}
% \\
\R[\intro{f{-}{\da}}]
    {\Gamma \piq{\Gamma', I \JB \Delta' \sep \cS} \Delta}
    {\Gamma, I \piq{\Gamma' \JB \Delta' \sep \cS} \Delta}
&
\R[\intro{f{+}{\da}}]
    {\Gamma \piq{\cS \sep \Gamma' \JB I, \Delta'} \Delta}
    {\Gamma \piq{\cS \sep \Gamma' \JB \Delta'} I, \Delta}
\\
\R[\intro{f{-}{+}{\da}}]
    {\Gamma \J (\Gamma', I \JB \Delta'), \Delta}
    {\Gamma, I \J (\Gamma' \JB \Delta'), \Delta}
&
\R[\intro{f{+}{-}{\da}}]
    {\Gamma, (\Gamma' \JB I, \Delta') \J \Delta}
    {\Gamma, (\Gamma' \JB \Delta') \J I, \Delta}
\\
\R[\intro{f{-}{-}{\ua}}]
    {\Gamma, I, (\Gamma' \JB \Delta') \J \Delta}
    {\Gamma, (\Gamma', I \JB \Delta') \J \Delta}
&
\R[\intro{f{+}{+}{\ua}}]
    {\Gamma \J (\Gamma' \JB \Delta'), I, \Delta}
    {\Gamma \J (\Gamma' \JB I, \Delta'), \Delta}
\\
\R[\intro{f{-}{+}{\ua}}]
    {\Gamma, I \J (\Gamma' \JB \Delta'), \Delta}
    {\Gamma \J (\Gamma', I \JB \Delta'), \Delta}
&
\R[\intro{f{+}{-}{\ua}}]
    {\Gamma, (\Gamma' \JB \Delta') \J I, \Delta}
    {\Gamma, (\Gamma' \JB I, \Delta') \J \Delta}
\\
\R[\intro{f{-}{-}{\da}}]
    {\Gamma, (\Gamma', I \JB \Delta') \J \Delta}
    {\Gamma, I, (\Gamma' \JB \Delta') \J \Delta}
&
\R[\intro{f{+}{+}{\da}}]
    {\Gamma \J (\Gamma' \JB I, \Delta'), \Delta}
    {\Gamma \J (\Gamma' \JB \Delta'), I, \Delta}
\end{array}
\and
\begin{array}{cc}
\multicolumn{2}{c}{\intro{\membrane}} \\[1em]

\multicolumn{2}{c}{
\R[\intro{p}]
    {\Gamma \piq{\cS} \Delta}
    {\Gamma \piq{\cS \sep {\piq{}}} \Delta}
} \\
\R[\intro{p{-}}]
    {\Gamma \piq{} \Delta}
    {\Gamma, (\piq{}) \seq \Delta}
&
\R[\intro{p{+}}]
    {\Gamma \piq{} \Delta}
    {\Gamma \seq (\piq{}), \Delta}
\\

\multicolumn{2}{c}{
\R[\intro{a}]
    {\Gamma \piq{S} \Delta}
    {\Gamma \piq{{\piq{S}}} \Delta}
} \\
\R[\intro{a{-}}]
    {\Gamma, S \J \Delta}
    {\Gamma, (\piq{S}) \J \Delta}
&
\R[\intro{a{+}}]
    {\Gamma \J S, \Delta}
    {\Gamma \J (\piq{S}), \Delta}
\\
\end{array}
\\
\begin{array}{c@{\quad}c}
\multicolumn{2}{c}{\intro{\heating}} \\[1em]

\R[\intro{\top{-}}]
    {\Gamma \J \Delta}
    {\Gamma, \top \J \Delta}
&
\R[\intro{\top{+}}]
    {\Gamma  \piq{} \Delta}
    {\Gamma \seq \top, \Delta}
\\
\R[\intro{\bot{-}}]
    {\Gamma \piq{} \Delta}
    {\Gamma, \bot \seq \Delta}
&
\R[\intro{\bot{+}}]
    {\Gamma \J \Delta}
    {\Gamma \J \bot, \Delta}
\\
\R[\intro{\land{-}}]
    {\Gamma, A, B \J \Delta}
    {\Gamma, A \land B \J \Delta}
&
\R[\intro{\land{+}}]
    {\Gamma \piq{\seq A \sep \seq B\ } \Delta}
    {\Gamma \seq A \land B, \Delta}
\\
\R[\intro{\lor{-}}]
    {\Gamma \piq{A \seq \sep B\seq} \Delta}
    {\Gamma, A \lor B \seq \Delta}
&
\R[\intro{\lor{+}}]
    {\Gamma \J A, B, \Delta}
    {\Gamma \J A \lor B, \Delta}
\\
\R[\intro{{\limp}{-}}]
    {\Gamma \piq{\seq A \sep B\seq} \Delta}
    {\Gamma, A \limp B \seq \Delta}
&
\R[\intro{{\limp}{+}}]
    {\Gamma \J (A \seq B), \Delta}
    {\Gamma \J A \limp B, \Delta}
\\
\R[\intro{{\lsub}{-}}]
    {\Gamma, (A \seq B) \J \Delta}
    {\Gamma, A \lsub B \J \Delta}
&
\R[\intro{{\lsub}{+}}]
    {\Gamma \piq{\seq A \sep B\seq} \Delta}
    {\Gamma \seq A \lsub B, \Delta}
\\
\R[\intro{\forall{-}}]
    {\Gamma, \subst{A}{t}{x} \J \Delta}
    {\Gamma, \forall x. A \J \Delta}
&
\R[\intro{\forall{+}}]
    {\Gamma \J A, \Delta}
    {\Gamma \J \forall x. A, \Delta}
\\
\R[\intro{\exists{-}}]
    {\Gamma, A \J \Delta}
    {\Gamma, \exists x. A \J \Delta}
&
\R[\intro{\exists{+}}]
    {\Gamma \J \subst{A}{t}{x}, \Delta}
    {\Gamma \J \exists x. A, \Delta}
\end{array}
\end{mathpar}

In the {\kl{\forall{+}}} and {\kl{\exists{-}}} rules, $x$ is not free in
$\Gamma$, $\Delta$ and $\J$.
\end{framed}

  \caption{Sequent-style presentation of the symmetric bubble calculus \sys{B}}
  \labfig{sequent-B}
\end{figure*}

\section{Soundness}\labsec{bubbles-soundness}

\section{Completeness}\labsec{bubbles-completeness}

\section{Invertible calculus}\labsec{invertible-calculus}

\section{Decomposable calculus}\labsec{decomposable-calculus}

\todo{IDEA: add the {\rnm{pl_2}} and {\rnm{pr_2}} rules of
\cite{clouston-annotation-free-2013} into (a non-invertible version of) the
calculus. Indeed, they allow taking red items outside of red bubbles: thus if
the proof can be made outside with a smaller context, it is more general and
immediately solves all subgoals, improving \emph{factorizability}.}
