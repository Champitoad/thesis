We are now going to prove the \emph{completeness} of the bi-intuitionistic (and
propositional) fragment $\sysBHB$ of system $\sysB$, by simulating the nested
sequent system \sys{DBiInt} of Postniece. In \sidecite{postniece_deep_2009} she
shows that this calculus is sound and complete with respect to another calculus
\sys{LBiInt}, and in Chapter 4 of her thesis \sidecite{postniece_proof_2010} she
proves that \sys{LBiInt} is sound and complete with respect to the Kripke
semantics of bi-intuitionistic logic. Importantly, the cut rule is shown to be
\emph{admissible} in both systems, through a syntactic process of
cut-elimination in \sys{LBiInt}. We will rely on this result to obtain
admissibility of the cut rule \rsf{i{\uparrow}} in $\sysBHB$, and by extension
in $\sysB$, $\sysBH$ and $\sysBB$. It might be interesting to have our own
internal cut-elimination procedure for system $\sysB$, notably to unveil its
computational content in the spirit of the Curry-Howard correspondence. But this
would lead us astray from the purpose of this thesis, and thus we leave this
task for future work.

\begin{definition}[Structure]
  The \emph{structures} of \sys{DBiInt} are generated by the following grammar:
  $$X, Y, Z \Coloneq \emptyset \mid A \mid (X,Y) \mid X \dseq Y$$
  The structural connective ``,'' (comma) is associative and commutative and
  $\emptyset$ is its unit. We always consider structures modulo these
  equivalences. To reduce parentheses, we assume that ``,'' binds tighter than
  $\dseq$. Thus, we write $X, Y \dseq Z$ to mean $(X, Y) \dseq Z$.
\end{definition}

\begin{definition}[Structure translation]
  The \emph{translation} $\dtrans{X}$ of a structure $X$ as a multiset of items
  $\Gamma$ is defined recursively as follows:
  \begin{align*}
    \dtrans{\emptyset} &= \emptyset &
    \dtrans{(X, Y)} &= \dtrans{X}, \dtrans{Y} \\
    \dtrans{A} &= A &
    \dtrans{(X \dseq Y)} &= \dtrans{X} \seq \dtrans{Y}
  \end{align*}
\end{definition}

Note that the translation $\dtrans{(-)}$ is clearly \emph{injective}: in fact
structures are isomorphic to multisets of items that contain only \emph{open}
subsolutions. Thus from now on, we will always apply the translation implicitly,
and rely on meta-variables $X, Y, Z$ to distinguish structures from arbitrary
solutions when necessary.

The rules of \sys{DBiInt} are given in \reffig{rules-dbiint}. Note that like
bubble calculi, \sys{DBiInt} is truly a \emph{deep inference} system, in the
sense that rules can be applied on sequents nested arbitrarily deep inside
structures. The main difference lies in the fact that proofs in \sys{DBiInt} are
\emph{trees} built up by composing traditional inference rules with multiple
premisses, while we use closed solutions (neutral bubbles) to internalize the
tree structure of proofs inside solutions. This gives a lot of expressive power
since closed solutions can themselves be nested in open solutions and thus
\emph{polarized}, a phenomemon which cannot be simulated in \sys{DBiInt}. This
is why we did not prove soundness in \refsec{bubbles-soundness} by simulating
directly system $\sysB$ in \sys{DBiInt}, and conversely this will explain the
ease with which \sys{DBiInt} can be simulated inside system $\sysB$.

\begin{figure*}
  % \renewcommand{\seq}{\dseq}
\begin{framed}
\renewcommand{\arraystretch}{2}
\begin{mathpar}
\begin{array}{c}
\text{\textsc{Identity}} \\[1em]
\R[\intro(dbiint){id}]
  {}
  {X, A \seq A, Y}
\end{array}
\and
\begin{array}{c@{\quad}c}
\multicolumn{2}{c}{\textsc{Propagation}} \\[1em]
\R[\rsf{\seq_{L1}}]
  {X, A, (X', A \seq Y') \seq Y}
  {X, (X', A \seq Y') \seq Y}
&
\R[\rsf{\seq_{R1}}]
  {X \seq (X' \seq A, Y'), A, Y}
  {X \seq (X' \seq A, Y'), Y}
\\
\R[\rsf{\seq_{L2}}]
  {X, A \seq (X', A \seq Y'), Y}
  {X, A \seq (X' \seq Y'), Y}
&
\R[\rsf{\seq_{R2}}]
  {X, (X' \seq A, Y') \seq A, Y}
  {X, (X' \seq Y') \seq A, Y}
\end{array}
\and
\begin{array}{c@{\quad}c}
\multicolumn{2}{c}{\text{\textsc{Logic}}} \\[1em]
\R[\rsf{\bot_L}]
  {}
  {X, \bot \seq Y}
&
\R[\rsf{\top_R}]
  {}
  {X \seq \top, Y}
\\
\R[\rsf{\land_L}]
  {X, A \land B, A, B \seq Y}
  {X, A \land B \seq Y}
&
\R[\rsf{\land_R}]
  {X \seq A, A \land B, Y}
  {X \seq B, A \land B, Y}
  {X \seq A \land B, Y}
\\
\R[\rsf{\lor_L}]
  {X, A \lor B, A \seq Y}
  {X, A \lor B, B \seq Y}
  {X, A \lor B \seq Y}
&
\R[\rsf{\lor_R}]
  {X \seq A, B, A \lor B, Y}
  {X \seq A \lor B, Y}
\\
\R[\rsf{\limp_L}]
  {X, A \limp B \seq A, Y}
  {X, A \limp B, B \seq Y}
  {X, A \limp B \seq Y}
&
\R[\rsf{\limp_R}]
  {X \seq (A \seq B), A \limp B, Y}
  {X \seq A \limp B, Y}
\\
\R[\rsf{\lsub_L}]
  {X, A \lsub B, (A \seq B) \seq Y}
  {X, A \lsub B \seq Y}
&
\R[\rsf{\lsub_R}]
  {X \seq A, A \lsub B, Y}
  {X, B \seq A \lsub B, Y}
  {X \seq A \lsub B, Y}
\end{array}
\end{mathpar}
\end{framed}

  \caption{Rules of the deep nested sequent system \sys{DBiInt}}
  \labfig{rules-dbiint}
\end{figure*}

\begin{definition}[Syntactic entailment]
  We say that $\Gamma$ \emph{entails} $\Delta$ in a fragment $\mathsf{F}$ of
  rules of system $\sysB$, written $\Gamma \prov{\mathsf{F}} \Delta$, if and
  only if $\Gamma \seq \Delta \steps_{\mathsf{F}} \piq{}$. Similarly, we say
  that $X$ entails $Y$ in a fragment $\mathsf{F}$ of rules of \sys{DBiInt},
  written $X \prov{\mathsf{F}} Y$, if and only if $X \seq Y$ has a proof in
  \sys{DBiInt} using only rules in $\mathsf{F}$.
\end{definition}

\begin{lemma}[Simulation of \sys{DBiInt}]\lablemma{simulation-dbiint} If $X
  \prov{\msys{DBiInt}} Y$ then $X \prov{\sysBHB \setminus
  \{\rsf{i{\uparrow}}\}} Y$.
\end{lemma}
\begin{proof}
  By induction on the derivation of $X \prov{\msys{DBiInt}} Y$.
\end{proof}

\begin{fact}[Completeness of \sys{DBiInt}]\labfact{completeness-dbiint}
  If $A \sementHB B$ then $A \prov{\msys{DBiInt}} B$.
\end{fact}

\begin{theorem}[Cut-free completeness]\labthm{bubbles-completeness}
  If $A \sementHB B$ then $A \prov{\sysBHB \setminus \{\rsf{i{\uparrow}}\}} B$.
\end{theorem}
\begin{proof}
  This follows immediately from \reffact{completeness-dbiint} and
  \reflemma{simulation-dbiint}.
\end{proof}

\begin{corollary}[Cut admissibility]\label{cor:cut-admissibility}

  If $\prov{\sysBHB} A$ then $\prov{\sysBHB \setminus \{\rsf{i{\uparrow}}\}} A$.
\end{corollary}
\begin{proof}
  This follows immediately from \refthm{bubbles-soundness} and
  \refthm{bubbles-completeness}.
\end{proof}

As in sequent calculus, every rule of system $\sysB$ other than
\rsf{i{\uparrow}} satisfies the \emph{subformula property}:

\begin{fact}[Subformula property]\label{cor:subformula-property}
  If $S \step_{\sysB \setminus \{\rsf{i{\uparrow}}\}} T$ and $A \subsol T$ then
  $A \subsol S$.
\end{fact}

Thanks to cut admissibility, we thus get that system \sys{B} is \emph{analytic}.
This has many nice consequences, a well-known one being that when searching for
a proof of a given solution $S$, one does not need to come up with or ``invent''
a formula that does not appear in $S$. This is crucial when designing
\emph{automated} decision procedures because it reduces drastically the search
space, but is also desirable in the setting of \emph{interactive} proof
building. Indeed with our Proof-by-Action interpretation of bubble calculi
(\refsec{bubbles-pba}), this means that all logical reasoning can be performed
by direct manipulation of \emph{what is already there}. Then the cut rule is
indispensable, but confined to a role of \emph{theory building}: it allows the
creation of \emph{lemmas}, in order to make proofs shorter and more tractable by
humans.

As noted in \cite{postniece_deep_2009}, one can simply ignore rules related to
the exclusion connective $\lsub$ to get a sound and complete system for
intuitionistic logic. In \sys{DBiInt}, these rules are the introduction rules
\rsf{{\lsub}_R} and \rsf{{\lsub}_L}, as well as the propagation rules
\rsf{{\seq}_{L1}} and \rsf{{\seq}_{R2}}. Indeed the latter are only useful in
combination with the former, since \rsf{{\lsub}_R} and \rsf{{\lsub}_L} are the
only rules of \sys{DBiInt} that can introduce nested sequents in negative
contexts. The situation is similar in system $\sysB$, and in fact the proof of
\reflemma{simulation-dbiint} shows that the intuitionistic fragment $\sysBH$ is
sufficient to simulate \sys{DBiInt} without the aforementioned rules. The dual
argument can be made for dual-intuitionistic logic, and thus we obtain
(cut-free) intuitionistic (resp. dual-intuitionistic) completeness of $\sysBH$
(resp. $\sysBB$):

\begin{corollary}[Intuitionistic completeness]
  ~\\\vspace{-1em}
  \begin{itemize}
    \item If $A \sementH B$ then $A \prov{\sysBH \setminus
    \{\rsf{i{\uparrow}}\}} B$.
    \item If $A \sementB B$ then $A \prov{\sysBB \setminus
    \{\rsf{i{\uparrow}}\}} B$.
  \end{itemize}
\end{corollary}

% Note that $\{\rsf{f{+}{-}{\downarrow}}, \rsf{f{-}{-}{\uparrow}},
% \rsf{{\lsub}{-}}, \rsf{{\lsub}{+}}\}$ are the only rules of $\sysBHB$ that
% involve negative solutions and/or exclusions.

\begin{marginfigure}
  $$
  \R[{\limp}{-}]
  {\R[{\limp}{+}]
  {\R[{\bot}{+}]
  {\R[{\bot}{-}]
  {\R[\rsf{p}]
  {\R[\rsf{f{+}{\downarrow}}]
  {\R[\rsf{f{+}{+}{\downarrow}}]
  {\R[\rsf{i{\downarrow}}]
  {\R[\rsf{p}{+}]
  {\R[\rsf{p}]
  {{} \piq{}}
  {{} \piq{\piq{}}}}
  {{} \piq{\seq (\piq{})}}}
  {{} \piq{\seq (A \seq A)}}}
  {{} \piq{\seq (A \seq), A}}}
  {{} \piq{\seq (A \seq)} A}}
  {{} \piq{\seq (A \seq) \sep \piq{}} A}}
  {{} \piq{\seq (A \seq) \sep \bot \seq} A}}
  {{} \piq{\seq (A \seq \bot) \sep \bot \seq} A}}
  {{} \piq{\seq \neg A \sep \bot \seq} A}}
  {\neg \neg A \seq A}
  $$
  \caption{Proof of DNE in system $\sysB$}
  \labfig{dne-bubbles}
\end{marginfigure}

\reffig{dne-bubbles} shows a proof of the double-negation elimination law
$\mathrm{DNE} \defeq \neg \neg A \seq A$ in system $\sysB$. Since $\sysBH$ is
intuitionistically complete, the well-known double-negation embedding of
classical logic into intuitionistic logic tells us that $\neg \neg A$ is
provable in $\sysBH$ (and a fortiori in $\sysB$) if $A$ is a theorem of
classical logic. Combining the two previous facts, we obtain the classical
completeness of system $\sysB$. In fact the proof of DNE only relies on the use
of the \rsf{f{+}{+}{\downarrow}} rule, so we can make the following stronger
statement:

\begin{corollary}[Classical completeness]\label{cor:bubbles-completeness-classical}
  If $A$ is a theorem of classical logic, then $\prov{\sysBH \cup
  \{\rsf{f{+}{+}{\downarrow}}\}} A$.
\end{corollary}
\begin{proof}
  By the double-negation embedding, we have $\prov{\sysBH} \neg \neg A$. Then we
  can build the following derivation:
  $$
  \R[\rsf{i{\uparrow}}]
  {\prftree[d]
  {\R[\rsf{p}]
  {\R[\rsf{f{+}{\downarrow}}]
  {\prftree[r][d]{DNE}  
  {\R[\rsf{p}]
  {{} \piq{}}
  {{} \piq{\piq{}}}}
  {{} \piq{\neg \neg A \seq A}}}
  {{} \piq{\neg \neg A \seq} A}}
  {{} \piq{\piq{} \sep \neg \neg A \seq} A}}
  {{} \piq{\seq \neg \neg A \sep \neg \neg A \seq} A}}
  {\seq A}
  $$
\end{proof}

Alas this argument makes use of the \rsf{i{\uparrow}} rule. Note however that
the reason we chose to prove completeness of $\sysBHB$ by simulating a rather
exotic system like \sys{DBiInt}, was that standard sequent calculi for
bi-intuitionistic logic like the one of Rauszer
\sidecite{rauszer_formalization_1974} are not \emph{cut-free} complete; and in
our literature review, \sys{DBiInt} was the cut-free system closest in its
syntax and rules to system $\sysB$. But for classical logic we do not have this
limitation, and thus it should be straightforward to simulate directly a
cut-free sequent calculus such as \sys{LK}.\todo{Do the proof in appendix,
should be quick}