An important thing to note, is that all the rules of \sys{DBiInt} are
\emph{invertible}\sidenote{Lemma 5.2.4 in Postniece's thesis
\cite{postniece_proof_2010}.}. Thus it follows immediately from
\reflemma{simulation-dbiint} that one can just take the translation of the rules
of \sys{DBiInt} in system $\sysB$, and get a complete, fully invertible
calculus. But this would be a waste of the expressive power and nice properties
of system $\sysB$, like linearity and locality.

Instead, we will target precisely the non-invertible rules of system $\sysB$,
and modify only those. From the proof of \reflemma{bubbles-local-soundness}, we
can identify which rules of system \sys{B} are invertible, and which are
probably not. Indeed if the soundness of a rule only relies on a chain of
equivalences, then it is necessarily invertible. On the contrary if it relies on
an inequality, then it is probably not invertible\sidenote{To ensure that it is
not invertible, we would need additionally to find a counter-model that
invalidates the converse inequality.}.

\begin{fact}[Invertibility of system \sys{B}]
  All rules in the fragment $\mathbb{I} \cup \{\rsf{w{-}},\rsf{w{+}}\} \cup
  \mathbb{M} \cup \mathbb{H} \setminus \{\rsf{{\limp}{-}},\rsf{{\lsub}{+}}\}$ of
  system \sys{B} are invertible.
\end{fact}

Thus the only remaining rules of system \sys{B} that are (most probably) not
invertible are the weakening rules $\{\rsf{w{-}},\rsf{w{+}}\}$, all the
$\mathbb{F}$-rules, and the $\mathbb{H}$-rules $\{\rsf{{\limp}{-}},
\rsf{{\lsub}{+}}\}$ that \emph{apply} an implication/exclusion\sidenote{For
quantifier rules, we conjecture that as in sequent calculus, the rules
$\{\rsf{\forall{+}}, \rsf{\exists{-}}\}$ are invertible, while the rules
$\{\rsf{\forall{-}}, \rsf{\exists{+}}\}$ are not.}. In \reffig{sequent-B-inv} we
define the \sys{B_{inv}} calculus, which results from the following
modifications to the previous rules:
\begin{description}
  \item[\textbf{Flow}]
  As shown by the simulation of \reflemma{simulation-dbiint}, the
  \emph{propagation} rules of \sys{DBiInt} combine an instance of
  \emph{contraction} followed by the application of a flow rule on the
  duplicated formula. Thus we can make all $\mathbb{F}$-rules of system $\sysB$
  invertible by systematically duplicating the moved formula, although this
  breaks \emph{linearity}.

  A downside of propagation rules in the style of \sys{DBiInt}, is that they
  create a lot of unnecessary copies of the moved formula $A$. Often, one will
  want to move $A$ in a subgoal/supergoal at a distance $n$ in the proof tree,
  with $n > 1$. Usually this would be performed by $n$ applications of
  $\mathbb{F}$-rules, which by linearity indeed just move the formula. But with
  propagation rules, $n$ copies of $A$ will be created, with one copy in each
  subgoal met on the path to the destination.

  To prevent this, one would need a way to copy formulas at an arbitrary
  distance. This can be done with inference rules that are \emph{doubly} deep,
  by encoding the path to the destination as a second solution context inside
  the context where the rule is applied\sidenote{Such rules are sometimes called
  \emph{super-switch} rules in the deep inference literature, see for instance
  Chapter 8, Section 2.1 of \cite{guenot_nested_2013}.}. It turns out to be hard
  to express in bubbles, because this requires a syntactic way to describe
  contexts that correspond to valid flow paths of arbitrary length\sidenote{This
  problem is solved trivially in the flower calculus (\refch{flowers}), by using
  so-called \emph{pollination} rules.}. But in principle it should be feasible,
  and would enable a more comfortable usa in a Proof-by-Action setting.

  Note also that we removed the \rsf{f{\uparrow}} rule of \reffig{sequent-B}.
  Indeed even after turning it into a propagation rule, the moved copy of the
  duplicated subgoal $S$ cannot be weakened because it lives in a neutral
  bubble. Thus the rule stays non-invertible, and cannot be included in
  \sys{B_{inv}}. Fortunately, we showed that it is admissible in
  \refthm{bubbles:cut-admissibility}, so this is not problematic.
  
  \item[\textbf{Weakening}]
  Here we follow a standard technique in sequent calculus, that merges the
  weakening rule in all \emph{terminal} rules of the calculus (i.e. rules with
  no premisses). In bubble calculi, the notion of premiss is captured by neutral
  bubbles; thus we incorporate weakenings in all rules that solve subgoals by
  closing solutions with no neutral bubbles. Those are the rules
  $\{\rsf{i{\downarrow}},\rsf{p{-}},\rsf{p{+}},\rsf{\bot{-},\rsf{\top{+}}}\}$.

  \item[\textbf{Implication/Exclusion}]
  The last source of non-invertibility is the $\mathbb{H}$-rules
  \rsf{{\limp}{-}} and \rsf{{\lsub}{+}}, that respectively allow to use an
  implication hypothesis, and prove an exclusion conclusion. Here we can just
  duplicate the implication/exclusion formula, as in the introduction rules of
  \sys{DBiInt}. Also like in \sys{DBiInt}, we removed the contraction rules
  \rsf{c{-}} and \rsf{c{+}}, which are now merged with these two rules as well
  as the $\mathbb{F}$-rules. Although contraction rules are invertible, they
  induce a lot of complexity in proof search, because it is hard to predict the
  (occurrences of) formulas that need to be duplicated. Thus it is preferable to
  design a calculus where they are admissible. But unlike what is done in
  \sys{DBiInt}, we did not incorporate contractions in other $\mathbb{H}$-rules.
  Thus we cannot simulate exactly all the introduction rules of \sys{DBiInt} in
  \sys{B_{inv}}.

\end{description}

These modifications only change superficially the proof of soundness, and thus
we do not redo it. As for completeness, we would need to prove that the
contraction rules are admissible, in order to solve the aforementioned problem
of simulating \sys{DBiInt}'s introduction rules:

\begin{lemma}[Admissibility of contraction]\lablemma{admissibility-contraction}
  % If $\prov{\sys{B_{inv}} \cup \{\rsf{c{-}},\rsf{c{+}}\}} S$, then
  % $\prov{\sys{B_{inv}}} S$.
  ~\\\vspace{-1em}
  \begin{itemize}
    \item If $\prov{\msys{B_{inv}}} S\select{\Gamma, A, A \J \Delta}$, then
    $\prov{\msys{B_{inv}}} S\select{\Gamma, A \J \Delta}$.
    \item If $\prov{\msys{B_{inv}}} S\select{\Gamma \J A, A, \Delta}$, then
    $\prov{\msys{B_{inv}}} S\select{\Gamma \J A, \Delta}$.
  \end{itemize}
\end{lemma}

Note that it is sufficient to prove admissibility of contraction on formulas,
rather than on arbitrary items. Indeed we only need it to simulate the
introduction rules of \sys{DBiInt}, which always duplicate formulas. For now we
only conjecture completeness of \sys{B_{inv}}, since it not clear what method
should be used to prove \reflemma{admissibility-contraction}. In her thesis
\cite{postniece_proof_2010} (Lemma 5.2.3), Postniece does a proof by recurrence
on the depth of the derivation, relying on the fact that all introduction rules
of \sys{DBiInt} preserve the principal formula; but this is precisely what we
are trying to avoid with our version of the rules. Of course, if we either give
up on this constraint or include contraction rules in \sys{B_{inv}}, then we
immediately get our desired result: \sys{B_{inv}} is a fully invertible
calculus, where the same fragments as system $\sysB$ capture intuitionistic,
dual-intuitionistic, bi-intuitionistic and classical logic.

\todo{discuss the standard proof-search procedure in (the intuitionistic
fragment of) \sys{B_{inv}}}

\todo{discuss the failure of full iconicity caused by the implicit contraction
in \rsf{{\limp}{-}}, and why this might not be solvable if we stick to the
symmetric bi-intuitionistic setting (because one needs a native way to represent
negative implications, which is done both in Guenot's system and the
flower calculus by breaking the symmetry and interpreting negative sequents as
implications rather than exclusions)}

\begin{figure*}
  \fontsize{10}{10.5}\selectfont
\begin{framed}
\renewcommand{\arraystretch}{2}
\begin{mathpar}
\begin{array}{c}
\identity \\[1em]

\R[\intro{i{\da}}]
    {\piq{}}
    {\Gamma, A \J A, \Delta}
\end{array}
\\
\begin{array}{c@{\quad}c}
\multicolumn{2}{c}{\flow} \\[1em]

% \R[\intro{s{-}}]
%     {\piq{S \seq} \mix \Gamma \J \Delta}
%     {\Gamma, (\piq{S}) \J \Delta}
% &
% \R[\intro{s{+}}]
%     {\Gamma \J \Delta \mix \piq{\seq S}}
%     {\Gamma \J (\piq{S}), \Delta}
% \\

\R[\intro{f{-}{\da}}]
    {\Gamma, I \piq{\Gamma', I \JB \Delta' \sep \cS} \Delta}
    {\Gamma, I \piq{\Gamma' \JB \Delta' \sep \cS} \Delta}
&
\R[\intro{f{+}{\da}}]
    {\Gamma \piq{\cS \sep \Gamma' \JB I, \Delta'} I, \Delta}
    {\Gamma \piq{\cS \sep \Gamma' \JB \Delta'} I, \Delta}
\\
\R[\intro{f{-}{+}{\da}}]
    {\Gamma, I \J (\Gamma', I \JB \Delta'), \Delta}
    {\Gamma, I \J (\Gamma' \JB \Delta'), \Delta}
&
\R[\intro{f{+}{-}{\da}}]
    {\Gamma, (\Gamma' \JB I, \Delta') \J I, \Delta}
    {\Gamma, (\Gamma' \JB \Delta') \J I, \Delta}
\\
\R[\intro{f{-}{-}{\ua}}]
    {\Gamma, I, (\Gamma', I \JB \Delta') \J \Delta}
    {\Gamma, (\Gamma', I \JB \Delta') \J \Delta}
&
\R[\intro{f{+}{+}{\ua}}]
    {\Gamma \J (\Gamma' \JB I, \Delta'), I, \Delta}
    {\Gamma \J (\Gamma' \JB I, \Delta'), \Delta}
\\
\R[\intro{f{-}{+}{\ua}}]
    {\Gamma, I \J (\Gamma', I \JB \Delta'), \Delta}
    {\Gamma \J (\Gamma', I \JB \Delta'), \Delta}
&
\R[\intro{f{+}{-}{\ua}}]
    {\Gamma, (\Gamma' \JB I, \Delta') \J I, \Delta}
    {\Gamma, (\Gamma' \JB I, \Delta') \J \Delta}
\\
\R[\intro{f{-}{-}{\da}}]
    {\Gamma, I, (\Gamma', I \JB \Delta') \J \Delta}
    {\Gamma, I, (\Gamma' \JB \Delta') \J \Delta}
&
\R[\intro{f{+}{+}{\da}}]
    {\Gamma \J (\Gamma' \JB I, \Delta'), I, \Delta}
    {\Gamma \J (\Gamma' \JB \Delta'), I, \Delta}
\end{array}
\and
\begin{array}{cc}
\multicolumn{2}{c}{\membrane} \\[1em]

\multicolumn{2}{c}{
\R[\intro{p}]
    {\Gamma \piq{\cS} \Delta}
    {\Gamma \piq{\cS \sep {\piq{}}} \Delta}
} \\
\R[\intro{p{-}}]
    {\piq{}}
    {\Gamma, (\piq{}) \J \Delta}
&
\R[\intro{p{+}}]
    {\piq{}}
    {\Gamma \J (\piq{}), \Delta}
\\

\multicolumn{2}{c}{
\R[\intro{a}]
    {\Gamma \piq{S} \Delta}
    {\Gamma \piq{{\piq{S}}} \Delta}
} \\
\R[\intro{a{-}}]
    {\Gamma, S \J \Delta}
    {\Gamma, (\piq{S}) \J \Delta}
&
\R[\intro{a{+}}]
    {\Gamma \J S, \Delta}
    {\Gamma \J (\piq{S}), \Delta}
\\
\end{array}
\\
\begin{array}{c@{\quad}c}
\multicolumn{2}{c}{\heating} \\[1em]

\R[\intro{\top{-}}]
    {\Gamma \J \Delta}
    {\Gamma, \top \J \Delta}
&
\R[\intro{\top{+}}]
    {\Gamma \J (\piq{}), \Delta}
    {\Gamma \J \top, \Delta}
\\
\R[\intro{\bot{-}}]
    {\Gamma, (\piq{}) \J \Delta}
    {\Gamma, \bot \J \Delta}
&
\R[\intro{\bot{+}}]
    {\Gamma \J \Delta}
    {\Gamma \J \bot, \Delta}
\\
\R[\intro{\land{-}}]
    {\Gamma, A, B \J \Delta}
    {\Gamma, A \land B \J \Delta}
&
\R[\intro{\land{+}}]
    {\Gamma \piq{\seq A \sep \seq B\ } \Delta}
    {\Gamma \seq A \land B, \Delta}
\\
\R[\intro{\lor{-}}]
    {\Gamma \piq{A \seq \sep B\seq} \Delta}
    {\Gamma, A \lor B \seq \Delta}
&
\R[\intro{\lor{+}}]
    {\Gamma \J A, B, \Delta}
    {\Gamma \J A \lor B, \Delta}
\\
\R[\intro{{\limp}{-}}]
    {\Gamma, A \limp B \piq{\seq A \sep B\seq} \Delta}
    {\Gamma, A \limp B \seq \Delta}
&
\R[\intro{{\limp}{+}}]
    {\Gamma \J (A \seq B), \Delta}
    {\Gamma \J A \limp B, \Delta}
\\
\R[\intro{{\lsub}{-}}]
    {\Gamma, (A \seq B) \J \Delta}
    {\Gamma, A \lsub B \J \Delta}
&
\R[\intro{{\lsub}{+}}]
    {\Gamma \piq{\seq A \sep B\seq} A \lsub B, \Delta}
    {\Gamma \seq A \lsub B, \Delta}
\\
\R[\intro{\forall{-}}]
    {\Gamma, \forall x. A, \subst{A}{t}{x} \J \Delta}
    {\Gamma, \forall x. A \J \Delta}
&
\R[\intro{\forall{+}}]
    {\Gamma \J A, \Delta}
    {\Gamma \J \forall x. A, \Delta}
\\
\R[\intro{\exists{-}}]
    {\Gamma, A \J \Delta}
    {\Gamma, \exists x. A \J \Delta}
&
\R[\intro{\exists{+}}]
    {\Gamma \J \subst{A}{t}{x}, \exists x. A, \Delta}
    {\Gamma \J \exists x. A, \Delta}
\end{array}
\end{mathpar}

In the {\kl{\forall{+}}} and {\kl{\exists{-}}} rules, $x$ is not free in
$\Gamma$, $\Delta$ and $\J$.
\end{framed}

  \caption{Rules for the invertible bubble calculus \sys{B_{inv}}}
  \labfig{sequent-B-inv}
\end{figure*}