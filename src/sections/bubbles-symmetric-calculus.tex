% \begin{figure*}
%   \begin{framed}
\renewcommand{\arraystretch}{1.25}
\begin{mathpar}
\begin{array}{r@{\quad}c@{\quad}lr}
  \multicolumn{4}{c}{\identity} \\[2em]

   \hypo{A}~~~\conc{A}
  &\step{}
  &
  &\mathsf{i}{\da} \\

   \conc{\Delta}
  &\step{}
  &\bubble{\conc{A}}~~~\bubble{\hypo{A}~~~\conc{\Delta}}
  &\mathsf{i}{\ua} \\
\end{array}
\and
\begin{array}{r@{\quad}c@{\quad}lr}
  \multicolumn{4}{c}{\resource} \\[2em]

    \hypo{A}
  &\step{}
  &
  &\mathsf{w} \\

    \hypo{A}
  &\step{}
  &\hypo{A~~~A}
  &\mathsf{c} \\
\end{array}
\vspace{2em}\\
\begin{array}{r@{\quad}c@{\quad}lr}
  \multicolumn{4}{c}{\flow} \\[2em]

    \hypo{A}~~~\bubble{\color{black}S}
  &\step{}
  &\bubble{\hypo{A}~~~S}
  &\mathsf{f{-}} \\
\end{array}
\and
\begin{array}{r@{\quad}c@{\quad}lr}
  \multicolumn{4}{c}{\membrane} \\[2em]

    \bubble{\phantom{S}}
  &\step{}
  &
  &\mathsf{p} \\
\end{array}
\vspace{2em}
\\
\begin{array}{r@{\quad}c@{\quad}lr@{\qquad\qquad}r@{\quad}c@{\quad}lr}
  \multicolumn{8}{c}{\heating} \\[2em]

    \hypo{\top}
  &\step{}
  &
  &\mathsf{\top{-}}

  &\conc{\top}
  &\step{}
  &
  &\mathsf{\top{+}} \\

    \hypo{\bot}~~~\conc{\Delta}
  &\step{}
  &
  &\mathsf{\bot{-}}

  &&&&\\

    \hypo{A \land B}
  &\step{}
  &\hypo{A}~~~\hypo{B}
  &\mathsf{\land{-}}

  &\conc{A \land B}
  &\step{}
  &\bubble{\conc{A}}~~~\bubble{\conc{B}}
  &\mathsf{\land{+}} \\

    \multirow{2}{*}{$\hypo{A \lor B}~~~\conc{\Delta}$}
  &\multirow{2}{*}{$\step{}$}
  &\multirow{2}{*}{$\bubble{\hypo{A}~~~\conc{\Delta}}~~~\bubble{\hypo{B}~~~\conc{\Delta}}$}
  &\multirow{2}{*}{$\mathsf{\lor{-}}$}

  &\conc{A \lor B}
  &\step{}
  &\conc{A}
  &\mathsf{\lor{+}_1} \\

  &&&

  &\conc{A \lor B}
  &\step{}
  &\conc{B}
  &\mathsf{\lor{+}_2} \\

    \hypo{A \limp B}~~~\conc{\Delta}
  &\step{}
  &\bubble{\conc{A}}~~~\bubble{\hypo{B}~~~\conc{\Delta}}
  &\mathsf{{\limp}{-}}

  &\conc{A \limp B}
  &\step{}
  &\hypo{A}~~~\conc{B}
  &\mathsf{{\limp}{+}} \\

    \hypo{\forall x. A}
  &\step{}
  &\hypo{\subst{A}{t}{x}}
  &\mathsf{\forall{-}}

  &\conc{\forall x. A}
  &\step{}
  &\conc{\subst{A}{y}{x}}
  &\mathsf{\forall{+}} \\

    \hypo{\exists x. A}
  &\step{}
  &\hypo{\subst{A}{y}{x}}
  &\mathsf{\exists{-}}

  &\conc{\exists x. A}
  &\step{}
  &\conc{\subst{A}{t}{x}}
  &\mathsf{\exists{+}} \\
\end{array}
\vspace{2em}
\end{mathpar}
In the {\rsf{i{\ua}}}, {\rnm{\bot{-}}}, {\rnm{\lor{-}}} and {\rnm{{\limp}{-}}} rules, $\Delta$
is either empty, or a singleton of one \kl{positive} ion.\\
In the {\rnm{\forall{+}}} and {\rnm{\exists{-}}} rules, $y$ is fresh.
\end{framed}

% \end{figure*}

\begin{figure*}
  \fontsize{10}{10.5}\selectfont
\begin{framed}
\renewcommand{\arraystretch}{2}
\begin{mathpar}
\begin{array}{r@{\quad}l}
\multicolumn{2}{c}{\identity} \\[1em]

\R[\mathsf{i{\da}}]
    {\Gamma \piq{} \Delta}
    {\Gamma, A \seq A, \Delta}
&
\R[\mathsf{i{\ua}}]
    {\Gamma \piq{\seq A \sep A \seq} \Delta}
    {\Gamma \seq \Delta}
\end{array}
\and
\begin{array}{c@{\quad}c}
\multicolumn{2}{c}{\resource} \\[1em]

\R[\mathsf{w{-}}]
    {\Gamma \J \Delta}
    {\Gamma, I \J \Delta}
&
\R[\mathsf{w{+}}]
    {\Gamma \J \Delta}
    {\Gamma \J I, \Delta}
\\
\R[\mathsf{c{-}}]
    {\Gamma, I, I \J \Delta}
    {\Gamma, I \J \Delta}
&
\R[\mathsf{c{+}}]
    {\Gamma \J I, I, \Delta}
    {\Gamma \J I, \Delta}
\end{array}
\\
\begin{array}{c@{\quad}c}
\multicolumn{2}{c}{\flow} \\[1em]

% \R[\mathsf{s{-}}]
%     {\piq{S \seq} \mix \Gamma \J \Delta}
%     {\Gamma, (\piq{S}) \J \Delta}
% &
% \R[\mathsf{s{+}}]
%     {\Gamma \J \Delta \mix \piq{\seq S}}
%     {\Gamma \J (\piq{S}), \Delta}
% \\

\multicolumn{2}{c}{
\R[\mathsf{f{\ua}}]
    {\Gamma \piq{\mathcal{S} \sep \Gamma' \piq{\mathcal{S'}} \Delta' \sep S} \Delta}
    {\Gamma \piq{\mathcal{S} \sep \Gamma' \piq{\mathcal{S'} \sep S} \Delta'} \Delta}
} \\

% \R[\mathsf{n{-}{\ua}}]
%     {(\Gamma, (\piq{\cS}) \J \Delta) \mix \piq{S}}
%     {\Gamma, (\piq{\cS \sep S}) \J \Delta}
% &
% \R[\mathsf{n{+}{\ua}}]
%     {\piq{S} \mix (\Gamma \J (\piq{\cS}), \Delta)}
%     {\Gamma  \J (\piq{\cS \sep S}), \Delta}
% \\
\R[\mathsf{f{-}{\da}}]
    {\Gamma \piq{\Gamma', I \JB \Delta' \sep \cS} \Delta}
    {\Gamma, I \piq{\Gamma' \JB \Delta' \sep \cS} \Delta}
&
\R[\mathsf{f{+}{\da}}]
    {\Gamma \piq{\cS \sep \Gamma' \JB I, \Delta'} \Delta}
    {\Gamma \piq{\cS \sep \Gamma' \JB \Delta'} I, \Delta}
\\
\R[\mathsf{f{-}{+}}{\da}]
    {\Gamma \J (\Gamma', I \JB \Delta'), \Delta}
    {\Gamma, I \J (\Gamma' \JB \Delta'), \Delta}
&
\R[\mathsf{f{+}{-}}{\da}]
    {\Gamma, (\Gamma' \JB I, \Delta') \J \Delta}
    {\Gamma, (\Gamma' \JB \Delta') \J I, \Delta}
\\
\R[\mathsf{f{-}{-}{\ua}}]
    {\Gamma, I, (\Gamma' \JB \Delta') \J \Delta}
    {\Gamma, (\Gamma', I \JB \Delta') \J \Delta}
&
\R[\mathsf{f{+}{+}{\ua}}]
    {\Gamma \J (\Gamma' \JB \Delta'), I, \Delta}
    {\Gamma \J (\Gamma' \JB I, \Delta'), \Delta}
\\
\R[\mathsf{f{-}{+}}{\ua}]
    {\Gamma, I \J (\Gamma' \JB \Delta'), \Delta}
    {\Gamma \J (\Gamma', I \JB \Delta'), \Delta}
&
\R[\mathsf{f{+}{-}}{\ua}]
    {\Gamma, (\Gamma' \JB \Delta') \J I, \Delta}
    {\Gamma, (\Gamma' \JB I, \Delta') \J \Delta}
\\
\R[\mathsf{f{-}{-}{\da}}]
    {\Gamma, (\Gamma', I \JB \Delta') \J \Delta}
    {\Gamma, I, (\Gamma' \JB \Delta') \J \Delta}
&
\R[\mathsf{f{+}{+}{\da}}]
    {\Gamma \J (\Gamma' \JB I, \Delta'), \Delta}
    {\Gamma \J (\Gamma' \JB \Delta'), I, \Delta}
\end{array}
\and
\begin{array}{cc}
\multicolumn{2}{c}{\membrane} \\[1em]

\multicolumn{2}{c}{
\R[\mathsf{p}]
    {\Gamma \piq{\cS} \Delta}
    {\Gamma \piq{\cS \sep \piq{}} \Delta}
} \\
\R[\mathsf{p{-}}]
    {\Gamma \piq{} \Delta}
    {\Gamma, (\piq{}) \seq \Delta}
&
\R[\mathsf{p{+}}]
    {\Gamma \piq{} \Delta}
    {\Gamma \seq (\piq{}), \Delta}
\\

\multicolumn{2}{c}{
\R[\mathsf{a}]
    {\Gamma \piq{S} \Delta}
    {\Gamma \piq{\piq{S}} \Delta}
} \\
\R[\mathsf{a{-}}]
    {\Gamma, S \J \Delta}
    {\Gamma, (\piq{S}) \J \Delta}
&
\R[\mathsf{a{+}}]
    {\Gamma \J S, \Delta}
    {\Gamma \J (\piq{S}), \Delta}
\\
\end{array}
\\
\begin{array}{c@{\quad}c}
\multicolumn{2}{c}{\heating} \\[1em]

\R[\top{-}]
    {\Gamma \J \Delta}
    {\Gamma, \top \J \Delta}
&
\R[\top{+}]
    {\Gamma  \piq{} \Delta}
    {\Gamma \seq \top, \Delta}
\\
\R[\bot{-}]
    {\Gamma \piq{} \Delta}
    {\Gamma, \bot \seq \Delta}
&
\R[\bot{+}]
    {\Gamma \J \Delta}
    {\Gamma \J \bot, \Delta}
\\
\R[\land{-}]
    {\Gamma, A, B \J \Delta}
    {\Gamma, A \land B \J \Delta}
&
\R[\land{+}]
    {\Gamma \piq{\seq A \sep \seq B {}} \Delta}
    {\Gamma \seq A \land B, \Delta}
\\
\R[\lor{-}]
    {\Gamma \piq{A \seq \sep B\seq} \Delta}
    {\Gamma, A \lor B \seq \Delta}
&
\R[\lor{+}]
    {\Gamma \J A, B, \Delta}
    {\Gamma \J A \lor B, \Delta}
\\
\R[{\limp}{-}]
    {\Gamma \piq{\seq A \sep B\seq} \Delta}
    {\Gamma, A \limp B \seq \Delta}
&
\R[{\limp}{+}]
    {\Gamma \J (A \seq B), \Delta}
    {\Gamma \J A \limp B, \Delta}
\\
\R[{\lsub}{-}]
    {\Gamma, (A \seq B) \J \Delta}
    {\Gamma, A \lsub B \J \Delta}
&
\R[{\lsub}{+}]
    {\Gamma \piq{\seq A \sep B\seq} \Delta}
    {\Gamma \seq A \lsub B, \Delta}
\\
\R[\forall{-}]
    {\Gamma, \subst{A}{t}{x} \J \Delta}
    {\Gamma, \forall x. A \J \Delta}
&
\R[\forall{+}]
    {\Gamma \J A, \Delta}
    {\Gamma \J \forall x. A, \Delta}
\\
\R[\exists{-}]
    {\Gamma, A \J \Delta}
    {\Gamma, \exists x. A \J \Delta}
&
\R[\exists{+}]
    {\Gamma \J \subst{A}{t}{x}, \Delta}
    {\Gamma \J \exists x. A, \Delta}
\end{array}
\end{mathpar}

In the {\rnm{\forall{+}}} and {\rnm{\exists{-}}} rules, $x$ is not free in
$\Gamma$, $\Delta$ and $\J$.
\end{framed}

  \caption{Sequent-style presentation of system \sys{B}}
  \labfig{sequent-B}
\end{figure*}

\begin{figure*}
  \fontsize{9.5}{10}\selectfont
\begin{framed}
  % \renewcommand{\arraystretch}{2}
  \begin{mathpar}
  \begin{array}{r@{\quad}c@{\quad}ll}
    \multicolumn{4}{c}{\kl{\identity}} \\[2em]
  
     \hypo{A}~~~\conc{A}
    &\step{}
    &\bsheet{\phantom{\hypo{A}~~~\conc{A}}}
    &\kl{i{\da}} \\
  
     
     \phantom{\bubble{\conc{A}}~~~\bubble{\hypo{A}}}
    &\step{}
    &\bsheet{\bubble{\conc{A}}~~~\bubble{\hypo{A}}}
    &\kl{i{\ua}} \\
  \end{array}
  \and
  \begin{array}{r@{\quad}c@{\quad}ll@{\qquad\qquad}r@{\quad}c@{\quad}ll}
    \multicolumn{8}{c}{\kl{\resource}} \\[2em]
  
     \gAsheet{\hypo{I}}
    &\step{}
    &\gAsheet{\phantom{\hypo{I}}}
    &\kl{w{-}} &

     \gAsheet{\conc{I}}
    &\step{}
    &\gAsheet{\phantom{\hypo{I}}}
    &\kl{w{+}} \\
  
     \gAsheet{\hypo{I}}
    &\step{}
    &\gAsheet{\hypo{I~~~I}}
    &\kl{c{-}} &

     \gAsheet{\conc{I}}
    &\step{}
    &\gAsheet{\conc{I~~~I}}
    &\kl{c{+}} \\
  \end{array}
  \vspace{2em}\\
  \begin{array}{r@{\quad}c@{\quad}ll@{\qquad\qquad}r@{\quad}c@{\quad}ll}
    \multicolumn{8}{c}{\kl{\flow}} \\[2em]

    %  \gAsheet{\hbbubble{\gBbubble{\color{black}S}}}
    % &\step{}
    % &\bsheet{\gAbubble{\hgBbubble{\color{black}S}}}
    % &\kl{s{-}} &

    %  \gAsheet{\cbbubble{\gBbubble{\color{black}S}}}
    % &\step{}
    % &\bsheet{\gAbubble{\cgBbubble{\color{black}S}}}
    % &\kl{s{+}} \\

    \multicolumn{8}{c}{
      \begin{array}{r@{\quad}c@{\quad}ll}
         \bsheet{\bbubble{\gBbubble{\color{black}T}~~~\color{black}S}}
        &\step{}
        &\bsheet{\gBbubble{\color{black}T}~~~\bbubble{\color{black}S}}
        &\kl{f{\ua}}
      \end{array}
    } \\[3em]

    %  \gAsheet{\hbbubble{\gBbubble{\color{black}S}~~~\color{black}T}}
    % &\step{}
    % &\bsheet{\gBbubble{\color{black}S}~~~\hbbubble{\color{black}T}}
    % &\kl{n{-}{\ua}} &

    %  \gAsheet{\cbbubble{\gBbubble{\color{black}S}~~~\color{black}T}}
    % &\step{}
    % &\bsheet{\gBbubble{\color{black}S}~~~\cbbubble{\color{black}T}}
    % &\kl{n{+}{\ua}} \\
  
     \bsheet{\hypo{I}~~~\gBbubble{\color{black}S}}
    &\step{}
    &\bsheet{\gBbubble{\hypo{I}~~~\color{black}S}}
    &\kl{f{-}{\da}} &

     \bsheet{\conc{I}~~~\gBbubble{\color{black}S}}
    &\step{}
    &\bsheet{\gBbubble{\conc{I}~~~\color{black}S}}
    &\kl{f{+}{\da}} \\

     \gAsheet{\hypo{I}~~~\cgBbubble{\color{black}S}}
    &\step{}
    &\gAsheet{\cgBbubble{\hypo{I}~~~\color{black}S}}
    &\kl{f{-}{+}{\da}} &

     \gAsheet{\conc{I}~~~\hgBbubble{\color{black}S}}
    &\step{}
    &\gAsheet{\hgBbubble{\conc{I}~~~\color{black}S}}
    &\kl{f{+}{-}{\da}} \\

     \gAsheet{\hgBbubble{\hypo{I}~~~\color{black}S}}
    &\step{}
    &\gAsheet{\hypo{I}~~~\hgBbubble{\color{black}S}}
    &\kl{f{-}{-}{\ua}} &

     \gAsheet{\cgBbubble{\conc{I}~~~\color{black}S}}
    &\step{}
    &\gAsheet{\conc{I}~~~\cgBbubble{\color{black}S}}
    &\kl{f{+}{+}{\ua}} \\

     \gAsheet{\cgBbubble{\hypo{I}~~~\color{black}S}}
    &\step{}
    &\gAsheet{\hypo{I}~~~\cgBbubble{\color{black}S}}
    &\kl{f{-}{+}{\ua}} &

     \gAsheet{\hgBbubble{\conc{I}~~~\color{black}S}}
    &\step{}
    &\gAsheet{\conc{I}~~~\hgBbubble{\color{black}S}}
    &\kl{f{+}{-}{\ua}} \\

     \gAsheet{\hypo{I}~~~\hgBbubble{\color{black}S}}
    &\step{}
    &\gAsheet{\hgBbubble{\hypo{I}~~~\color{black}S}}
    &\kl{f{-}{-}{\da}} &

     \gAsheet{\conc{I}~~~\cgBbubble{\color{black}S}}
    &\step{}
    &\gAsheet{\cgBbubble{\conc{I}~~~\color{black}S}}
    &\kl{f{+}{+}{\da}} \\
  \end{array}
  \and
  \begin{array}{r@{\quad}c@{\quad}ll}
    \multicolumn{4}{c}{\kl{\membrane}} \\[2em]

     \bsheet{\bbubble{\phantom{S}}}
    &\step{}
    &\bsheet{\phantom{\bbubble{\phantom{S}}}}
    &\kl{p} \\

     \hbbubble{\phantom{S}}
    &\step{}
    &\bsheet{\phantom{\bbubble{\phantom{S}}}}
    &\kl{p{-}} \\

     \cbbubble{\phantom{S}}
    &\step{}
    &\bsheet{\phantom{\bbubble{\phantom{S}}}}
    &\kl{p{+}} \\

     \bsheet{\bbubble{\gBbubble{\color{black}S}}}
    &\step{}
    &\bsheet{\gBbubble{\color{black}S}}
    &\kl{a} \\

     \gAsheet{\hbbubble{\gBbubble{\color{black}S}}}
    &\step{}
    &\gAsheet{\hgBbubble{\color{black}S}}
    &\kl{a{-}} \\

     \gAsheet{\cbbubble{\gBbubble{\color{black}S}}}
    &\step{}
    &\gAsheet{\cgBbubble{\color{black}S}}
    &\kl{a{+}} \\
  \end{array}
  \vspace{2em}\\
  \begin{array}{r@{\quad}c@{\quad}ll@{\qquad\qquad}r@{\quad}c@{\quad}ll}
    \multicolumn{8}{c}{\kl{\heating}} \\[2em]
  
     \gAsheet{\hypo{\top}}
    &\step{}
    &\gAsheet{\phantom{\hypo{\top}}}
    &\kl{\top{-}}
  
    &\conc{\top}
    &\step{}
    &\bsheet{\phantom{\top}}
    &\kl{\top{+}} \\
  
     \hypo{\bot}
    &\step{}
    &\bsheet{\phantom{\bot}}
    &\kl{\bot{-}}

    &\gAsheet{\conc{\bot}}
    &\step{}
    &\gAsheet{\phantom{\conc{\bot}}}
    &\kl{\bot{+}} \\
  
     \gAsheet{\hypo{A \land B}}
    &\step{}
    &\gAsheet{\hypo{A}~~~\hypo{B}}
    &\kl{\land{-}}
  
    &\conc{A \land B}
    &\step{}
    &\bsheet{\bubble{\conc{A}}~~~\bubble{\conc{B}}}
    &\kl{\land{+}} \\
  
     \hypo{A \lor B}
    &\step{}
    &\bsheet{\bubble{\hypo{A}}~~~\bubble{\hypo{B}}}
    &\kl{\lor{-}}

    &\gAsheet{\conc{A \lor B}}
    &\step{}
    &\gAsheet{\conc{A}~~~\conc{B}}
    &\kl{\lor{+}} \\
  
     \hypo{A \limp B}
    &\step{}
    &\bsheet{\bubble{\conc{A}}~~~\bubble{\hypo{B}}}
    &\kl{{\limp}{-}}
  
    &\gAsheet{\conc{A \limp B}}
    &\step{}
    &\gAsheet{\cbubble{\hypo{A}~~~\conc{B}}}
    &\kl{{\limp}{+}} \\
  
     \gAsheet{\hypo{A \lsub B}}
    &\step{}
    &\gAsheet{\hbubble{\hypo{A}~~~\conc{B}}}
    &\kl{{\lsub}{-}}

    &\conc{A \lsub B}
    &\step{}
    &\bsheet{\bubble{\conc{A}}~~~\bubble{\hypo{B}}}
    &\kl{{\lsub}{+}} \\
  
     \gAsheet{\hypo{\forall x. A}}
    &\step{}
    &\gAsheet{\hypo{\subst{A}{t}{x}}}
    &\kl{\forall{-}}
  
    &\gAsheet{\conc{\forall x. A}}
    &\step{}
    &\gAsheet{\conc{\subst{A}{y}{x}}}
    &\kl{\forall{+}} \\
  
     \gAsheet{\hypo{\exists x. A}}
    &\step{}
    &\gAsheet{\hypo{\subst{A}{y}{x}}}
    &\kl{\exists{-}}
  
    &\gAsheet{\conc{\exists x. A}}
    &\step{}
    &\gAsheet{\conc{\subst{A}{t}{x}}}
    &\kl{\exists{+}} \\
  \end{array}
  \vspace{2em}
  \end{mathpar}
  In the {\rnm{\forall{+}}} and {\rnm{\exists{-}}} rules, $y$ is fresh.
  \end{framed}
  
  \caption{Graphical presentation of system \sys{B}}
  \labfig{graphical-B}
\end{figure*}

As for the asymmetric bubble calculus \sys{BJ}, the rules of our full symmetric
bubble calculus system $\sysB$ enjoy both a sequent-style and a graphical
presentation, given respectively in \reffig{sequent-B} and \reffig{graphical-B}.
The presence of closed and open solutions complicates quite a bit the graphical
representation of rules, thus some explanations are in order:
\begin{description}
  \item[Closed solutions] In \refsec{branching}, we mentioned that closed
solutions with no neutral bubbles can be distinguished visually from open
solutions by painting their background in a different color; we chose a light
green, to suggest that they denote \emph{solved} subgoals. In
\reffig{graphical-B}, we emphasize systematically the distinction by extending
this convention to all closed solutions.
  \item[Generic statuses] As can be seen in \reffig{sequent-B}, many rules of
system $\sysB$ are \emph{generic} over branching operators $\J, \JB$, which
determine whether a solution is closed or open, i.e. its \emph{status}. The
challenge is thus to find an iconic counterpart to the symbols $\J, \JB$, that
fulfills the same function of \emph{meta-variable} ranging over solution
statuses. Since we already use the background color to represent the status of
concrete solutions, we chose to do the same with abstract ones: each new color
other than green will stand for the status of the solution associated to the
given location of the canvas. For instance in the \rsf{f{-}{+}{\downarrow}}
rule, the status of the ambient solution where the rule is applied is denoted by
a light yellow background, while the status of the solution $S$ enclosed in a
red bubble is denoted by the light pink background.
  \item[Status changes] Last but not least, many rules like \rsf{i{\downarrow}}
change the status of the ambient solution from open to closed: graphically, this
means that the background must become green \emph{everywhere}, not only in the
portion of the canvas depicted by the rule. At first it might appear as breaking
locality, but it should rather be understood as the result of a perfectly local
and continuous process: one can imagine a literal \emph{drop} of green paint
that soaks a growing portion of the canvas, until it reaches an enclosing bubble
--- for the sake of metaphor, let us say a cut in the papersheet --- that stops
its progression\sidenote{We will come back to this \emph{cuts in a sheet}
metaphor, first introduced by C. S. Peirce, in \refch{flowers}. When closing the
top-level solution --- Peirce called it the \emph{sheet of assertions}, the drop
expansion process becomes \emph{infinite}. I find it to be a beautiful allegory
of the \emph{unreachability} of global, unconditional truth: it is only by being
confined to a finite, well-delimited space, that we can affirm unequivocally our
certainty.
% What lays behind the fences (possibly some hidden assumptions!), out of our
% grasp will stay.
As Wittgenstein famously said at the end of the Tractatus: \textit{``Whereof one
cannot speak, thereof one must be silent''}.}.
\end{description}

We will now analyze the various groups of rules of system \sys{B}, by
comparing them to those of the \sys{BJ} calculus:
\begin{description}
  \item[\textbf{\identity}] 
  A first difference, that we will find in most rules of system \sys{B}, is that
  we rely on the distributive interpretation of conclusions in solutions. For
  instance in the \rsf{i{\uparrow}} rule, $\Delta$ is available potentially in
  both subgoals, and we do not need to move it manually: this will be the role
  of the flow rules for red items.
  
  A second difference is that the rules are not applicable in arbitrary
  subsolutions, but only \emph{open} ones. This will also be the case of some
  membrane and heating rules. In the case of the \rsf{i{\downarrow}} rule, it
  guarantees its \emph{locality}: if the conclusion was $\Gamma, A \piq{\cS} A,
  \Delta$, then the distributive semantics would entail that all subgoals in
  $\cS$ must be solved at once, despite the fact that they are not directly
  related to $A$\sidenote{If we were to give up on locality, we could opt for
  this variant, which gives better \emph{factorizability}. In fact we will
  precisely do that in \refsec{invertible-calculus}.}. As for the
  \rsf{i{\uparrow}} rule, restricting to open solutions makes the rule
  \emph{invertible}, without sacrificing locality. This will in fact be the case
  of all rules that create multiple subgoals.

  \item[\textbf{\resource}] 
  Here we still have weakening and contraction for negative items (hypotheses),
  and we also allow them for positive items (conclusions). Note that contrary to
  the $\mathbb{I}$-rules which apply only to a formula $A$, $\mathbb{R}$-rules
  apply to an arbitrary item $I$, which can either be a formula or a solution.
  Combined to the fact that the ambient solution can be either open or closed,
  this gives the most general and expressive formulation of the rules. We
  believe that like in CoS, the atomic version where $I$ is restricted to an
  atomic formula might be sufficient for completeness.

  \item[\textbf{\flow}]

  Compared to \sys{BJ} where we only had neutral bubbles, the presence of
  polarized bubbles in system $\sysB$ creates a mini-combinatorial explosion in
  the number of possible $\mathbb{F}$-rules. Indeed, the general scheme is to
  consider what types of items are allowed to flow through bubbles, either
  inwards or outwards. With $i$ item types and $b$ bubble types, this makes for
  a total of $i \times b \times 2$ possible rules. In \sys{BJ} items consisted
  only of polarized formulas and neutral bubbles ($i = 3$ and $b = 1$), thus we
  had a total of $6$ possible $\mathbb{F}$-rules. It turns out that only the
  \rsf{f{-}{\downarrow}} rule was necessary, and it is also present in system
  $\sysB$. Now with positive and negative bubbles added to the mix ($b = 3$), we
  get up to a total of $18$ possible $\mathbb{F}$-rules in system $\sysB$. Out
  of these, $11$ were identified as being sound logically, and thus we decided
  to include all of them in system $\sysB$.

  \begin{marginfigure}
    % \hspace{0.5em}
    \stkfig{1.3}{bubbles-porosity}
    \caption{Porosity of bubbles in system $\sysB$}
    \labfig{bubbles-porosity}
  \end{marginfigure}
  
  Some of them we have already encountered in \refsec{colors}: first the
  \rsf{f{+}{\downarrow}} rule for distributing conclusions in subgoals, which
  would not have made sense with the asymmetric interpretation of solutions
  (\refdef{ainterp}); but also the \rsf{f{-}{+}{\downarrow}} and
  \rsf{f{+}{-}{\downarrow}} rules, which allow a polarized item to flow
  \emph{into} a bubble of \emph{opposite} polarity. However to get
  \emph{cut-free} completeness, we will also need a sort of dual of these rules,
  \rsf{f{+}{+}{\uparrow}} and \rsf{f{-}{-}{\uparrow}}, which allow a polarized
  item to flow \emph{out} of a bubble with the \emph{same} polarity. Thus in
  addition to the duality that \emph{swaps} polarities
  (\rsf{f{-}{+}{\downarrow}} versus \rsf{f{+}{-}{\downarrow}}), we have this new
  duality which \emph{reverses} at the same time the \emph{direction} of the
  flow, and the \emph{relationship} between polarities
  (\rsf{f{-}{+}{\downarrow}} versus \rsf{f{+}{+}{\uparrow}}).

  Taken together, these $6$ rules capture provability in
  \emph{bi-intuitionistic} logic, as will be demonstrated by the soundness and
  completeness theorems for system $\sysB$. By adding any one of the converses
  to the $4$ rules that define the porosity of polarized bubbles
  (\rsf{f{-}{+}{\uparrow}}, \rsf{f{+}{-}{\uparrow}}, \rsf{f{-}{-}{\downarrow}},
  \rsf{f{+}{+}{\downarrow}}), the system collapses to \emph{classical} logic.
  This situation is summarized in \reffig{bubbles-porosity}: as in
  \reffig{bubbles-flow}, green and orange arrows represent respectively valid
  and invalid moves, but in bi-intuitionistic rather than intuitionistic logic.
  To recover the latter, one can just ignore all arrows that cross the blue
  bubble, which are only useful in dual-intuitionistic logic. Then the purple
  arrows represent moves that are valid only in classical logic. The reader can
  easily check that there is a total of $18$ arrows, and map the green and
  purple arrows back to the corresponding $\mathbb{F}$-rules of
  \reffig{graphical-B}.

  \begin{remark}
    Since all items can freely go in and out of polarized bubbles in classical
    logic, the latter are useless. In fact, one could restrict the syntax of
    solutions to neutral bubbles and only one polarity of formulas, say
    conclusions. This corresponds to the possibility of having one-sided
    formulations of sequent calculi for classical logic, by restricting negation
    to atomic formulas and extending it to arbitrary formulas through De Morgan
    dualities\sidenote{See for instance the one-sided sequent calculus in
    \cite{girard:hal-01322183}.}
  \end{remark}
  
  In their graphical representation, the bi-intuitionistic $\mathbb{F}$-rules of
  system $\sysB$ are equivalent to the three following \emph{topological laws},
  that we call the \emph{$\mathbb{F}$-laws}\sidenote{Hopefully, those are not
  \emph{flaws} of our \emph{flow} rules, but rather the opposite\dots}:
  \begin{fact}[$\mathbb{F}$-laws]\labfact{bubbles-flaws}
    \sbr
    \begin{enumerate}
      \item Polarized bubbles trap (resp. repel) items with a different (resp.
      identical) polarity.
      \item Neutral bubbles trap (resp. repel) polarized (resp. neutral) items.
      \item Polarized bubbles both trap and repel neutral bubbles.
    \end{enumerate}
  % For any bubble $S$, and any trajectory $\mathfrak{T}$ in space of any item $I$
  % that crosses $S$:
  % \begin{align*}
  %   \text{$\mathfrak{T}$ crosses $S$ inwards if and only if $\pol{I} \not= \pol{S}$.} \\
  %   \text{Conversely, $\mathfrak{T}$ crosses $S$ outwards if and only if $\pol{I} = \pol{S}$.}
  % \end{align*}
  % where we denote the polarity of any item $J$ in a solution by $\pol{J}$.
  \end{fact}
  In \reffig{bubbles-porosity}, the ability of bubbles to trap (resp. repel)
  items corresponds to outward (resp. inward) orange arrows. $\mathbb{F}$-laws
  are thus the ``negative'' counterpart --- in the grammatical sense --- of
  $\mathbb{F}$-rules, represented by green arrows. The fact that purple arrows
  are demoted to orange arrows in bi-intuitionistic logic, can be interpreted as
  resulting from their violation of the first $\mathbb{F}$-law. The second and
  third $\mathbb{F}$-laws characterize the behavior of neutral bubbles, and are
  respected by all rules of system $\sysB$.
  
  In particular, they suggest the addition of a new $\mathbb{F}$-rule
  \rsf{f{\uparrow}}, which allows to move neutral bubbles out of other neutral
  bubbles. When looking at it as a graphical rewrite rule in
  \reffig{graphical-B}, it can be seen as the act of \emph{abstracting} the
  subgoal $T$ from its parent subgoal $S$, since the hypotheses and conclusions
  of $S$ cannot be brought to interact with those of $T$ anymore. More generally
  in bi-intuitionistic logic, all flow rules can be understood as
  \emph{abstraction} moves, that strengthen the goal by moving irreversibly an
  item $I$ out of its subgoal $S$. In the case of outward rules (whose name ends
  with $\uparrow$), $I$ is brought closer to the \emph{root} of the proof tree;
  and in the case of inward rules (whose name ends with $\downarrow$), $I$ is
  brought closer to the \emph{leaves} of the proof tree.
  % More generally, all outwards flow rules (whose name ends with $\uparrow$) can
  % be understood as \emph{abstraction} moves that strengthen the goal, by moving
  % an item closer to the root of the proof tree; and dually, all inwards flow
  % rules (whose name ends with $\downarrow$) correspond to \emph{concretization}
  % moves that weaken the goal, by moving an item towards the leaves of the proof
  % tree.
  
  It would be interesting to try to formalize $\mathbb{F}$-laws, and more
  generally the graphical presentation of system $\sysB$, with the rigorous
  tools of mathematical topology. This has been done for instance in
  \sidecite[-18em]{brady_categorical_2000} for the existential graphs of C. S.
  Peirce (see \refch{flowers}).

  \item[\textbf{\membrane}] 
  We still have the popping rule \rsf{p} of \sys{BJ}, which is now restricted to
  closed empty bubbles. We add two popping rules \rsf{p{-}} and \rsf{p{+}} for
  popping respectively negative and positive closed empty bubbles. Like the
  \rsf{i{\downarrow}} rule, these have the effect of closing the ambient
  solutions, and for the same reasons we thus restrict them to open ambient
  solutions.

  \begin{marginfigure}
    $$
\R[\kl{{\limp}{+}}]
{\R[\kl{\land{+}}]
{\R[\kl{f{-}{\da}}]
{\R[\kl{i{\da}}]
{\R[\kl{p}]
{\R[\kl{a{+}}]
{\R[\kl{f{+}{+}{\da}}]
{\R[\kl{w{+}}]
{\R[\kl{{\lsub}{+}}]
{\R[\kl{f{-}{\da}}]
{\R[\kl{f{+}{\da}}]
{\R[\kl{i{\da}}]
{\R[\kl{i{\da}}]
{\R[\kl{p}]
{\R[\kl{p}]
{{\piq{}}}
{{\piq{{\piq{}}}}}}
{{\piq{{\piq{}} \sep {\piq{}}}}}}
{{\piq{{\piq{}} \sep q \seq q}}}}
{{\piq{p \seq p \sep q \seq q}}}}
{\piq{p \seq p \sep q \seq} q}}
{p \piq{\seq p \sep q \seq} q}}
{p \seq q, p \lsub q}}
{p \seq q, p \lsub q, (\seq)}}
{p \seq q, (\seq p \lsub q)}}
{p \seq q, (\piq{\seq p \lsub q})}}
{p \seq q, (\piq{\seq p \lsub q \sep {\piq{}}})}}
{p \seq q, (\piq{\seq p \lsub q \sep r \seq r})}}
{p \seq q, (r \piq{\seq p \lsub q \sep \seq r})}}
{p \seq q, (r \seq ((p \lsub q) \land r))}}
{p \seq q, r \limp ((p \lsub q) \land r)}
$$
    \caption{A proof of Uustalu's formula in system $\sysB$}
    \labfig{bubbles-uustalu}
  \end{marginfigure}

  The novelty compared to \sys{BJ} is that we also add so-called
  \emph{absorption rules} $\{\rsf{a},\rsf{a{-}},\rsf{a{+}}\}$ for membranes.
  These rules state that when a bubble contains only a single neutral bubble,
  the membrane of the latter can be absorbed into the membrane of the former.
  This is mainly useful when one wants to apply an outward $\mathbb{F}$-rule to
  an item that has the same polarity as the outer bubble, as witnessed by the
  use of the \rsf{a{+}} rule in the proof of Uustalu's formula in
  \reffig{bubbles-uustalu}. This formula was first introduced in
  \sidecite{hutchison_proof_2009} as a counter-example to the cut-elimination
  theorem of Rauszer's sequent calculus for bi-intuitionistic logic
  \sidecite{rauszer_formalization_1974}, and our initial motivation for
  introducing absorption rules was precisely to provide a cut-free proof of this
  formula in system $\sysB$.

  Later, we realized that there is an interesting \emph{symmetry} at play
  between popping rules and absorption rules. As mentioned in
  \refsec{bubbles-pba}, popping rules can be understood as resulting from a
  process of \emph{contraction} of membranes into a single point. Dually,
  absorption rules can be seen as the result of a process of \emph{expansion} of
  the inner bubble towards the outer bubble. While contraction gets stuck on
  polarized items because they cannot cross neutral membranes outwards,
  expansion gets stuck on neutral items because they cannot cross neutral
  membranes inwards. Thus there is a very natural interplay between
  $\mathbb{M}$-rules, and the $\mathbb{F}$-laws induced by $\mathbb{F}$-rules.

  \item[\textbf{\heating}] 
  Like the \rsf{i{\uparrow}} rule, the \rsf{\bot{-}}, \rsf{\lor{-}} and
  \rsf{{\limp}{-}} rules become truly local in system \sys{B} by letting
  $\mathbb{F}$-rules handle the distribution of conclusions in subgoals.
  Together with their dual rules \rsf{\top{+}}, \rsf{\land{+}} and
  \rsf{{\limp}{+}}, they constitute the \emph{closing} $\mathbb{H}$-rules of
  system $\sysB$. All other $\mathbb{H}$-rules work in arbitrary solutions just
  as in \sys{BJ}. But thanks to the ability to have multiple conclusions
  (\refsec{branching}) and positive bubbles (\refsec{colors}), both the
  \rsf{\lor{+}} and \rsf{{\limp}{+}} rules are now \emph{invertible}: this was
  the initial motivation for designing the symmetric bubble calculus.
\end{description}