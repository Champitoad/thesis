\subsection{Heyting and Brouwer algebras}

\begin{figure*}
  \tikzfig{1}{0.83}{venn-algebras}
  \caption{Relationship between the various algebras interpreting system \sys{B}}
  \labfig{venn-algebras}
\end{figure*}

We are now going to prove the soundness of system \sys{B} with respect to
various classes of \emph{algebras}. While the full system is classical and thus
sound only in \emph{Boolean} algebras, most rules are sound in larger classes of
algebras, namely: \emph{Heyting} algebras for intuitionistic logic,
\emph{Brouwer} algebras for dual-intuitionistic logic, and
\emph{Heyting-Brouwer} algebras for bi-intuitionistic logic. These 4 classes are
all instances of \emph{bounded lattices}, and their relationship is summarized
in the Venn diagram of \reffig{venn-algebras}.

First we recall the definition of a bounded lattice:

\begin{definition}[Bounded lattice]\labdef{bounded-lattice}
  A \emph{bounded lattice} is a structure $(\mathcal{A}, \sement, \ltop, \lbot, \lmeet,
  \ljoin)$ such that:
  \begin{itemize}
    \item $(\mathcal{A}, \sement)$ is a partial order, i.e. for every $a, b, c
    \in \mathcal{A}$ we have:
      \begin{itemize}
        \item $a \sement a$;
        \item if $a \sement b$ and $b \sement a$ then $a = b$;
        \item if $a \sement b$ and $b \sement c$ then $a \sement c$.
      \end{itemize}
    \item $\lbot$ and $\ltop$ are respectively the smallest and greatest
    elements of $(\mathcal{A}, \sement)$, i.e. for every $a \in \mathcal{A}$ we
    have $\bot \sement a$ and $a \sement \ltop$;
    \item For every pair of elements $a, b \in \mathcal{A}$, $a \ljoin b$ is
    their join (least upper bound) and $a \lmeet b$ their meet (greatest lower
    bound), that is:
    \begin{itemize}
      \item $a \sement a \ljoin b$, $b \sement a \ljoin b$ and $a \ljoin b \sement c$ for all $c \in \mathcal{A}$ s.t. $a \sement c$ and $b \sement c$;
      \item $a \lmeet b \sement a$, $a \lmeet b \sement b$ and $c \sement a \lmeet b$ for all $c \in \mathcal{A}$ s.t. $c \sement a$ and $c \sement b$.
    \end{itemize}
  \end{itemize}
\end{definition}

\begin{remark}
  For lack of time, we only conjecture the soundness of rules for quantifiers:
  they would make the semantics more involved, by requiring to consider
  \emph{complete} lattices, i.e. with meets and joins for arbitrary sets rather
  than just pairs\sidenote{see for instance section 4 of
  \cite{forster_completeness_2021} for a concise treatment of the soundness and
  completeness of intuitionistic and classical natural deduction for first-order
  logic with respect to algebraic semantics.}.
\end{remark}

As the notation strongly suggests, the greatest and smallest elements $\top$ and
$\bot$ will model respectively truth and falsity, while the meet $\lmeet$ and
join $\ljoin$ will model conjunction and disjunction. In fact the conditions of
\refdef{bounded-lattice} are very close to the rules of natural deduction for
these connectives, by replacing the sequent operator $\seq$ with the partial
order relation $\sement$. The same idea can be applied to the implication
connective, and adding a corresponding \emph{exponential} operation $\lexp$
indeed gives the definition of a Heyting algebra:

\begin{definition}[Heyting algebra]
  A \emph{Heyting algebra} is a structure $(\mathcal{A}, \sement, \ltop, \lbot,
  \lmeet, \ljoin, \lexp)$ such that $(\mathcal{A}, \sement, \ltop, \lbot,
  \lmeet, \ljoin)$ is a bounded lattice and for every pair $a, b \in
  \mathcal{A}$, the \emph{exponential} $a \lexp b$ is the greatest element of
  the set $\compr{c \in \mathcal{A}}{c \lmeet a \sement b}$. That is, $(a \lexp
  b) \lmeet a \sement b$ and $c \sement a \lexp b$ for all $c \in \mathcal{A}$
  s.t. $c \lmeet a \sement b$.
\end{definition}

By dualizing this definition, we get a \emph{co-exponential} operation $\lcoexp$
that models the exclusion connective, and thus dual-intuitionistic logic in
so-called Brouwer algebras:

\begin{definition}[Brouwer algebra]
  A \emph{Brouwer algebra} is a structure $(\mathcal{A}, \sement, \ltop, \lbot,
  \lmeet, \ljoin, \lcoexp)$ such that $(\mathcal{A}, \sement, \ltop, \lbot,
  \lmeet, \ljoin)$ is a bounded lattice and for every pair $a, b \in
  \mathcal{A}$, the \emph{co-exponential} $a \lcoexp b$ is the smallest element
  of the set $\compr{c \in \mathcal{A}}{b \sement a \ljoin c}$. That is, $b
  \sement a \ljoin (b \lcoexp a)$ and $b \lcoexp a \sement c$ for all $c \in
  \mathcal{A}$ s.t. $b \sement a \ljoin c$.
\end{definition}

Then we can model bi-intuitionistic logic, which comprises both implication and
exclusion, by just taking pairs of a Heyting and a Brouwer algebra on the same
bounded lattice:

\begin{definition}[Heyting-Brouwer algebra]
  A \emph{Heyting-Brouwer algebra} is a structure $(\mathcal{A}, \sement, \ltop,
  \lbot, \lmeet, \ljoin, \lexp, \lcoexp)$ such that $(\mathcal{A}, \sement,
  \ltop, \lbot, \lmeet, \ljoin, \lexp)$ is a Heyting algebra and $(\mathcal{A},
  \sement, \ltop, \lbot, \lmeet, \ljoin, \lcoexp)$ is a Brouwer algebra.
\end{definition}

Finally, we recover classical logic by collapsing exponentials and
co-exponentials to their classical definitions, giving a characterization of
Boolean algebras:

\begin{definition}\labdef{boolean-algebra}
  A \emph{Boolean algebra} is a Heyting-Brouwer algebra $(\mathcal{A}, \sement,
  \ltop, \lbot, \lmeet, \ljoin, \lexp, \lcoexp)$ such that for every $a, b \in
  \mathcal{A}$, $a \lexp b = (\ltop \lcoexp a) \ljoin b$ and $a \lcoexp b = a
  \lmeet (b \lexp \bot)$.
\end{definition}

\begin{remark}
\refdef{boolean-algebra} can be shown equivalent to more
usual definitions of Boolean algebras, that are based only on lattice operations
and a primitive complement operation modelling negation; but including the proof
here would lead us out of the scope of this chapter.
% \todo{Proof in appendix?}
\end{remark}

In the rest of this chapter, we will freely assimilate formulas and their
interpretation in the various algebras. Indeed, since we only consider the
abstract classes of all algebras and never deal with a particular instance,
they will stand in perfect bijection.

\subsection{Local soundness}

\begin{lemma}[Sharing]
  $\psint{\Gamma \seq \Delta} \sement \psint{\Gamma \piq{\mathcal{S}} \Delta}$.
\end{lemma}

\begin{lemma}[Weakening]
  $\psint{S} \sement \psint{S \mix (\seq I)}$ and $\psint{S} \sement \psint{S
  \mix (\seq I)}$.
\end{lemma}

\begin{lemma}[Contraction]
  $\psint{S \mix (\seq I, I)} \sement \psint{S \mix (\seq I)}$ and
  $\psint{S \mix (I, I \seq)} \sement \psint{S \mix (I \seq)}$.
\end{lemma}

\begin{lemma}[Currying]
  $\psint{\Gamma, I \J \Delta} \semequiv \nsint{I} \limp \psint{\Gamma \J
  \Delta}$.
\end{lemma}

\begin{lemma}[Co-currying]
  $\nsint{\Gamma \J I, \Delta} \semequiv \nsint{\Gamma \J \Delta} \lsub
  \psint{I}$.
\end{lemma}

\begin{lemma}[Local soundness]\lablemma{bubbles-local-soundness}
  
  If $S \lstep T$ then $\psint{T \mix (\Gamma \seq \Delta)} \sement \psint{S
  \mix (\Gamma \seq \Delta)}$.
  % and $\nsint{S \mix (\Gamma \seq \Delta)} \sement
  % \nsint{T \mix (\Gamma \seq \Delta)}$.
\end{lemma}

\subsection{Duality}

\begin{definition}[Dual formula]\labdef{dual-formula}
  The \emph{dual formula} $\soldual{A}$ of a formula $A$ is defined recursively
  as follows:
  \begin{align*}
    \soldual{a} &= a &\\
    \soldual{\top} &= \bot &
    \soldual{\bot} &= \top \\
    \soldual{A \land B} &= \soldual{A} \lor \soldual{B} &
    \soldual{A \lor B} &= \soldual{A} \land \soldual{B} \\
    \soldual{A \limp B} &= \soldual{B} \lsub \soldual{A} &
    \soldual{A \lsub B} &= \soldual{B} \limp \soldual{A}
  \end{align*}
\end{definition}

\begin{fact}[Duality]\labfact{duality}
  In a Heyting-Brouwer algebra, if $A \sement B$ then $\soldual{B} \sement
  \soldual{A}$.
\end{fact}

We omit the proof of \reffact{duality}, but this can easily be obtained from the
soundness and completeness of a symmetric sequent calculus for bi-intuitionistic
logic, see for instance Lemma 2 of \sidecite{restall_extending_1997}.

\begin{definition}[Dual solution]\labdef{dual-solution}
  The \emph{dual solution} $\soldual{S}$ of a solution $S$ is defined
  mutually recursively as follows:
  \begin{align*}
    \soldual{\Gamma \J \Delta} &= \soldual{\Delta} \mathbin{\soldual{\J}} \soldual{\Gamma} &
    \soldual{S_1 \sep \ldots \sep S_n} &= \soldual{S_1} \sep \ldots \sep \soldual{S_n} \\
    \soldual{A} &= \soldual{A} &
    \soldual{I_1, \ldots, I_n} &= \soldual{I_1}, \ldots, \soldual{I_n} \\
    \soldual{\seq} &= {\seq} &
    \soldual{\piq{\mathcal{S}}} &= \piq{\soldual{\mathcal{S}}}
  \end{align*}
  For solution contexts, the hole is self-dual: $\soldual{\hole} = \hole$. This
  entails in particular that $\soldual{S}\select{\soldual{T}} =
  \soldual{S\select{T}}$.
\end{definition}

Graphically, the dual of a solution $S$ is $S$ where the colors of items have
been swapped --- i.e. blue items become red and red items become blue --- and
formulas have been dualized (\refdef{dual-formula}).

\begin{definition}
  The \emph{depth} $\soldepth{I}$ of an item $I$ is defined recursively as
  follows:
  \begin{align*}
    \soldepth{A} &= 0 \\
    \soldepth{\Gamma \seq \Delta} &= 1 + \max_{J \in \Gamma \cup \Delta}{\soldepth{J}} \\
    \soldepth{\Gamma \piq{\mathcal{S}} \Delta} &= 1 + \max_{J \in \Gamma \cup \mathcal{S} \cup \Delta}{\soldepth{J}}
  \end{align*}
\end{definition}

\begin{lemma}[Involutivity]\lablemma{involutivity}
  $\soldual{\soldual{I}} = I$.
\end{lemma}
\begin{proof}
  By recurrence on $\soldepth{I}$.
  \begin{itemize}
    \item[\textbf{Formula}] Suppose $I = A$. Then we conclude by a
    straightforward induction on $A$.
    \item[\textbf{Open solution}] Suppose $I = \Gamma \seq \Delta$. Then by
    definition we have $\soldual{\soldual{\Gamma \seq \Delta}} =
    \soldual{\soldual{\Delta} \seq \soldual{\Gamma}} =
    \soldual{\soldual{\Gamma}} \seq \soldual{\soldual{\Delta}}$, and we conclude
    by IH.
    \item[\textbf{Closed solution}] Suppose $I = \Gamma \piq{\mathcal{S}}
    \Delta$. Then by definition we have $\soldual{\soldual{\Gamma
    \piq{\mathcal{S}} \Delta}} = \soldual{\soldual{\Delta}
    \piq{\soldual{\mathcal{S}}} \soldual{\Gamma}} =
    \soldual{\soldual{\Gamma}} \piq{\soldual{\soldual{\mathcal{S}}}}
    \soldual{\soldual{\Delta}}$, and we conclude by IH.
  \end{itemize}
\end{proof}

\begin{lemma}[Local rule duality]\lablemma{local-rule-duality}
  If $S \lstep T$ then $\soldual{S} \lstep \soldual{T}$.
\end{lemma}
\begin{proof}
  There is a bijection among the rules of system \sys{B}, that matches each rule
  $\irule{r}{S}{T}$ to its dual $\irule{\soldual{r}}{\soldual{S}}{\soldual{T}}$.
  By involutivity (\reflemma{involutivity}), this bijection is self-inverse:
  $\soldual{\soldual{\mathsf{r}}} = \mathsf{r}$. It is most easily observed in
  the graphical presentation of the rules (\reffig{graphical-B}), where looking
  for the dual rule boils down to swapping red and blue (and mirroring logical
  connectives). The mapping goes as follows:
  \begin{mathpar}
  \begin{array}{r@{\quad\leftrightarrow\quad}l}
    \mathsf{i{\downarrow}} & \mathsf{i{\downarrow}} \\
    \mathsf{i{\uparrow}} & \mathsf{i{\uparrow}} \\
  \end{array}
  \and
  \begin{array}{r@{\quad\leftrightarrow\quad}l}
    \mathsf{w{-}} & \mathsf{w{+}} \\
    \mathsf{c{-}} & \mathsf{c{+}} \\
  \end{array}
  \\
  \begin{array}{r@{\quad\leftrightarrow\quad}l}
    \mathsf{s{-}} & \mathsf{s{+}} \\
    \mathsf{f{-}} & \mathsf{f{+}} \\
    \mathsf{f{-}{+}{\downarrow}} & \mathsf{f{+}{-}{\downarrow}} \\
    \mathsf{f{-}{-}{\uparrow}} & \mathsf{f{+}{+}{\uparrow}} \\
    \mathsf{f{-}{+}{\uparrow}} & \mathsf{f{+}{-}{\uparrow}} \\
    \mathsf{f{-}{-}{\downarrow}} & \mathsf{f{+}{+}{\downarrow}} \\
  \end{array}
  \and
  \begin{array}{r@{\quad\leftrightarrow\quad}l}
    \mathsf{p} & \mathsf{p} \\
    \mathsf{p{-}} & \mathsf{p{+}} \\
  \end{array}
  \\
  \begin{array}{r@{\quad\leftrightarrow\quad}l}
    \top{-} & \bot{+} \\
    \bot{-} & \top{+} \\
    \land{-} & \lor{+} \\
    \lor{-} & \land{+} \\
    {\limp}{-} & {\lsub}{+} \\
    {\lsub}{-} & {\limp}{+} \\
    \forall{-} & \exists{+} \\
    \exists{-} & \forall{+} \\
  \end{array}
  \end{mathpar}
  Notice that some rules are self-dual, namely the identity rules
  {\rnmsf{i{\downarrow}}} and {\rnmsf{i{\uparrow}}}, and the popping rule
  {\rnmsf{p}}.
\end{proof}

\begin{lemma}[Rule duality]\lablemma{rule-duality}
  If $S \step T$ then $\soldual{S} \step \soldual{T}$.
\end{lemma}
\begin{proof}
  Let $U\hole$, $S_0$ and $T_0$ such that $S = U\select{S_0}$, $T =
  U\select{T_0}$ and $S_0 \lstep T_0$. By \reflemma{local-rule-duality} we have
  $\soldual{S_0} \lstep \soldual{T_0}$, and thus
  $\soldual{U}\select{\soldual{S_0}} \step \soldual{U}\select{\soldual{T_0}}$,
  or equivalently $\soldual{U\select{S_0}} \step \soldual{U\select{T_0}}$.
\end{proof}

\begin{lemma}[Interpretation duality]\lablemma{int-duality}
  $\soldual{\psint{I}} = \nsint{\soldual{I}}$ and $\soldual{\nsint{I}} =
  \psint{\soldual{I}}$.
\end{lemma}
\begin{proof}
  By a straightforward recurrence on $\soldepth{I}$.
\end{proof}

\begin{lemma}\lablemma{int-invert}
  If $\psint{\soldual{S}} \sement \psint{\soldual{T}}$ then $\nsint{T} \sement
  \nsint{S}$.
\end{lemma}
\begin{proof}
  By duality (\reffact{duality}) we have $\soldual{\psint{\soldual{T}}} \sement
  \soldual{\psint{\soldual{S}}}$, and then by \reflemma{int-duality}
  $\nsint{\soldual{\soldual{T}}} \sement \nsint{\soldual{\soldual{S}}}$. We
  conclude by involutivity (\reflemma{involutivity}).
\end{proof}

\subsection{Contextual soundness}

\begin{definition}
  The \emph{depth} $\bradepth{\J}$ of a branching operator $\J$ is defined
  recursively as follows:
  \begin{align*}
    \bradepth{\seq} &= 0 \\
    \bradepth{\piq{\mathcal{S}}} &= 1 + \max_{S \in \mathcal{S}}{\bradepth{S}}
  \end{align*}
\end{definition}

\begin{lemma}[Functoriality]\lablemma{bubbles-functoriality}
  ~\\
  \begin{itemize}
    \item $\psint{I} \sement \psint{J}$ implies $\psint{(\seq I) \mix S} \sement
    \psint{(\seq J) \mix S}$
    % \item $\nsint{I} \sement \nsint{J}$ implies $\nsint{(\seq I) \mix S} \sement
    % \nsint{(\seq J) \mix S}$
    \item $\nsint{J} \sement \nsint{I}$ implies $\psint{(I \seq) \mix S} \sement
    \psint{(J \seq) \mix S}$
    % \item $\psint{J} \sement \psint{I}$ implies $\nsint{(I
    % \seq) \mix S} \sement \nsint{(J \seq) \mix S}$
  \end{itemize}
\end{lemma}
\begin{proof}
  Let $S = \Gamma \J \Delta$. We proceed by recurrence on $\bradepth{\J}$.
  \begin{itemize}
    \item[\textbf{Base case}] Suppose $\bradepth{\J} = 0$. Then $\J = {\seq}$,
    and we have
    $$
    \begin{array}{rcll}
      \psint{(\seq I) \mix S}
      &=& \psint{\Gamma \seq I, \Delta} &\text{(\refdef{mixop})}\\
      &=& \bigwedge_{K \in \Gamma}{\nsint{K}} \limp \psint{I} \lor \bigvee_{K \in \Delta}{\psint{K}} &\text{(\refdef{pol-sinterp})}\\
      &\sement& \bigwedge_{K \in \Gamma}{\nsint{K}} \limp \psint{J} \lor \bigvee_{K \in \Delta}{\psint{K}} &\text{(Hypothesis)}\\
      &=& \psint{\Gamma \seq J, \Delta} &\text{(\refdef{pol-sinterp})}\\
      &=& \psint{(\seq J) \mix S} &\text{(\refdef{mixop})}
    \end{array}
    $$
    $$
    \begin{array}{rcll}
      \psint{(I \seq) \mix S}
      &=& \psint{\Gamma, I \seq \Delta} &\text{(\refdef{mixop})}\\
      &=& \bigwedge_{K \in \Gamma}{\nsint{K}} \land \nsint{I} \limp \bigvee_{K \in \Delta}{\psint{K}} &\text{(\refdef{pol-sinterp})}\\
      &\sement& \bigwedge_{K \in \Gamma}{\nsint{K}} \land \nsint{J} \limp \bigvee_{K \in \Delta}{\psint{K}} &\text{(Hypothesis)} \\
        % &\begin{array}{rl}
        %    &\text{Contravariant functoriality of $\_{\limp}$} \\
        %   +&\text{Functoriality of $\_{\lor}$} \\
        %   +&\text{Hypothesis}
        % \end{array}\\
      &=& \psint{\Gamma, J \seq \Delta} &\text{(\refdef{pol-sinterp})}\\
      &=& \psint{(J \seq) \mix S} &\text{(\refdef{mixop})}
    \end{array}
    $$
    \item[\textbf{Recursive case}] Suppose $\bradepth{\J} > 0$. Then $\J =
    {\piq{\mathcal{S}}}$, and for all $S_0 = \Gamma_0 \JB \Delta_0 \in
    \mathcal{S}$ we have that $\bradepth{\JB} < \bradepth{\J}$. Thus we have
    $$
    \begin{array}{rcll}
      \psint{(\seq I) \mix S}
      &=& \psint{\Gamma \piq{\mathcal{S}} I, \Delta} &\text{(\refdef{mixop})} \\
      &=& \bigwedge_{S_0 \in \mathcal{S}}{\psint{(\Gamma \seq I, \Delta) \mix S_0}} &\text{(\refdef{pol-sinterp})} \\
      &=& \bigwedge_{S_0 \in \mathcal{S}}{\psint{(\seq I) \mix ((\Gamma \seq \Delta) \mix S_0)}} &\text{(\refdef{mixop})} \\
      &\sement& \bigwedge_{S_0 \in \mathcal{S}}{\psint{(\seq J) \mix ((\Gamma \seq \Delta) \mix S_0)}} &\text{(IH)} \\
      &=& \bigwedge_{S_0 \in \mathcal{S}}{\psint{(\Gamma \seq J, \Delta) \mix S_0}} &\text{(\refdef{mixop})} \\
      &=& \psint{\Gamma \piq{\mathcal{S}} J, \Delta} &\text{(\refdef{pol-sinterp})} \\
      &=& \psint{(\seq J) \mix S} &\text{(\refdef{mixop})}
    \end{array}
    $$
    $$
    \begin{array}{rcll}
      \psint{(I \seq) \mix S}
      &=& \psint{\Gamma, I \piq{\mathcal{S}} \Delta} &\text{(\refdef{mixop})} \\
      &=& \bigwedge_{S_0 \in \mathcal{S}}{\psint{(\Gamma, I \seq \Delta) \mix S_0}} &\text{(\refdef{pol-sinterp})} \\
      &=& \bigwedge_{S_0 \in \mathcal{S}}{\psint{(I \seq) \mix ((\Gamma \seq \Delta) \mix S_0)}} &\text{(\refdef{mixop})} \\
      &\sement& \bigwedge_{S_0 \in \mathcal{S}}{\psint{(J \seq) \mix ((\Gamma \seq \Delta) \mix S_0)}} &\text{(IH)} \\
      &=& \bigwedge_{S_0 \in \mathcal{S}}{\psint{(\Gamma, J \seq \Delta) \mix S_0}} &\text{(\refdef{mixop})} \\
      &=& \psint{\Gamma, J \piq{\mathcal{S}} \Delta} &\text{(\refdef{pol-sinterp})} \\
      &=& \psint{(J \seq) \mix S} &\text{(\refdef{mixop})}
    \end{array}
    $$
  \end{itemize}
\end{proof}

In order to ease reasoning by induction on solution contexts, we give a
formulation equivalent to \refdef{solution-context} as a context-free grammar:
\begin{fact}
  Solution contexts $S\hole$ are generated by the following grammar:
  $$
    S\hole \Coloneq \hole \mid \Gamma \J S\hole, \Delta
                          \mid \Gamma, S\hole \J \Delta
                          \mid \Gamma \piq{\mathcal{S} \sep S\hole} \Delta
  $$
\end{fact}

\begin{definition}
The \emph{depth} $\soldepth{S\hole}$ of a solution context $S\hole$ is defined
recursively as follows:
\begin{align*}
  \soldepth{\hole} &= 0 \\
  \soldepth{\Gamma \J S\hole, \Delta} = \soldepth{\Gamma, S\hole \J \Delta} =
  \soldepth{\Gamma \piq{\mathcal{S} \sep S\hole} \Delta} &= 1 + \soldepth{S\hole}
\end{align*}
\end{definition}

\begin{lemma}[Contextual soundness]\lablemma{bubbles-ctx-soundness}

  If $S \lstep T$ then $\psint{U\select{T} \mix (\Gamma \seq \Delta)} \sement
  \psint{U\select{S} \mix (\Gamma \seq \Delta)}$.
  % and $\nsint{S \mix (\Gamma \seq \Delta)} \sement
  % \nsint{T \mix (\Gamma \seq \Delta)}$.
\end{lemma}
\begin{proof}
  By recurrence on $\soldepth{U\hole}$.
  \begin{itemize}
    \item[\textbf{Base case}] Suppose $\soldepth{U\hole}$ = 0. Then $U\hole =
    \hole$, and we conclude by local soundness
    (\reflemma{bubbles-local-soundness}).
    \item[\textbf{Positive case}] Suppose $\soldepth{U\hole} > 0$ and $U\hole =
    \Gamma' \J U_0\hole, \Delta'$. Then by IH we have $\psint{U_0\select{T}}
    \sement \psint{U_0\select{S}}$, and thus
    $$
    \begin{array}{rcll}
      \psint{(\Gamma' \J U_0\select{T}, \Delta') \mix (\Gamma \seq \Delta)}
      &=& \psint{(\seq U_0\select{T}) \mix (\Gamma, \Gamma' \J \Delta', \Delta)} &\text{(\refdef{mixop})}\\
      &\sement& \psint{(\seq U_0\select{S}) \mix (\Gamma, \Gamma' \J \Delta', \Delta)} &\text{(\reflemma{bubbles-functoriality})}\\
      &=& \psint{(\Gamma' \J U_0\select{S}, \Delta') \mix (\Gamma \seq \Delta)} &\text{(\refdef{mixop})}
    \end{array}
    $$

    \item[\textbf{Negative case}] Suppose $\soldepth{U\hole} > 0$ and $U\hole =
    \Gamma', U_0\hole \J \Delta'$. Then by \reflemma{local-rule-duality} we have
    $\soldual{S} \lstep \soldual{T}$, and thus by IH
    $\psint{\soldual{U_0}\select{\soldual{T}}} \sement
    \psint{\soldual{U_0}\select{\soldual{S}}}$, or equivalently
    $\psint{\soldual{U_0\select{T}}} \sement \psint{\soldual{U_0\select{S}}}$.
    Then by \reflemma{int-invert} we get $\nsint{U_0\select{S}} \sement
    \nsint{U_0\select{T}}$, and thus
    $$
    \begin{array}{rcll}
      \psint{(\Gamma', U_0\select{T} \J \Delta') \mix (\Gamma \seq \Delta)}
      &=& \psint{(U_0\select{T} \seq) \mix (\Gamma, \Gamma' \J \Delta', \Delta)} &\text{(\refdef{mixop})}\\
      &\sement& \psint{(U_0\select{S} \seq) \mix (\Gamma, \Gamma' \J \Delta', \Delta)} &\text{(\reflemma{bubbles-functoriality})}\\
      &=& \psint{(\Gamma', U_0\select{S} \J \Delta') \mix (\Gamma \seq \Delta)} &\text{(\refdef{mixop})}
    \end{array}
    $$

    \item[\textbf{Neutral case}] Suppose $\soldepth{U\hole} > 0$ and $U\hole =
    \Gamma \piq{\mathcal{S} \sep U_0\hole} \Delta$. Then by IH we have
    $\psint{U_0\select{T} \mix (\Gamma \seq \Delta)} \sement
    \psint{U_0\select{S} \mix (\Gamma \seq \Delta)}$, and thus
    $$
    \begin{array}{rcll}
      \psint{\Gamma \piq{\mathcal{S} \sep U_0\select{T}} \Delta}
      &=& \psint{\Gamma \piq{\mathcal{S}} \Delta} \land \psint{U_0\select{T} \mix (\Gamma \seq \Delta)} &\text{(\refdef{pol-sinterp})}\\
      &\sement& \psint{\Gamma \piq{\mathcal{S}} \Delta} \land \psint{U_0\select{S} \mix (\Gamma \seq \Delta)} &\\
      &=& \psint{\Gamma \piq{\mathcal{S} \sep U_0\select{S}} \Delta} &\text{(\refdef{pol-sinterp})}
    \end{array}
    $$
  \end{itemize}
\end{proof}

\begin{theorem}[Soundness]
  If $S \step T$ then $\psint{T} \sement \psint{S}$.
\end{theorem}
\begin{proof}
  By definition of $\step$, and then applying \reflemma{bubbles-ctx-soundness}
  with $\Gamma = \Delta = \emptyset$.
\end{proof}

We also get for free soundness with respect to the negative interpretation,
which we call \emph{co-soundness}:

\begin{theorem}[Co-soundness]
  If $S \step T$ then $\nsint{S} \sement \nsint{T}$.
\end{theorem}
\begin{proof}
  By \reflemma{rule-duality} we have $\soldual{S} \step \soldual{T}$,
  and thus by soundness $\psint{\soldual{T}} \sement \psint{\soldual{S}}$. Then
  we can conclude by \reflemma{int-invert}.
\end{proof}