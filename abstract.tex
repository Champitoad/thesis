Proof assistants are software systems that allow for the precise checking of
mathematical reasoning. They can be general purpose (like Coq, Lean,
Isabelle...) or more specialized like EasyCrypt. They enable a level of accuracy
which certifies that no error can occur, but remain difficult to use.

We propose a new paradigm for constructing formal proofs through actions
performed in a graphical user interface (GUI), in order to enable a more
comfortable and intuitive use. Our paradigm builds upon direct manipulation
principles, combining both old (Proof-by-Pointing) and new (Proof-by-Linking)
interaction techniques that exploit recent advances in deep inference proof
theory. We implement the paradigm in a web-based GUI called Actema, which we
subsequently integrate into the Coq proof assistant by developing the coq-actema
plugin.

We then explore a series of deep inference proof systems that give more
structure to the notion of logical goal. These systems share the ability to
represent goals in two alternative ways: either textually through a standard
inductive syntax, or graphically using a metaphorical notation well-suited
to direct manipulation.

The first family of systems, called bubble calculi, is a topological
reformulation of the theory of nested sequents. It allows for efficient
sharing of hypotheses and conclusions among subgoals, facilitating the
factorization of both forward and backward proof steps. The second system,
called flower calculus, is an intuitionistic refinement of C. S. Peirce's
theory of existential graphs. Both types of systems are shown to be analytic
and fully invertible, making them amenable to proof automation techniques.

We finally go back to practical experimentation by designing and implementing
the Flower Prover, another web-based GUI for interactive proof building based on
the flower calculus. An innovative feature of the Flower Prover is that it works
well on modern mobile devices, thanks to its responsive layout and first-class
support for touch interactions.