Les assistants de preuve sont des logiciels permettant de vérifier rigoureusement des raisonnements mathématiques. Ils peuvent être généraux (comme \kl{Coq} \cite{the_coq_development_team_2022_7313584}, \kl{Lean} \cite{10.1007/978-3-030-79876-5_37}, \kl{Isabelle} \cite{nipkow2002isabelle}...) ou plus spécialisés (comme \kl{EasyCrypt} \cite{Barthe2014}). Ils permettent un niveau de précision qui certifie qu'aucune erreur ne peut se produire, mais restent difficiles d'utilisation.

Nous proposons un nouveau paradigme de construction de preuves formelles par actions effectuées dans une interface graphique, afin de permettre une utilisation plus confortable et plus intuitive. Intitulé \kl{Proof-by-Action} (\kl{PbA}), notre paradigme s'appuie sur des principes de manipulation directe, combinant des techniques d'interactions anciennes comme \kl{Proof-by-Pointing} (\kl{PbP}) \cite{PbP} et plus récentes comme \kl{Proof-by-Linking} (\kl{PbL}) \cite{chaudhuri_certifying_2022}. Toutes les techniques étudiées dans cette thèse sont fondées sur une branche récente de la théorie de la démonstration dite par ``\kl{inférence profonde}'', introduite pour la première fois sous ce nom par Alessio Guglielmi dans le cadre de son \kl{calcul des structures} \cite{Guglielmi1999ACO}.

\section*{Manipulations symboliques}

Dans la première partie de cette thèse, nous développons le paradigme \kl{PbA} dans le contexte des représentations traditionnelles des buts logiques, en introduisant plusieurs techniques de manipulation directe des formules symboliques dans les séquents.

Nous commençons au Chapitre \ref{ch:pba} par une introduction à \kl{PbP} et \kl{PbL}. Nous décrivons comment raisonner avec des connecteurs logiques, des quantificateurs et des équations par des clics et des actions de glisser-déposer (\kl{DnD}) dans un prototype d'interface graphique appelé \kl{Actema} \cite{Actema:link}. En particulier, les actions \kl{DnD} permettent de lier deux sous-formules arbitraires du but courant pour les faire interagir, et peuvent être vues comme une généralisation des tactiques \texttt{apply} et \texttt{rewrite} communes aux langages de preuve impératifs.

Nous établissons au Chapitre \ref{ch:sfl} la sémantique des actions \kl{DnD} dans la théorie de la démonstration par inférence profonde, en concevant une variante intuitionniste du calcul des structures pour la liaison de sous-formules introduit par Chaudhuri \cite{Chaudhuri2013}. Notre approche diffère de celle de Chaudhuri principalement par notre notion de validité d'une liaison, qui permet de filtrer les actions \kl{DnD} non productives en autorisant uniquement les liaisons entre sous-formules unifiables.

% Les principaux résultats des chapitres \ref{ch:pba} et \ref{ch:sfl} ont été publiés dans un article de conférence \cite{10.1145/3497775.3503692}.

Nous présentons au Chapitre \ref{ch:advanced} des techniques plus avancées du paradigme \kl{PbA}, qui traitent de formes de raisonnement pervasives dans la pratique mathématique telles que l'utilisation de définitions, le raisonnement par induction et la simplification d'expressions par calcul automatique. Nous illustrons cela au travers de trois études de cas sur des problèmes logiques et mathématiques basiques, qui pourraient être donnés en exercices dans un cours d'initiation à la démonstration formelle.

Nous étudions au Chapitre \ref{ch:sfl-classical} une extension de \kl{PbA} aux séquents comportant plusieurs conclusions, par opposition aux séquents à conclusion unique que l'on trouve dans l'interface de la plupart des assistants de preuve. Nous soutenons que l'utilisation de la manipulation directe facilite grandement la gestion des conclusions multiples, et introduisons un opérateur d'interaction dite ``parallèle'' pour modéliser le raisonnement en logique classique impliquant l'interaction de deux conclusions.

Enfin, nous présentons au Chapitre \ref{ch:plugin} \kl{coq-actema}, un plugin qui intègre l'application web \kl{Actema} en tant que vue de preuve interactive dans \kl{Coq}. Nous nous concentrons sur l'architecture et les protocoles qui connectent les différents composants du système, et donnons un aperçu de la stratégie de compilation qui transforme les actions graphiques effectuées dans \kl{Actema} en termes de preuve \kl{Coq}. Nous discutons également des lacunes actuelles de notre approche et des pistes d'amélioration futures, en particulier concernant la question de l'évolution et de la maintenance des démonstrations.

\section*{Manipulations iconiques}

Dans la deuxième partie de cette thèse, nous explorons une série de systèmes de démonstration par inférence profonde qui donnent plus de structure à la notion de but logique. Ces systèmes partagent la capacité de représenter les buts de deux manières alternatives : soit textuellement au travers d'une syntaxe inductive standard, soit graphiquement à l'aide d'une notation métaphorique adaptée à la manipulation directe. La première peut être utilisée comme représentation machine dans le backend d'un assistant de preuve, et la seconde comme substrat pour une interface graphique dans le frontend.

Nous introduisons dans les deux premiers chapitres une famille de systèmes appelés calculs de bulles. Ils constituent une extension de la théorie des séquents imbriqués introduite par Brünnler \cite{brunnler_deep_2009}, que nous reformulons comme des systèmes de réécriture locale disposant d'une interprétation graphique et topologique. Les calculs de bulles permettent un partage efficace des hypothèses et conclusions entre sous-buts, facilitant la factorisation des étapes de raisonnement avant et arrière dans les démonstrations.

Nous présentons au Chapitre \ref{ch:bubbles} le calcul de bulles asymétrique \kl{BJ} pour la logique intuitionniste, modelé sur le calcul des séquents intuitionniste \kl{LJ} de Gentzen. Nous introduisons la métaphore des bulles comme moyen de représenter diagrammatiquement la séparation et le partage des hypothèses et conclusions entre sous-buts.

Puis au Chapitre \ref{ch:bubbles-symm} nous affinons \kl{BJ} en un calcul plus général et symétrique pour la logique classique appelé \kl{système~B}, où les bulles peuvent être polarisées en plus des formules. Les logiques intuitionniste, dual-intuitionniste et bi-intuitionniste peuvent être récupérées comme fragments de \kl{système~B}, en interdisant certaines règles d'inférence qui caractérisent la porosité des bulles. Nous concevons également une variante entièrement réversible de \kl{système~B}, que nous conjecturons complète.

Dans les deux derniers chapitres, nous étudions deux systèmes basés sur les \kl{graphes existentiels} de C. S. Peirce, ce qui nous permet d'atteindre une pleine \kl{iconicité} : chaque construction logique possède une représentation diagrammatique associée, éliminant ainsi l'usage des connecteurs et quantificateurs symboliques. Cela devrait enlever une première barrière dans l'apprentissage de la logique formelle, qui réside dans la correspondance arbitraire entre les symboles et leur signification.

Nous effectuons au Chapitre \ref{ch:eg} un examen approfondi des systèmes originaux de \kl{graphes existentiels} proposés par Peirce pour la logique classique propositionnelle et du premier ordre, qui furent systématiquement négligés dans la littérature sur la théorie de la démonstration. Nous proposons en particulier une nouvelle caractérisation inductive de la syntaxe des \kl{graphes existentiels}, ainsi que la première identification d'un fragment analytique du système \kl{Alpha} pour la logique propositionnelle qui est complet pour la prouvabilité.

Enfin, nous introduisons au Chapitre \ref{ch:flowers} le \kl{calcul des fleurs}, une variante intuitionniste des \kl{graphes existentiels} où les énoncés sont représentés métaphoriquement comme des fleurs. Nous partitionnons le système en un fragment ``\kl{naturel}'' où chaque règle est à la fois analytique et réversible, et un fragment ``\kl{culturel}'' où chaque règle est irréversible. Nous prouvons que le fragment \kl{culturel} est admissible grâce à une preuve de complétude pour le fragment \kl{naturel} vis-a-vis d'une sémantique de Kripke. Nous exploitons ces résultats méta-théoriques pour concevoir le \kl{Flower\,\,Prover}, un prototype d'interface graphique dans le paradigme \kl{PbA} qui vise à unifier les concepts de but et de théorie au sein d'une interface modale : les buts correspondent à des fleurs manipulées avec des règles \kl{naturelles} en mode \kl{Démonstration}, tandis que les théories correspondent aux mêmes fleurs manipulées avec des règles \kl{culturelles} en mode \kl{Édition}. À notre connaissance, le \kl{Flower\,\,Prover} est également la première interface d'assistant de preuve conçue pour appareils mobiles.