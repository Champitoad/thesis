\setchapterpreamble[u]{\margintoc}
\chapter{Proof-by-Pointing}
\labch{pbp}

In this chapter, I will introduce arguably the first application of a deep
inference principle to the construction of proofs by direct manipulation: the
Proof-by-Pointing (hereafter ``PbP'') algorithm. Introduced in the '90s by the
team of G. Kahn at INRIA \cite{PbP}, it demonstrates a powerful idea that will
be pervasive throughout this thesis: the simple act of \emph{pointing} at parts
of a logical expression can provide enough information to synthesize
\emph{complex} reasoning steps in a \emph{controlled} way.

By ``complex'', I do not mean that it comes up with particulary long or clever
demonstrations, as can be found in proofs of advanced mathematical theorems.
Indeed the PbP algorithm is deceptively simple, following deterministically a
small set of trivial rules of inference, which analyse only the purely logical
parts of the expression involved\sidenote{In fact, the rules are exactly that of
the \emph{sequent calculus} presented in \refch{structural-proof-theory}.}.
Rather, its power comes from its ability to handle not only formulas of
\emph{arbitrary shape}, but also locations at an \emph{arbitrary depth} inside a
formula. The latter is what gives it its \emph{deep inference} flavor.

The aforementioned properties have three important consequences: from its simple
rule-based specification, logical soundness of the PbP algorithm is immediate.
From its ability to handle arbitrary logical connectives, inherited from sequent
calculus, it is also \emph{complete} with respect to provability in
propositional logic\sidenote{By supposing an additional input mechanism to
specify \emph{witnesses} when instantiating quantifiers, it can also be shown
complete for full first-order logic.}. And because of its \emph{deep} nature,
PbP can express compound reasoning steps of arbitrary length, while only
requiring a constant amount of information from the user: pointing at a given
location.

Regarding the last point, this comes in stark contrast with the basic commands
provided by usual proof languages, which implement low-level rules of inference
that must be combined manually in an explicit textual syntax\todo{concrete
examples, e.g. Isar for declarative proof languages, and Ltac for imperative
tactic languages}. Furthermore, this syntax is often idiosyncratic, varying from
one proof assistant to the other for no particularly good reason\todo{other
example comparing various syntaxes}. The main benefit brought forth by PbP, but
more generally by the direct manipulation paradigm, is that it provides a
\emph{universal} language for interaction, abstracting from inessential details
related to the static representation of procedural data\todo{add citation of
some work about the universal aspect of direct manipulation interfaces}. It has
been argued in \cite{Chaudhuri2013} that it is especially adapted to the setting
of logical reasoning, as universality is often considered a core property in
philosophical and epistemological accounts of logic\todo{citation idoine,
peut-être en regardant la biblio du cours de philo de la logique du LMFI ?}.