% !TEX root =index.tex
\setchapterpreamble[u]{\margintoc}
\chapter{Asymmetric Bubble Calculus}
\labch{bubbles}

\epigraph{Leibniz sought to make the form of a symbol reflect its content. ``In
signs,'' he wrote, ``one sees an advantage for discovery that is greatest when
they express the exact nature of a thing briefly and, as it were, picture it;
then, indeed, the labor of thought is wonderfully diminished.''}
{\textbf{Frederick Kreiling}, \textit{Leibniz}, 1968}


\begin{scope}\knowledgeimport{bubble}


We introduce a new kind of \kl{nested sequent} \kl{proof system} dubbed
\intro{bubble calculus}. Inspired by the \emph{membrane} mechanism of the
\intro{chemical abstract machine} (\reintro{\cham})
\sidecite[5em]{berry_chemical_1989}, so-called \emph{\kl{bubbles}} internalize
the notion of \kl{subgoal} inside \kl{sequents}, rather than through the tree
structure induced by traditional \kl{inference rules}. This allows for a more
hierarchical representation of the \kl{proof state}, where
\kl(sequent){contexts} can be shared between different \kl{subgoals}. In
addition to the usual textual syntax, the \kl{bubble calculus} can be expressed
in a graphical syntax, where logical meaning is captured by \emph{physical}
constraints on \kl{diagrammatic} manipulations, instead of \emph{virtual}
restrictions on available \kl{inference rules}.

% We introduce a new kind of nested sequent \kl{proof systems} dubbed \emph{bubble
% calculi}. Inspired by the \emph{membrane} mechanism of the chemical abstract
% machine ({\cham} hereafter) \sidecite[25em]{berry_chemical_1989}, so-called
% \emph{bubbles} internalize the notion of \emph{subgoal} inside sequents, rather
% than through the tree structure induced by \kl{inference rules}. This allows for a
% more hierarchical representation of the proof state, where contexts can be
% shared between different subgoals. In addition to the usual textual syntax,
% bubble calculi can be expressed in a graphical syntax, where logical meaning is
% captured by \emph{physical} constraints on diagrammatic manipulations, instead
% of \emph{virtual} restrictions on available \kl{inference rules}. In the chemical
% metaphor, \emph{intuitionism} is then characterized as the phenomenon of
% \emph{repulsion} between objects that have the same polarity.

We start in \refsec{chemical} with the genesis of the idea of \kl{bubble
calculus}, coming from the observation that our \kl{Proof-by-Action} paradigm
(\refch{pba}) lends itself quite naturally to a \kl{metaphorical}
interpretation, where actions are seen as \emph{chemical} reactions. In
\refsec{bubbles} we introduce the concept of \emph{bubble} as a way to control
the scope of hypotheses inside \kl{nested sequents} that we call
\emph{\kl{solutions}}. In \refsec{asymmetric} we describe our \kl{proof system} for
\kl{intuitionistic} logic dubbed \emph{asymmetric \kl{bubble calculus}}, based
on multiset \kl{rewriting rules} over \kl{solutions} comprising at most one
conclusion. Finally in \refsec{bubbles-pba}, we import ideas from this
\kl{bubble calculus} back to the realm of \kl{GUIs} for interactive proof
building, analyzing their possible impact for \kl{UX} improvements.

\section{The chemical metaphor}\labsec{chemical}

The \kl{Proof-by-Action} paradigm introduced in \refch{pba} offers multiple ways to
the user to attack the proof of a theorem. \kl{DnD} actions for \kl{subformula linking}
and equality rewriting are the main mechanism, but they only work in a goal
comprising multiple \kl{items}. Since it is customary in \kl{proof assistants} to specify
the \kl{goal} to be proved as a single logical formula, one needs a way to decompose
it into many \kl{items} for further processing through \kl{DnD}. This is precisely what
the \kl{introduction rules} for logical connectives in \kl{sequent calculus} do, and
following the \kl{Proof-by-Pointing} paradigm \cite{PbP} we map them to click
actions (see \refsec{clicks}).

So visually, a proof in \kl{Actema} consists in breaking logical \kl{items} into
subitems positioned freely in space, and then bringing those subitems together
to make them interact and produce a new \kl{item}. This is quite evocative of a
\emph{chemical reaction} controlled by the user, where logical formulas are akin
to molecules made of propositional atoms linked together by logical
connectives\sidenote{This precise \kl{metaphor} about the molecular structure of
propositions can already be found in Russell's introduction to Wittgenstein's
\textit{Tractatus Logico-Philosophicus} \cite[p.~11]{tractatus}, which was the
main inspiration to his philosophy of \emph{logical atomism}
\cite{klement_russells_2020}. Even earlier in the history of logic, C. S. Peirce
took inspiration from chemical \kl{diagrams} to devise his \emph{\kl{existential
graphs}} --- see \cite[pp.~17--18]{Roberts+1973} or our own presentation in
\refsec{beta} for more details.}. Click actions are then a mean to ``heat''
molecules to the point of breaking these chemical bonds. The most canonical
examples are the \kl{right introduction rule} for implication \kl{{\limp}R},
which breaks a conclusion/\kl{positive} \kl{ion} into a new
hypothesis/\kl{negative} \kl{ion} and a new conclusion; and the \kl{left
introduction rule} for conjunction \kl{\land L}, which breaks a hypothesis into
two hypotheses. In fact, we strongly conjecture that these are the only click
actions needed to obtain a complete deductive system for propositional logic:
breaking red implications allows for \kl(dnd){backward} \kl{DnDs}, and blue
conjunctions for \kl(dnd){forward} \kl{DnDs}\sidenote{In \kl{predicate logic},
one would also need the \kl[right introduction rule]{right} (resp. \kl[left
introduction rule]{left}) introduction rule for $\forall$ (resp. $\exists$). It
might also be the case that \kl(dnd){backward} \kl{DnDs} alone are sufficient
for completeness, since a \kl{linkage} of the form $A \back \select{B} \limp C$
will involve a \kl(dnd){forward} phase. In this case only the \kl{right
introduction rules} for $\limp$ and $\forall$ would be required.}.
% \sidenote{Interestingly,
% those rules are the basis for the adjunction between $\land$ and $\limp$ in the
% interpretation of IPL into cartesian closed categories.}

Rather than completeness, the issue here is \emph{consistency} of the user
interface: if the user is allowed to decompose red $\conc{\limp}$ and blue
$\hypo{\land}$, she will assume naturally that she can also decompose blue
$\hypo{\limp}$ and red $\conc{\land}$, as well as $\lor$ of any color. While red
$\conc{\lor}$ can be handled by pointing directly at the disjunct to be proved,
other configurations correspond to rules of \kl{sequent calculus} with multiple
premisses. In \kl{Actema}, this corresponds to creating a new \kl{subgoal} for
each premise, where \kl{subgoals} are displayed one at a time in different
\emph{tabs}: this new interface mechanism breaks the chemical \kl{metaphor}. The
root cause lies in the way \kl{sequent calculus} implements
\emph{context-scoping}: each \kl{subgoal} will share the same initial context of
hypotheses, but future hypotheses ``buried'' in the conclusions must be
available only in their respective \kl{subgoals}. The tabs mechanism implements
this by forcing the user to focus on exactly one tab/\kl{subgoal}, thus making
it impossible to display \kl{items} from different \kl{subgoals} on the same
screen: this renders interaction between items \emph{physically} impossible.


\section{Bubbles and solutions}\labsec{bubbles}

In order to accomodate context-scoping within the chemical \kl{metaphor}, we
were led to explore a notion of \emph{\kl{bubble}} inspired by the
\emph{membranes} of the \kl{\cham} \sidecite{berry_chemical_1989}. The latter
are used to delineate zones of \emph{local} interaction, which are still porous
to external data. This is precisely what we want to do here: let us consider
that the user tries to prove the \kl{sequent} $\hypo{\Gamma} \seq \conc{A \land
B}$. By clicking on the red \kl{item} $\conc{A \land B}$, she will break it into
two \kl{bubbles} $\bubbleT{\seq \conc{A}}$ and $\bubbleT{\seq \conc{B}}$. Then
she might decompose $\conc{A}$ and $\conc{B}$ further into \kl{sequents}
$\sigma_A = \hypo{\Gamma_A} \seq \conc{C_A}$ and $\sigma_B = \hypo{\Gamma_B}
\seq \conc{C_B}$, and use hypotheses from $\hypo{\Gamma}$ by dragging them
inside either $\bubbleT{\sigma_A}$ or $\bubbleT{\sigma_B}$. However, hypotheses
from $\hypo{\Gamma_A}$ and $\hypo{\Gamma_B}$ cannot be dragged out from their
respective bubble, since then they could be used in the other \kl{bubble} and
violate context-scoping.

This situation is illustrated in \reffig{bubbles-flow}, where \kl{bubbles} are
represented by gray circles, and possible drag moves of formulas by arrows. More
specifically, green and orange arrows \kl{symbolize} respectively valid and invalid
moves. Notice how this graphical depiction of \kl{bubbles} exhibits their
\emph{topological} behavior: while objects can enter inside \kl{bubbles} from the
outside, they get blocked by the membrane in the opposite direction. Indeed the
only relevant feature of the circle representation is that it divides the space
into an \emph{interior} and an \emph{exterior}. Then the \emph{nesting} of
circles and the \emph{positions} of formulas relative to them encode
respectively the \emph{tree} structure of the proof, and the scope of hypotheses
in it.

\begin{figure}
\stkfig{1.5}{bubbles-flow}
\caption{Context-scoping in \kl{bubbles} as topological constraints}
\labfig{bubbles-flow}
\end{figure}
\kl{Bubbles} can also be seen as a way to internalize in the syntax of
\kl{sequents} the notion of \emph{subgoal}, which requires in turn to allow
nesting of \kl{sequents} inside each other. The \kl{proof state} is not a set of
\kl{subgoals} anymore, but a single \kl{nested sequent} of this sort, that we
call a \emph{\kl{solution}}\sidenote{The term ``solution'' refers here to the
\kl{metaphor} of a \emph{chemical solution} made up of an unordered collection
of molecules. Which is quite ironic, since we use it to denote \kl{goals}
waiting to be proved, that is problems lacking a solution\dots}. In textual
syntax, \kl{solutions} $S$ are generated by the following grammar:
\begin{mathpar}
S, T, U \Coloneq \Gamma \intro*\piq{S_1 \sep \ldots \sep S_n} \Delta
\and
\Gamma, \Delta \Coloneq A_1, \ldots, A_n
\end{mathpar}
where the left-hand $\Gamma$ and right-hand $\Delta$ in \kl{solutions} represent
respectively \emph{hypotheses} and \emph{conclusions}, and the $A_i$ are usual
formulas of \kl{FOL}. Thus \kl{solutions} are just like sequents, except that we add
a collection of nested \kl{solutions} $S_i$ that will represent \kl{subgoals}, or
premisses of usual \kl{inference rules}. To be more precise, the collections of
formulas $A_i$ and \kl{solutions} $S_i$ are \emph{multisets}, which gives the
following mutually recursive definitions:
\begin{definition}[Ion]
An \intro{ion} is a formula charged either \intro{negatively} (hypothesis) or
\intro{positively} (conclusion).
\end{definition}
\begin{definition}[Bubble]
A \intro{bubble} is a \kl{solution} enclosed in a membrane.
\end{definition}
\begin{definition}[Solution]\labdef{solution}
  A \intro{solution} $S$ is a multiset of \kl{ions} and \kl{bubbles}. It is
  \intro{single-conclusion} if it contains at most one \kl{positive} \kl{ion}. We
  will use letters $\cS, \mathcal{T}, \mathcal{U}$ to denote multisets of
  \kl{solutions}.
\end{definition}
Note that in the above definitions, \kl{bubbles} play a purely \kl{metaphorical} role and
could be dispensed with. But it will be useful later on to distinguish them
conceptually from \kl{solutions}.

\section{Asymmetric calculus}\labsec{asymmetric}

\subsection{Interpreting solutions}

A natural way to give logical meaning to a \kl{solution} is to translate it into a
formula. In the following we provide one such translation, which will play a
determining role in the design of \kl{inference rules} for manipulating \kl{solutions}. We
qualify it of \emph{asymmetric} because it only works for \kl{single-conclusion}
\kl{solutions}, in the same way that \kl{LJ} only works for single-conclusion
sequents.

\begin{remark}
In this section we only deal with \kl{single-conclusion} \kl{solutions}, but the more
general case will be studied in \refch{bubbles-symm}.
\end{remark}

Just like a sequent, a \kl{solution} is semantically equivalent to an implication,
except that we add the \emph{conjunction} of all \kl{subgoals} to the consequent:

\begin{definition}[Asymmetric interpretation]\labdef{ainterp}
  The \intro{asymmetric interpretation} $\intro*\aint{-}$ on \kl{solutions}
  is defined recursively by:
$$\aint{\Gamma \piq{S_1 \sep \ldots \sep S_n} \Delta} = \bigwedge \Gamma
  \limp \left(\bigwedge \Delta \land \bigwedge_i{\aint{S_i}}\right)$$
\end{definition}

Note that we join formulas in $\Delta$ conjunctively: since we do not consider
\kl{solutions} with more than one conclusion, this is just to handle the case where
$\Delta = \emptyset$, and thus $\bigwedge \Delta = \top$. This subtle detail is
in fact essential to the way we encode the tree structure of proofs inside
\kl{solutions}:
\begin{itemize}
\item a \kl{solution} with one conclusion corresponds to a \emph{leaf} of the proof
tree, i.e. a \kl{subgoal};
\item a \kl{solution} with no conclusion corresponds to a \emph{node} of the proof
tree, i.e. a branching point where we created multiple \kl{subgoals}.
\end{itemize}
This will soon become clearer with examples of derivations in our calculus. In
\refsec{symm-interp}, we will consider a different interpretation of \kl{solutions} that
entails a different encoding of the proof structure in them.

\subsection{Sequent-style rules}

\begin{figure*}
\begin{framed}
\renewcommand{\arraystretch}{3}
\begin{mathpar}
\begin{array}{r@{\quad}l}
\multicolumn{2}{c}{\intro{\identity}} \\[1em]

\R[\intro{i{\da}}]
    {\Gamma \piq{\cS} {}}
    {\Gamma, A \piq{\cS} A}
&
\R[\intro{i{\ua}}]
    {\Gamma \piq{\cS \sep {} \piq{} A \sep A \piq{} \Delta} {}}
    {\Gamma \piq{\cS} \Delta} \\
\end{array}
\and
\begin{array}{r@{\quad}l}
\multicolumn{2}{c}{\intro{\resource}} \\[1em]

\R[\intro{w}]
    {\Gamma \piq{\cS} \Delta}
    {\Gamma, A \piq{\cS} \Delta}
&
\R[\intro{c}]
    {\Gamma, A, A \piq{\cS} \Delta}
    {\Gamma, A \piq{\cS} \Delta} \\
\end{array}
\\
\begin{array}{r}
\multicolumn{1}{c}{\intro{\flow}} \\[1em]

\R[\intro{f{-}}]
    {\Gamma \piq{\cS \sep \Gamma', A \piq{\mathcal{S'}} \Delta'} \Delta}
    {\Gamma, A \piq{\cS \sep \Gamma' \piq{\mathcal{S'}} \Delta'} \Delta}
\end{array}
\and
\begin{array}{r}
\multicolumn{1}{c}{\intro{\membrane}} \\[1em]

\R[\intro{p}]
    {\Gamma \piq{\cS} \Delta}
    {\Gamma \piq{\cS \sep \piq{}} \Delta}
\end{array}
\\
\begin{array}{c@{\quad}c}
\multicolumn{2}{c}{\intro{\heating}} \\[1em]

\R[\intro{\top{-}}]
    {\Gamma \piq{\cS} \Delta}
    {\Gamma, \top \piq{\cS} \Delta}
&
\R[\intro{\top{+}}]
    {\Gamma \piq{\cS} {}}
    {\Gamma \piq{\cS} \top}
\\
\R[\intro{\bot{-}}]
    {\Gamma \piq{\cS} {}}
    {\Gamma, \bot \piq{\cS} \Delta}
&\\
\R[\intro{\land{-}}]
    {\Gamma, A, B \piq{\cS} \Delta}
    {\Gamma, A \land B \piq{\cS} \Delta}
&
\R[\intro{\land{+}}]
    {\Gamma \piq{\cS \sep {} \piq{} A \sep {} \piq{} B} {}}
    {\Gamma \piq{\cS} A \land B}
\\
\multirow{2}{*}{
\R[\intro{\lor{-}}]
    {\Gamma \piq{\cS \sep A \piq{} \Delta \sep B \piq{} \Delta} {}}
    {\Gamma, A \lor B \piq{\cS} \Delta}}
&
\R[\intro{\lor{+}_1}]
    {\Gamma \piq{\cS} A}
    {\Gamma \piq{\cS} A \lor B}
\\&
\R[\intro{\lor{+}_2}]
    {\Gamma \piq{\cS} B}
    {\Gamma \piq{\cS} A \lor B}
\\
\R[\intro{{\limp}{-}}]
    {\Gamma \piq{\cS \sep {} \piq{} A \sep B \piq{} \Delta}}
    {\Gamma, A \limp B \piq{\cS} \Delta}
&
\R[\intro{{\limp}{+}}]
    {\Gamma, A \piq{\cS} B}
    {\Gamma \piq{\cS} A \limp B}
\\
\R[\intro{\forall{-}}]
    {\Gamma, \subst{A}{t}{x} \piq{\cS} \Delta}
    {\Gamma, \forall x. A \piq{\cS} \Delta}
&
\R[\intro{\forall{+}}]
    {\Gamma \piq{\cS} A}
    {\Gamma \piq{\cS} \forall x. A}
\\
\R[\intro{\exists{-}}]
    {\Gamma, A \piq{\cS} \Delta}
    {\Gamma, \exists x. A \piq{\cS} \Delta}
&
\R[\intro{\exists{+}}]
    {\Gamma \piq{\cS} \subst{A}{t}{x}}
    {\Gamma \piq{\cS} \exists x. A}
\end{array}
\end{mathpar}

In the \kl{\forall{+}} and \kl{\exists{-}} rules, $x$ is not free in $\Gamma$,
$\Delta$ and $\cS$.
\end{framed}
\caption{Sequent-style presentation of the asymmetric \kl{bubble calculus} \kl{BJ}}
\labfig{sequent-BJ}
\end{figure*}

Our initial idea for a \kl{proof system} based on \kl{solutions} was quite simple: we
take the \kl{inference rules} of \kl{LJ}, and turn them each into an unary rule
by encoding premisses as bubbles. This gives the basis for the set of rules
presented in \reffig{sequent-BJ}, which defines our asymmetric \kl{bubble
calculus} for \kl{intuitionistic} logic dubbed \intro{BJ}. It is divided in five
groups:
\begin{itemize}
\item The {\identity}, {\resource} and {\heating} groups correspond respectively
to the identity, structural and logical rules of \kl{sequent calculus},
following the terminology of \sidecite{girard:hal-01322183}. More precisely,
rules {\kl{i{\da}}} and {\kl{i{\ua}}} correspond to the \kl(rule){axiom} and
\kl(rule){cut} rules; rules {\kl{w}} and {\kl{c}} to the \kl{weakening} and
\kl{contraction} rules; and every rule of the form $\mcirc{-}$ (resp.
$\mcirc{+}$) to the \kl{left introduction rule} (resp. \kl{right introduction
rule}) for the logical connective $\mcirc$.
\item The {\flow} and {\membrane} groups are new, and define the behavior of
bubbles. More specifically, $\mathbb{F}$-rules characterize how information
flows inside \kl{solutions} by specifying what kinds of objects can traverse
bubbles, and in which direction. They play the same role as \emph{switch} rules
in formalisms based on \kl{CoS} \cite{Guglielmi1999ACO}, which includes our own
\kl{subformula linking} rules (\reffig{DISL}). In the asymmetric \kl{bubble
calculus} there is only one $\mathbb{F}$-rule {\kl{f{-}}} allowing hypotheses to
flow inside bubbles.

As their name suggests, $\mathbb{M}$-rules handle the behavior of the
\emph{membrane} of bubbles, but independently from other \kl{items} as opposed to
$\mathbb{F}$-rules. In the asymmetric \kl{bubble calculus} there is only one
$\mathbb{M}$-rule {\kl{p}} allowing to \emph{pop} any empty bubble, which
can be interpreted as the action of dismissing a solved \kl{subgoal}. In \kl{CoS} it
would correspond to congruence rules handling the truth unit $\top$, and in
\kl{subformula linking} to the unit elimination rules (\reffig{DISL-U}).
\end{itemize}

Now that we have rules for manipulating \kl{solutions}, and since \kl{solutions}
can be nested through bubbles, we need a notion of \emph{context} for applying
rules on subsolutions of arbitrary depth:

\begin{definition}[Context]\labdef{solution-context}
% \emph{Solution contexts} are defined by the following grammar:
% $$S\hole \Coloneq \hole \mid \Gamma \piq{\cS \sep S\hole \sep
% \mathcal{T}} \Delta$$
A \intro{context} $S\hole$ is a \kl{solution} which contains exactly one
occurrence of a special \kl{solution} written $\hole$, called its \reintro{hole}.
Given another \kl{solution} $T$, we write $S\select{T}$ to denote the \kl{solution} equal
to $S\hole$ where $\hole$ has been replaced by $T$.
\end{definition}

Then every rule of \reffig{sequent-BJ} is applicable in any
\kl{context} $U\hole$. That is:
$$\vcenter{\R{S}{T}} \quad \text{should be read as} \quad
\vcenter{\R{U\select{S}}{U\select{T}}} \quad \text{for all $U\hole$}$$

\begin{definition}[Derivation]\labdef{BJ-deriv}
  
  We write $S \intro*\step{} T$ to indicate a \emph{rewrite step}, that is an
  instance of some rule from \reffig{sequent-BJ} with $T$ as premiss and $S$ as
  conclusion\sidenote[][-18cm]{The direction of the arrow is from conclusion to
  premiss, to stay consistent with our interactive proof building setting where
  \kl{inference rules} are seen as \kl{goal}-modifying actions.}. A
  \emph{derivation} $S \reintro*\nsteps{n}{} T$ is a sequence of rewrite steps
  $S_0 \step{} S_1 \ldots \step{} S_n$ with $S_0 = S$, $S_n = T$ and $n \geq 0$.
  Generally the length $n$ of the derivation does not matter, and we just write
  $S \reintro*\steps{} T$. Finally, derivations are closed under arbitrary
  \kl{contexts}: for every \kl{context} $U\hole$, $S \step{} T$ implies
  $\cfill{U}{S} \step{} \cfill{U}{T}$. We write $S \intro*\lstep{} T$ to denote
  a \emph{shallow} step, i.e. a direct instance of a rule in the empty
  \kl{context}.
\end{definition}

\begin{definition}[Proof]\labdef{BJ-proof}
  A \emph{proof} of a \kl{solution} $S$ in \kl{BJ} is a derivation $S \steps{}
\piq{}$ that reduces $S$ to the empty \kl{solution}, which denotes the \kl{proof
state} where there are no \kl{subgoals} left.
\end{definition}

\begin{marginfigure}
\begin{mathpar}
  \R[{\limp}{+}]
  {\R[{\land}{+}]
  {\R[{\limp}{+}]
  {\R[{\limp}{+}]
  {\R[\mathsf{c}]
  {\R[\mathsf{f}{-}]
  {\R[\mathsf{f}{-}]
  {\R[{\limp}{-}]
  {\R[\mathsf{i}{\da}]
  {\R[\mathsf{p}]
  {\R[{\lor}{+}_1]
  {\R[\mathsf{f}{-}]
  {\R[\mathsf{i}{\da}]
  {\R[\mathsf{p}]
  {\R[\mathsf{p}]
  {\R[{\limp}{-}]
  {\R[\mathsf{i}{\da}]
  {\R[\mathsf{p}]
  {\R[{\lor}{+}_2]
  {\R[\mathsf{f}{-}]
  {\R[\mathsf{i}{\da}]
  {\R[\mathsf{p}]
  {\R[\mathsf{p}]
  {\piq{}}
  {\select{\piq{\piq{}}}}}
  {\piq{\select{\piq{\piq{}}}}}}
  {\piq{\piq{\select{B \piq{} B}} {}}}}
  {\piq{\select{B \piq{{} \piq{} B}} {}}}}
  {\piq{B \piq{\select{{} \piq{} A \lor B}} {}}}}
  {\piq{\select{B \piq{{} \piq{} A \lor B \sep \piq{}} {}}}}}
  {\piq{B \piq{{} \piq{} A \lor B \sep \select{C \piq{} C}} {}}}}
  {\piq{\select{B, A \lor B \limp C \piq{} C {}}}}}
  {\select{\piq{\piq{} \sep B, A \lor B \limp C \piq{} C} {}}}}
  {\piq{\select{\piq{\piq{}}} \sep B, A \lor B \limp C \piq{} C} {}}}
  {\piq{\piq{\select{A \piq{} A}} \sep B, A \lor B \limp C \piq{} C} {}}}
  {\piq{\select{A \piq{{} \piq{} A} {}} \sep B, A \lor B \limp C \piq{} C} {}}}
  {\piq{A \piq{\select{{} \piq{} A \lor B} {}} \sep B, A \lor B \limp C \piq{} C} {}}}
  {\piq{\select{A \piq{{} \piq{} A \lor B \sep \piq{}} {}} \sep B, A \lor B \limp C \piq{} C} {}}}
  {\piq{A \piq{{} \piq{} A \lor B \sep \select{C \piq{} C} {}} \sep B, A \lor B \limp C \piq{} C} {}}}
  {\piq{\select{A, A \lor B \limp C \piq{} C} \sep B, A \lor B \limp C \piq{} C {}}}}
  {\select{A \lor B \limp C \piq{A, A \lor B \limp C \piq{} C \sep B \piq{} C} {}}}}
  {\select{A \lor B \limp C, A \lor B \limp C \piq{A \piq{} C \sep B \piq{} C} {}}}}
  {\select{A \lor B \limp C \piq{A \piq{} C \sep B \piq{} C} {}}}}
  {A \lor B \limp C \piq{A \piq{} C \sep \select{{} \piq{} B \limp C}} {}}}
  {A \lor B \limp C \piq{\select{{} \piq{} A \limp C} \sep {} \piq{} B \limp C} {}}}
  {\select{A \lor B \limp C \piq{} (A \limp C) \land (B \limp C)}}}
  {\select{{} \piq{} (A \lor B \limp C) \limp (A \limp C) \land (B \limp C)}}
\end{mathpar}

\caption{Example of sequent-style proof in \kl{BJ}}
\labfig{ex-seq-BJ}
\end{marginfigure}

\subsection{Proof-as-trace}

An example of proof in \kl{BJ} is shown in \reffig{ex-seq-BJ}, where the
focused subsolution is squared for each inference. Notice that many rules could
have been applied in a different order: for instance all applications of the
{\kl{p}} rule could have been postponed to the top/end of the derivation.
This is generally true of all formalisms based on \kl{CoS}, which is known in the
\kl{deep inference} literature for its ``bureaucracy''. In \kl{BJ},
$\mathbb{H}$-rules aggravate the matter by adding all inessential rule
permutations from \kl{sequent calculus} to those of \kl{CoS}. As our wording suggests,
this is usually perceived negatively in \kl{deep inference} \kl{proof theory}, where a
central question is that of finding \emph{canonical} representations of proof
objects \sidecite{strasburger-problem-2019}.

However in our interactive proof building setting, it should rather be seen as a
\emph{desirable} property of the system. Indeed, one consequence is that the
user has more freedom to organize her reasoning in whichever order she wants, in
an incremental and guided way. One should remember that in the
\kl{Proof-by-Action} paradigm, the focus is not the proof object, which is
implicit and hidden to the user, but the \emph{process} of building it. Then a
\kl{BJ}-derivation is better understood as the \emph{trace} of this building
process, rather than the constructed proof\sidenote{The idea of
\emph{proof-as-trace} is relatively common in logic programming
\cite{miller_survey_2022}, but not so much in \kl{deep inference} \kl{proof
theory}. It is Jean-Baptiste Joinet who shared with us his idea of applying it
in this setting, based on his own work interpreting the \kl{CoS} for \kl{MLL} as
a system for building \emph{multiplicative proof nets}
\cite{joinet_completeness_2007}.}. And the fact that this trace corresponds, or
can be transformed into a more canonical representation is of no concern to the
user. What matters for a good proof building interface is to be as flexible as
possible, in order to match the user's own mental process of argumentation.

Of course flexibility comes at a price, and the rules of \kl{BJ} are probably
too numerous and low-level to be mapped directly into individual proof actions
in a user interface. Some of these concerns will be tackled in
\refsubsec{bubbles-search}, but we think a better answer might have been found
with the \kl{proof system} introduced in \refch{flowers}, and its associated
prototype of \kl{GUI} presented in \refsec{flowers-prover}.

\subsection{Graphical rules}\labsec{bubbles-graphical-rules}

\begin{figure*}
  \begin{framed}
\renewcommand{\arraystretch}{1.25}
\begin{mathpar}
\begin{array}{r@{\quad}c@{\quad}lr}
  \multicolumn{4}{c}{\identity} \\[2em]

   \hypo{A}~~~\conc{A}
  &\step{}
  &
  &\kl{i{\da}} \\

   \conc{\Delta}
  &\step{}
  &\bubble{\conc{A}}~~~\bubble{\hypo{A}~~~\conc{\Delta}}
  &\kl{i{\ua}} \\
\end{array}
\and
\begin{array}{r@{\quad}c@{\quad}lr}
  \multicolumn{4}{c}{\resource} \\[2em]

    \hypo{A}
  &\step{}
  &
  &\kl{w} \\

    \hypo{A}
  &\step{}
  &\hypo{A~~~A}
  &\kl{c} \\
\end{array}
\vspace{2em}\\
\begin{array}{r@{\quad}c@{\quad}lr}
  \multicolumn{4}{c}{\flow} \\[2em]

    \hypo{A}~~~\bubble{\color{black}S}
  &\step{}
  &\bubble{\hypo{A}~~~\color{black}{S}}
  &\kl{f{-}} \\
\end{array}
\and
\begin{array}{r@{\quad}c@{\quad}lr}
  \multicolumn{4}{c}{\membrane} \\[2em]

    \bubble{\phantom{S}}
  &\step{}
  &
  &\kl{p} \\
\end{array}
\vspace{2em}
\\
\begin{array}{r@{\quad}c@{\quad}lr@{\qquad\qquad}r@{\quad}c@{\quad}lr}
  \multicolumn{8}{c}{\heating} \\[2em]

    \hypo{\top}
  &\step{}
  &
  &\kl{\top{-}}

  &\conc{\top}
  &\step{}
  &
  &\kl{\top{+}} \\

    \hypo{\bot}~~~\conc{\Delta}
  &\step{}
  &
  &\kl{\bot{-}}

  &&&&\\

    \hypo{A \land B}
  &\step{}
  &\hypo{A}~~~\hypo{B}
  &\kl{\land{-}}

  &\conc{A \land B}
  &\step{}
  &\bubble{\conc{A}}~~~\bubble{\conc{B}}
  &\kl{\land{+}} \\

    \multirow{2}{*}{$\hypo{A \lor B}~~~\conc{\Delta}$}
  &\multirow{2}{*}{$\step{}$}
  &\multirow{2}{*}{$\bubble{\hypo{A}~~~\conc{\Delta}}~~~\bubble{\hypo{B}~~~\conc{\Delta}}$}
  &\multirow{2}{*}{$\kl{\lor{-}}$}

  &\conc{A \lor B}
  &\step{}
  &\conc{A}
  &\kl{\lor{+}_1} \\

  &&&

  &\conc{A \lor B}
  &\step{}
  &\conc{B}
  &\kl{\lor{+}_2} \\

    \hypo{A \limp B}~~~\conc{\Delta}
  &\step{}
  &\bubble{\conc{A}}~~~\bubble{\hypo{B}~~~\conc{\Delta}}
  &\kl{{\limp}{-}}

  &\conc{A \limp B}
  &\step{}
  &\hypo{A}~~~\conc{B}
  &\kl{{\limp}{+}} \\

    \hypo{\forall x. A}
  &\step{}
  &\hypo{\subst{A}{t}{x}}
  &\kl{\forall{-}}

  &\conc{\forall x. A}
  &\step{}
  &\conc{\subst{A}{y}{x}}
  &\kl{\forall{+}} \\

    \hypo{\exists x. A}
  &\step{}
  &\hypo{\subst{A}{y}{x}}
  &\kl{\exists{-}}

  &\conc{\exists x. A}
  &\step{}
  &\conc{\subst{A}{t}{x}}
  &\kl{\exists{+}} \\
\end{array}
\vspace{2em}
\end{mathpar}
In the rules {\kl{i{\ua}}}, {\kl{\bot{-}}}, {\kl{\lor{-}}} and {\kl{{\limp}{-}}}, $\Delta$
is either empty, or a singleton of one \kl{positive} \kl{ion}.\\
In the rules {\kl{\forall{+}}} and {\kl{\exists{-}}}, $y$ is fresh.
\end{framed}

  \caption{Graphical presentation of the asymmetric \kl{bubble calculus} \kl{BJ}}
  \labfig{graphical-BJ}
\end{figure*}

While the sequent-style presentation of \kl{BJ} clearly shows its filiation with
\kl{sequent calculus}, its syntax is quite heavy, and obscures an important
property of the rules: they almost always preserve the \kl(sequent){contexts}
$\Gamma, \Delta$ of formulas and $\cS$ of bubbles. That is, the rules of \kl{BJ}
are \emph{local}. This enables a more economical and graphical presentation of
the rules in \reffig{graphical-BJ}, where \kl{BJ} is seen as a multiset
\kl{rewriting system} just like the \kl{\cham} thanks to \refdef{solution}.
Instead of relying on a notion of \kl{context}, we define formally what it means
to be a \emph{subsolution}:

\begin{definition}[Subsolution]\labdef{subsolution}
  $S$ is a \intro{subsolution} of $T$, written $S \intro*\subsol T$, if either $S
  \subseteq T$ or $S \subsol T_0$ for some $T_0 \in T$, where $\subseteq$
  denotes multiset inclusion. 
\end{definition}

Then a multiset \kl{rewriting rule} $\rrule{r}{S}{T}$ can be applied in a
\kl{solution} $U$ whenever $S \subsol U$, by replacing one occurrence of $S$ by $T$
inside $U$. The notions of derivation (\refdef{BJ-deriv}) and proof
(\refdef{BJ-proof}) stay unchanged, by observing that the \kl{rewriting rule}
$\rrule{r}{S}{T}$ from $S$ to $T$ and the \kl{inference rule}
$\irule{r}{S}{T}$ with premiss $T$ and conclusion $S$ denote the same
rule $r$.

\reffig{ex-gra-BJ} shows the graphical presentation of the same \kl{BJ}-proof
as in \reffig{ex-seq-BJ}. Whereas in \reffig{ex-seq-BJ} we squared the whole
\kl{subsolutions} corresponding to the conclusions of \kl{inference rules}, here we
squared on each line the redex modified by the associated \kl{rewriting rule}. This
example highlights the greater locality of the rewriting approach, by indicating
more precisely which parts of the \kl{proof state} are changed by the rules.

But it still over-approximates the modifications that really need to be
performed to carry the transformations. Indeed, by only exposing the data of a
redex $S$ and a reddendum $T$, a \kl{rewriting rule} $\rrule{r}{S}{T}$ can only
be interpreted as the deletion of $S$ followed by the insertion $T$. Taking for
instance the {\kl{{\limp}{-}}} rule in \reffig{graphical-BJ}, one can describe
its graphical behavior more finely as resulting from the following sequence of
\emph{edits}:
\begin{enumerate}
  \item Erase the $\hypo{{\limp}}$ connective;
  \item Change the \kl[positive]{polarity} of $\hypo{A}$ from hypothesis to conclusion;
  \item Insert a new empty bubble;
  \item Move $\conc{A}$ in this bubble;
  \item Insert a new empty bubble;
  \item Move $\hypo{B}$ in this bubble;
  \item If $\conc{\Delta}$ is not empty, also move $\conc{\Delta}$ in this bubble.
\end{enumerate}
It would be interesting to consider the question of finding a minimal set of
edit operations like these, that can simulate all the rules of
\kl{BJ}\sidenote{As will become apparent in \refsec{bubbles-completeness},
\kl{BJ} itself provides a finer-grained simulation of the rules of \kl{sequent
calculus}, which in turn is known to be a more detailed variant of
\emph{\kl{natural deduction}}. Interestingly through the \kl{Curry-Howard
isomorphism}, this would correspond to a \emph{chain of compilation}, starting
from the higher-level \kl{$\lambda$-calculus} (\kl{natural deduction}), going
into abstract machines (\kl{sequent calculus}) \cite{downen_sequent_2016}, down
to something akin to \emph{assembly language} with \texttt{jump} instructions
\cite[Section~6.3.1]{guenot_nested_2013}.}. Note however that most of the above
edits are \emph{unsound} as reasoning steps. If not for logical insight, such an
edit calculus could still be relevant \emph{computationally}, typically by
enabling efficient implementations of the rules with a small memory footprint.

\begin{figure*}
  \setlength{\fboxsep}{2pt}
\renewcommand{\arraystretch}{1.3}
$$
\begin{array}{r@{\qquad}|@{\qquad}l}
\begin{array}[t]{rlr}
        &\select{\conc{(A \lor B \limp C) \limp (A \limp C) \land (B \limp C)}} &\mathsf{{\limp}{+}} \\
  \step{} &\hypo{A \lor B \limp C}~~~\select{\conc{(A \limp C) \land (B \limp C)}} &\mathsf{{\land}{+}} \\
  \step{} &\hypo{A \lor B \limp C}~~~\bubble{\select{\conc{A \limp C}}}~~~\bubble{\conc{B \limp C}} &\mathsf{{\limp}{+}} \\
  \step{} &\hypo{A \lor B \limp C}~~~\bubble{\hypo{A}~~~\conc{C}}~~~\bubble{\select{\conc{B \limp C}}} &\mathsf{{\limp}{+}} \\
  \step{} &\select{\hypo{A \lor B \limp C}}~~~\bubble{\hypo{A}~~~\conc{C}}~~~\bubble{\hypo{B}~~~\conc{C}} &\mathsf{c} \\
  \step{} &\select{\hypo{A \lor B \limp C}~~~\bubble{\hypo{A}~~~\conc{C}}}~~~\hypo{A \lor B \limp C}~~~\bubble{\hypo{B}~~~\conc{C}} &\mathsf{f{-}} \\
  \step{} &~~~{\bubble{
      \begin{array}{@{}c@{}}
        \hypo{A}~~~\conc{C} \\
        \hypo{A \lor B \limp C}
      \end{array}}}
      ~~~\select{\hypo{A \lor B \limp C}
      ~~~\bubble{\hypo{B}~~~\conc{C}}} &\mathsf{f{-}} \\
  \step{} &\bubble{
      \begin{array}{@{}c@{}}
        \hypo{A} \\
        \select{\hypo{A \lor B \limp C}~~~\conc{C}}
      \end{array}}
      ~~~\bubble{
        \begin{array}{@{}c@{}}
          \hypo{B}~~~\conc{C} \\
          \hypo{A \lor B \limp C}
        \end{array}} &\mathsf{{\limp}{-}} \\
  \step{} &\bubble{
      \begin{array}{@{}c@{}}
        \hypo{A} \\
        \bubble{\conc{A \lor B}}~~~\bubble{\select{\hypo{C}~~~\conc{C}}}
      \end{array}}
      ~~~\bubble{
        \begin{array}{@{}c@{}}
          \hypo{B}~~~\conc{C} \\
          \hypo{A \lor B \limp C}
        \end{array}} &\mathsf{{i}{\da}} \\
  \step{} &\bubble{
      \begin{array}{@{}c@{}}
        \hypo{A} \\
        {\bubble{\conc{A \lor B}}}~~~\select{\bubble{\phantom{\hypo{C}~~~\conc{C}}}}
      \end{array}}
      ~~~\bubble{
        \begin{array}{@{}c@{}}
          \hypo{B}~~~\conc{C} \\
          \hypo{A \lor B \limp C}
        \end{array}} &\mathsf{p} \\
  \step{} &\bubble{\hypo{A}~~~\bubble{\select{\conc{A \lor B}}}}
      ~~~\bubble{
        \begin{array}{@{}c@{}}
          \hypo{B}~~~\conc{C} \\
          \hypo{A \lor B \limp C}
        \end{array}} &\mathsf{\lor{+}_1} \\
  \step{} &\bubble{\select{\hypo{A}~~~\bubble{\conc{A}}}}
      ~~~\bubble{
        \begin{array}{@{}c@{}}
          \hypo{B}~~~\conc{C} \\
          \hypo{A \lor B \limp C}
        \end{array}} &\mathsf{f{-}} \\
  \step{} &\bubble{\bubble{\select{\hypo{A}~~~\conc{A}}}}
      ~~~\bubble{
        \begin{array}{@{}c@{}}
          \hypo{B}~~~\conc{C} \\
          \hypo{A \lor B \limp C}
        \end{array}} &\mathsf{i{\da}} \\
\end{array}
&
\begin{array}[t]{rlr}
  \step{} &\bubble{\select{\bubble{\phantom{\hypo{A}~~~\conc{A}}}}}
      ~~~\bubble{
        \begin{array}{@{}c@{}}
          \hypo{B}~~~\conc{C} \\
          \hypo{A \lor B \limp C}
        \end{array}} &\mathsf{p} \\
  \step{} &\select{\bubble{\phantom{\bubble{\phantom{\hypo{A}~~~\conc{A}}}}}}
      ~~~\bubble{
        \begin{array}{@{}c@{}}
          \hypo{B}~~~\conc{C} \\
          \hypo{A \lor B \limp C}
        \end{array}} &\mathsf{p} \\
  \step{} &
      ~~~\bubble{
        \begin{array}{@{}c@{}}
          \hypo{B} \\
          \select{\hypo{A \lor B \limp C}~~~\conc{C}}
        \end{array}} &\mathsf{{\limp}{-}} \\
  \step{} &
      ~~~\bubble{
        \begin{array}{@{}c@{}}
          \hypo{B} \\
          {\bubble{\conc{A \lor B}}}~~~\bubble{\select{\hypo{C}~~~\conc{C}}}
        \end{array}} &\mathsf{i{\da}} \\
  \step{} &
      ~~~\bubble{
        \begin{array}{@{}c@{}}
          \hypo{B} \\
          {\bubble{\conc{A \lor B}}}~~~\select{\bubble{\phantom{\hypo{C}~~~\conc{C}}}}
        \end{array}} &\mathsf{p} \\
  \step{} &
      ~~~\bubble{\hypo{B}~~~{\bubble{\select{\conc{A \lor B}}}}} &\mathsf{{\lor}{+}_2} \\
  \step{} &
      ~~~\bubble{\select{\hypo{B}~~~{\bubble{\conc{B}}}}} &\mathsf{f{-}} \\
  \step{} &
      ~~~\bubble{{\bubble{\select{\hypo{B}~~~\conc{B}}}}} &\mathsf{i{\da}} \\
  \step{} &
      ~~~\bubble{{\select{\bubble{\phantom{\hypo{B}~~~\conc{B}}}}}} &\mathsf{p} \\
  \step{} &
      ~~~\select{\bubble{{\phantom{\bubble{\phantom{\hypo{B}~~~\conc{B}}}}}}} &\mathsf{p} \\
  \step{} && \\
\end{array}
\end{array}
$$
  \caption{Example of graphical proof in \kl{BJ}}
  \labfig{ex-gra-BJ}
\end{figure*}

\section{Back to Proof-by-Action}\labsec{bubbles-pba}

When looking at the \kl{BJ}-proof of \reffig{ex-gra-BJ}, the astute reader might
have been reminded of the \kl{Proof-by-Action} paradigm as introduced in
\refch{pba}, by seeing redexes as the \kl{items} involved in a graphical action
--- there are always at most two such items. More precisely, $\mathbb{H}$-rules
correspond to \emph{click} actions on blue ({\rnm{\mcirc{-}}} rules) or red
\kl{items} ({\rnm{\mcirc{+}}} rules), and the {\kl{i{\da}}} rule corresponds to
the most basic \kl{DnD} action between dual occurrences of a formula.

As mentioned earlier when comparing \kl{BJ} to \kl{LJ}, the novelty here lies
with $\mathbb{H}$-rules, $\mathbb{F}$-rules and $\mathbb{M}$-rules that deal
with \emph{bubbles}. Remember that the goal behind the idea of \kl{bubble calculus}
was precisely to provide a new way to manipulate \kl{subgoals} through \kl{bubbles}
instead of tabs, which are more in line with the chemical \kl{metaphor}. It is quite
easy to imagine a \kl{GUI} presenting the \kl{proof state} as a \kl{solution}, in a graphical
layout close to that of \reffig{ex-gra-BJ}\sidenote{Although there might be some
challenges in implementing an efficient layouting algorithm for bubbles,
typically to make \kl{solutions} fit into the screen.}. Like formulas in blue and red
items, whole \kl{subgoals} could now be shown on the same screen in their respective
bubbles, and be freely moved around with a pointing device. Following are some
ideas for mapping the remaining rules of \kl{BJ} in such a \kl{GUI}:

% \begin{marginfigure}
%   $$
%   \begin{array}{rcll}
%     \conc{t = t} &\step{} &&{=}{+} \\
%     \hypo{t = u}~~~A &\step{} &\subst{A}{u}{t} &{=}{-}1 \\
%     \hypo{t = u}~~~A &\step{} &\subst{A}{t}{u} &{=}{-}2 \\
%   \end{array}
%   $$
%   \caption{Rules for equality in \kl{BJ}}
%   \labfig{bubbles-eq}
% \end{marginfigure}

% \begin{marginfigure}
%   \begin{center}
$$
\begin{array}{rcll}
  &\step{} &\dvar{\ldef{x}{t}} &\mathsf{d{\ua}} \\[1em]

  \dvar{\ldef{x}{t}} &\step{} &\dvar{\ldef{x}{t}}~~~\hypo{x = t} &\mathsf{cd} \\[1em]

  % \dvar{\delta}~~~{\bubble{S}} &\step{} &\bubble{\dvar{\delta}~~~S} &\mathsf{f}\delta \\[1em]

  \conc{\forall x. A} &\step{} &\dvar{\ldef{y}~~~\conc{\subst{A}{y}{x}} &\forall{+} \\
  \hypo{\exists x. A} &\step{} &\dvar{\ldef{y}~~~\hypo{\subst{A}{y}{x}} &\exists{-} \\[1em]

  % \dvar{y}~~~{\hypo{\forall x. A}} &\step{} &\hypo{\subst{A}{y}{x}} &\forall{-}\mathsf{h} \\
  % \dvar{y}~~~{\conc{\exists x. A}} &\step{} &\conc{\subst{A}{y}{x}} &\exists{+}\mathsf{h} \\[1em]

  \dvar{\ldef{y}{\beta}}~~~{\hypo{\forall x. A}} &\step{} &\hypo{\subst{A}{y}{x}} &\forall{-} \\
  \dvar{\ldef{y}{\beta}}~~~{\conc{\exists x. A}} &\step{} &\conc{\subst{A}{y}{x}} &\exists{+} \\
\end{array}
$$
In the {\rnm{\forall{+}}} and {\rnm{\exists{-}}} rules, $y$ is fresh.
\end{center}
%   \caption{Rules for variables and definitions in \kl{BJ}}
%   \labfig{bubbles-vars}
% \end{marginfigure}

\begin{description}
  \item[\textbf{\flow}]
    The {\kl{f{-}}} rule plays a special role, in that it would not be mapped to
    any particular action. Indeed it captures the way information flows in
    \kl{solutions}, and we already described in \refsec{bubbles} how this is
    reflected in the topological behavior of \kl{bubbles}. Thus it could be
    implemented in the graphical interface as a kind of \emph{physics engine},
    like those found in video games: when dragging an \kl{item} around the
    \kl{proof canvas}, it would get stuck on the membrane of \kl{bubbles},
    except when the \kl{item} is blue and the drag movement goes inward. This of
    course would provide a level of interactivity unseen before in a proving
    interface, making it very discoverable and playful. It also combines nicely
    with \kl{DnD} actions in general: for instance a sequence of applications of
    {\kl{f{-}}} followed by {\kl{i{\da}}} could be performed as a single
    \kl{DnD} action, where the dragged hypothesis crosses successively the
    various \kl{bubbles} on the way.
    
  \item[\textbf{\membrane}]
    The {\kl{p}} rule can be mapped very straightforwardly to the action of
    clicking on the area of an empty bubble, in order to pop it. It could also
    be entirely automated, by letting the proof engine eagerly pop empty
    \kl{bubbles} as soon as they appear in a \kl{solution}. Note that in this
    graphical setting, the {\kl{p}} rule can be understood as resulting from a
    process of \emph{contraction}\sidenote{Not to be confused with the
    \kl{contraction} rule \kl{c}.} of the membrane into a single point: if the
    \kl{bubble} contains some \kl{items} $\Delta$, then this process fails
    because the membrane gets stuck on the boundaries of $\Delta$. This is a
    topological way to check the emptiness of a bubble, which has the benefit of
    being completely \emph{continous}, in addition to being very clear visually.

  \item[\textbf{\resource}]
    The \kl{contraction} rule {\kl{c}} could be mapped to a specific triggering
    input when starting to drag a blue \kl{item} $\hypo{A}$ (e.g. a shortkey if
    a keyboard is available, or a long press on the \kl{item} on a touchscreen),
    which has the effect of keeping a copy of $A$ at its original location in
    addition to moving the \kl{item}\sidenote{This mechanism is quite standard
    in \kl{GUIs} that manipulate duplicable resources like file managers, where
    one maintains the \texttt{CTRL} key to enable copy mode. It was also chosen
    by K. Chaudhuri to implement \kl{contraction} in his \kl{Profint} prototype
    for \kl{subformula linking} in \kl{intuitionistic} logic \cite{ProfInt}.}.
    As for the \kl{weakening} rule {\kl{w}}, it could be available as a
    \kl{contextual action} when selecting blue \kl{items}.

  \item[\textbf{\identity}]
    \begin{marginfigure}
      $$
      \begin{array}{rcll}
        \hypo{A}~~~\hypo{B} &\step{} &\hypo{A \forw B} &\forw \\
        \hypo{A}~~~\conc{B} &\step{} &\conc{A \back B} &\back \\
      \end{array}
      $$
      \caption{\kl{Linkage} creation rules in \kl{BJ}}
      \labfig{bubbles-linkage}
    \end{marginfigure}

    Although the {\kl{i{\da}}} rule only corresponds to the base case of
    \kl{DnD} actions, it would be easy to integrate the full \kl{SFL} semantics
    of \kl{DnD} actions directly in \kl{BJ}. Indeed our \kl{SFL} rules
    (\reffig{DISL}) are already expressed as \kl{rewriting rules}, just like the
    graphical rules of \kl{BJ} (\reffig{graphical-BJ}). Thus it is just a matter
    of adding \kl{linkage} creation rules like those of
    \refsec{dnd-completeness}, but between adjacent formulas in a \kl{solution}
    (\reffig{bubbles-linkage}).

    The \kl(rule){cut} rule was handled in \kl{Actema} with a separate
    \textsf{+hyp} button, which adds the cut formula $A$ (input by the user in a
    dialog box) as a new hypothesis in the current \kl{goal}, and as the
    conclusion in a new \kl{subgoal} (see \refsubsec{pba-layout}). Since
    \kl{subgoals} are now reified as bubbles, the {\kl{i{\ua}}} rule could be
    mapped instead to a contextual action available on any red \kl{item}
    $\conc{\Delta}$, which would have the effect of spawning a \kl{bubble}
    around it with a blue \kl{item} $\hypo{A}$, and another \kl{bubble} nearby
    it with a red \kl{item} $\conc{A}$.

  \item[\textbf{\heating}]
    For $\mathbb{H}$-rules that spawn \kl{bubbles} like {\kl{\land{+}}}, it is
    important that \kl{bubbles} stay close to the \kl{item} being clicked, in order to
    make the transformation visually clear. One could even imagine a small
    animation that smoothly turns the main connective into bubbles, to convey
    more effectively the intuition that heating rules break logical connectives
    seen as chemical bonds.
\end{description}

Beyond the recovered uniformity of the user interface in terms of the chemical
\kl{metaphor}, \kl{BJ} exhibits some features that are interesting both on the
\kl{proof-theoretical} and user-experience levels:
\begin{description}
  \item[\textbf{Factorization}] It implements a form of \emph{context-sharing}
    between \kl{subgoals}: that is, one can perform transformations on shared
    hypotheses (\kl{forward} reasoning) without going back to a \kl{proof state}
    anterior to the splitting of said \kl{subgoals}. This should simplify the
    navigation in the proof as it is being constructed, by avoiding the need to
    locate these splitting points. In fact often beginners (but also
    occasionally seasoned users) do not have the reflex to do this, precisely
    because the interface makes it difficult. This results in proofs with a lot
    of duplicated arguments, since splitting \kl{goals} systematically
    duplicates the \kl(sequent){context} of hypotheses. Thus \kl{bubbles} can be
    seen as a mechanism that favors by default a style of proof with better
    factorization of subproofs.
  
  \item[\textbf{Parallelism}]
    The locality of \kl{rewriting rules} makes it possible for multiple users to
    reason on different \kl{subgoals} of the same \kl{proof state} \emph{at the
    same time}, without compromising soundness. Combined with the above
    factorization property, this enables \emph{asynchronous} collaborative
    setups, where various users can work on the same proof in parallel (e.g.
    through an online web interface), while still benefitting from the knowledge
    built by collaborators in shared \kl{contexts}.

  \item[\textbf{Navigation}] The tree structure of \kl{subgoals} is immediately
    apparent in the \kl{proof state} through the nesting of bubbles. Thus part of the
    information on the proof construction process, which was made implicit and
    temporal in the \kl{proof state} history, is now made explicit and spatial in the
    \kl{proof state} itself\sidenote{This concern of finding an explicit graphical
    representation of the ``motions of reasoning \emph{in actu}'', and not only
    the states of mind, can be found already in the works of Peirce on his
    \kl{existential graphs} \cite[pp.~112--113]{Roberts+1973}. We will come back to
    this in \refch{eg}.}.
    
    There are multiple ways to visualize trees on a planar surface, but if we
    are to maintain the \kl{bubble} \kl{metaphor}, \intro{zoomable user
    interfaces} (\reintro{ZUI}) seem to be a right fit: they allow for efficient
    space management and navigation, and \emph{zooming in} intuitively conveys
    the idea of focusing on a specific \kl{subgoal}. One could also \emph{zoom
    out} to have an overview of the different \kl{subgoals} and their shared
    \kl(sequent){context}, something which is hard to do in current \kl{proof
    assistants}.
    
    When zoomed in on a \kl{subgoal}, the shared \kl(sequent){contexts} around
    it will not be visible anymore. While this is useful to focus attention and
    avoid being distracted by other \kl{subgoals}, it can quickly become
    cumbersome for the user to always have to zoom out in order to retrieve
    hypotheses from these shared \kl(sequent){contexts}. One solution would be
    to rebrand the \kl(sequent){context} zone of \kl{Actema} as a \emph{global}
    \kl(sequent){context} zone, where all the shared \kl(sequent){contexts}
    available in the \kl{subgoal} under focus are merged in a single list, and
    immediately accessible for manipulation. Of course actions performed in the
    \kl(sequent){context} zone would be reflected in the \kl{proof canvas}, and
    vice versa.

  \item[\textbf{Goal diffing}] From a user perspective, the locality of rules
    means that applying some action to one or two \kl{items} will not involve
    other \kl{items}\sidenote{The only exceptions are clicks on blue
    $\hypo{\bot}$, $\hypo{\lor}$ and ${\hypo{\limp}}$, but the only extra
    \kl{item} they involve is the conclusion.}. Non-local rules are less natural
    for a beginner because they modify a global state (here other \kl{items} and
    \kl{subgoals}) which is not clearly correlated to the transformed data,
    often because it is not immediately visible.

    For instance in \kl{Actema}, many users have reported difficulties in
    understanding the effect of click actions that create new \kl{subgoals}. A first
    reason that can easily be remedied, is that there was not enough visual
    feedback to indicate the newly created tabs. But a deeper limitation is that
    the user needs to explicitly focus on these \kl{subgoals} to show their content,
    which they might not do immediately. And then it gets difficult to keep
    track of the origin of said \kl{subgoals} without a way to visualize the tree
    structure of the proof.

    All these concerns can be addressed within the \kl{bubble} \kl{metaphor}:
    since \kl{bubbles} are \kl{items} freely positioned on the \kl{proof
    canvas}, all the new \kl{items} produced by an action can stay near the
    location where the action was initiated (i.e. the click or drop location);
    and since all transformations are local, all \kl{items} not involved in the
    action can have their locations preserved. In other words, \kl{bubbles} make
    it easier to understand the \emph{difference} between a \kl{goal} and
    \kl{subgoals} generated by a proof action, which is crucial when learning
    the semantics of actions through practice.
    % This is of limited importance however in our case, because \kl{sequent calculus}
    % rules always perform the same trivial operation on the global state:
    % duplicating the whole context of hypotheses.
\end{description}


\end{scope}