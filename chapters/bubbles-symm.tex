% !TEX root =index.tex
\setchapterpreamble[u]{\margintoc}
\chapter{Symmetric Bubble Calculi}
\labch{bubbles-symm}


\begin{scope}\knowledgeimport{bubble}


In this chapter, we explore to what extent the \kl{bubble calculus} of
\refch{bubbles} can be made more \emph{symmetric}, by relaxing the restriction
that solutions must contain at most one conclusion. At a surface level, our
approach is similar to that of Gentzen, who went from his single-conclusion
\kl{sequent calculus} \kl{LJ} to the multi-conclusion calculus \sys{LK}. Like
him, we will uncover beautiful dualities that were hidden by the asymmetry of
the initial calculus. But by sticking unwaveringly to intuitionism, we will be
led to the exotic territory of \kl{bi-intuitionistic} logic, an intermediate
logic that conservatively extends \kl{intuitionistic} logic, but does not prove the
\kl{law of excluded middle}. An underlying thread of our investigation will be the
quest for a \emph{fully \kl{iconic}} \kl{proof system}, where all logical connectives
can be replaced by appropriate (new) kinds of bubbles. This will lead us to
rediscover many principles already studied in the \kl{deep inference}
literature, with topological intuitions of the \kl{bubble} \kl{metaphor} shedding a new
light on them. We will end up with two symmetric \kl{bubble calculi}, each with their
own tradeoffs on the properties satisfied by \kl{inference rules}. In particular,
their ability to \emph{factorize} both \kl{forward} and \kl{backward} proof steps might
prove useful to build concise proofs, all through \kl{direct manipulation}.

The chapter is organized as follows: in \refsec{non-determinism} we motivate our
quest for a system where all \kl{introduction rules} for logical connectives are
\emph{\kl{invertible}}, to reduce non-determinism in proof search and enable a fully
\emph{\kl{iconic}} approach to proof building. To that effect, we relax in
\refsec{branching} the restriction to single-conclusion solutions, which
requires a new distinction between \emph{closed} and \emph{open} solutions. This
gives rise in \refsec{colors} to an extension of the syntax of solutions, where
\kl{bubbles} can themselves be \emph{\kl{polarized}}. In \refsec{design-props} we identify
key properties that will guide the design of \kl{inference rules}, some of which were
already aimed for implicitly through the evolution of our concept of bubble. In
\refsec{symmetric-calculus} we introduce a core \emph{symmetric \kl{bubble calculus}}
for \kl{classical} logic called \kl{system~B}, in reference to the symmetric
\kl{system L} of Herbelin \sidecite[10em]{herbelin_duality_nodate}. Then in
\refsec{bubbles-soundness} we prove the soundness of \kl{system~B}, and show
that by removing selectively among \kl{inference rules} that define the
\emph{porosity} of \kl{polarized} bubbles, one gets \kl{intuitionistic},
\kl{dual-intuitionistic} and \kl{bi-intuitionistic} logic as fragments. In
\refsec{bubbles-completeness} we support this claim by showing that the
\kl{bi-intuitionistic} fragment is not only sound, but also \emph{cut-free complete}
with respect to the cut-free nested \kl{sequent calculus} \kl{DBiInt} of Postniece
\cite{postniece_deep_2009}. Finally in \refsec{invertible-calculus}, we
introduce a fully \kl{invertible} variant of system $\sysB$ that we conjecture to be
complete, and present a canonical way to search for proofs in this system.
Unfortunately, invertibility does not entail the full \kl{iconicity} of the system,
and we reflect on the fundamental reasons that might prevent any variant of
system $\sysB$ from being fully \kl{iconic}.

% In \refsec{invertible-calculus} we present a fully invertible variant of system
% \kl{B}, whose completeness follows naturally from the proof of
% \refsec{bubbles-completeness}. Despite the invertibility of \kl{introduction rules},
% it turns out that this variant does not satisfy the \emph{decomposability}
% property. We fix this defect in \refsec{decomposable-calculus} with another
% variant of the system conjectured complete, finally achieving full iconicity.

\begin{remark}
  Although we include rules for quantifiers, in this thesis we only treat the
soundness and completeness of \kl{bubble calculi} for \emph{propositional} logic.
Indeed quantifiers would make the algebraic semantics more involved when proving
soundness, and during our literature review we found very few \kl{proof systems}
for \kl{bi-intuitionistic} logic supporting them, at least none suitable for our
syntactic completeness proof. More generally, \kl{bi-intuitionistic} logic has
received less attention in the setting of \kl{FOL}, probably because it is
\emph{not} a conservative extension of \kl{intuitionistic} \kl{FOL}, but only of
\emph{constant-domain} \kl{intuitionistic} \kl{FOL} (see
\cite{crolard_subtractive_2001,aschieri_natural_2018}).
\end{remark}

\section{Non-determinism and iconicity}\labsec{non-determinism}

In all known \kl{sequent calculus} formulations of \kl{intuitionistic} logic, there are at
least two rules which are invariably \emph{non-\kl{invertible}}:
\begin{enumerate}
  \item a \kl{left introduction rule} for $\limp$ (there might be many ones, as in
  the calculus \kl{LJT} of \sidecite{dyckhoff_contraction-free_1992});
  \item the right introduction for either:
    \begin{itemize}
      \item $\lor$ when \kl{sequents} have at most or exactly one conclusion;
      \item $\limp$ when \kl{sequents} have multiple conclusions, e.g. in the
        multi-conclusion variant of \kl{LJT} in
        \cite{dyckhoff_contraction-free_1992}.
    \end{itemize}
\end{enumerate}
In \kl{BJ}, this means that click actions on blue $\hypo{\limp}$ and red
$\conc{\lor}$ need to be performed in a specific order to be able to complete
proofs.

In his thesis \cite{guenot_nested_2013}, Guenot introduced a specific kind of
\kl{nested sequent} system, where like in \kl{BJ} \kl{inference rules} can be expressed
as \kl{rewriting rules}. An interesting feature of these systems is that they satisfy
a \emph{decomposability} property: all \kl{introduction rules} for connectives are
\emph{\kl{invertible}}, and formulas can be completely decomposed with them until
atoms are reached, before applying other rules. Thus \kl{introduction rules} are
\emph{\kl{admissible}} in these systems, because every formula can be translated into
an equivalent pure \kl{nested sequent} with the same number of atoms\sidenote{As far
as we know, the admissibility of \kl{introduction rules} is not proved, let alone
mentioned in \cite{guenot_nested_2013}. This is our own observation which lacks
a proper formal proof, and is thus subject to caution.}. Non-determinism then
arises in the choice of atoms that are to be connected in axioms, as well as the
choice of sub-sequents to be duplicated for reuse.

In our graphical setting, this would translate to an interface where all click
actions are redundant. Although we already considered this possibility in
\refsec{dnd-completeness}, here it goes further by making even \emph{logical
connectives} superfluous, since all other rules work purely on the structure of
sequents. This means that all logical connectives could be replaced by
\kl{metaphorical} constructs like bubbles, which suggest \emph{physically} the
possible transformations on the \kl{proof state}.
% We call this property of a proof system \emph{iconicity}, following a
% terminology introduced by C. S. Peirce in his \emph{semiotics}
% \sidecite{noth_peircean_1999}, which he also applied to his diagrammatic proof
% system of \emph{existential graphs} \sidecite{10.7551/mitpress/3633.001.0001}.
Unfortunately, the systems in \cite{guenot_nested_2013} only handle \kl{classical}
logic, and the implicative fragment of \kl{intuitionistic} logic. Thus began our
quest for a \kl{bubble calculus} in the style of Guenot capturing full \kl{intuitionistic}
logic\sidenote{Other \kl{nested sequent} systems for full \kl{intuitionistic} logic exist
\cite{postniece_deep_2009,fitting-nested-2014}, but they are based on
tree-shaped proofs, and thus ignore the whole \emph{raison d'être} of our
concept of bubble.}.


\section{Conclusions and branching}\labsec{branching}

The first direction we followed was to relax the constraint that solutions must
be single-conclusion. Indeed as already noted in \refsec{sfl-backtracking}, a
notable property of \kl{sequent calculi} with multiple conclusions is that their
\kl{right introduction rule} for $\lor$ is \kl{invertible}.

The main difficulty lies in the way one should interpret a multi-conclusion
solution $S$ as a formula $\sint{S}$. If we just take the asymmetric
interpretation (\refdef{ainterp}) and group conclusions disjunctively instead of
conjunctively, we get
$$
\sint{\Gamma \piq{\cS} \Delta} =
\bigwedge \Gamma \limp \bigvee \Delta \land \bigwedge_{S \in \cS}{\sint{S}}
$$
But this interpretation breaks on the 0-ary case when $\Delta$ is empty: instead
of seeing $\Gamma \piq{\cS}$ as a node of the proof tree with hypotheses
$\Gamma$ and \kl{subgoals} $\cS$, it trivializes it to $\sint{\Gamma
\piq{\cS}} = \bigwedge \Gamma \limp \bot$, i.e. a \kl{goal} where one has to
find a contradiction in $\Gamma$; which is obviously not what we have in mind.

\begin{marginfigure}
  $$
  \R[\land R*]
    {\Gamma \seq A, \Delta}
    {\Gamma \seq B, \Delta}
    {\Gamma \seq A \land B, \Delta}
  $$
  \caption{Multi-conclusion \kl{right introduction rule} for conjunction}
  \labfig{multi-and-intro}
\end{marginfigure}

A key observation was that in the rules of multi-conclusion \kl{sequent calculi}, one
usually distributes the \kl(sequent){context} $\Delta$ of conclusions in all premisses: this
restores a perfect symmetry with respect to the \kl(sequent){context} of hypotheses $\Gamma$,
as illustrated by the {\rnm{\land R*}} rule (\reffig{multi-and-intro}). Then our
idea was that instead of implementing distribution/sharing of conclusions inside
\kl{inference rules}, we could do it implicitly in the interpretation of solutions.
This is already what happens in the asymmetric interpretation for hypotheses
(\refdef{ainterp}); indeed the \kl(sequent){context} $\Gamma$ is shared among \kl{subgoals},
because:
\begin{enumerate}
  \item it appears on the left of an implication $\limp$
  \item \kl{bubbles} are joined conjunctively, and
  \item implication distributes over conjunction thanks to the equivalence $A
  \limp B \land C \semequiv (A \limp B) \land (A \limp C)$.
\end{enumerate}
But what does it mean precisely to share conclusions among \kl{subgoals}? If we
consider the two following solutions:
\begin{equation}\label{eq:concdistr}
\underbrace{\bubble{\hypo{A}~~~\conc{B}}~~~\bubble{\hypo{C}~~~\conc{D}}~~~\conc{E}}_{S} \qquad\qquad
\underbrace{\bubble{\hypo{A}~~~\conc{B}~~~\conc{E}}~~~\bubble{\hypo{C}~~~\conc{D}~~~\conc{E}}}_{T}
\end{equation}
we would like to have $\sint{S} \semequiv \sint{T} \semequiv (A \limp B
\lor E) \land (C \limp D \lor E)$. Since disjunction distributes over
conjunction, a first naive try would give the following interpretation, where we
just replaced $\land$ by $\lor$ compared to the previous attempt:
$$
\sint{\Gamma \piq{\cS} \Delta} =
\bigwedge{\Gamma} \limp \bigwedge_{S \in \cS}{\sint{S}} \lor \bigvee \Delta
$$
But this immediately fails whenever $\cS = \emptyset$, because it
trivializes to $\bigwedge \Gamma \limp \top \lor \bigvee \Delta \semequiv \top$
instead of $\bigwedge \Gamma \limp \bigvee \Delta$. The only way we found around
this defect was to internalize \emph{syntactically} a distinction between two
kinds of solutions, by assigning them one of two \emph{statuses}\sidenote{In the
terminology of Martin-Löf, we could say that we now have two distinct forms of
\emph{judgment}.}:
\begin{itemize}
  \item \emph{closed} solutions $\Gamma \piq{\cS} \Delta$ correspond
  to branching nodes in the proof tree, or to closed leaves when $\cS =
  \emptyset$ (i.e. solved \kl{subgoals}). Thus it becomes sensical to have
  $\sint{\Gamma \piq{} \Delta} = \top$. In the asymmetric interpretation,
  closed solutions were encoded by solutions with no conclusions;
  \item \emph{open} solutions $\Gamma \seq \Delta$ correspond to open leaves
  in the proof tree (i.e. unsolved \kl{subgoals}). In the asymmetric interpretation,
  they were encoded by solutions with one conclusion.
\end{itemize}
Then we keep the last proposed interpretation for closed solutions, and
interpret open solutions like usual sequents:
$$\sint{\Gamma \seq \Delta} = \bigwedge{\Gamma} \limp \bigvee{\Delta}$$ To be
able to abstract from the particular kind of solution at hand, we reframe the
syntax of solutions with so-called \emph{branching} operators $\J$:
\begin{align*}
  S, T, U &\Coloneq \Gamma \J \Delta \\
  \J, \JB &\Coloneq {\seq} \mid \piq{\cS}
\end{align*}
Graphically, closed solutions with no \kl{bubbles} can be distinguished from open
solutions by painting their \emph{background} on the proof canvas in green, the
intent being to suggest that they have already been solved. A pathological
example is the distinction between the closed empty \kl{bubble}
$\bbubble{\phantom{a}}$ and the open empty \kl{bubble} $\bubble{\phantom{a}}$, who
are interpreted respectively by $\sint{\piq{\piq{}}} = \top$ and
$\sint{\piq{\seq}} = \bot$.

Now coming back to our target example,
% we must explicitly assign a status to each subsolution:
% $$
% \underbrace{\bsheet{\bubble{\hypo{A}~~~\conc{B}}~~~\bubble{\hypo{C}~~~\conc{D}}~~~\conc{E}}}_{S}
% \qquad\qquad
% \underbrace{\bsheet{\bubble{\hypo{A}~~~\conc{B}~~~\conc{E}}~~~\bubble{\hypo{C}~~~\conc{D}~~~\conc{E}}}}_{T}
% $$
% However
the interpretation still fails, because we associate two non-equivalent
formulas to $S$ and $T$. To show this, let us try to derive the equivalence
through some algebraic developments:
\begin{align}
  \sint{S} &= \top \limp ((A \limp B) \land (C \limp D)) \lor E \nonumber\\
              &\semequiv ((A \limp B) \land (C \limp D)) \lor E \nonumber\\
              &\semequiv ((A \limp B) \lor E) \land ((C \limp D) \lor E) \nonumber\\
              &\semequiv (A \limp B \lor E) \land (C \limp D \lor E) \labeq{grishin}\\
              &\semequiv ((A \limp B) \land (C \limp D)) \lor E \nonumber\\
  \sint{T} &= \top \limp ((A \limp B \lor E) \land (C \limp D \lor E)) \lor \bot \nonumber
\end{align}
Wait, we did manage to prove it! The trick resides in \refeq{grishin}, which
uses twice the equivalence $(A \limp B) \lor C \semequiv A \limp (B \lor C)$. It
turns out that this equivalence is true in \kl{classical} logic, but \emph{not} in
\kl{intuitionistic} logic. More precisely, it is the implication $G \defeq (A \limp
(B \lor C)) \limp ((A \limp B) \lor C)$ which is not provable
\kl{intuitionistically}, since it can easily be shown equivalent to the \kl{law of
excluded middle}\sidenote{This was already noticed in
\cite{clouston-annotation-free-2013}, with the linear version $(A \multimap (B
\parr C)) \multimap ((A \multimap B) \parr C)$ of $G$ called Grishin (a) and its
converse Grishin (b). More precisely, it is affirmed that while Grishin (b) is
valid in \kl{FILL}, the restriction of the classical multiplicative linear
logic \kl{MLL} to single-conclusion sequents, adding Grishin (a) makes
\kl{FILL} collapse to \sys{MLL}.}. Thus according to this interpretation, $S$
entails $T$ but $T$ does not entail $S$, which means that it is not able to
account for the \emph{factorization} of common conclusions in distinct \kl{subgoals}.

To remedy this situation, we opted for a different strategy: instead of finding
a logical formula capturing the distributive semantics of conclusions over
sub\kl{goals}, we hardcode the latter by defining the interpretation function on
closed solutions through \emph{non-structural} recursion. This gives the
following final definitions:

\begin{definition}[Mix operator]\labdef{mixop}
  The commutative \emph{mix operator} $\mix$ on solutions is defined by:
  \begin{align*}
    (\Gamma \J \Delta) \mix (\Gamma' \seq \Delta') &=
      \Gamma, \Gamma' \J \Delta, \Delta' \\
    (\Gamma \piq{\cS} \Delta) \mix (\Gamma' \piq{\mathcal{S'}} \Delta') &=
      \Gamma, \Gamma' \piq{\cS \sep \mathcal{S'}} \Delta, \Delta' \\
  \end{align*}
\end{definition}

\begin{definition}[Symmetric interpretation]\labdef{sinterp}
  The \emph{symmetric interpretation} of a solution is defined recursively by:
  \begin{align*}
    \sint{\Gamma \piq{\cS} \Delta} &=
      \bigwedge_{S \in \cS} \sint{S \mix (\Gamma \seq \Delta)} \\
    \sint{\Gamma \seq \Delta} &=
      \bigwedge \Gamma \limp \bigvee \Delta
  \end{align*}
\end{definition}

This is the right approach for interpreting solutions with multiple conclusions,
as will be demonstrated formally in \refsec{bubbles-soundness}.

\section{Coloring bubbles}\labsec{colors}

\subsection{Red bubbles}

\begin{marginfigure}
  $$
  \R[\mathsf{{\limp}{+}c}]
    {\Gamma, A \J B, \Delta}
    {\Gamma \J A \limp B, \Delta}
  $$
  \caption{\kl{Classical} multi-conclusion version of ${\limp}{+}$}
  \labfig{wrong-imp-pos}
\end{marginfigure}

With our new symmetric interpretation, we can start generalizing the rules of
\kl{BJ} to multiple conclusions. While for most rules one just has to replace
single-conclusion (resp. no-conclusion) solutions with open (resp. closed) ones
(more details will be given in the next section), the ${\limp}{+}$ rule stands
out as particularly problematic. Indeed if we content ourselves with the natural
generalization {\rsf{{\limp}{+}c}} of \reffig{wrong-imp-pos}, then we can
easily build a proof of the excluded middle like in \reffig{lk-tnd}, and thus
collapse to \kl{classical} logic. This fact is well-known in the literature on
multi-conclusion \kl{intuitionistic} \kl{sequent calculi}, and the solution is usually to
discard the \kl(sequent){context} of conclusions $\Delta$, as in the {\rnm{{\limp}R{*}i}} rule
of \reffig{multi-imp-intro}. But this would make our rule both non-local and
non-\kl{invertible}.

\begin{marginfigure}
  $$
  \begin{array}{rclr}
    \hypo{A}~~~\cbubble{\color{black}S} &\step{} &\cbubble{\hypo{A}~~~\color{black}S} &\mathsf{f}{-}{+}{\da} \vspace{1em}\\
    % \conc{A}~~~\cbubble{\color{black}S} &\step{} &\cbubble{\conc{A}~~~\color{black}S} &\mathsf{f}{+}{+}{\da} \\
  \end{array}
  $$
  \caption{$\mathbb{F}$-rule for red bubbles}
  \labfig{flow-red-bubbles}
\end{marginfigure}

A better solution comes from the \kl{nested sequent} systems of Fitting
\sidecite{fitting-nested-2014} and Clouston et al.
\sidecite{clouston-annotation-free-2013}, where \kl{sequents} can appear as
\emph{conclusions} of other sequents. In our chemical \kl{metaphor}, this corresponds
to having \emph{red bubbles}. Then the key idea is to allow hypotheses to flow
into \kl{sequents} that appear as conclusions\sidenote{This corresponds to the
{\rnm{Lift}} rule of \cite{fitting-nested-2014} and {\rnm{pl_1}} rule of
\cite{clouston-annotation-free-2013}.}, but \emph{not other conclusions}.
Graphically, this means that blue \kl{items} can enter red \kl{bubbles} (rule
{\rsf{f{-}{+}}} of \reffig{flow-red-bubbles}), but red \kl{items} cannot: this is
reminiscent of the electromagnetic phenomemon of \emph{repulsion} between
objects charged with the same polarity.

\begin{figure*}
  \setlength{\fboxsep}{2pt}
\setlength{\arraycolsep}{0pt}
\newcommand{\vsp}{\vspace{2em}}
$$
\begin{array}[t]{rcr@{\qquad}|@{\qquad}rcr@{\vsp}}
       &\text{\textbf{Grishin (b)}} &&
       &\text{\textbf{Grishin (a)}} & \\

       &\stkfig{1}{bubbles-grishin-b-0} &{\limp}{+}, {\lor}{+} &
       &\stkfig{1}{bubbles-grishin-a-0} &{\lor}{+}, {\limp}{+} \\

\steps &\stkfig{1}{bubbles-grishin-b-1} &{\mathsf{f{-}{+}{\da}}} &
\steps &\stkfig{1}{bubbles-grishin-a-1} &{\mathsf{f{-}{+}{\da}}} \\

\step  &\stkfig{1}{bubbles-grishin-b-2} &{\lor}{-}, {\limp}{-} &
\step  &\stkfig{1}{bubbles-grishin-a-2} &{\limp}{-}, {\lor}{-} \\

\steps &\stkfig{1}{bubbles-grishin-b-3} &\mathsf{f{-}{\da}}, \mathsf{f{+}{\da}} &
\steps &\stkfig{1}{bubbles-grishin-a-3} &\mathsf{f{-}{\da}}, \mathsf{f{+}{\da}} \\

\steps &\stkfig{1}{bubbles-grishin-b-4} & &
\steps &\stkfig{1}{bubbles-grishin-a-4} &
\end{array}
$$

  \caption{Proof attempts for Grishin (a) and Grishin (b)}
  \labfig{bubbles-grishin}
\end{figure*}

To illustrate why this works, let us consider how one can manipulate with red
\kl{bubbles} the \kl{classical} equivalence $ (A \limp B) \lor C \semequiv A \limp (B \lor
C)$, that we already stumbled upon in the previous section. The begginings of
the proofs for both directions of the equivalence are depicted parallely in
\reffig{bubbles-grishin}. Indeed both proofs have a very similar structure:
\begin{enumerate}
  \item the first step is to decompose the conclusion with the new version of
  the rules {\rnm{{\lor}{+}}} and {\rnm{{\limp}{+}}}. While the former simply
  splits disjunctions in two, the latter encapsulates the antecedant and
  consequent of implications in a red bubble: the \kl{goal} is to forbid the use of
  the antecedant to prove conclusions other than the consequent, as will become
  apparent later;
  \item then in both cases we want to apply the hypothesis $\hypo{A}$ in a
  \kl{forward} step, either with $\hypo{A \limp B}$ or $\hypo{A \limp (B \lor C)}$.
  To do so, we need to bring the two hypotheses together in the same solution.
  And since \kl{items} are trapped within bubbles, the only way to go is to move the
  blue $\hypo{A}$ inside the red \kl{bubble} with the {\rsf{f{-}{+}}} rule;
  \item this time we decompose the hypothesis with the new version of the rules
  {\rnm{{\lor}{-}}} and {\rnm{{\limp}{-}}}. They are basically a local variant
  of those of \kl{BJ}: we encapsulate both subformulas in separate bubbles, but
  without touching to the conclusions of the ambient solution;
  \item now that all formulas have been decomposed, it only remains to bring
  together dual atoms for annihilation, and pop all empty bubbles. In Grishin
  (b) this is easy, because all necessary movements (indicated by green arrows)
  are valid: they only cross gray \kl{bubbles} inward. In Grishin (a) this works for
  $\hypo{A}$ and $\conc{B}$, but not for $\conc{C}$ (orange dotted arrow): it
  would cross the red bubble, which is expressedly forbidden.
\end{enumerate}
Thus in order to prove Grishin (a) and recover \kl{classical} logic, it suffices
either to add the {\rsf{f{+}{+}}} rule allowing red \kl{items} to enter red \kl{bubbles}
(\reffig{flow-red-bubbles}), or to use the {\rsf{{\limp}{+}c}} rule which
avoids red \kl{bubbles} altogether. In the following we will settle for the first
option: we find it more elegant, because it explains the distinction between
\kl{intuitionistic} and \kl{classical} logic as a kind of \emph{physical law} independent
of logical connectives.

\subsection{Blue bubbles}

Now it is only natural to wonder: since \kl{bubbles} can be colored in red, or
charged \kl{positively}, would it also make sense to have \emph{blue} \kl{bubbles} charged
\emph{\kl{negatively}}? The answer is \emph{yes}, but we need to broaden our logical
view and consider more exotic beasts: the adequately named
\emph{\kl{dual-intuitionistic}} logic, and \emph{\kl{bi-intuitionistic} logic}.

\begin{marginfigure}
  $$
  \begin{array}{rclr}
    \conc{A}~~~\hbubble{\color{black}S} &\step{} &\hbubble{\conc{A}~~~\color{black}S} &\mathsf{f}{+}{-}{\da} \vspace{1em}\\
    % \hypo{A}~~~\hbubble{\color{black}S} &\step{} &\hbubble{\hypo{A}~~~\color{black}S} &\mathsf{f}{-}{-}{\da} \\
  \end{array}
  $$
  \caption{$\mathbb{F}$-rule for blue bubbles}
  \labfig{flow-blue-bubbles}
\end{marginfigure}

But for now let us stay at a purely syntactic level. The idea is very simple,
and can be summarized in two words: \emph{color swap}. Thus the law that ``blue
items can enter red bubbles, but red \kl{items} cannot'' becomes a new law that ``red
items can enter blue bubbles, but blue \kl{items} cannot'', which is enforced by
allowing only the use of the {\rsf{f{+}{-}}} rule in
\reffig{flow-blue-bubbles}. Well this is neat, but will not be of much use if
there is no way to spawn blue bubbles. Be it as it may: we can just craft a new
logical connective! Since red \kl{bubbles} are produced by the implication connective
$A \limp B$, we define a dual \emph{exclusion} connective $A \lsub B$ (read
``$A$ excludes $B$''\sidenote{We ask for the reader's leniency regarding our
choice of \kl{symbol} and terminology: in \kl{set theory} this would be total nonsense,
since $A \subset B$ would read ``$A$ is included in $B$''. Even worse, in the
boolean algebra induced by set operations, $A \subset B$ is interpreted as $A$
\emph{implies} $B$\ldots~But all the arrow \kl{symbols} were already taken, and we
want to emphasize the duality between exclusion and implication by mirroring the
\kl{symbol}, as it is traditionally done with conjunction $\land$ and disjunction
$\lor$.}), whose heating rules are those of $\limp$ with swapped colors
(\reffig{heating-exclusion}).

\begin{marginfigure}
  $$
  \begin{array}{rclr}
    \hypo{A \lsub B} &\step{} &\hbubble{\hypo{A}~~~\conc{B}} &{\lsub}{-}\vspace{1em}\\
    \conc{A \lsub B} &\step{} &{\bubble{\conc{A}}}~~~\bubble{\hypo{B}} &{\lsub}{+}
  \end{array}
  $$
  \caption{$\mathbb{H}$-rules for exclusion $\lsub$}
  \labfig{heating-exclusion}
\end{marginfigure}

Not very surprisingly, the exclusion connective has already been studied in the
literature on \kl{intuitionistic} logic, starting with the seminal paper of
Rauszer on \emph{Heyting-Brouwer logic}, i.e. \kl{intuitionistic} logic to which
we add exclusion \sidecite{Rauszer1974-RAUSAA}. In this paper, exclusion was
called \emph{pseudo-difference}, to evoke its close connection with
\kl{set-theoretical} difference. Indeed given two sets $A$ and $B$, one can
define the set $A \setminus B$ by comprehension as $\{x \mid x \in A \land x
\not\in B\}$, which is the set $A$ from which all elements of $B$ have been
\emph{excluded}. With an interpretation in boolean algebras, this corresponds to
the \kl{classical} connective defined by the truth table of $A \land \neg B$,
which is dual to the truth table of $\neg A \lor B$ defining material
implication.

While the first paper of Rauszer \cite{Rauszer1974-RAUSAA} belongs to the Polish
tradition of algebraic logic, she also explored in later works the
\kl{proof-theoretic} \sidecite{rauszer_formalization_1974} and \kl{model-theoretic}
\sidecite{rauszer_applications_1977} sides of the question. Many authors have
then deepened the \kl{proof theory} of exclusion, whether in isolation from
implication in \intro{dual-intuitionistic} logic
\sidecite{urbas_dual-intuitionistic_1996,gore_dual_2000}, or with both
connectives in \intro{bi-intuitionistic} logic as in Rauszer's original
work\sidenote{Crolard \cite{crolard_subtractive_2001} and Aschieri
\cite{aschieri_natural_2018} have also explored the computational counterpart of
exclusion through the \kl{Curry-Howard correspondence}, which is claimed by the first
author to be a typing operator for \emph{first-class coroutines}.}
\sidecite{postniece_proof_2010,pinto_relating_2011}. In particular, we are going
to rely in \refsec{bubbles-completeness} on the \kl{deep inference} calculus
developed by Postniece in her thesis \cite{postniece_proof_2010} to get
completeness and cut admissibility of our symmetric \kl{bubble calculus} introduced
in the next section.

\subsection{Polarized interpretation}

Let us now extend the formal definition of \kl{bubbles} so that they can be colored:

\begin{definition}[Bubble]\labdef{pol-bubble}
  A \emph{bubble} is a solution enclosed in a membrane, which can be either
  \intro{unpolarized} (\reintro{neutral}), charged \reintro{positively}, or
  charged \reintro{negatively}.
\end{definition}

\kl{Neutral} \kl{bubbles} are the usual ones depicted in gray, while \kl{positive} and \kl{negative}
\kl{bubbles} correspond respectively to red and blue bubbles. We also update the
definition of solutions, which can now be open or closed:

\begin{definition}[Solution]\labdef{pol-solution}
  
  A \emph{solution} is a multiset of ions and bubbles. Its \emph{status} is
  either \emph{closed} or \emph{open}, and open solutions cannot contain \kl{neutral}
  bubbles. Solutions $S$ can be represented textually with the following syntax:
  \begin{align*}
    S, T, U &\Coloneq \Gamma \J \Delta &
    \cS &\Coloneq S_1 \sep \ldots \sep S_n \\
    I, J, K &\Coloneq A \mid S &
    \Gamma, \Delta &\Coloneq I_1, \ldots, I_n \\
    \J, \JB &\Coloneq {\seq} \mid {\piq{\cS}} &&
  \end{align*}
\end{definition}

Note that in the textual syntax, \kl{bubbles} are identified with \emph{subsolutions}
(\refdef{subsolution}), and their \kl{polarity} is determined by their position
relative to branching operators; that is, for any solutions $S, T, U$ such that
$T \subsol U$, $S$ is either:
\begin{itemize}
  \item \emph{\kl{neutral}} if $T = \Gamma \piq{\cS} \Delta$ and $S \in
  \cS$;
  \item \emph{\kl{positive}} if $T = \Gamma \J \Delta$ and $S \in \Delta$;
  \item \emph{\kl{negative}} if $T = \Gamma \J \Delta$ and $S \in \Gamma$.
\end{itemize}

Then we need to split our symmetric interpretation accordingly, so that \kl{positive}
\kl{bubbles} are mapped to implications, and \kl{negative} \kl{bubbles} to
exclusions\sidenote{Here we took inspiration from the work of Clouston et al. on
\kl{nested sequents} for \kl{FILL} \cite{clouston-annotation-free-2013}.}:

\begin{definition}[Polarized symmetric interpretation]\labdef{pol-sinterp}
  The \emph{positive} and \emph{negative symmetric interpretations} of solutions
  $\psint{-}$ and $\nsint{-}$ are defined by mutual recursion as
  follows:
  \begin{align*}
    \psint{A} &= A &
    \nsint{A} &= A \\
    \psint{\Gamma \piq{\cS} \Delta} &=
      \bigwedge_{S \in \cS} \psint{S \mix \Gamma \seq \Delta} &
    \nsint{\Gamma \piq{\cS} \Delta} &=
      \bigvee_{S \in \cS} \nsint{S \mix \Gamma \seq \Delta} \\
    \psint{\Gamma \seq \Delta} &=
      \nsint{\Gamma} \limp \psint{\Delta} &
    \nsint{\Gamma \seq \Delta} &=
      \nsint{\Gamma} \lsub \psint{\Delta} \\
    \psint{\Gamma} &= \bigvee_{I \in \Gamma}{\psint{I}} &
    \nsint{\Gamma} &= \bigwedge_{I \in \Gamma}{\nsint{I}}
  \end{align*}
\end{definition}

One can easily check that the interpretation of a solution that has no \kl{negative}
(resp. \kl{positive}) subsolution will not contain any occurrence of the exclusion
(resp. implication) connective. This will be crucial later to represent proofs
of both \kl{intuitionistic}, \kl{dual-intuitionistic} and \kl{bi-intuitionistic} logic in the
same system.

\section{Designing for properties}\labsec{design-props}

With our new syntax and interpretation of solutions at hand, we can design a new
proof calculus including the rules previously discussed for manipulating
\kl{polarized} bubbles. The rich structure of solutions offers many possibilities in
the precise formulation of rules, depending on the properties we expect from the
calculus. We identified \emph{six} of these properties, whose consequences range
from aesthetic and theoretical considerations on paper, to concrete usability
matters in a graphical proof-building interface. Let us summarize them in order
of priorization relatively to the latter:
\begin{description}
  \item[Invertibility]
    A rule is \kl{invertible} when it could in principle be applied in the converse
    direction, while staying logically sound\sidenote{The \textit{``in
    principle''} part is important: more often than not, adding the converse of
    a rule only brings unnecessary complexity in proof search, especially in a
    user interface that aims for simplicity.}. In other words, it corresponds to
    a logical \emph{equivalence}: when all rules in a (bubble) calculus are
    \kl{invertible}, we get that $S \step{} T$ implies $\sint{S} \semequiv
    \sint{T}$. This entails in particular that a user can apply the rule
    without fear of turning a provable \kl{goal} into an unprovable
    one\sidenote{Assuming that the calculus is \emph{complete}
    (\refsec{bubbles-completeness}).}, eliminating an important source of
    non-determinism in proof search: the need for
    \emph{backtracking}\sidenote{See also \refsec{sfl-backtracking} for a
    discussion on this matter.}.
  \item[Decomposability]
    We already mentioned this property in \refsec{non-determinism} as one of the
    main motivations for this chapter: the ability to decompose all logical
    connectives ``for free'', and thus reason solely on solutions that comprise
    only \kl{bubbles} and atomic formulas. As far as we know, it has never been
    identified explicitly in the literature before, although it can loosely be
    seen as an extension of the decomposition procedures of existing \kl{deep
    inference} systems\sidenote{One could argue that more ``semantic'' approaches
    in \kl{proof theory} have achieved connective-free explanations of proofs, like
    strategies in game semantics or the combinatorial proofs of D. Hughes
    \cite{heijltjes_intuitionistic_2019}. But this is more of a side effect than
    a goal of these approaches, which intentionally abstract from the syntactic
    process of building proofs. A notable exception is the Girardian line of
    works starting from \emph{ludics} \cite{girard_locus_2001} and culminating
    in \emph{transcendental syntax} \cite{eng_exegesis_2023}, where both
    frameworks are founded upon the syntactic mechanisms of proof search
    (\kl{focusing} in \kl{sequent calculus}, and unification in the resolution algorithm
    of Robinson, respectively). Here the aim to rid proofs of connectives is
    greatly emphasized by Girard, but the focus is again on \emph{proofs} and
    not \emph{\kl{proof states}}. Also Girard embraces the full space of incomplete
    but also \emph{incorrect} proofs, while we still want a framework where
    proofs are correct by construction.}. One reason is that logical connectives
    are widely considered as \emph{primitive} in the tradition of \kl{mathematical
    logic}: they \emph{are} the objects of the reasoning activity, rather than a
    tool for representing and structuring arguments. Thus the idea of an
    alternative does not even occur. But even if it does, it is not clear that
    it would bring any interesting viewpoint on the problems usually studied in
    \kl{proof theory}. In our case, it was brought by a very concrete application:
    making formal proofs accessible to a broader audience, by replacing \kl{symbolic}
    and linguistic means of representation by \kl{iconic} and directly manipulable
    ones.
  \item[Factorizability]
    We say that a proof calculus is \emph{factorizable} when it makes it easier
    to avoid duplicating arguments in subproofs. In \refsec{bubbles-pba}, we
    already remarked that the ability to share hypotheses between \kl{subgoals} in
    \kl{BJ} enables the factorization of \emph{\kl{forward}} reasoning steps at any
    stage of the proof construction. With our new symmetric interpretation of
    multi-conclusion solutions, we will now be able to factorize \emph{\kl{backward}}
    reasoning steps as well, which was in fact the main motivation behind
    Example \ref{eq:concdistr} in \refsec{branching}.
  \item[Locality]
    There does not seem to be a general consensus on what it means precisely for
    an \kl{inference rule} to be \emph{local}. This terminology has been
    employed by various authors in \kl{proof theory}, in ways that are often
    hard to compare. For instance in \sidecite{negri_structural_2001}, rules are
    said to be local because the \kl(sequent){contexts} of hypotheses involved in a rule are
    located in the \kl{sequents} of that rule, by opposition to \kl{natural
    deduction} rules in their labelled presentation where hypotheses are located
    in arbitrary distant leaves of the derivation. In the setting of \kl{deep
    inference}, local rules are those that can be applied without ``inspection
    of expressions of arbitrary size''\sidenote{Definition 2.1.1 in
    \cite{tubella:hal-02390267}. The same definition is used in
    \cite{tiu_local_2006}.}. Finally in his transcendental syntax, Girard evokes
    a related but more elusive notion, concerned with the \emph{genericity} of
    logical objects involved in a rule\sidenote{See the section \emph{Globality
    and locality in logical systems} in \cite[Chapter 6]{eng_exegesis_2023}.}.
    
    Our conception of locality is related to all the previous ones, although it
    is guided by the idea of \kl{direct manipulation} of logical entities by humans,
    rather than purely \kl{proof-theoretical} considerations. For instance,
    \kl{BJ} has some locality in the \kl{deep inference} sense because all rules
    are applicable in arbitrary \kl{contexts}; but we relax the \emph{atomicity}
    constraint that reduces $\mathbb{I}$-rules and $\mathbb{R}$-rules to their
    atomic version, because it would be unnecessarily restrictive for the
    purpose of building proofs manually. Still, we want to avoid as much as
    possible referring to generic objects that are not directly related to the
    manipulated data, in the spirit of Girard's locality. A typical example is
    the \kl{elimination rule} \kl{\lor e} for disjunction in \kl{natural
    deduction}, corresponding to the {\rsf{{\lor}{-}}} rule of \kl{BJ} that
    involves an arbitrary conclusion $\Delta$. The benefits of locality from a
    \kl{UX} point of view have already been discussed at the end of
    \refsec{bubbles-pba}.
  \item[Linearity] 
    We consider an \kl{inference rule} to be \emph{linear} when it preserves the
    number of atomic formulas in solutions. This is a strong requirement, which
    for instance excludes the identity rules of \kl{BJ} since they can insert
    or remove (even numbers of) atoms. Thus we cannot achieve full linearity in
    that sense, but it is still interesting to maximize it. The first reason is
    \emph{methodological}: by the words of its creator A. Guglielmi,
    \textit{``[...] \kl{deep inference} is obtained by applying some of the main
    concepts behind linear logic to the formalisms, i.e., to the rules by which
    \kl{proof systems} are designed.''} \sidecite{deep_inference}. The second reason
    is \emph{computational}: it can enable a measure on solutions that is
    strictly decreasing with the application of rules, avoiding infinite loops
    during proof search as in the calculus \kl{LJT} of R. Dyckhoff
    \cite{dyckhoff_contraction-free_1992}. The third reason is ergonomical: as
    already remarked by the authors of the \kl{Proof-by-Pointing}
    paradigm\sidenote{Section 4.1 of \cite{PbP}.}, rules that systematically
    duplicate formulas can quickly overload the \kl{goal} with useless copies, making
    it harder to read and navigate.
  \item[Symmetry] 
    In \kl{classical} logic, both \kl{sequent calculi} like \kl{LK} and \kl{deep
    inference} systems like \kl{CoS} are known for their very rich
    \emph{symmetries}. In fact, one of our ambitions with \kl{bubbles} was to bring
    back the symmetry of \kl{classical} logic in a constructive setting, without
    resorting to linear logic. This chapter stems in great part from our lack of
    satisfaction with the asymmetry at work in the \kl{BJ} calculus, which
    looked quite unnatural. Of course we will not be able to completely
    eliminate it, but it will be distilled into the flow rules governing the
    \emph{porosity} of \kl{bubbles} that were hinted at in \refsec{colors}, rather
    than through the arbitrary restriction of \kl{sequents} to one
    conclusion\sidenote{Whether it is enforced in the syntax of \kl{sequents} themselves, or through restriction on rules that manipulate conclusions like
    \kl{contraction} or the \kl{right introduction rule} for $\limp$.}. Our
    treatment of \kl{dual-intuitionistic} and \kl{bi-intuitionistic} logic
    through blue \kl{bubbles} is also motivated by this quest for symmetry. It should
    be noted that although we use naming conventions for rules that resemble
    those of \kl{CoS} (e.g. with the identity rules), we do not aim for a
    perfect symmetry where one can get a complete calculus by simply taking the
    dual of each rule.
    % \sidenote{In fact it is not even clear what would be the dual of a
    % solution, but the {\rsf{i{\da}}} and {\rsf{i{\ua}}} rules
    % suggest that open and closed solutions may be dual to eachother.}
    Thus we will content ourselves with the hypothesis/conclusion symmetry
    coming from \kl{sequent calculus}. Interestingly, the calculus \kl{ISgq} of Tiu
    for \kl{intuitionistic} \kl{predicate logic} does the opposite, by having a perfect
    dual system \kl{cISgq} but no symmetries among its \kl{switch rules} (the
    equivalent of our flow rules) \cite{tiu_local_2006}.
\end{description}

In the next section we present a core calculus called \kl{system~B} that
maximizes \emph{symmetry}, \emph{linearity} and \emph{locality}. In our opinion
this makes for a good \kl{proof-theoretical} foundation, around which variant
calculi with different tradeoffs can be designed.

% In \refsec{invertible-calculus} we introduce such a variant that focuses on
% \emph{invertibility} at the cost of \emph{linearity}, and in
% \refsec{decomposable-calculus} on a refinement of the latter that achieves
% \emph{decomposability} and \emph{factorizability}, losing some of its
% \emph{locality} and \emph{symmetry} along the way.

\section{Symmetric calculus}\labsec{symmetric-calculus}

% \begin{figure*}
%   \begin{framed}
\renewcommand{\arraystretch}{1.25}
\begin{mathpar}
\begin{array}{r@{\quad}c@{\quad}lr}
  \multicolumn{4}{c}{\identity} \\[2em]

   \hypo{A}~~~\conc{A}
  &\step{}
  &
  &\mathsf{i}{\da} \\

   \conc{\Delta}
  &\step{}
  &\bubble{\conc{A}}~~~\bubble{\hypo{A}~~~\conc{\Delta}}
  &\mathsf{i}{\ua} \\
\end{array}
\and
\begin{array}{r@{\quad}c@{\quad}lr}
  \multicolumn{4}{c}{\resource} \\[2em]

    \hypo{A}
  &\step{}
  &
  &\mathsf{w} \\

    \hypo{A}
  &\step{}
  &\hypo{A~~~A}
  &\mathsf{c} \\
\end{array}
\vspace{2em}\\
\begin{array}{r@{\quad}c@{\quad}lr}
  \multicolumn{4}{c}{\flow} \\[2em]

    \hypo{A}~~~\bubble{\color{black}S}
  &\step{}
  &\bubble{\hypo{A}~~~S}
  &\mathsf{f{-}} \\
\end{array}
\and
\begin{array}{r@{\quad}c@{\quad}lr}
  \multicolumn{4}{c}{\membrane} \\[2em]

    \bubble{\phantom{S}}
  &\step{}
  &
  &\mathsf{p} \\
\end{array}
\vspace{2em}
\\
\begin{array}{r@{\quad}c@{\quad}lr@{\qquad\qquad}r@{\quad}c@{\quad}lr}
  \multicolumn{8}{c}{\heating} \\[2em]

    \hypo{\top}
  &\step{}
  &
  &\mathsf{\top{-}}

  &\conc{\top}
  &\step{}
  &
  &\mathsf{\top{+}} \\

    \hypo{\bot}~~~\conc{\Delta}
  &\step{}
  &
  &\mathsf{\bot{-}}

  &&&&\\

    \hypo{A \land B}
  &\step{}
  &\hypo{A}~~~\hypo{B}
  &\mathsf{\land{-}}

  &\conc{A \land B}
  &\step{}
  &\bubble{\conc{A}}~~~\bubble{\conc{B}}
  &\mathsf{\land{+}} \\

    \multirow{2}{*}{$\hypo{A \lor B}~~~\conc{\Delta}$}
  &\multirow{2}{*}{$\step{}$}
  &\multirow{2}{*}{$\bubble{\hypo{A}~~~\conc{\Delta}}~~~\bubble{\hypo{B}~~~\conc{\Delta}}$}
  &\multirow{2}{*}{$\mathsf{\lor{-}}$}

  &\conc{A \lor B}
  &\step{}
  &\conc{A}
  &\mathsf{\lor{+}_1} \\

  &&&

  &\conc{A \lor B}
  &\step{}
  &\conc{B}
  &\mathsf{\lor{+}_2} \\

    \hypo{A \limp B}~~~\conc{\Delta}
  &\step{}
  &\bubble{\conc{A}}~~~\bubble{\hypo{B}~~~\conc{\Delta}}
  &\mathsf{{\limp}{-}}

  &\conc{A \limp B}
  &\step{}
  &\hypo{A}~~~\conc{B}
  &\mathsf{{\limp}{+}} \\

    \hypo{\forall x. A}
  &\step{}
  &\hypo{\subst{A}{t}{x}}
  &\mathsf{\forall{-}}

  &\conc{\forall x. A}
  &\step{}
  &\conc{\subst{A}{y}{x}}
  &\mathsf{\forall{+}} \\

    \hypo{\exists x. A}
  &\step{}
  &\hypo{\subst{A}{y}{x}}
  &\mathsf{\exists{-}}

  &\conc{\exists x. A}
  &\step{}
  &\conc{\subst{A}{t}{x}}
  &\mathsf{\exists{+}} \\
\end{array}
\vspace{2em}
\end{mathpar}
In the {\rsf{i{\ua}}}, {\rnm{\bot{-}}}, {\rnm{\lor{-}}} and {\rnm{{\limp}{-}}} rules, $\Delta$
is either empty, or a singleton of one \kl{positive} ion.\\
In the {\rnm{\forall{+}}} and {\rnm{\exists{-}}} rules, $y$ is fresh.
\end{framed}

% \end{figure*}

\begin{figure*}
  \fontsize{10}{10.5}\selectfont
\begin{framed}
\renewcommand{\arraystretch}{2}
\begin{mathpar}
\begin{array}{r@{\quad}l}
\multicolumn{2}{c}{\identity} \\[1em]

\R[\mathsf{i{\da}}]
    {\Gamma \piq{} \Delta}
    {\Gamma, A \seq A, \Delta}
&
\R[\mathsf{i{\ua}}]
    {\Gamma \piq{\seq A \sep A \seq} \Delta}
    {\Gamma \seq \Delta}
\end{array}
\and
\begin{array}{c@{\quad}c}
\multicolumn{2}{c}{\resource} \\[1em]

\R[\mathsf{w{-}}]
    {\Gamma \J \Delta}
    {\Gamma, I \J \Delta}
&
\R[\mathsf{w{+}}]
    {\Gamma \J \Delta}
    {\Gamma \J I, \Delta}
\\
\R[\mathsf{c{-}}]
    {\Gamma, I, I \J \Delta}
    {\Gamma, I \J \Delta}
&
\R[\mathsf{c{+}}]
    {\Gamma \J I, I, \Delta}
    {\Gamma \J I, \Delta}
\end{array}
\\
\begin{array}{c@{\quad}c}
\multicolumn{2}{c}{\flow} \\[1em]

% \R[\mathsf{s{-}}]
%     {\piq{S \seq} \mix \Gamma \J \Delta}
%     {\Gamma, (\piq{S}) \J \Delta}
% &
% \R[\mathsf{s{+}}]
%     {\Gamma \J \Delta \mix \piq{\seq S}}
%     {\Gamma \J (\piq{S}), \Delta}
% \\

\multicolumn{2}{c}{
\R[\mathsf{f{\ua}}]
    {\Gamma \piq{\mathcal{S} \sep \Gamma' \piq{\mathcal{S'}} \Delta' \sep S} \Delta}
    {\Gamma \piq{\mathcal{S} \sep \Gamma' \piq{\mathcal{S'} \sep S} \Delta'} \Delta}
} \\

% \R[\mathsf{n{-}{\ua}}]
%     {(\Gamma, (\piq{\cS}) \J \Delta) \mix \piq{S}}
%     {\Gamma, (\piq{\cS \sep S}) \J \Delta}
% &
% \R[\mathsf{n{+}{\ua}}]
%     {\piq{S} \mix (\Gamma \J (\piq{\cS}), \Delta)}
%     {\Gamma  \J (\piq{\cS \sep S}), \Delta}
% \\
\R[\mathsf{f{-}{\da}}]
    {\Gamma \piq{\Gamma', I \JB \Delta' \sep \cS} \Delta}
    {\Gamma, I \piq{\Gamma' \JB \Delta' \sep \cS} \Delta}
&
\R[\mathsf{f{+}{\da}}]
    {\Gamma \piq{\cS \sep \Gamma' \JB I, \Delta'} \Delta}
    {\Gamma \piq{\cS \sep \Gamma' \JB \Delta'} I, \Delta}
\\
\R[\mathsf{f{-}{+}}{\da}]
    {\Gamma \J (\Gamma', I \JB \Delta'), \Delta}
    {\Gamma, I \J (\Gamma' \JB \Delta'), \Delta}
&
\R[\mathsf{f{+}{-}}{\da}]
    {\Gamma, (\Gamma' \JB I, \Delta') \J \Delta}
    {\Gamma, (\Gamma' \JB \Delta') \J I, \Delta}
\\
\R[\mathsf{f{-}{-}{\ua}}]
    {\Gamma, I, (\Gamma' \JB \Delta') \J \Delta}
    {\Gamma, (\Gamma', I \JB \Delta') \J \Delta}
&
\R[\mathsf{f{+}{+}{\ua}}]
    {\Gamma \J (\Gamma' \JB \Delta'), I, \Delta}
    {\Gamma \J (\Gamma' \JB I, \Delta'), \Delta}
\\
\R[\mathsf{f{-}{+}}{\ua}]
    {\Gamma, I \J (\Gamma' \JB \Delta'), \Delta}
    {\Gamma \J (\Gamma', I \JB \Delta'), \Delta}
&
\R[\mathsf{f{+}{-}}{\ua}]
    {\Gamma, (\Gamma' \JB \Delta') \J I, \Delta}
    {\Gamma, (\Gamma' \JB I, \Delta') \J \Delta}
\\
\R[\mathsf{f{-}{-}{\da}}]
    {\Gamma, (\Gamma', I \JB \Delta') \J \Delta}
    {\Gamma, I, (\Gamma' \JB \Delta') \J \Delta}
&
\R[\mathsf{f{+}{+}{\da}}]
    {\Gamma \J (\Gamma' \JB I, \Delta'), \Delta}
    {\Gamma \J (\Gamma' \JB \Delta'), I, \Delta}
\end{array}
\and
\begin{array}{cc}
\multicolumn{2}{c}{\membrane} \\[1em]

\multicolumn{2}{c}{
\R[\mathsf{p}]
    {\Gamma \piq{\cS} \Delta}
    {\Gamma \piq{\cS \sep \piq{}} \Delta}
} \\
\R[\mathsf{p{-}}]
    {\Gamma \piq{} \Delta}
    {\Gamma, (\piq{}) \seq \Delta}
&
\R[\mathsf{p{+}}]
    {\Gamma \piq{} \Delta}
    {\Gamma \seq (\piq{}), \Delta}
\\

\multicolumn{2}{c}{
\R[\mathsf{a}]
    {\Gamma \piq{S} \Delta}
    {\Gamma \piq{\piq{S}} \Delta}
} \\
\R[\mathsf{a{-}}]
    {\Gamma, S \J \Delta}
    {\Gamma, (\piq{S}) \J \Delta}
&
\R[\mathsf{a{+}}]
    {\Gamma \J S, \Delta}
    {\Gamma \J (\piq{S}), \Delta}
\\
\end{array}
\\
\begin{array}{c@{\quad}c}
\multicolumn{2}{c}{\heating} \\[1em]

\R[\top{-}]
    {\Gamma \J \Delta}
    {\Gamma, \top \J \Delta}
&
\R[\top{+}]
    {\Gamma  \piq{} \Delta}
    {\Gamma \seq \top, \Delta}
\\
\R[\bot{-}]
    {\Gamma \piq{} \Delta}
    {\Gamma, \bot \seq \Delta}
&
\R[\bot{+}]
    {\Gamma \J \Delta}
    {\Gamma \J \bot, \Delta}
\\
\R[\land{-}]
    {\Gamma, A, B \J \Delta}
    {\Gamma, A \land B \J \Delta}
&
\R[\land{+}]
    {\Gamma \piq{\seq A \sep \seq B {}} \Delta}
    {\Gamma \seq A \land B, \Delta}
\\
\R[\lor{-}]
    {\Gamma \piq{A \seq \sep B\seq} \Delta}
    {\Gamma, A \lor B \seq \Delta}
&
\R[\lor{+}]
    {\Gamma \J A, B, \Delta}
    {\Gamma \J A \lor B, \Delta}
\\
\R[{\limp}{-}]
    {\Gamma \piq{\seq A \sep B\seq} \Delta}
    {\Gamma, A \limp B \seq \Delta}
&
\R[{\limp}{+}]
    {\Gamma \J (A \seq B), \Delta}
    {\Gamma \J A \limp B, \Delta}
\\
\R[{\lsub}{-}]
    {\Gamma, (A \seq B) \J \Delta}
    {\Gamma, A \lsub B \J \Delta}
&
\R[{\lsub}{+}]
    {\Gamma \piq{\seq A \sep B\seq} \Delta}
    {\Gamma \seq A \lsub B, \Delta}
\\
\R[\forall{-}]
    {\Gamma, \subst{A}{t}{x} \J \Delta}
    {\Gamma, \forall x. A \J \Delta}
&
\R[\forall{+}]
    {\Gamma \J A, \Delta}
    {\Gamma \J \forall x. A, \Delta}
\\
\R[\exists{-}]
    {\Gamma, A \J \Delta}
    {\Gamma, \exists x. A \J \Delta}
&
\R[\exists{+}]
    {\Gamma \J \subst{A}{t}{x}, \Delta}
    {\Gamma \J \exists x. A, \Delta}
\end{array}
\end{mathpar}

In the {\rnm{\forall{+}}} and {\rnm{\exists{-}}} rules, $x$ is not free in
$\Gamma$, $\Delta$ and $\J$.
\end{framed}

  \caption{Sequent-style presentation of \kl{system~B}}
  \labfig{sequent-B}
\end{figure*}

\begin{figure*}
  \fontsize{9.5}{10}\selectfont
\begin{framed}
  % \renewcommand{\arraystretch}{2}
  \begin{mathpar}
  \begin{array}{r@{\quad}c@{\quad}ll}
    \multicolumn{4}{c}{\kl{\identity}} \\[2em]
  
     \hypo{A}~~~\conc{A}
    &\step{}
    &\bsheet{\phantom{\hypo{A}~~~\conc{A}}}
    &\kl{i{\da}} \\
  
     
     \phantom{\bubble{\conc{A}}~~~\bubble{\hypo{A}}}
    &\step{}
    &\bsheet{\bubble{\conc{A}}~~~\bubble{\hypo{A}}}
    &\kl{i{\ua}} \\
  \end{array}
  \and
  \begin{array}{r@{\quad}c@{\quad}ll@{\qquad\qquad}r@{\quad}c@{\quad}ll}
    \multicolumn{8}{c}{\kl{\resource}} \\[2em]
  
     \gAsheet{\hypo{I}}
    &\step{}
    &\gAsheet{\phantom{\hypo{I}}}
    &\kl{w{-}} &

     \gAsheet{\conc{I}}
    &\step{}
    &\gAsheet{\phantom{\hypo{I}}}
    &\kl{w{+}} \\
  
     \gAsheet{\hypo{I}}
    &\step{}
    &\gAsheet{\hypo{I~~~I}}
    &\kl{c{-}} &

     \gAsheet{\conc{I}}
    &\step{}
    &\gAsheet{\conc{I~~~I}}
    &\kl{c{+}} \\
  \end{array}
  \vspace{2em}\\
  \begin{array}{r@{\quad}c@{\quad}ll@{\qquad\qquad}r@{\quad}c@{\quad}ll}
    \multicolumn{8}{c}{\kl{\flow}} \\[2em]

    %  \gAsheet{\hbbubble{\gBbubble{\color{black}S}}}
    % &\step{}
    % &\bsheet{\gAbubble{\hgBbubble{\color{black}S}}}
    % &\kl{s{-}} &

    %  \gAsheet{\cbbubble{\gBbubble{\color{black}S}}}
    % &\step{}
    % &\bsheet{\gAbubble{\cgBbubble{\color{black}S}}}
    % &\kl{s{+}} \\

    \multicolumn{8}{c}{
      \begin{array}{r@{\quad}c@{\quad}ll}
         \bsheet{\bbubble{\gBbubble{\color{black}T}~~~\color{black}S}}
        &\step{}
        &\bsheet{\gBbubble{\color{black}T}~~~\bbubble{\color{black}S}}
        &\kl{f{\ua}}
      \end{array}
    } \\[3em]

    %  \gAsheet{\hbbubble{\gBbubble{\color{black}S}~~~\color{black}T}}
    % &\step{}
    % &\bsheet{\gBbubble{\color{black}S}~~~\hbbubble{\color{black}T}}
    % &\kl{n{-}{\ua}} &

    %  \gAsheet{\cbbubble{\gBbubble{\color{black}S}~~~\color{black}T}}
    % &\step{}
    % &\bsheet{\gBbubble{\color{black}S}~~~\cbbubble{\color{black}T}}
    % &\kl{n{+}{\ua}} \\
  
     \bsheet{\hypo{I}~~~\gBbubble{\color{black}S}}
    &\step{}
    &\bsheet{\gBbubble{\hypo{I}~~~\color{black}S}}
    &\kl{f{-}{\da}} &

     \bsheet{\conc{I}~~~\gBbubble{\color{black}S}}
    &\step{}
    &\bsheet{\gBbubble{\conc{I}~~~\color{black}S}}
    &\kl{f{+}{\da}} \\

     \gAsheet{\hypo{I}~~~\cgBbubble{\color{black}S}}
    &\step{}
    &\gAsheet{\cgBbubble{\hypo{I}~~~\color{black}S}}
    &\kl{f{-}{+}{\da}} &

     \gAsheet{\conc{I}~~~\hgBbubble{\color{black}S}}
    &\step{}
    &\gAsheet{\hgBbubble{\conc{I}~~~\color{black}S}}
    &\kl{f{+}{-}{\da}} \\

     \gAsheet{\hgBbubble{\hypo{I}~~~\color{black}S}}
    &\step{}
    &\gAsheet{\hypo{I}~~~\hgBbubble{\color{black}S}}
    &\kl{f{-}{-}{\ua}} &

     \gAsheet{\cgBbubble{\conc{I}~~~\color{black}S}}
    &\step{}
    &\gAsheet{\conc{I}~~~\cgBbubble{\color{black}S}}
    &\kl{f{+}{+}{\ua}} \\

     \gAsheet{\cgBbubble{\hypo{I}~~~\color{black}S}}
    &\step{}
    &\gAsheet{\hypo{I}~~~\cgBbubble{\color{black}S}}
    &\kl{f{-}{+}{\ua}} &

     \gAsheet{\hgBbubble{\conc{I}~~~\color{black}S}}
    &\step{}
    &\gAsheet{\conc{I}~~~\hgBbubble{\color{black}S}}
    &\kl{f{+}{-}{\ua}} \\

     \gAsheet{\hypo{I}~~~\hgBbubble{\color{black}S}}
    &\step{}
    &\gAsheet{\hgBbubble{\hypo{I}~~~\color{black}S}}
    &\kl{f{-}{-}{\da}} &

     \gAsheet{\conc{I}~~~\cgBbubble{\color{black}S}}
    &\step{}
    &\gAsheet{\cgBbubble{\conc{I}~~~\color{black}S}}
    &\kl{f{+}{+}{\da}} \\
  \end{array}
  \and
  \begin{array}{r@{\quad}c@{\quad}ll}
    \multicolumn{4}{c}{\kl{\membrane}} \\[2em]

     \bsheet{\bbubble{\phantom{S}}}
    &\step{}
    &\bsheet{\phantom{\bbubble{\phantom{S}}}}
    &\kl{p} \\

     \hbbubble{\phantom{S}}
    &\step{}
    &\bsheet{\phantom{\bbubble{\phantom{S}}}}
    &\kl{p{-}} \\

     \cbbubble{\phantom{S}}
    &\step{}
    &\bsheet{\phantom{\bbubble{\phantom{S}}}}
    &\kl{p{+}} \\

     \bsheet{\bbubble{\gBbubble{\color{black}S}}}
    &\step{}
    &\bsheet{\gBbubble{\color{black}S}}
    &\kl{a} \\

     \gAsheet{\hbbubble{\gBbubble{\color{black}S}}}
    &\step{}
    &\gAsheet{\hgBbubble{\color{black}S}}
    &\kl{a{-}} \\

     \gAsheet{\cbbubble{\gBbubble{\color{black}S}}}
    &\step{}
    &\gAsheet{\cgBbubble{\color{black}S}}
    &\kl{a{+}} \\
  \end{array}
  \vspace{2em}\\
  \begin{array}{r@{\quad}c@{\quad}ll@{\qquad\qquad}r@{\quad}c@{\quad}ll}
    \multicolumn{8}{c}{\kl{\heating}} \\[2em]
  
     \gAsheet{\hypo{\top}}
    &\step{}
    &\gAsheet{\phantom{\hypo{\top}}}
    &\kl{\top{-}}
  
    &\conc{\top}
    &\step{}
    &\bsheet{\phantom{\top}}
    &\kl{\top{+}} \\
  
     \hypo{\bot}
    &\step{}
    &\bsheet{\phantom{\bot}}
    &\kl{\bot{-}}

    &\gAsheet{\conc{\bot}}
    &\step{}
    &\gAsheet{\phantom{\conc{\bot}}}
    &\kl{\bot{+}} \\
  
     \gAsheet{\hypo{A \land B}}
    &\step{}
    &\gAsheet{\hypo{A}~~~\hypo{B}}
    &\kl{\land{-}}
  
    &\conc{A \land B}
    &\step{}
    &\bsheet{\bubble{\conc{A}}~~~\bubble{\conc{B}}}
    &\kl{\land{+}} \\
  
     \hypo{A \lor B}
    &\step{}
    &\bsheet{\bubble{\hypo{A}}~~~\bubble{\hypo{B}}}
    &\kl{\lor{-}}

    &\gAsheet{\conc{A \lor B}}
    &\step{}
    &\gAsheet{\conc{A}~~~\conc{B}}
    &\kl{\lor{+}} \\
  
     \hypo{A \limp B}
    &\step{}
    &\bsheet{\bubble{\conc{A}}~~~\bubble{\hypo{B}}}
    &\kl{{\limp}{-}}
  
    &\gAsheet{\conc{A \limp B}}
    &\step{}
    &\gAsheet{\cbubble{\hypo{A}~~~\conc{B}}}
    &\kl{{\limp}{+}} \\
  
     \gAsheet{\hypo{A \lsub B}}
    &\step{}
    &\gAsheet{\hbubble{\hypo{A}~~~\conc{B}}}
    &\kl{{\lsub}{-}}

    &\conc{A \lsub B}
    &\step{}
    &\bsheet{\bubble{\conc{A}}~~~\bubble{\hypo{B}}}
    &\kl{{\lsub}{+}} \\
  
     \gAsheet{\hypo{\forall x. A}}
    &\step{}
    &\gAsheet{\hypo{\subst{A}{t}{x}}}
    &\kl{\forall{-}}
  
    &\gAsheet{\conc{\forall x. A}}
    &\step{}
    &\gAsheet{\conc{\subst{A}{y}{x}}}
    &\kl{\forall{+}} \\
  
     \gAsheet{\hypo{\exists x. A}}
    &\step{}
    &\gAsheet{\hypo{\subst{A}{y}{x}}}
    &\kl{\exists{-}}
  
    &\gAsheet{\conc{\exists x. A}}
    &\step{}
    &\gAsheet{\conc{\subst{A}{t}{x}}}
    &\kl{\exists{+}} \\
  \end{array}
  \vspace{2em}
  \end{mathpar}
  In the {\rnm{\forall{+}}} and {\rnm{\exists{-}}} rules, $y$ is fresh.
  \end{framed}
  
  \caption{Graphical presentation of \kl{system~B}}
  \labfig{graphical-B}
\end{figure*}

As for the asymmetric \kl{bubble calculus} \kl{BJ}, the rules of our full symmetric
\kl{bubble calculus} system $\sysB$ enjoy both a sequent-style and a graphical
presentation, given respectively in \reffig{sequent-B} and \reffig{graphical-B}.
The presence of closed and open solutions complicates quite a bit the graphical
representation of rules, thus some explanations are in order:
\begin{description}
  \item[Closed solutions] In \refsec{branching}, we mentioned that closed
solutions with no \kl{neutral} \kl{bubbles} can be distinguished visually from open
solutions by painting their background in a different color; we chose a light
green, to suggest that they denote \emph{solved} \kl{subgoals}. In
\reffig{graphical-B}, we emphasize systematically the distinction by extending
this convention to all closed solutions.
  \item[Generic statuses] As can be seen in \reffig{sequent-B}, many rules of
system $\sysB$ are \emph{generic} over branching operators $\J, \JB$, which
determine whether a solution is closed or open, i.e. its \emph{status}. The
challenge is thus to find an \kl{iconic} counterpart to the \kl{symbols} $\J, \JB$, that
fulfills the same function of \emph{meta-variable} ranging over solution
statuses. Since we already use the background color to represent the status of
concrete solutions, we chose to do the same with abstract ones: each new color
other than green will stand for the status of the solution associated to the
given location of the canvas. For instance in the \rsf{f{-}{+}{\da}}
rule, the status of the ambient solution where the rule is applied is denoted by
a light yellow background, while the status of the solution $S$ enclosed in a
red \kl{bubble} is denoted by the light pink background.
  \item[Status changes] Last but not least, many rules like \rsf{i{\da}} change
the status of the ambient solution from open to closed: graphically, this means
that the background must become green \emph{everywhere}, not only in the portion
of the canvas depicted by the rule. At first it might appear as breaking
locality, but it should rather be understood as the result of a perfectly local
and continuous process: one can imagine a literal \emph{drop} of green paint
that soaks a growing portion of the canvas, until it reaches an enclosing \kl{bubble}
--- for the sake of \kl{metaphor}, let us say a cut in the papersheet --- that
stops its progression\sidenote{We will come back to this \emph{cuts in a sheet}
\kl{metaphor}, first introduced by C. S. Peirce, in \refch{eg}. When closing the
top-level solution --- Peirce called it the \emph{\kl{sheet of assertion}}, the
drop expansion process becomes \emph{infinite}. I find it to be a beautiful
allegory of the \emph{unreachability} of global, unconditional truth: it is only
by being confined to a finite, well-delimited space, that we can affirm
unequivocally our certainty.
% What lays behind the fences (possibly some hidden assumptions!), out of our
% grasp will stay.
As Wittgenstein famously said at the end of the Tractatus: \textit{``Whereof one
cannot speak, thereof one must be silent''}.}.
\end{description}

We will now analyze the various groups of rules of \kl{system~B}, by
comparing them to those of the \kl{BJ} calculus:
\begin{description}
  \item[\textbf{\identity}] 
  A first difference, that we will find in most rules of \kl{system~B}, is that
  we rely on the distributive interpretation of conclusions in solutions. For
  instance in the \rsf{i{\ua}} rule, $\Delta$ is available potentially in
  both \kl{subgoals}, and we do not need to move it manually: this will be the role
  of the flow rules for red \kl{items}.
  
  A second difference is that the rules are not applicable in arbitrary
  subsolutions, but only \emph{open} ones. This will also be the case of some
  membrane and heating rules. In the case of the \rsf{i{\da}} rule, it
  guarantees its \emph{locality}: if the conclusion was $\Gamma, A \piq{\cS} A,
  \Delta$, then the distributive semantics would entail that all \kl{subgoals} in
  $\cS$ must be solved at once, despite the fact that they are not directly
  related to $A$\sidenote{If we were to give up on locality, we could opt for
  this variant, which gives better \emph{factorizability}. In fact we will
  precisely do that in \refsec{invertible-calculus}.}. As for the
  \rsf{i{\ua}} rule, restricting to open solutions makes the rule
  \emph{\kl{invertible}}, without sacrificing locality. This will in fact be the case
  of all rules that create multiple \kl{subgoals}.

  \item[\textbf{\resource}] 
  Here we still have \kl{weakening} and \kl{contraction} for \kl{negative} \kl{items} (hypotheses),
  and we also allow them for \kl{positive} \kl{items} (conclusions). Note that contrary to
  the $\mathbb{I}$-rules which apply only to a formula $A$, $\mathbb{R}$-rules
  apply to an arbitrary \kl{item} $I$, which can either be a formula or a solution.
  Combined to the fact that the ambient solution can be either open or closed,
  this gives the most general and expressive formulation of the rules. We
  believe that like in \kl{CoS}, the atomic version where $I$ is restricted to an
  atomic formula might be sufficient for completeness.

  \item[\textbf{\flow}]

  Compared to \kl{BJ} where we only had \kl{neutral} bubbles, the presence of
  \kl{polarized} \kl{bubbles} in system $\sysB$ creates a mini-combinatorial explosion in
  the number of possible $\mathbb{F}$-rules. Indeed, the general scheme is to
  consider what types of \kl{items} are allowed to flow through bubbles, either
  inwards or outwards. With $i$ \kl{item} types and $b$ \kl{bubble} types, this makes for
  a total of $i \times b \times 2$ possible rules. In \kl{BJ} \kl{items} consisted
  only of \kl{polarized} formulas and \kl{neutral} \kl{bubbles} ($i = 3$ and $b = 1$), thus we
  had a total of $6$ possible $\mathbb{F}$-rules. It turns out that only the
  \rsf{f{-}{\da}} rule was necessary, and it is also present in system
  $\sysB$. Now with \kl{positive} and \kl{negative} \kl{bubbles} added to the mix ($b = 3$), we
  get up to a total of $18$ possible $\mathbb{F}$-rules in system $\sysB$. Out
  of these, $11$ were identified as being sound logically, and thus we decided
  to include all of them in system $\sysB$.

  \begin{marginfigure}
    % \hspace{0.5em}
    \stkfig{1.3}{bubbles-porosity}
    \caption{Porosity of \kl{bubbles} in system $\sysB$}
    \labfig{bubbles-porosity}
  \end{marginfigure}
  
  Some of them we have already encountered in \refsec{colors}: first the
  \rsf{f{+}{\da}} rule for distributing conclusions in \kl{subgoals}, which
  would not have made sense with the asymmetric interpretation of solutions
  (\refdef{ainterp}); but also the \rsf{f{-}{+}{\da}} and
  \rsf{f{+}{-}{\da}} rules, which allow a \kl{polarized} \kl{item} to flow
  \emph{into} a \kl{bubble} of \emph{opposite} \kl{polarity}. However to get
  \emph{cut-free} completeness, we will also need a sort of dual of these rules,
  \rsf{f{+}{+}{\ua}} and \rsf{f{-}{-}{\ua}}, which allow a \kl{polarized}
  \kl{item} to flow \emph{out} of a \kl{bubble} with the \emph{same} \kl{polarity}. Thus in
  addition to the duality that \emph{swaps} \kl{polarities}
  (\rsf{f{-}{+}{\da}} versus \rsf{f{+}{-}{\da}}), we have this new
  duality which \emph{reverses} at the same time the \emph{direction} of the
  flow, and the \emph{relationship} between \kl{polarities}
  (\rsf{f{-}{+}{\da}} versus \rsf{f{+}{+}{\ua}}).

  Taken together, these $6$ rules capture provability in
  \emph{\kl{bi-intuitionistic}} logic, as will be demonstrated by the soundness and
  completeness theorems for system $\sysB$. By adding any one of the converses
  to the $4$ rules that define the porosity of \kl{polarized} \kl{bubbles} (\rsf{f{-}{+}{\ua}}, \rsf{f{+}{-}{\ua}}, \rsf{f{-}{-}{\da}},
  \rsf{f{+}{+}{\da}}), the system collapses to \emph{classical} logic.
  This situation is summarized in \reffig{bubbles-porosity}: as in
  \reffig{bubbles-flow}, green and orange arrows represent respectively valid
  and invalid moves, but in \kl{bi-intuitionistic} rather than \kl{intuitionistic} logic.
  To recover the latter, one can just ignore all arrows that cross the blue
  bubble, which are only useful in \kl{dual-intuitionistic} logic. Then the purple
  arrows represent moves that are valid only in \kl{classical} logic. The reader can
  easily check that there is a total of $18$ arrows, and map the green and
  purple arrows back to the corresponding $\mathbb{F}$-rules of
  \reffig{graphical-B}.

  \begin{remark}
    Since all \kl{items} can freely go in and out of \kl{polarized} \kl{bubbles} in \kl{classical}
    logic, the latter are useless. In fact, one could restrict the syntax of
    solutions to \kl{neutral} \kl{bubbles} and only one \kl{polarity} of formulas, say
    conclusions. This corresponds to the possibility of having one-sided
    formulations of \kl{sequent calculi} for \kl{classical} logic, by restricting negation
    to atomic formulas and extending it to arbitrary formulas through De Morgan
    dualities\sidenote{See for instance the one-sided \kl{sequent calculus} in
    \cite{girard:hal-01322183}.}
  \end{remark}
  
  In their graphical representation, the \kl{bi-intuitionistic} $\mathbb{F}$-rules of
  system $\sysB$ are equivalent to the three following \emph{topological laws},
  that we call the \emph{$\mathbb{F}$-laws}\sidenote{Hopefully, those are not
  \emph{flaws} of our \emph{flow} rules, but rather the opposite\dots}:
  \begin{fact}[$\mathbb{F}$-laws]\labfact{bubbles-flaws}
    \sbr
    \begin{enumerate}
      \item \kl{Polarized} \kl{bubbles} trap (resp. repel) \kl{items} with a different (resp.
      identical) \kl{polarity}.
      \item \kl{Neutral} \kl{bubbles} trap (resp. repel) \kl{polarized} (resp. \kl{neutral}) \kl{items}.
      \item \kl{Polarized} \kl{bubbles} both trap and repel \kl{neutral} bubbles.
    \end{enumerate}
  % For any bubble $S$, and any trajectory $\mathfrak{T}$ in space of any item $I$
  % that crosses $S$:
  % \begin{align*}
  %   \text{$\mathfrak{T}$ crosses $S$ inwards if and only if $\pol{I} \not= \pol{S}$.} \\
  %   \text{Conversely, $\mathfrak{T}$ crosses $S$ outwards if and only if $\pol{I} = \pol{S}$.}
  % \end{align*}
  % where we denote the polarity of any item $J$ in a solution by $\pol{J}$.
  \end{fact}
  In \reffig{bubbles-porosity}, the ability of \kl{bubbles} to trap (resp. repel)
  \kl{items} corresponds to outward (resp. inward) orange arrows. $\mathbb{F}$-laws
  are thus the ``negative'' counterpart --- in the grammatical sense --- of
  $\mathbb{F}$-rules, represented by green arrows. The fact that purple arrows
  are demoted to orange arrows in \kl{bi-intuitionistic} logic, can be interpreted as
  resulting from their violation of the first $\mathbb{F}$-law. The second and
  third $\mathbb{F}$-laws characterize the behavior of \kl{neutral} bubbles, and are
  respected by all rules of system $\sysB$.
  
  In particular, they suggest the addition of a new $\mathbb{F}$-rule
  \rsf{f{\ua}}, which allows to move \kl{neutral} \kl{bubbles} out of other \kl{neutral}
  bubbles. When looking at it as a graphical \kl{rewriting rule} in
  \reffig{graphical-B}, it can be seen as the act of \emph{abstracting} the
  \kl{subgoal} $T$ from its parent \kl{subgoal} $S$, since the hypotheses and conclusions
  of $S$ cannot be brought to interact with those of $T$ anymore. More generally
  in \kl{bi-intuitionistic} logic, all flow rules can be understood as
  \emph{abstraction} moves, that strengthen the \kl{goal} by moving irreversibly an
  \kl{item} $I$ out of its \kl{subgoal} $S$. In the case of outward rules (whose name ends
  with $\ua$), $I$ is brought closer to the \emph{root} of the proof tree;
  and in the case of inward rules (whose name ends with $\da$), $I$ is
  brought closer to the \emph{leaves} of the proof tree.
  % More generally, all outwards flow rules (whose name ends with $\ua$) can
  % be understood as \emph{abstraction} moves that strengthen the goal, by moving
  % an item closer to the root of the proof tree; and dually, all inwards flow
  % rules (whose name ends with $\da$) correspond to \emph{concretization}
  % moves that weaken the goal, by moving an item towards the leaves of the proof
  % tree.
  
  It would be interesting to try to formalize $\mathbb{F}$-laws, and more
  generally the graphical presentation of system $\sysB$, with the rigorous
  tools of mathematical topology. This has been done for instance in
  \sidecite[-18em]{brady_categorical_2000} for the existential graphs of C. S.
  Peirce (see \refch{flowers}).

  \item[\textbf{\membrane}] 
  We still have the popping rule \rsf{p} of \kl{BJ}, which is now restricted to
  closed empty bubbles. We add two popping rules \rsf{p{-}} and \rsf{p{+}} for
  popping respectively \kl{negative} and \kl{positive} closed empty bubbles. Like the
  \rsf{i{\da}} rule, these have the effect of closing the ambient
  solutions, and for the same reasons we thus restrict them to open ambient
  solutions.

  \begin{marginfigure}
    $$
\R[\kl{{\limp}{+}}]
{\R[\kl{\land{+}}]
{\R[\kl{f{-}{\da}}]
{\R[\kl{i{\da}}]
{\R[\kl{p}]
{\R[\kl{a{+}}]
{\R[\kl{f{+}{+}{\da}}]
{\R[\kl{w{+}}]
{\R[\kl{{\lsub}{+}}]
{\R[\kl{f{-}{\da}}]
{\R[\kl{f{+}{\da}}]
{\R[\kl{i{\da}}]
{\R[\kl{i{\da}}]
{\R[\kl{p}]
{\R[\kl{p}]
{{\piq{}}}
{{\piq{{\piq{}}}}}}
{{\piq{{\piq{}} \sep {\piq{}}}}}}
{{\piq{{\piq{}} \sep q \seq q}}}}
{{\piq{p \seq p \sep q \seq q}}}}
{\piq{p \seq p \sep q \seq} q}}
{p \piq{\seq p \sep q \seq} q}}
{p \seq q, p \lsub q}}
{p \seq q, p \lsub q, (\seq)}}
{p \seq q, (\seq p \lsub q)}}
{p \seq q, (\piq{\seq p \lsub q})}}
{p \seq q, (\piq{\seq p \lsub q \sep {\piq{}}})}}
{p \seq q, (\piq{\seq p \lsub q \sep r \seq r})}}
{p \seq q, (r \piq{\seq p \lsub q \sep \seq r})}}
{p \seq q, (r \seq ((p \lsub q) \land r))}}
{p \seq q, r \limp ((p \lsub q) \land r)}
$$
    \caption{A proof of Uustalu's formula in system $\sysB$}
    \labfig{bubbles-uustalu}
  \end{marginfigure}

  The novelty compared to \kl{BJ} is that we also add so-called
  \emph{absorption rules} $\{\rsf{a},\rsf{a{-}},\rsf{a{+}}\}$ for membranes.
  These rules state that when a \kl{bubble} contains only a single \kl{neutral} bubble,
  the membrane of the latter can be absorbed into the membrane of the former.
  This is mainly useful when one wants to apply an outward $\mathbb{F}$-rule to
  an \kl{item} that has the same \kl{polarity} as the outer bubble, as witnessed by the
  use of the \rsf{a{+}} rule in the proof of Uustalu's formula in
  \reffig{bubbles-uustalu}. This formula was first introduced in
  \sidecite{hutchison_proof_2009} as a counter-example to the \kl{cut-elimination}
  theorem of Rauszer's \kl{sequent calculus} for \kl{bi-intuitionistic} logic
  \sidecite{rauszer_formalization_1974}, and our initial motivation for
  introducing absorption rules was precisely to provide a cut-free proof of this
  formula in system $\sysB$.

  Later, we realized that there is an interesting \emph{symmetry} at play
  between popping rules and absorption rules. As mentioned in
  \refsec{bubbles-pba}, popping rules can be understood as resulting from a
  process of \emph{contraction} of membranes into a single point. Dually,
  absorption rules can be seen as the result of a process of \emph{expansion} of
  the inner \kl{bubble} towards the outer bubble. While contraction gets stuck on
  \kl{polarized} \kl{items} because they cannot cross \kl{neutral} membranes outwards,
  expansion gets stuck on \kl{neutral} \kl{items} because they cannot cross \kl{neutral}
  membranes inwards. Thus there is a very natural interplay between
  $\mathbb{M}$-rules, and the $\mathbb{F}$-laws induced by $\mathbb{F}$-rules.

  \item[\textbf{\heating}] 
  Like the \rsf{i{\ua}} rule, the \rsf{\bot{-}}, \rsf{\lor{-}} and
  \rsf{{\limp}{-}} rules become truly local in \kl{system~B} by letting
  $\mathbb{F}$-rules handle the distribution of conclusions in \kl{subgoals}.
  Together with their dual rules \rsf{\top{+}}, \rsf{\land{+}} and
  \rsf{{\limp}{+}}, they constitute the \emph{closing} $\mathbb{H}$-rules of
  system $\sysB$. All other $\mathbb{H}$-rules work in arbitrary solutions just
  as in \kl{BJ}. But thanks to the ability to have multiple conclusions
  (\refsec{branching}) and \kl{positive} \kl{bubbles} (\refsec{colors}), both the
  \rsf{\lor{+}} and \rsf{{\limp}{+}} rules are now \emph{\kl{invertible}}: this was
  the initial motivation for designing the symmetric \kl{bubble calculus}.
\end{description}

\section{Soundness}\labsec{bubbles-soundness}

\subsection{Heyting and Brouwer algebras}

\begin{figure*}
  \tikzfig{1}{0.83}{venn-algebras}
  \caption{Relationship between the various algebras interpreting \kl{system~B}}
  \labfig{venn-algebras}
\end{figure*}

We are now going to prove the soundness of \kl{system~B} with respect to
various classes of \emph{algebras}. While the full system is \kl{classical} and thus
sound only in \emph{Boolean} algebras, most rules are sound in larger classes of
algebras, namely: \emph{Heyting} algebras for \kl{intuitionistic} logic,
\emph{Brouwer} algebras for \kl{dual-intuitionistic} logic, and
\emph{Heyting-Brouwer} algebras for \kl{bi-intuitionistic} logic. These 4 classes are
all instances of \emph{bounded lattices}, and their relationship is summarized
in the Venn \kl{diagram} of \reffig{venn-algebras}.

First we recall the definitions of the various algebras:

\begin{definition}[Bounded lattice]\labdef{bounded-lattice}
  A \emph{bounded lattice} is a structure $(\mathcal{A}, \sement, \ltop, \lbot, \lmeet,
  \ljoin)$ such that:
  \begin{itemize}
    \item $(\mathcal{A}, \sement)$ is a partial order, i.e. for every $a, b, c
    \in \mathcal{A}$ we have:
      \begin{itemize}
        \item $a \sement a$;
        \item if $a \sement b$ and $b \sement a$ then $a = b$;
        \item if $a \sement b$ and $b \sement c$ then $a \sement c$.
      \end{itemize}
    \item $\lbot$ and $\ltop$ are respectively the smallest and greatest
    elements of $(\mathcal{A}, \sement)$, i.e. for every $a \in \mathcal{A}$ we
    have $\bot \sement a$ and $a \sement \ltop$;
    \item For every pair of elements $a, b \in \mathcal{A}$, $a \ljoin b$ is
    their join (least upper bound) and $a \lmeet b$ their meet (greatest lower
    bound), that is:
    \begin{itemize}
      \item $a \sement a \ljoin b$, $b \sement a \ljoin b$ and $a \ljoin b \sement c$ for all $c \in \mathcal{A}$ s.t. $a \sement c$ and $b \sement c$;
      \item $a \lmeet b \sement a$, $a \lmeet b \sement b$ and $c \sement a \lmeet b$ for all $c \in \mathcal{A}$ s.t. $c \sement a$ and $c \sement b$.
    \end{itemize}
  \end{itemize}
\end{definition}

\begin{remark}
  As mentioned in the introduction, we only conjecture the soundness of rules
  for quantifiers: this would require considering \emph{complete} lattices, i.e.
  with meets and joins for arbitrary sets rather than just pairs\sidenote{see
  for instance section 4 of \cite{forster_completeness_2021} for a concise
  treatment of the soundness and completeness of \kl{intuitionistic} and \kl{classical}
  \kl{natural deduction} for \kl{first-order logic} with respect to algebraic semantics.}.
\end{remark}

As the notation strongly suggests, the greatest and smallest elements $\top$ and
$\bot$ will model respectively truth and absurdity, while the meet $\lmeet$ and
join $\ljoin$ will model conjunction and disjunction. In fact the conditions of
\refdef{bounded-lattice} are very close to the rules of \kl{natural deduction}
for these connectives, by replacing the \kl{sequent} operator $\seq$ with the partial
order relation $\sement$. The same idea can be applied to the implication
connective, and adding a corresponding \emph{exponential} operation $\lexp$
indeed gives the definition of a Heyting algebra:

\begin{definition}[Heyting algebra]
  A \emph{Heyting algebra} is a structure $(\mathcal{A}, \sement, \ltop, \lbot,
  \lmeet, \ljoin, \lexp)$ such that $(\mathcal{A}, \sement, \ltop, \lbot,
  \lmeet, \ljoin)$ is a bounded lattice and for every pair $a, b \in
  \mathcal{A}$, the \emph{exponential} $a \lexp b$ is the greatest element of
  the set $\compr{c \in \mathcal{A}}{c \lmeet a \sement b}$. That is, $(a \lexp
  b) \lmeet a \sement b$ and $c \sement a \lexp b$ for all $c \in \mathcal{A}$
  s.t. $c \lmeet a \sement b$.
\end{definition}

By dualizing this definition, we get a \emph{co-exponential} operation $\lcoexp$
that models the exclusion connective, and thus \kl{dual-intuitionistic} logic in
so-called Brouwer algebras:

\begin{definition}[Brouwer algebra]
  A \emph{Brouwer algebra} is a structure $(\mathcal{A}, \sement, \ltop, \lbot,
  \lmeet, \ljoin, \lcoexp)$ such that $(\mathcal{A}, \sement, \ltop, \lbot,
  \lmeet, \ljoin)$ is a bounded lattice and for every pair $a, b \in
  \mathcal{A}$, the \emph{co-exponential} $a \lcoexp b$ is the smallest element
  of the set $\compr{c \in \mathcal{A}}{b \sement a \ljoin c}$. That is, $b
  \sement a \ljoin (b \lcoexp a)$ and $b \lcoexp a \sement c$ for all $c \in
  \mathcal{A}$ s.t. $b \sement a \ljoin c$.
\end{definition}

Then we can model \kl{bi-intuitionistic} logic, which comprises both implication and
exclusion, by just taking pairs of a Heyting and a Brouwer algebra on the same
bounded lattice:

\begin{definition}[Heyting-Brouwer algebra]
  A \emph{Heyting-Brouwer algebra} is a structure $(\mathcal{A}, \sement, \ltop,
  \lbot, \lmeet, \ljoin, \lexp, \lcoexp)$ such that $(\mathcal{A}, \sement,
  \ltop, \lbot, \lmeet, \ljoin, \lexp)$ is a Heyting algebra and $(\mathcal{A},
  \sement, \ltop, \lbot, \lmeet, \ljoin, \lcoexp)$ is a Brouwer algebra.
\end{definition}

Finally, we recover \kl{classical} logic by collapsing exponentials and
co-exponentials to their \kl{classical} definitions, giving a characterization of
Boolean algebras:

\begin{definition}\labdef{boolean-algebra}
  A \emph{Boolean algebra} is a Heyting-Brouwer algebra $(\mathcal{A}, \sement,
  \ltop, \lbot, \lmeet, \ljoin, \lexp, \lcoexp)$ such that for every $a, b \in
  \mathcal{A}$, $a \lexp b = (\ltop \lcoexp a) \ljoin b$ and $a \lcoexp b = a
  \lmeet (b \lexp \bot)$.
\end{definition}

\begin{remark}
\refdef{boolean-algebra} can be shown equivalent to more
usual definitions of Boolean algebras, that are based only on lattice operations
and a primitive complement operation modelling negation; but including the proof
here would lead us out of the scope of this chapter.
% \todo{Proof in appendix?}
\end{remark}

In the rest of this chapter, we will freely assimilate formulas and their
interpretation in the various algebras. Indeed, since we only consider the
abstract classes of all algebras and never deal with a particular instance,
they will stand in perfect bijection.

\begin{definition}[Semantic entailment]
  We will write $A \sement_{\mathcall{X}} B$ (resp. $A \semequiv_{\mathcall{X}}
  B$) to express that $A \sement B$ (resp. $A \sementX B$ and $B \sementX A$) in
  every algebra of the class $\mathcall{X}$. More precisely, $\mathcall{X}$ can
  be one of $\Lattice$, $\Heyting$, $\Brouwer$, $\HeytingBrouwer$ or $\Boolean$,
  which stand respectively for bounded lattices, Heyting, Brouwer,
  Heyting-Brouwer and Boolean algebras. We will write $A \sement B$ (resp. $A
  \semequiv B$) as a shorthand for $A \sementH B$ (resp. $A \semequiv_{\Heyting}
  B$).
\end{definition}

\subsection{Duality}

We now prove a number of lemmas that characterize \emph{duality} both
semantically, typically between Heyting and Brouwer algebras, and syntactically
in the rules of system $\sysB$. This will be useful later on to shorten some
proofs.

\begin{definition}[Dual formula]\labdef{dual-formula}
  The \emph{dual formula} $\soldual{A}$ of a formula $A$ is defined recursively
  as follows:
  \begin{align*}
    \soldual{a} &= a &\\
    \soldual{\top} &= \bot &
    \soldual{\bot} &= \top \\
    \soldual{A \land B} &= \soldual{A} \lor \soldual{B} &
    \soldual{A \lor B} &= \soldual{A} \land \soldual{B} \\
    \soldual{A \limp B} &= \soldual{B} \lsub \soldual{A} &
    \soldual{A \lsub B} &= \soldual{B} \limp \soldual{A}
  \end{align*}
\end{definition}

\begin{fact}[Duality]\labfact{duality}
  \sbr
  \begin{itemize}
    \item $A \sementH B$ if and only if $\soldual{B} \sementB \soldual{A}$
    \item $A \sementB B$ if and only if $\soldual{B} \sementH \soldual{A}$
    \item $A \sementX B$ if and only if $\soldual{B} \sementX \soldual{A}$ when $\ACVar
    \in \{\HeytingBrouwer, \Boolean\}$.
  \end{itemize}
\end{fact}

We omit the proof of \reffact{duality}, but this can easily be obtained from the
soundness and completeness of a symmetric \kl{sequent calculus} for \kl{bi-intuitionistic}
logic, see for instance Lemma 2 of \sidecite{restall_extending_1997}.

\begin{definition}[Dual solution]\labdef{dual-solution}
  The \emph{dual solution} $\soldual{S}$ of a \kl{solution} $S$ is defined
  mutually recursively as follows:
  \begin{align*}
    \soldual{\Gamma \J \Delta} &= \soldual{\Delta} \mathbin{\soldual{\J}} \soldual{\Gamma} &
    \soldual{S_1 \sep \ldots \sep S_n} &= \soldual{S_1} \sep \ldots \sep \soldual{S_n} \\
    \soldual{A} &= \soldual{A} &
    \soldual{I_1, \ldots, I_n} &= \soldual{I_1}, \ldots, \soldual{I_n} \\
    \soldual{\seq} &= {\seq} &
    \soldual{\piq{\cS}} &= \piq{\soldual{\cS}}
  \end{align*}
  For \kl{solution} \kl{contexts}, the \kl{hole} is self-dual: $\soldual{\hole} = \hole$. This
  entails in particular that $\soldual{S}\select{\soldual{T}} =
  \soldual{S\select{T}}$.
\end{definition}

Graphically, the dual of a \kl{solution} $S$ is $S$ where the colors of \kl{items} have
been swapped --- i.e. blue \kl{items} become red and red \kl{items} become blue --- and
formulas have been dualized (\refdef{dual-formula}).

\begin{definition}\labdef{item-depth}
  The \emph{depth} $\sdepth{I}$ of an \kl{item} $I$ is defined recursively as
  follows:
  \begin{align*}
    \sdepth{A} &= 0 \\
    \sdepth{\Gamma \seq \Delta} &= 1 + \max_{J \in \Gamma \cup \Delta}{\sdepth{J}} \\
    \sdepth{\Gamma \piq{\cS} \Delta} &= 1 + \max_{J \in \Gamma \cup \cS \cup \Delta}{\sdepth{J}}
  \end{align*}
\end{definition}

\begin{lemma}[Involutivity]\lablemma{involutivity}
  $\soldual{\soldual{I}} = I$.
\end{lemma}
\begin{proof}
  By recurrence on $\sdepth{I}$.
  \begin{itemize}
    \item[\textbf{Formula}] Suppose $I = A$. Then we conclude by a
    straightforward induction on $A$.
    \item[\textbf{Open solution}] Suppose $I = \Gamma \seq \Delta$. Then by
    definition we have $\soldual{\soldual{\Gamma \seq \Delta}} =
    \soldual{\soldual{\Delta} \seq \soldual{\Gamma}} =
    \soldual{\soldual{\Gamma}} \seq \soldual{\soldual{\Delta}}$, and we conclude
    by IH.
    \item[\textbf{Closed solution}] Suppose $I = \Gamma \piq{\cS}
    \Delta$. Then by definition we have $\soldual{\soldual{\Gamma
    \piq{\cS} \Delta}} = \soldual{\soldual{\Delta}
    \piq{\soldual{\cS}} \soldual{\Gamma}} =
    \soldual{\soldual{\Gamma}} \piq{\soldual{\soldual{\cS}}}
    \soldual{\soldual{\Delta}}$, and we conclude by IH.
  \end{itemize}
\end{proof}

\begin{lemma}[Local rule duality]\lablemma{local-rule-duality}
  If $S \lstep{} T$ then $\soldual{S} \lstep{} \soldual{T}$.
\end{lemma}
\begin{proof}
  There is a bijection among the rules of \kl{system~B}, that matches each rule
  $\irule{r}{S}{T}$ to its dual $\irule{\soldual{r}}{\soldual{S}}{\soldual{T}}$.
  By involutivity (\reflemma{involutivity}), this bijection is self-inverse:
  $\soldual{\soldual{r}} = r$. It is most easily observed in
  the graphical presentation of the rules (\reffig{graphical-B}), where looking
  for the dual rule boils down to swapping red and blue (and mirroring logical
  connectives). The mapping goes as follows:
  \begin{mathpar}
  \begin{array}{r@{\quad\leftrightarrow\quad}l}
    \mathsf{i{\da}} & \mathsf{i{\da}} \\
    \mathsf{i{\ua}} & \mathsf{i{\ua}} \\
  \end{array}
  \and
  \begin{array}{r@{\quad\leftrightarrow\quad}l}
    \mathsf{w{-}} & \mathsf{w{+}} \\
    \mathsf{c{-}} & \mathsf{c{+}} \\
  \end{array}
  \\
  \begin{array}{r@{\quad\leftrightarrow\quad}l}
    \mathsf{f{-}} & \mathsf{f{+}} \\
    \mathsf{f{-}{+}{\da}} & \mathsf{f{+}{-}{\da}} \\
    \mathsf{f{-}{-}{\ua}} & \mathsf{f{+}{+}{\ua}} \\
    \mathsf{f{-}{+}{\ua}} & \mathsf{f{+}{-}{\ua}} \\
    \mathsf{f{-}{-}{\da}} & \mathsf{f{+}{+}{\da}} \\
  \end{array}
  \and
  \begin{array}{r@{\quad\leftrightarrow\quad}l}
    \mathsf{p} & \mathsf{p} \\
    \mathsf{p{-}} & \mathsf{p{+}} \\
    \mathsf{a} & \mathsf{a} \\
    \mathsf{a{-}} & \mathsf{a{+}} \\
  \end{array}
  \\
  \begin{array}{r@{\quad\leftrightarrow\quad}l}
    \top{-} & \bot{+} \\
    \bot{-} & \top{+} \\
    \land{-} & \lor{+} \\
    \lor{-} & \land{+} \\
    {\limp}{-} & {\lsub}{+} \\
    {\lsub}{-} & {\limp}{+} \\
    \forall{-} & \exists{+} \\
    \exists{-} & \forall{+} \\
  \end{array}
  \end{mathpar}
  Notice that some rules are self-dual, namely the identity rules
  {\rsf{i{\da}}} and {\rsf{i{\ua}}}, and the membrane rules
  {\rsf{p}} and {\rsf{a}}.
\end{proof}

\begin{lemma}[Rule duality]\lablemma{rule-duality}
  If $S \step{} T$ then $\soldual{S} \step{} \soldual{T}$.
\end{lemma}
\begin{proof}
  Let $U\hole$, $S_0$ and $T_0$ such that $S = U\select{S_0}$, $T =
  U\select{T_0}$ and $S_0 \lstep{} T_0$. By \reflemma{local-rule-duality} we have
  $\soldual{S_0} \lstep{} \soldual{T_0}$, and thus
  $\soldual{U}\select{\soldual{S_0}} \step{} \soldual{U}\select{\soldual{T_0}}$,
  or equivalently $\soldual{U\select{S_0}} \step{} \soldual{U\select{T_0}}$.
\end{proof}

\begin{lemma}[Interpretation duality]\lablemma{int-duality}
  $\soldual{\psint{I}} = \nsint{\soldual{I}}$ and $\soldual{\nsint{I}} =
  \psint{\soldual{I}}$.
\end{lemma}
\begin{proof}
  By a straightforward recurrence on $\sdepth{I}$.
\end{proof}

\begin{lemma}\lablemma{int-invert}
  $\psint{\soldual{S}} \sementX \psint{\soldual{T}}$ if and only if $\nsint{T} \sementX
  \nsint{S}$ when $\ACVar \in \{\HeytingBrouwer, \Boolean\}$.
\end{lemma}
\begin{proof}
  By duality (\reffact{duality}) we have $\soldual{\psint{\soldual{T}}} \sementX
  \soldual{\psint{\soldual{S}}}$, and then by \reflemma{int-duality}
  $\nsint{\soldual{\soldual{T}}} \sementX \nsint{\soldual{\soldual{S}}}$. We
  conclude by involutivity (\reflemma{involutivity}).
\end{proof}

\subsection{Local soundness}

In the following we give a number of (in)equalities that hold in the various
classes of algebras. They can easily be checked by building derivations in an
adequate \kl{sequent calculus}.

\begin{fact}[Commutativity]\labfact{lattice-commutativity}
  $A \lor B \semequiv_{\Lattice} B \lor A$ and $A \land B \semequiv_{\Lattice} B \land A$.
\end{fact}

\begin{fact}[Idempotency]\labfact{idempotency}
  $A \lor A \semequiv_{\Lattice} A$ and $A \land A \semequiv_{\Lattice} A$.
\end{fact}

\begin{fact}[Currying]\labfact{currying}
  \begin{align*}
    A \limp (B \limp C) &\semequiv (A \land B) \limp C \\
    (A \lsub B) \lsub C &\semequiv_{\Brouwer} A \lsub (B \lor C)
  \end{align*}
\end{fact}

\begin{fact}[Distributivity]\labfact{distributivity}
  \begin{align*}
    A \land (B \lor C) &\semequiv_{\Lattice} (A \land B) \lor (A \land C) \\
    A \lor (B \land C) &\semequiv_{\Lattice} (A \lor B) \land (A \lor C) \\
    A \limp B \land C &\semequiv (A \limp B) \land (A \limp C) \\
    A \lor B \limp C &\semequiv (A \limp B) \land (A \limp C) \\
    A \lor B \lsub C &\semequiv_{\Brouwer} (A \lsub B) \lor (A \lsub C) \\
    A \lsub B \land C &\semequiv_{\Brouwer} (A \lsub B) \lor (A \lsub C)
  \end{align*}
\end{fact}

\begin{fact}[Weak distributivity]\labfact{weakdistrib}
  \begin{align*}
    (A \limp B) \lor C &\sement A \limp (B \lor C) \\
    A \limp (B \lor C) &\sementC (A \limp B) \lor C \\
    (A \land B) \lsub C &\sementB A \land (B \lsub C) \\
    A \land (B \lsub C) &\sementC (A \land B) \lsub C
  \end{align*}
\end{fact}

\begin{fact}\labfact{gencut}
  \begin{align*}
  (A \lor B) \land (C \limp D) &\sement (A \limp C) \limp (B \lor D) \\
  (A \lor B) \land (C \limp D) &\sementHB (A \lsub C) \lor (B \lor D) \\
  (A \lor B) \land (A \limp B) &\semequiv B
  \end{align*}
\end{fact}

\begin{fact}\labfact{impsub}
  $(A \lsub B) \limp C \sementHB A \limp B \lor C$.
\end{fact}

The following definition will be used pervasively to reason by induction on the
tree structure induced by branching operators:

\begin{definition}
  The \emph{depth} $\sdepth{\J}$ of a branching operator $\J$ is defined
  recursively as follows:
  \begin{align*}
    \sdepth{\seq} &= 0 \\
    \sdepth{\piq{\cS}} &= 1 + \max_{S \in \cS}{\sdepth{S}}
  \end{align*}
\end{definition}

% \begin{lemma}\lablemma{bubbles-multiweak}
%   $\psint{\Gamma \seq \Delta} \sement \psint{\Gamma', \Gamma \seq \Delta, \Delta'}$.
% \end{lemma}
% \begin{proof}
%   $$
%   \begin{array}{rcll}
%     \psint{\Gamma \seq \Delta}
%     &=& \nsint{\Gamma} \limp \psint{\Delta} & \\
%     &\sement& \nsint{\Gamma'} \wedge \nsint{\Gamma} \limp \psint{\Delta}} &\text{(Weakening)} \\
%     &=& \psint{\Gamma', \Gamma \seq \Delta, \Delta'} & \\
%   \end{array}
%   $$
% \end{proof}

% \begin{lemma}[Sharing]
%   $\psint{\Gamma \seq \Delta} \sement \psint{\Gamma \piq{\cS} \Delta}$.
% \end{lemma}
% \begin{proof}
%   Let $\cS = S_1 \sep \ldots \sep S_n$. We proceed by recurrence on
%   $\sdepth{\piq{\cS}}$.
%   \begin{itemize}proof
%     \item[\bcase] Suppose $\sdepth{\piq{\cS}} = 1$, and
%     let $1 \leq i \leq n$. Then we know that $S_i = \Gamma_i \seq \Delta_i$, and
%     by \reflemma{bubbles-multiweak} we get $\psint{\Gamma \seq \Delta} \sement
%     \psint{S_i \mix (\Gamma \seq \Delta)}$. Thus we have
%     $$
%     \begin{array}{rcll}
%       \psint{\Gamma \seq \Delta}
%       &\sement& \bigwedge_{1 \leq i \leq n}{\psint{S_i \mix (\Gamma \seq \Delta)}} & \\
%       &=& \psint{\Gamma \piq{\cS} \Delta}
%     \end{array}
%     $$
%     \item[\rcase] Suppose $\sdepth{\piq{\cS}} > 1$.

%   \end{itemize}
% \end{proof}

Now we can give a few lemmas that generalize some semantic (in)equalities to the
interpretation of \kl{solutions} with arbitrary branching operators. All detailed
proofs are available in appendix (\refsec{app:bubbles-soundness}).

\begin{lemma}[Generalized weakening]\lablemma{bubbles-weakening}
  $\psint{S} \sement \psint{S \mix (\Gamma \seq \Delta)}$.
\end{lemma}
\begin{proof}
  By recurrence on $\sdepth{\J}$, with $S = \Gamma' \J \Delta'$.
\end{proof}

\begin{lemma}[Generalized contraction]\lablemma{bubbles-contraction}
  $\psint{S \mix (\seq I, I)} \semequiv \psint{S \mix (\seq I)}$ and
  $\psint{S \mix (I, I \seq)} \semequiv \psint{S \mix (I \seq)}$.
\end{lemma}
\begin{proof}
  By recurrence on $\sdepth{\J}$, with $S = \Gamma \J \Delta$.
\end{proof}

\begin{lemma}[Generalized weak distributivity]\lablemma{bubbles-weakdistrib}
  \begin{align}
    \psint{\Gamma \J \Delta} \lor \psint{I} &\sement \psint{\Gamma \J I, \Delta} \label{eqn:weakdistrib-one} \\
    \psint{\Gamma \J I, \Delta} &\sementC \psint{\Gamma \J \Delta} \lor \psint{I} \label{eqn:weakdistrib-two} \\
    \nsint{\Gamma, I \J \Delta} &\sementB \nsint{I} \land \nsint{\Gamma \J \Delta} \label{eqn:weakdistrib-three} \\
    \nsint{I} \land \nsint{\Gamma \J \Delta} &\sementC \nsint{\Gamma, I \J \Delta} \label{eqn:weakdistrib-four}
  \end{align}
\end{lemma}
\begin{proof}
  (\ref{eqn:weakdistrib-one}) holds by recurrence on $\sdepth{\J}$, using the
  corresponding inequality from \reffact{weakdistrib}. The proof of
  (\ref{eqn:weakdistrib-two}) is the same, except that we use the converse
  inequality of \reffact{weakdistrib} that holds in Boolean algebras.
  (\ref{eqn:weakdistrib-three}) and (\ref{eqn:weakdistrib-four}) hold by duality
  from (\ref{eqn:weakdistrib-one}) and (\ref{eqn:weakdistrib-two}).
\end{proof}

\begin{lemma}[Generalized currying]\lablemma{bubbles-currying}
  \begin{align}
    \psint{\Gamma, I \J \Delta} &\semequiv \nsint{I} \limp \psint{\Gamma \J \Delta} \label{eqn:currying-one} \\
    \nsint{\Gamma \J I, \Delta} &\semequiv_{\Brouwer} \nsint{\Gamma \J \Delta} \lsub \psint{I} \label{eqn:currying-two}
  \end{align}
\end{lemma}
\begin{proof}
  (\ref{eqn:currying-one}) holds by recurrence on $\sdepth{\J}$, and
  (\ref{eqn:currying-two}) by duality.
\end{proof}

% \begin{lemma}[Mix]\lablemma{bubbles-mix}
%   $\psint{S} \land \psint{T} \sement \psint{S \mix T}$.
% \end{lemma}

% \begin{lemma}\lablemma{bubbles-piq}
%   $\psint{S \mix T} \sement \psint{S \mix \piq{T}}$.
% \end{lemma}

Lastly, we mention a technical property of rules that will be necessary for the
final proof of soundness to go through:

\begin{fact}[Top-level genericity]\labfact{bubbles-top-level}
  If $S \lstep{} T$, then $S \mix (\Gamma \seq \Delta) \lstep{} T \mix (\Gamma \seq \Delta)$.
\end{fact}

All the previous facts and lemmas can now be used to prove \emph{local
soundness}, i.e. that the interpretation of each rule of system $\sysB$ maps to
an (in)equality in some class of algebras:

\begin{lemma}[Local soundness]\lablemma{bubbles-local-soundness}
  
  If $S \lstep{} T$ then $\psint{T \mix (\Gamma \seq \Delta)} \sementC \psint{S
  \mix (\Gamma \seq \Delta)}$.
  % and $\nsint{S \mix (\Gamma \seq \Delta)} \sement
  % \nsint{T \mix (\Gamma \seq \Delta)}$.
\end{lemma}
\begin{proof}
  $S \lstep{} T$ implies $\psint{T} \sementC \psint{S}$, which is shown by
  inspection of each rule of \kl{system~B} (see \refsec{bubbles-soundness}).
  That we can mix an arbitrary top-level \kl(sequent){context} $\Gamma \seq \Delta$ into $S$
  and $T$ follows from \reffact{bubbles-top-level}.
\end{proof}

Since some rules only hold \kl{classically}, the statement for the full system is
relative to Boolean algebras. But from the detailed proof in
\refsec{app:bubbles-soundness}, we can identify two fragments $\sysBH$ and
$\sysBHB$ of system $\sysB$ that are sound with respect to Heyting and
Heyting-Brouwer algebras:

\begin{corollary}\labcor{lsoundness}
  Let
  \begin{align*}
    \sysBHB &\defeq \sysB \setminus \{\rsf{f{-}{+}{\ua}}, \rsf{f{+}{+}{\da}}, \rsf{f{+}{-}{\ua}}, \rsf{f{-}{-}{\da}}\} \\
    \sysBH &\defeq \sysBHB \setminus \{\rsf{f{+}{-}{\da}, \rsf{f{-}{-}{\ua}, \rsf{{\lsub}{-}}, \rsf{{\lsub}{+}}}}\}
  \end{align*}
  Then we have:
  \begin{itemize}
    \item $S \lstep{\sysBH} T$ implies $\psint{T} \sement \psint{S}$
    \item $S \lstep{\sysBHB} T$ implies $\psint{T} \sementHB \psint{S}$
    % \item $S \lstep{\text{\sysB}} T$ implies $\psint{T} \sementC \psint{S}$
  \end{itemize}
\end{corollary}

In order to get the last missing fragment $\sysBB$ sound with respect to Brouwer
algebras, we need dual lemmas that are relative to the negative interpretation
$\nsint{\cdot}$ instead of the positive interpretation $\psint{\cdot}$, since
implication is replaced by exclusion. To avoid verbosity, we only formulate the
main lemma, and assume that its proof will go through mechanically:

\begin{lemma}[Local co-soundness]\lablemma{bubbles-local-cosoundness}
  If $S \lstep{} T$ then $\nsint{S \mix (\Gamma \seq \Delta)} \sementC \nsint{T
  \mix (\Gamma \seq \Delta)}$.
\end{lemma}

Then from the (assumed) proof of \reflemma{bubbles-local-cosoundness} we get:
\begin{corollary}\labcor{lcosoundness}
  Let $\sysBB \defeq \sysBHB \setminus \{\rsf{f{-}{+}{\da}},
  \rsf{f{+}{+}{\ua}}, \rsf{{\limp}{-}}, \rsf{{\limp}{+}}\}$. Then $S
  \lstep{\sysBB} T$ implies $\nsint{S} \sementB \nsint{T}$.
\end{corollary}

The full situation is summarized in \reffig{venn-algebras}.

\subsection{Contextual soundness}

\begin{lemma}[Functoriality]\lablemma{bubbles-functoriality}
  Let $\ACVar \in \{\Heyting, \HeytingBrouwer, \Boolean\}$.
  \sbr
  \begin{itemize}
    \item $\psint{I} \sementX \psint{J}$ implies $\psint{(\seq I) \mix S} \sementX
    \psint{(\seq J) \mix S}$
    % \item $\nsint{I} \sementX \nsint{J}$ implies $\nsint{(\seq I) \mix S} \sementX
    % \nsint{(\seq J) \mix S}$
    \item $\nsint{J} \sementX \nsint{I}$ implies $\psint{(I \seq) \mix S}
    \sementX \psint{(J \seq) \mix S}$
    % \item $\psint{J} \sementX \psint{I}$ implies $\nsint{(I
    % \seq) \mix S} \sementX \nsint{(J \seq) \mix S}$
  \end{itemize}
\end{lemma}
\begin{proof}
  Let $S = \Gamma \J \Delta$. We proceed by recurrence on $\sdepth{\J}$.
  \begin{itemize}
    \item[\bcase] Suppose $\sdepth{\J} = 0$. Then $\J = {\seq}$,
    and we have
    $$
    \begin{array}{rcll}
      \psint{(\seq I) \mix S}
      &=& \psint{\Gamma \seq I, \Delta} &\\
      &=& \nsint{\Gamma} \limp \psint{I} \lor \psint{\Delta} &\\
      &\sementX& \nsint{\Gamma} \limp \psint{J} \lor \psint{\Delta} &\text{(Hypothesis)}\\
      &=& \psint{\Gamma \seq J, \Delta} &\\
      &=& \psint{(\seq J) \mix S} &
    \end{array}
    $$
    $$
    \begin{array}{rcll}
      \psint{(I \seq) \mix S}
      &=& \psint{\Gamma, I \seq \Delta} &\\
      &=& \nsint{\Gamma} \land \nsint{I} \limp \psint{\Delta} &\\
      &\sementX& \nsint{\Gamma} \land \nsint{J} \limp \psint{\Delta} &\text{(Hypothesis)} \\
        % &\begin{array}{rl}
        %    &\text{Contravariant functoriality of $\_{\limp}$} \\
        %   +&\text{Functoriality of $\_{\lor}$} \\
        %   +&\text{Hypothesis}
        % \end{array}\\
      &=& \psint{\Gamma, J \seq \Delta} &\\
      &=& \psint{(J \seq) \mix S} &
    \end{array}
    $$
    \item[\rcase] Suppose $\sdepth{\J} > 0$. Then $\J =
    {\piq{\cS}}$, and for all $S_0 = \Gamma_0 \JB \Delta_0 \in
    \cS$ we have that $\sdepth{\JB} < \sdepth{\J}$. Thus we have
    $$
    \begin{array}{rcll}
      \psint{(\seq I) \mix S}
      &=& \psint{\Gamma \piq{\cS} I, \Delta} & \\
      &=& \bigwedge_{S_0 \in \cS}{\psint{(\Gamma \seq I, \Delta) \mix S_0}} & \\
      &=& \bigwedge_{S_0 \in \cS}{\psint{(\seq I) \mix ((\Gamma \seq \Delta) \mix S_0)}} & \\
      &\sementX& \bigwedge_{S_0 \in \cS}{\psint{(\seq J) \mix ((\Gamma \seq \Delta) \mix S_0)}} &\text{(IH)} \\
      &=& \bigwedge_{S_0 \in \cS}{\psint{(\Gamma \seq J, \Delta) \mix S_0}} & \\
      &=& \psint{\Gamma \piq{\cS} J, \Delta} & \\
      &=& \psint{(\seq J) \mix S} &
    \end{array}
    $$
    $$
    \begin{array}{rcll}
      \psint{(I \seq) \mix S}
      &=& \psint{\Gamma, I \piq{\cS} \Delta} & \\
      &=& \bigwedge_{S_0 \in \cS}{\psint{(\Gamma, I \seq \Delta) \mix S_0}} & \\
      &=& \bigwedge_{S_0 \in \cS}{\psint{(I \seq) \mix ((\Gamma \seq \Delta) \mix S_0)}} & \\
      &\sementX& \bigwedge_{S_0 \in \cS}{\psint{(J \seq) \mix ((\Gamma \seq \Delta) \mix S_0)}} &\text{(IH)} \\
      &=& \bigwedge_{S_0 \in \cS}{\psint{(\Gamma, J \seq \Delta) \mix S_0}} & \\
      &=& \psint{\Gamma, J \piq{\cS} \Delta} & \\
      &=& \psint{(J \seq) \mix S} &
    \end{array}
    $$
  \end{itemize}
\end{proof}

In order to ease reasoning by induction on \kl{solution} \kl{contexts}, we give a
formulation equivalent to \refdef{solution-context} as a context-free grammar:
\begin{fact}
  \kl{Solution} \kl{contexts} $S\hole$ are generated by the following grammar:
  $$
    S\hole \Coloneq \hole \mid \Gamma \J S\hole, \Delta
                          \mid \Gamma, S\hole \J \Delta
                          \mid \Gamma \piq{\cS \sep S\hole} \Delta
  $$
\end{fact}

\begin{definition}\labdef{solctx-depth}
The \emph{depth} $\sdepth{S\hole}$ of a \kl{solution} \kl{context} $S\hole$ is defined
recursively as follows:
\begin{align*}
  \sdepth{\hole} &= 0 \\
  \sdepth{\Gamma \J S\hole, \Delta} = \sdepth{\Gamma, S\hole \J \Delta} =
  \sdepth{\Gamma \piq{\cS \sep S\hole} \Delta} &= 1 + \sdepth{S\hole}
\end{align*}
\end{definition}

\begin{lemma}[Contextual soundness]\lablemma{bubbles-ctx-soundness}

  If $S \lstep{} T$ then $\psint{U\select{T} \mix (\Gamma \seq \Delta)} \sementC
  \psint{U\select{S} \mix (\Gamma \seq \Delta)}$.
  % and $\nsint{S \mix (\Gamma \seq \Delta)} \sement
  % \nsint{T \mix (\Gamma \seq \Delta)}$.
\end{lemma}
\begin{proof}
  By recurrence on $\sdepth{U\hole}$.
  \begin{itemize}
    \item[\bcase] Suppose $\sdepth{U\hole}$ = 0. Then $U\hole =
    \hole$, and we conclude by local soundness
    (\reflemma{bubbles-local-soundness}).
    \item[\textbf{Positive case}] Suppose $\sdepth{U\hole} > 0$ and $U\hole =
    \Gamma' \J U_0\hole, \Delta'$. Then by IH we have $\psint{U_0\select{T}}
    \sementC \psint{U_0\select{S}}$, and thus
    $$
    \begin{array}{rcll}
      \psint{(\Gamma' \J U_0\select{T}, \Delta') \mix (\Gamma \seq \Delta)}
      &=& \psint{(\seq U_0\select{T}) \mix (\Gamma, \Gamma' \J \Delta', \Delta)} &\\
      &\sementC& \psint{(\seq U_0\select{S}) \mix (\Gamma, \Gamma' \J \Delta', \Delta)} &\text{(\reflemma{bubbles-functoriality})}\\
      &=& \psint{(\Gamma' \J U_0\select{S}, \Delta') \mix (\Gamma \seq \Delta)} &
    \end{array}
    $$

    \item[\textbf{Negative case}] Suppose $\sdepth{U\hole} > 0$ and $U\hole =
    \Gamma', U_0\hole \J \Delta'$. Then by \reflemma{local-rule-duality} we have
    $\soldual{S} \lstep{} \soldual{T}$, and thus by IH
    $\psint{\soldual{U_0}\select{\soldual{T}}} \sementC
    \psint{\soldual{U_0}\select{\soldual{S}}}$, or equivalently
    $\psint{\soldual{U_0\select{T}}} \sementC \psint{\soldual{U_0\select{S}}}$.
    Then by \reflemma{int-invert} we get $\nsint{U_0\select{S}} \sementC
    \nsint{U_0\select{T}}$, and thus
    $$
    \begin{array}{rcll}
      \psint{(\Gamma', U_0\select{T} \J \Delta') \mix (\Gamma \seq \Delta)}
      &=& \psint{(U_0\select{T} \seq) \mix (\Gamma, \Gamma' \J \Delta', \Delta)} &\\
      &\sementC& \psint{(U_0\select{S} \seq) \mix (\Gamma, \Gamma' \J \Delta', \Delta)} &\text{(\reflemma{bubbles-functoriality})}\\
      &=& \psint{(\Gamma', U_0\select{S} \J \Delta') \mix (\Gamma \seq \Delta)} &
    \end{array}
    $$

    \item[\textbf{Neutral case}] Suppose $\sdepth{U\hole} > 0$ and $U\hole =
    \Gamma \piq{\cS \sep U_0\hole} \Delta$. Then by IH we have
    $\psint{U_0\select{T} \mix (\Gamma \seq \Delta)} \sementC
    \psint{U_0\select{S} \mix (\Gamma \seq \Delta)}$, and thus
    $$
    \begin{array}{rcll}
      \psint{\Gamma \piq{\cS \sep U_0\select{T}} \Delta}
      &=& \psint{\Gamma \piq{\cS} \Delta} \land \psint{U_0\select{T} \mix (\Gamma \seq \Delta)} &\\
      &\sementC& \psint{\Gamma \piq{\cS} \Delta} \land \psint{U_0\select{S} \mix (\Gamma \seq \Delta)} &\\
      &=& \psint{\Gamma \piq{\cS \sep U_0\select{S}} \Delta} &
    \end{array}
    $$
  \end{itemize}
\end{proof}

\begin{theorem}[Soundness]\labthm{bubbles-soundness}
  If $S \step{} T$ then $\psint{T} \sementC \psint{S}$.
\end{theorem}
\begin{proof}
  By definition of $\step{}$, and then applying \reflemma{bubbles-ctx-soundness}
  with $\Gamma = \Delta = \emptyset$.
\end{proof}

We also get for free soundness with respect to the negative interpretation,
which we call \emph{co-soundness}:

\begin{theorem}[Co-soundness]\labthm{bubbles-cosoundness}
  If $S \step{} T$ then $\nsint{S} \sementC \nsint{T}$.
\end{theorem}
\begin{proof}
  By \reflemma{rule-duality} we have $\soldual{S} \step{} \soldual{T}$,
  and thus by soundness $\psint{\soldual{T}} \sementC \psint{\soldual{S}}$. Then
  we can conclude by \reflemma{int-invert}.
\end{proof}

As for local soundness (\refcor{lsoundness} and \refcor{lcosoundness}), we can easily generalize the proof of
\reflemma{bubbles-ctx-soundness} to Heyting and Heyting-Brouwer algebras, and
thus extend our soundness result to \kl{intuitionistic} and
\kl{bi-intuitionistic} logic:

\begin{corollary}\labcor{soundness}
  \sbr
  \begin{itemize}
    \item $S \step{\sysBH} T$ implies $\psint{T} \sement \psint{S}$
    \item $S \step{\sysBHB} T$ implies $\psint{T} \sementHB \psint{S}$
  \end{itemize}
\end{corollary}
\begin{proof}
  \reflemma{bubbles-local-soundness} is the only lemma used in
  \reflemma{bubbles-ctx-soundness} that relies on Boolean algebras. Thus we can
  easily replace it by \refcor{lsoundness} to get soundness in Heyting-Brouwer
  algebras.

  For soundness in Heyting algebras, we know that the negative case will never
  happen because formulas cannot contain exclusions. The other cases only depend
  on \reflemma{bubbles-local-soundness}, thus we can again replace it by
  \refcor{lsoundness}.
\end{proof}

Once again in order to extend contextual soundness to \kl{dual-intuitionistic} logic,
we need to dualize lemmas to the negative interpretation:

\begin{lemma}[Co-functoriality]\lablemma{bubbles-cofunctoriality}
  Let $\ACVar \in \{\Brouwer, \HeytingBrouwer, \Boolean\}$.
  \sbr
  \begin{itemize}
    \item $\nsint{I} \sementX \nsint{J}$ implies $\nsint{(\seq I) \mix S}
    \sementX \nsint{(\seq J) \mix S}$
    \item $\psint{J} \sementX \psint{I}$ implies $\nsint{(I \seq) \mix S}
    \sementX \nsint{(J \seq) \mix S}$
  \end{itemize}
\end{lemma}

\begin{lemma}[Contextual co-soundness]\lablemma{bubbles-ctx-cosoundness}
  If $S \lstep{} T$ then $\nsint{U\select{S} \mix (\Gamma \seq \Delta)} \sementC
  \nsint{U\select{T} \mix (\Gamma \seq \Delta)}$.
\end{lemma}

From the assumed proof of \reflemma{bubbles-ctx-cosoundness}, we finally get:

\begin{corollary}\labcor{cosoundness}
  $S \step{\sysBB} T$ implies $\nsint{S} \sementB \nsint{T}$.
\end{corollary}

Combined with the completeness proof of \refsec{bubbles-completeness}, this will
give us our main result that $\sysBH$, $\sysBB$, $\sysBHB$ and $\sysB$ capture
exactly provability in \kl{intuitionistic}, \kl{dual-intuitionistic}, \kl{bi-intuitionistic}
and \kl{classical} logic.

\section{Completeness}\labsec{bubbles-completeness}

We are now going to prove the \emph{completeness} of the \kl{bi-intuitionistic} (and
propositional) fragment $\sysBHB$ of system $\sysB$, by simulating the nested
sequent system \kl{DBiInt} of Postniece. In \sidecite{postniece_deep_2009} she
shows that this calculus is sound and complete with respect to another calculus
\kl{LBiInt}, and in Chapter 4 of her thesis \sidecite{postniece_proof_2010} she
proves that \kl{LBiInt} is sound and complete with respect to the Kripke
semantics of \kl{bi-intuitionistic} logic. Importantly, the cut rule is shown to be
\emph{admissible} in both systems, through a syntactic process of
cut-elimination in \kl{LBiInt}. We will rely on this result to obtain
admissibility of the cut rule \rsf{i{\ua}} in $\sysBHB$, and by extension
in $\sysB$, $\sysBH$ and $\sysBB$. It might be interesting to have our own
internal cut-elimination procedure for system $\sysB$, notably to unveil its
computational content in the spirit of the \kl{Curry-Howard correspondence}. But this
would lead us astray from the purpose of this thesis, and thus we leave this
task for future work.

\begin{definition}[Structure]
  The \emph{structures} of \kl{DBiInt} are generated by the following grammar:
  $$X, Y \Coloneq \emptyset \mid A \mid (X,Y) \mid X \dseq Y$$ The
  structural connective ``,'' (comma) is associative and commutative and
  $\emptyset$ is its unit. We always consider structures modulo these
  equivalences.
\end{definition}

\begin{definition}[Structure translation]
  The \emph{translation} $\dtrans{X}$ of a structure $X$ as a multiset of items
  $\Gamma$ is defined recursively as follows:
  \begin{align*}
    \dtrans{\emptyset} &= \emptyset &
    \dtrans{(X, Y)} &= \dtrans{X}, \dtrans{Y} \\
    \dtrans{A} &= A &
    \dtrans{(X \dseq Y)} &= \dtrans{X} \seq \dtrans{Y}
  \end{align*}
\end{definition}

Note that the translation $\dtrans{(-)}$ is clearly \emph{injective}: in fact
structures are isomorphic to multisets of items that contain only \emph{open}
subsolutions. Thus from now on, we will always apply the translation implicitly,
and rely on meta-variables $X, Y$ to distinguish structures from arbitrary
solutions when necessary.

The rules of \kl{DBiInt} are given in \reffig{rules-dbiint}. Note that like
bubble calculi, \kl{DBiInt} is truly a \emph{\kl{deep inference}} system, in the
sense that rules can be applied on sequents nested arbitrarily deep inside
structures\sidenote{Our presentation of rules is slightly different from
\cite{postniece_deep_2009}: the contexts in which rules apply are left implicit,
and thus we do not rely on their polarity. The counterpart is that rules always
apply on sequents and never on formulas, which makes them more verbose. Also we
do not rely on the notion of ``top-level formulas'' of a structure, making the
propagation rules yet more verbose.}. The main difference lies in the fact that
proofs in \kl{DBiInt} are \emph{trees} built up by composing traditional
\kl{inference rules} with multiple premisses, while we use closed solutions (neutral
bubbles) to internalize the tree structure of proofs inside solutions. This
gives a lot of expressive power since closed solutions can themselves be nested
in open solutions and thus \emph{polarized}, a phenomemon which cannot be
simulated in \kl{DBiInt}. This is why we did not prove soundness in
\refsec{bubbles-soundness} by simulating directly system $\sysB$ in
\kl{DBiInt}, and conversely this will explain the ease with which \sys{DBiInt}
can be simulated inside system $\sysB$.

\begin{figure*}
  % \renewcommand{\seq}{\dseq}
\begin{framed}
\renewcommand{\arraystretch}{2}
\begin{mathpar}
\begin{array}{c}
\text{\textsc{Identity}} \\[1em]
\R[\intro(dbiint){id}]
  {}
  {X, A \seq A, Y}
\end{array}
\and
\begin{array}{c@{\quad}c}
\multicolumn{2}{c}{\textsc{Propagation}} \\[1em]
\R[\rsf{\seq_{L1}}]
  {X, A, (X', A \seq Y') \seq Y}
  {X, (X', A \seq Y') \seq Y}
&
\R[\rsf{\seq_{R1}}]
  {X \seq (X' \seq A, Y'), A, Y}
  {X \seq (X' \seq A, Y'), Y}
\\
\R[\rsf{\seq_{L2}}]
  {X, A \seq (X', A \seq Y'), Y}
  {X, A \seq (X' \seq Y'), Y}
&
\R[\rsf{\seq_{R2}}]
  {X, (X' \seq A, Y') \seq A, Y}
  {X, (X' \seq Y') \seq A, Y}
\end{array}
\and
\begin{array}{c@{\quad}c}
\multicolumn{2}{c}{\text{\textsc{Logic}}} \\[1em]
\R[\rsf{\bot_L}]
  {}
  {X, \bot \seq Y}
&
\R[\rsf{\top_R}]
  {}
  {X \seq \top, Y}
\\
\R[\rsf{\land_L}]
  {X, A \land B, A, B \seq Y}
  {X, A \land B \seq Y}
&
\R[\rsf{\land_R}]
  {X \seq A, A \land B, Y}
  {X \seq B, A \land B, Y}
  {X \seq A \land B, Y}
\\
\R[\rsf{\lor_L}]
  {X, A \lor B, A \seq Y}
  {X, A \lor B, B \seq Y}
  {X, A \lor B \seq Y}
&
\R[\rsf{\lor_R}]
  {X \seq A, B, A \lor B, Y}
  {X \seq A \lor B, Y}
\\
\R[\rsf{\limp_L}]
  {X, A \limp B \seq A, Y}
  {X, A \limp B, B \seq Y}
  {X, A \limp B \seq Y}
&
\R[\rsf{\limp_R}]
  {X \seq (A \seq B), A \limp B, Y}
  {X \seq A \limp B, Y}
\\
\R[\rsf{\lsub_L}]
  {X, A \lsub B, (A \seq B) \seq Y}
  {X, A \lsub B \seq Y}
&
\R[\rsf{\lsub_R}]
  {X \seq A, A \lsub B, Y}
  {X, B \seq A \lsub B, Y}
  {X \seq A \lsub B, Y}
\end{array}
\end{mathpar}
\end{framed}

  \caption{Rules of the deep nested sequent system \kl{DBiInt}}
  \labfig{rules-dbiint}
\end{figure*}

\begin{definition}[Syntactic entailment]
  We say that $\Gamma$ \emph{entails} $\Delta$ in a fragment $\mathsf{F}$ of
  rules of system $\sysB$, written $\Gamma \prov{\mathsf{F}} \Delta$, if and
  only if $\Gamma \seq \Delta \step{\mathsf{F}} \piq{}$. Similarly, we say
  that $X$ entails $Y$ in a fragment $\mathsf{F}$ of rules of \kl{DBiInt},
  written $X \prov{\mathsf{F}} Y$, if and only if $X \seq Y$ has a proof in
  \kl{DBiInt} using only rules in $\mathsf{F}$.
\end{definition}

\begin{lemma}[Simulation of \kl{DBiInt}]\lablemma{simulation-dbiint}
  
  If $X \prov{\kl{DBiInt}} Y$ then $X \prov{\sysBHB \setminus
  \{\rsf{i{\ua}}\}} Y$.
\end{lemma}
\begin{proof}
  By induction on the derivation of $X \prov{\kl{DBiInt}} Y$. The detailed
  proof is available in appendix (\refsec{app:bubbles-completeness}). 
\end{proof}

Assuming that the consequence relation of the Kripke semantics used by Postniece
to prove the completeness of \kl{DBiInt} coincides with the order relation of
Heyting-Brouwer algebras, we get the following fact:

\begin{fact}[Completeness of \kl{DBiInt}]\labfact{completeness-dbiint}
  If $A \sementHB B$ then $A \prov{\kl{DBiInt}} B$.
\end{fact}

Combined with the simulation of \kl{DBiInt} from \reflemma{simulation-dbiint},
this gives us the \emph{cut-free} completeness of $\sysBHB$:

\begin{theorem}[Cut-free completeness]\labthm{bubbles-completeness}
  If $A \sementHB B$ then $A \prov{\sysBHB \setminus \{\rsf{i{\ua}}\}} B$.
\end{theorem}
% \begin{proof}
%   This follows immediately from \reffact{completeness-dbiint} and
%   \reflemma{simulation-dbiint}.
% \end{proof}

In fact there are other rules of $\sysBHB$ that were not used in the simulation,
namely the $\mathbb{F}$-rule \rsf{f{\ua}}, and all $\mathbb{M}$-rules other
than \rsf{p}. Combined with the soundness of $\sysBHB$ (Corollary
\ref{cor:soundness}), this gives us the following \emph{admissibility} theorem:

\begin{theorem}[Admissibility]\labthm{bubbles:cut-admissibility}

  If $\prov{\sysBHB} A$ then $\prov{\sysBHB \setminus
  \{\rsf{i{\ua}},\rsf{f{\ua}},\rsf{p{-}},\rsf{p{+}},\rsf{a},\rsf{a{-}},\rsf{a{+}}\}}
  A$.
\end{theorem}

Although these rules are admissible, they do not seem to be derivable from other
rules. We believe that they might help in making proofs more \emph{compact} by
improving \emph{factorizability}, just like the cut rule does.
% We also conjecture that they become necessary (except \rsf{i{\ua}}) if one
% wants to show the completeness of $\sysBHB$ without any $\mathbb{H}$-rule ---
% and thus the admissibility of \emph{logical connectives}. But this has yet to be
% demonstrated.

As in \kl{sequent calculus}, every rule of system $\sysB$ other than
\rsf{i{\ua}} satisfies the \emph{subformula property}:

\begin{fact}[Subformula property]\label{cor:subformula-property} If $S
  \step{\sysB \setminus \{\rsf{i{\ua}}\}} T$ and $A \subsol T$, then there is a
  formula $B$ such that $A$ is a subformula of $B$ and $B \subsol S$.
\end{fact}

Thanks to cut admissibility, we thus get that $\sysBHB$ is \emph{analytic}. This
has many nice consequences, a well-known one being that when searching for a
proof of a given solution $S$, one does not need to come up with or ``invent'' a
formula that does not appear in $S$. This is crucial when designing
\emph{automated} decision procedures because it reduces drastically the search
space, but is also desirable in the setting of \emph{interactive} proof
building. Indeed with our \kl{Proof-by-Action} interpretation of bubble calculi
(\refsec{bubbles-pba}), this means that all logical reasoning can be performed
by direct manipulation of \emph{what is already there}. Then the cut rule is
indispensable, but confined to a role of \emph{theory building}: it allows the
creation of \emph{lemmas}, in order to make proofs shorter and more tractable by
humans.

As noted in \cite{postniece_deep_2009}, one can simply ignore rules related to
the exclusion connective $\lsub$ to get a sound and complete system for
\kl{intuitionistic} logic. In \kl{DBiInt}, these rules are the \kl{introduction rules}
\rsf{{\lsub}_R} and \rsf{{\lsub}_L}, as well as the propagation rules
\rsf{{\seq}_{L1}} and \rsf{{\seq}_{R2}}. Indeed the latter are only useful in
combination with the former, since \rsf{{\lsub}_L} is the only rule of
\kl{DBiInt} that can introduce nested sequents in negative contexts. The
situation is similar in system $\sysB$, and in fact the proof of
\reflemma{simulation-dbiint} shows that the \kl{intuitionistic} fragment $\sysBH$ is
sufficient to simulate \kl{DBiInt} without the aforementioned rules. The dual
argument can be made for \kl{dual-intuitionistic} logic, and thus we obtain
(cut-free) \kl{intuitionistic} (resp. \kl{dual-intuitionistic}) completeness of $\sysBH$
(resp. $\sysBB$):

\begin{corollary}[Intuitionistic completeness]
  \sbr
  \begin{itemize}
    \item If $A \sementH B$ then $A \prov{\sysBH \setminus
    \{\rsf{i{\ua}}\}} B$.
    \item If $A \sementB B$ then $A \prov{\sysBB \setminus
    \{\rsf{i{\ua}}\}} B$.
  \end{itemize}
\end{corollary}

% Note that $\{\rsf{f{+}{-}{\da}}, \rsf{f{-}{-}{\ua}},
% \rsf{{\lsub}{-}}, \rsf{{\lsub}{+}}\}$ are the only rules of $\sysBHB$ that
% involve negative solutions and/or exclusions.

\begin{marginfigure}
  $$
  \R[{\limp}{-}]
  {\R[{\limp}{+}]
  {\R[{\bot}{+}]
  {\R[{\bot}{-}]
  {\R[\rsf{p}]
  {\R[\rsf{f{+}{\da}}]
  {\R[\rsf{f{+}{+}{\da}}]
  {\R[\rsf{i{\da}}]
  {\R[\rsf{p}{+}]
  {\R[\rsf{p}]
  {{} \piq{}}
  {{} \piq{\piq{}}}}
  {{} \piq{\seq (\piq{})}}}
  {{} \piq{\seq (A \seq A)}}}
  {{} \piq{\seq (A \seq), A}}}
  {{} \piq{\seq (A \seq)} A}}
  {{} \piq{\seq (A \seq) \sep \piq{}} A}}
  {{} \piq{\seq (A \seq) \sep \bot \seq} A}}
  {{} \piq{\seq (A \seq \bot) \sep \bot \seq} A}}
  {{} \piq{\seq \neg A \sep \bot \seq} A}}
  {\neg \neg A \seq A}
  $$
  \caption{Proof of DNE in system $\sysB$}
  \labfig{dne-bubbles}
\end{marginfigure}

\reffig{dne-bubbles} shows a proof of the double-negation elimination law
$\mathrm{DNE} \defeq \neg \neg A \seq A$ in system $\sysB$. Since $\sysBH$ is
\kl{intuitionistically} complete, the well-known double-negation embedding of
\kl{classical} logic into \kl{intuitionistic} logic tells us that $\neg \neg A$ is
provable in $\sysBH$ (and a fortiori in $\sysB$) if $A$ is a theorem of
\kl{classical} logic. Combining the two previous facts, we obtain the \kl{classical}
completeness of system $\sysB$. In fact the proof of DNE only relies on the use
of the \rsf{f{+}{+}{\da}} rule, so we can make the following stronger
statement:

\begin{corollary}[\kl{Classical} completeness]\label{cor:bubbles-completeness-classical}
  If $A$ is a theorem of \kl{classical} logic, then $\prov{\sysBH \cup
  \{\rsf{f{+}{+}{\da}}\}} A$.
\end{corollary}
\begin{proof}
  By the double-negation embedding, we have $\prov{\sysBH} \neg \neg A$. Then we
  can build the following derivation:
  $$
  \R[\rsf{i{\ua}}]
  {\prftree[dotted]
  {\R[\rsf{p}]
  {\R[\rsf{f{+}{\da}}]
  {\prftree[r][d]{DNE}  
  {\R[\rsf{p}]
  {{} \piq{}}
  {{} \piq{\piq{}}}}
  {{} \piq{\neg \neg A \seq A}}}
  {{} \piq{\neg \neg A \seq} A}}
  {{} \piq{\piq{} \sep \neg \neg A \seq} A}}
  {{} \piq{\seq \neg \neg A \sep \neg \neg A \seq} A}}
  {\seq A}
  $$
\end{proof}

Alas this argument makes use of the \rsf{i{\ua}} rule. Note however that
the reason we chose to prove completeness of $\sysBHB$ by simulating a rather
exotic system like \kl{DBiInt}, was that standard \kl{sequent calculi} for
\kl{bi-intuitionistic} logic like the one of Rauszer
\sidecite{rauszer_formalization_1974} are not \emph{cut-free} complete; and in
our literature review, \kl{DBiInt} was the cut-free system closest in its
syntax and rules to system $\sysB$. But for \kl{classical} logic we do not have this
limitation, and thus it is straightforward to simulate directly a cut-free
\kl{sequent calculus} such as \kl{G3cp} \sidecite{negri_structural_2001}:

\begin{lemma}[Simulation of \kl{G3cp}]
  If $\Gamma \prov{\kl{G3cp}} \Delta$, then $\Gamma \prov{\sysBH \cup
  \{\rsf{f{+}{+}}\} \setminus \{\rsf{i{\ua}}\}} \Delta$.
\end{lemma}
\begin{proof}
  By induction on the \kl{G3cp} derivation, see
  \refsec{app:bubbles-completeness} for the detailed proof.
\end{proof}

Lastly, let us mention a recent result of Goré and Shillito
\cite{gore_bi-intuitionistic_2020}, where they uncover a distinction between a
\emph{weak} and a \emph{strong} consequence relation in the semantics of
\kl{bi-intuitionistic} logic. Although they define the same set of theorems, these
two relations have different properties at the meta-level, and thus the authors
argue that they define two distinct logics, called respectively \kl{wBIL} and
\kl{sBIL}. At the end of the article, they conjecture that the various existing
calculi in the literature are sound and complete for \kl{wBIL}, including a
calculus designed by Postniece. Since our completeness proof is by simulation of
the system \kl{DBiInt} by the same author, we follow this conjecture regarding
the completeness of the \kl{bi-intuitionistic} fragment $\sysBHB$ of system $\sysB$.
For soundness, we would need to clarify the relationship between Heyting-Brouwer
algebras and these consequence relations, which stem instead from an analysis of
the Kripke semantics of \kl{bi-intuitionistic} logic. Since system $\sysB$ offers a
very expressive syntax, it would be interesting to investigate its ability to
capture both \kl{wBIL} and \sys{sBIL}, maybe by using distinct sets of flow
rules. Goré and Shillito suggest that a framework that captures both
\emph{provability} and \emph{refutability} ``in one shot'' would be needed, and
we believe system $\sysB$ might just provide this: indeed a derivation $S \steps{}
\piq{}$ can be read both as a \emph{proof} of $\psint{S}$, and a
\emph{refutation} of $\nsint{S}$.

\section{Invertible calculus}\labsec{invertible-calculus}

\subsection{Modifying rules}

\begin{scope}\knowledgeimport{bubble-inv}

\begin{figure*}
  \fontsize{10}{10.5}\selectfont
\begin{framed}
\renewcommand{\arraystretch}{2}
\begin{mathpar}
\begin{array}{c}
\identity \\[1em]

\R[\intro{i{\da}}]
    {\piq{}}
    {\Gamma, A \J A, \Delta}
\end{array}
\\
\begin{array}{c@{\quad}c}
\multicolumn{2}{c}{\flow} \\[1em]

% \R[\intro{s{-}}]
%     {\piq{S \seq} \mix \Gamma \J \Delta}
%     {\Gamma, (\piq{S}) \J \Delta}
% &
% \R[\intro{s{+}}]
%     {\Gamma \J \Delta \mix \piq{\seq S}}
%     {\Gamma \J (\piq{S}), \Delta}
% \\

\R[\intro{f{-}{\da}}]
    {\Gamma, I \piq{\Gamma', I \JB \Delta' \sep \cS} \Delta}
    {\Gamma, I \piq{\Gamma' \JB \Delta' \sep \cS} \Delta}
&
\R[\intro{f{+}{\da}}]
    {\Gamma \piq{\cS \sep \Gamma' \JB I, \Delta'} I, \Delta}
    {\Gamma \piq{\cS \sep \Gamma' \JB \Delta'} I, \Delta}
\\
\R[\intro{f{-}{+}{\da}}]
    {\Gamma, I \J (\Gamma', I \JB \Delta'), \Delta}
    {\Gamma, I \J (\Gamma' \JB \Delta'), \Delta}
&
\R[\intro{f{+}{-}{\da}}]
    {\Gamma, (\Gamma' \JB I, \Delta') \J I, \Delta}
    {\Gamma, (\Gamma' \JB \Delta') \J I, \Delta}
\\
\R[\intro{f{-}{-}{\ua}}]
    {\Gamma, I, (\Gamma', I \JB \Delta') \J \Delta}
    {\Gamma, (\Gamma', I \JB \Delta') \J \Delta}
&
\R[\intro{f{+}{+}{\ua}}]
    {\Gamma \J (\Gamma' \JB I, \Delta'), I, \Delta}
    {\Gamma \J (\Gamma' \JB I, \Delta'), \Delta}
\\
\R[\intro{f{-}{+}{\ua}}]
    {\Gamma, I \J (\Gamma', I \JB \Delta'), \Delta}
    {\Gamma \J (\Gamma', I \JB \Delta'), \Delta}
&
\R[\intro{f{+}{-}{\ua}}]
    {\Gamma, (\Gamma' \JB I, \Delta') \J I, \Delta}
    {\Gamma, (\Gamma' \JB I, \Delta') \J \Delta}
\\
\R[\intro{f{-}{-}{\da}}]
    {\Gamma, I, (\Gamma', I \JB \Delta') \J \Delta}
    {\Gamma, I, (\Gamma' \JB \Delta') \J \Delta}
&
\R[\intro{f{+}{+}{\da}}]
    {\Gamma \J (\Gamma' \JB I, \Delta'), I, \Delta}
    {\Gamma \J (\Gamma' \JB \Delta'), I, \Delta}
\end{array}
\and
\begin{array}{cc}
\multicolumn{2}{c}{\membrane} \\[1em]

\multicolumn{2}{c}{
\R[\intro{p}]
    {\Gamma \piq{\cS} \Delta}
    {\Gamma \piq{\cS \sep {\piq{}}} \Delta}
} \\
\R[\intro{p{-}}]
    {\piq{}}
    {\Gamma, (\piq{}) \J \Delta}
&
\R[\intro{p{+}}]
    {\piq{}}
    {\Gamma \J (\piq{}), \Delta}
\\

\multicolumn{2}{c}{
\R[\intro{a}]
    {\Gamma \piq{S} \Delta}
    {\Gamma \piq{{\piq{S}}} \Delta}
} \\
\R[\intro{a{-}}]
    {\Gamma, S \J \Delta}
    {\Gamma, (\piq{S}) \J \Delta}
&
\R[\intro{a{+}}]
    {\Gamma \J S, \Delta}
    {\Gamma \J (\piq{S}), \Delta}
\\
\end{array}
\\
\begin{array}{c@{\quad}c}
\multicolumn{2}{c}{\heating} \\[1em]

\R[\intro{\top{-}}]
    {\Gamma \J \Delta}
    {\Gamma, \top \J \Delta}
&
\R[\intro{\top{+}}]
    {\Gamma \J (\piq{}), \Delta}
    {\Gamma \J \top, \Delta}
\\
\R[\intro{\bot{-}}]
    {\Gamma, (\piq{}) \J \Delta}
    {\Gamma, \bot \J \Delta}
&
\R[\intro{\bot{+}}]
    {\Gamma \J \Delta}
    {\Gamma \J \bot, \Delta}
\\
\R[\intro{\land{-}}]
    {\Gamma, A, B \J \Delta}
    {\Gamma, A \land B \J \Delta}
&
\R[\intro{\land{+}}]
    {\Gamma \piq{\seq A \sep \seq B\ } \Delta}
    {\Gamma \seq A \land B, \Delta}
\\
\R[\intro{\lor{-}}]
    {\Gamma \piq{A \seq \sep B\seq} \Delta}
    {\Gamma, A \lor B \seq \Delta}
&
\R[\intro{\lor{+}}]
    {\Gamma \J A, B, \Delta}
    {\Gamma \J A \lor B, \Delta}
\\
\R[\intro{{\limp}{-}}]
    {\Gamma, A \limp B \piq{\seq A \sep B\seq} \Delta}
    {\Gamma, A \limp B \seq \Delta}
&
\R[\intro{{\limp}{+}}]
    {\Gamma \J (A \seq B), \Delta}
    {\Gamma \J A \limp B, \Delta}
\\
\R[\intro{{\lsub}{-}}]
    {\Gamma, (A \seq B) \J \Delta}
    {\Gamma, A \lsub B \J \Delta}
&
\R[\intro{{\lsub}{+}}]
    {\Gamma \piq{\seq A \sep B\seq} A \lsub B, \Delta}
    {\Gamma \seq A \lsub B, \Delta}
\\
\R[\intro{\forall{-}}]
    {\Gamma, \forall x. A, \subst{A}{t}{x} \J \Delta}
    {\Gamma, \forall x. A \J \Delta}
&
\R[\intro{\forall{+}}]
    {\Gamma \J A, \Delta}
    {\Gamma \J \forall x. A, \Delta}
\\
\R[\intro{\exists{-}}]
    {\Gamma, A \J \Delta}
    {\Gamma, \exists x. A \J \Delta}
&
\R[\intro{\exists{+}}]
    {\Gamma \J \subst{A}{t}{x}, \exists x. A, \Delta}
    {\Gamma \J \exists x. A, \Delta}
\end{array}
\end{mathpar}

In the {\kl{\forall{+}}} and {\kl{\exists{-}}} rules, $x$ is not free in
$\Gamma$, $\Delta$ and $\J$.
\end{framed}

  \caption{Rules for the \kl{invertible} \kl{bubble calculus} \kl{Binv}}
  \labfig{sequent-B-inv}
\end{figure*}

An important thing to note, is that all the rules of \kl{DBiInt} are
\emph{\kl{invertible}}\sidenote{Lemma 5.2.4 in Postniece's thesis
  \cite{postniece_proof_2010}.}. Thus it follows immediately from
\reflemma{simulation-dbiint} that one can just take the translation of the rules
of \kl{DBiInt} in \kl{system~B}, and get a complete, fully \kl{invertible}
calculus. But this would be a waste of the expressive power and nice properties
of \kl{system~B}, like linearity and locality.

Instead, we will target precisely the non-\kl{invertible} rules of \kl{system~B},
and modify only those. From the proof of \reflemma{bubbles-local-soundness}, we
can identify which rules of \kl{system~B} are \kl{invertible}, and which are
probably not. Indeed if the soundness of a rule only relies on a chain of
equivalences, then it is necessarily \kl{invertible}. On the contrary if it relies on
an inequality, then it is probably not \kl{invertible}\sidenote{To ensure that it is
not \kl{invertible}, we would need additionally to find a counter-model that
invalidates the converse inequality.}.

\begin{fact}[Invertibility of \kl{system~B}]
  All rules in the fragment $\mathbb{I} \cup \{\kl{c{-}},\kl{c{+}}\} \cup
    \mathbb{M} \cup \mathbb{H} \setminus \{\kl{{\limp}{-}},\kl{{\lsub}{+}}\}$ of
  \kl{system~B} are \kl{invertible}.
\end{fact}

Thus the only remaining rules of \kl{system~B} that are (most probably) not
\kl{invertible} are the \kl{weakening} rules $\{\kl{w{-}},\kl{w{+}}\}$, all the
\kl{$\mathbb{F}$-rules}, and the \kl{$\mathbb{H}$-rules} $\{\kl{{\limp}{-}},
\kl{{\lsub}{+}}\}$ that \emph{apply} an implication/\kl{exclusion}\sidenote{For
quantifier rules, we conjecture that as in \kl{sequent calculus}, the rules
$\{\kl{\forall{+}}, \kl{\exists{-}}\}$ are \kl{invertible}, while the rules
$\{\kl{\forall{-}}, \kl{\exists{+}}\}$ are not. And as in \kl{sequent calculus},
this can be remedied by systematically duplicating the instantiated formula.}.
In \reffig{sequent-B-inv} we define the \intro{Binv} calculus, which results
from the following modifications to the previous rules:
\begin{description}
  \item[Weakening]
    Here we follow a standard technique in \kl{sequent calculus}, that merges
    the \kl{weakening} rule in all \emph{terminal} rules of the calculus (i.e.
    rules with no premisses). In \kl{bubble calculi}, the notion of premiss is
    captured by \kl{neutral} bubbles; thus we incorporate \kl{weakenings} in all
    rules that solve \kl{subgoals} by \kl[saturated]{saturating} \kl{solutions}.
    Those are the rules $\{\kl{i{\da}},\kl{p{-}},\kl{p{+}}\}$, which for the
    occasion have also been generalized to arbitrary \kl{solutions}. Indeed in
    \kl{system~B} we restricted them to \kl{unsaturated} \kl{solutions} to make
    them \emph{local}, but here the \kl{weakenings} break locality anyway, and
    the general version improves \emph{factorizability} by solving instantly all
    \kl{subgoals} inside $\J$.

  \item[Flow]
    As shown by the simulation of \reflemma{simulation-dbiint}, the
    \emph{propagation} rules of \kl{DBiInt} combine an instance of
    \emph{\kl{contraction}} followed by the application of a \kl{$\mathbb{F}$-rule}
    on the duplicated formula. Thus we can make all \kl{$\mathbb{F}$-rules} of
    \kl{system~B} \kl{invertible} by systematically duplicating the moved
    formula, although this breaks \emph{linearity}.

    A downside of \kl{propagation rules} in the style of \kl{DBiInt}, is that they
    create a lot of unnecessary copies of the moved formula $A$. Often, one will
    want to move $A$ in a \kl{subgoal}/supergoal at a distance $n$ in the proof tree,
    with $n > 1$. Usually this would be performed by $n$ applications of
    \kl{$\mathbb{F}$-rules}, which by linearity indeed just move the formula. But with
    \kl{propagation rules}, $n$ copies of $A$ will be created, with one copy in each
    \kl{subgoal} met on the path to the destination.

    To prevent this, one would need a way to copy formulas at an arbitrary
    distance. This can be done with \kl{inference rules} that are \emph{doubly}
    deep, by encoding the path to the destination as a second \kl{context}
    inside the \kl{context} where the rule is applied\sidenote{Such rules are
    sometimes called \emph{super-switch} rules in the \kl{deep inference}
    literature, see for instance \cite[Chapter~8,
    Section~2.1]{guenot_nested_2013}.}. It turns out to be hard to express in
    bubbles, because this requires a syntactic way to describe \kl{contexts}
    that correspond to valid flow paths of arbitrary length\sidenote{This
    problem is solved trivially in the flower calculus (\refch{flowers}), by
    formulating a so-called \emph{\kl{pollination}} relation.}. But in principle
    it should be feasible, and would enable a more comfortable use in a
    \kl{Proof-by-Action} setting.

    Note also that we removed the \kl{f{\ua}} rule of \reffig{sequent-B}.
    Indeed even after turning it into a \kl{propagation rule}, the moved copy of the
    duplicated \kl{subgoal} $S$ cannot be weakened because it lives in a \kl{neutral}
    bubble. Thus the rule stays non-\kl{invertible}, and cannot be included in
    \kl{Binv}. Fortunately, we showed that it is \kl{admissible} in
    \refthm{bubbles:cut-admissibility}, so this is not problematic.

  \item[Implication/Exclusion]
    The last source of non-invertibility is the \kl{$\mathbb{H}$-rules}
    \kl{{\limp}{-}} and \kl{{\lsub}{+}}, that respectively allow to use an
    implication hypothesis, and prove an \kl{exclusion} conclusion.
    Here we can just duplicate the implication/\kl{exclusion} formula, as in the
    \kl{introduction rules} of \kl{DBiInt}. Also like in \kl{DBiInt}, we removed
    the \kl{contraction} rules \kl{c{-}} and \kl{c{+}}, which are now merged with
    these two rules as well as the \kl{$\mathbb{F}$-rules}. Although \kl{contraction}
    rules are \kl{invertible}, they induce a lot of complexity in proof search,
    because it is hard to predict the (occurrences of) formulas that need to be
    duplicated, and one can duplicate \emph{ad infinitum}. Thus it is preferable
    to design a calculus where they are \kl{admissible}. But unlike what is done in
    \kl{DBiInt}, we did not incorporate \kl{contraction} in other
    \kl{$\mathbb{H}$-rules}. Thus we cannot simulate exactly all the \kl{introduction
    rules} of \kl{DBiInt} in \kl{Binv}.

\end{description}

\begin{remark}
  We also changed the \kl{\bot{-}} and \kl{\top{+}} rules, so that they create
  \kl{polarized}, \kl{saturated} empty \kl{solutions}. This makes them both
  local, and generic with respect to the \kl[saturated]{saturation} status of
  the ambient \kl{solution}. The previous version can then be simulated by
  combination with the popping rules \kl{p{-}} and \kl{p{+}}.
\end{remark}

These modifications only change superficially the proof of soundness, and thus
we do not redo it. As for completeness, we would need to prove that the
\kl{contraction} rules are \kl{admissible}, in order to solve the aforementioned problem
of simulating \kl{DBiInt}'s \kl{introduction rules}:

\begin{lemma}[Admissibility of contraction]\lablemma{admissibility-contraction}
  \sbr
  \begin{itemize}
    \item If $\prov{\kl{Binv}} S\select{\Gamma, A, A \J \Delta}$, then
          $\prov{\kl{Binv}} S\select{\Gamma, A \J \Delta}$.
    \item If $\prov{\kl{Binv}} S\select{\Gamma \J A, A, \Delta}$, then
          $\prov{\kl{Binv}} S\select{\Gamma \J A, \Delta}$.
  \end{itemize}
\end{lemma}

Note that it is sufficient to prove admissibility of \kl{contraction} on
formulas, rather than on arbitrary \kl{items}. Indeed we only need it to
simulate the \kl{introduction rules} of \kl{DBiInt}, which always duplicate
formulas. For now we only conjecture completeness of \kl{Binv}, since it not
clear what method should be used to prove \reflemma{admissibility-contraction}.
In her thesis \cite[Lemma~5.2.3]{postniece_proof_2010}, Postniece does a proof
by induction on the depth of the derivation, relying on the fact that all
\kl{introduction rules} of \kl{DBiInt} preserve the principal formula; but this
is precisely what we are trying to avoid with our version of the rules. Of
course, if we either give up on this constraint or include \kl{contraction}
rules in \kl{Binv}, then we immediately get our desired result: \kl{Binv}
is a fully \kl{invertible} calculus, where the same fragments as \kl{system~B}
capture \kl{intuitionistic}, \kl{dual-intuitionistic}, \kl{bi-intuitionistic}
and \kl{classical} logic.

\subsection{Semi-automated proof search}\labsubsec{bubbles-search}

In the \kl{intuitionistic} (propositional) fragment of \kl{Binv}, a canonical
way to search for a proof of a formula $A$ consists in the following $5$
\emph{phases}, applied successively in a loop until the \kl{saturated} empty
\kl{solution} is reached:
\begin{description}
  \item[Decomposition] Decompose $A$ by applying recursively
    \kl{$\mathbb{H}$-rules}, until either atoms, \kl{negative} implications
    $\hypo{\limp}$, \kl{negative} disjunctions $\hypo{\lor}$, or \kl{positive}
    conjunctions $\conc{\land}$ are reached.
    
    Indeed since the \kl{{\limp}{-}} rule duplicates the implication, it cannot
    be used to decompose it. Regarding the \kl{\lor{-}} and \kl{\land{+}} rules,
    they can only be applied when the formula is in an \emph{\kl{unsaturated}}
    \kl{solution}. A first option is to let the system automatically distribute
    them in all \kl{unsaturated} \kl{subsolutions} that are reachable, so that
    it can keep decomposing them. But this might create an explosion in the
    number of created \kl{subgoals}. Another option is to let the user manually
    decompose them. We believe this second option is preferable, if one wants to
    keep control over the proof search process. Indeed, it is only natural that
    the user should be able to choose which \emph{cases} to consider when
    building a proof.

  \item[Absorption] Apply the absorption rules
    $\{\kl{a},\kl{a{-}},\kl{a{+}}\}$ wherever possible. This will prevent
    atoms from being unnecessarily stuck on \kl{neutral} membranes in the next phase.
    This phase can also be trivially automated.

  \item[Linking] Try to bring together every pair of dual atoms, so
    that they annihilate each other in an instance of the \kl{i{\da}} rule. This
    is reminiscent of our \emph{drag-and-drop} actions of \refch{pba}. In a
    touch-based \kl{GUI}, rather than dragging a complex formula onto another
    complex formula, one could \emph{pinch} together the two atoms: if there is
    no $\mathbb{F}$-law stucking one of the atoms on some membrane (orange
    arrows in \reffig{bubbles-porosity}), then the pinch succeeds, and the
    system \kl[saturated]{saturates} the \kl{subsolution} at the location where
    the pinch ends by applying the \kl{i{\da}} rule. Thus the user can choose
    the \kl{subgoal} to solve by controlling the destination of the pinch, which
    can be seen as a more symmetric and powerful version of \kl{DnD} actions.
    Generally though, one will want to apply the following \emph{rule of thumb}
    (pun intended):
    \begin{fact}[Rule of thumb]
      When linking a pair of dual atoms, follow these steps:
      \begin{enumerate}
        \item put your \emph{thumb} on the \emph{outermost} atom, and your
              \emph{index} on the \emph{innermost} atom;
        \item try to bring your index to your thumb;
        \item if you get stuck on a membrane, try to bring your thumb to your
              index;
        \item if you again get stuck, then give up on this pair.
      \end{enumerate}
    \end{fact}
    The point of this heuristic, is that it should maximize the
    \emph{factorization} of the proof: when it succeeds, it will solve the
    \kl{subgoal} that is located closest to the root of the \kl{goal}, maximizing the size
    of the pruned branch, and thus the number of \kl{subgoals} solved in one go. It
    can also be used to completely automate this phase.

  \item[Popping] Pop every \kl{saturated} empty \kl{bubble} in the \kl{goal} with the
  rules $\{\kl{p},\kl{p{-}},\kl{p{+}}\}$. This phase can also be trivially
  automated, and corresponds to the unit elimination phase in \kl{subformula linking}
  (\refsec{action}).

  \item[Application] When there are no more pairs of dual atoms, or all
    the remaining pairs have been given up (last step of the rule of thumb), let
    $S$ be the current \kl{goal}, and
    $$\intro*\imps{S} \defeq \compr{(S_0\hole, A, B, \J, \Delta)}{S = S_0\select{\Gamma,
          A \limp B \J \Delta} \text{ for some $\Gamma$}}$$

    If $\imps{S} = \emptyset$, then $S$ should not be provable, and we can stop
    the proof search procedure. Otherwise for each $(S_0\hole, A, B, \J, \Delta)
    \in \imps{S}$, we might need to apply the \kl{{\limp}{-}} rule on $\hypo{A
    \limp B}$, either directly in $S_0\hole$ if $\J = {\seq}$ and $\max_{I \in
    \Delta}{\sdepth{I}} = 0$, otherwise in some subgoal $T\select{U} \in \J \cup
    \Delta$. This is where the proof needs \emph{insight}, because it is not
    clear if the antecedant $A$ will be provable with the \kl{context} available
    in $S_0\hole$ or in one of the $T\hole$, or if the hypothesis $B$ is even
    needed at all.

    A first possibility is to let the user rely on her intuition, by choosing
    manually a specific \kl{subsolution} in $\imps{S}$ to apply the \kl{{\limp}{-}}
    rule upon. Additionally, she might need to determine a \kl{subgoal} $T\select{U}$
    in which $A$ is provable, and first duplicate $\hypo{A\limp B}$ in $T\hole$
    before applying \kl{{\limp}{-}}. This will always be possible with the
    \kl{$\mathbb{F}$-rules} \kl{f{-}{\da}} and \kl{f{-}{+}{\da}}.
    Ideally, she would also pick the most general $T\hole$ to factorize the
    proof, by minimizing its depth $\cdepth{T\hole}$ (\refdef{solctx-depth}).

    A second possibility is to duplicate eagerly every $\hypo{A \limp B}$ of
    $\imps{S}$ in every \kl{unsaturated} \kl{subsolution} of $S$ where it can be so, and then
    apply \kl{{\limp}{-}} on all the newly created copies. To avoid an
    explosion of the size of $S$, the system should mark all the copies as
    \emph{used}, so that during the next \textbf{Application} phases, all the
    already \emph{used} copies are ignored, and only the original occurrence of
    $\hypo{A \limp B}$ is considered.

    Then we can restart the procedure, by applying the \textbf{Decomposition}
    phase to every copy of $A$ and $B$.
\end{description}

\begin{remark}
  By adopting \kl{$\mathbb{H}$-rules} in the style of \kl{DBiInt}'s \kl{introduction
  rules}, we would make the \textbf{Decomposition} phase, and thus the whole
  procedure inoperable, since the \textbf{Linking} phase depends crucially on
  it. Allowing \kl{contraction} rules would also jeopardize the potential
  completeness of the procedure, because \kl{contraction} might be needed at
  unpredictable moments, and on unpredictable formulas.
\end{remark}

A strength of our proof search procedure, compared to the state-of-the-art in
other formalisms, is that most of its automation preserves the \emph{size}
(number of atoms) and the \emph{structure} of the \kl{goal}:
\begin{itemize}
  \item In the \textbf{Decomposition} phase, if we opt out of the automatic
  distribution of \kl{negative} $\hypo{\lor}$ and \kl{positive} $\conc{\land}$,
  then the system will only apply \kl{$\mathbb{H}$-rules} that split logical
  connectives, and create a \emph{partition} of the atoms of the \kl{goal} by
  enclosing them in bubbles. Thus the size of the \kl{goal} is kept intact, and
  the structure modified but in a controlled, local way.

  \item In the \textbf{Absorption} phase, we simply merge some membranes
  together, preserving both the size and the structure of the \kl{goal}.

  \item In the \textbf{Linking} phase, the particular way in which we use
  \kl{$\mathbb{F}$-rules} ensures that we only decrease the number of atoms. Indeed,
  if we assume as discussed earlier that we have ``super-flow'' rules that copy
  at a distance, then either:
  \begin{enumerate}
    \item the link is successful, and the two created copies of atoms are
  immediately destroyed by the \kl{i{\da}} rule. Then the solved
  \kl{subsolution} is entirely pruned out, decreasing the size of the \kl{goal}; or
    \item the link fails, but then we can instantly ``undo'' it. Or rather, one
  should consider that rules are applied only when the link is successful.
  \end{enumerate}

  \item In the \textbf{Popping} phase, entire branches of the \kl{goal} are pruned
  out, decreasing the size of the \kl{goal}.
\end{itemize}

Then only the automation of the \textbf{Application} phase (and part of the
\textbf{Decomposition} phase) is susceptible of both significantly increasing
the size of the \kl{goal}, and altering its global structure. But as is the case for
every phase, the user can easily opt out of this automation, and do the
reasoning manually when it is necessary to keep the \kl{goal} understandable by
humans. Typically in an educational setting, it should be quite instructive to
have the ability to perform \textbf{Decomposition} and \textbf{Linking} by hand
(literally).

\subsection{Failure of full iconicity}

Because of the implicit \kl{contraction} in the rules \kl{{\limp}{-}} and
\kl{{\lsub}{+}}, one cannot fully decompose a formula into an equivalent
\kl{solution} by deterministically applying a sequence of \kl{$\mathbb{H}$-rules} (and
possibly \kl{$\mathbb{F}$-rules}, to distribute \kl{positive} conjunctions and \kl{negative}
disjunctions). Thus \kl{Binv} fails to be \emph{fully \kl{iconic}}, because it
relies on the \emph{\kl{symbolic}} connectives $\limp$ and $\lsub$ to represent
logical statements.

\begin{marginfigure}
  $$
\begin{array}{r@{\quad\mapsto\quad}l@{\vspace{0.75em}}}
  \hypo{\top} & \hbubble{\phantom{\top}} \\
  \hypo{\bot} & \hbbubble{\phantom{\bot}} \\
  \hypo{A \land B} & \hbubble{\hypo{A}~~~\hypo{B}} \\
  \hypo{A \lor B} & \hbbubble{\bubble{\hypo{A}}~~~\bubble{\hypo{B}}} \\
  \hypo{A \limp B} & ? \\
  \hypo{A \lsub B} & \hbubble{\hypo{A}~~~\conc{B}} \\
\end{array}
$$
$$
\begin{array}{r@{\quad\mapsto\quad}l@{\vspace{0.75em}}}
  \conc{\top} & \cbbubble{\phantom{\top}} \\
  \conc{\bot} & \cbubble{\phantom{\bot}} \\
  \conc{A \land B} & \cbbubble{\bubble{\conc{A}}~~~\bubble{\conc{B}}} \\
  \conc{A \lor B} & \cbubble{\conc{A}~~~\conc{B}} \\
  \conc{A \limp B} & \cbubble{\hypo{A}~~~\conc{B}} \\
  \conc{A \lsub B} & ?
\end{array}
$$
  \caption{Mapping of formulas to equivalent \kl{solutions}}
  \labfig{bubbles-native}
\end{marginfigure}

This can be understood as resulting from the inability of \kl{solutions} to
represent natively \emph{\kl{negative} implications} and \emph{\kl{positive}
subtractions}, although they can represent natively all other polarizations of
connectives. This is illustrated by the mapping of \reffig{bubbles-native} from
\kl{polarized} formulas to equivalent \kl{solutions}, which is really just the
\kl{$\mathbb{H}$-rules} of \kl{system~B} (\reffig{graphical-B}) where the right-hand
\kl{solution} is enclosed in a \kl{bubble} of the corresponding \kl{polarity}.
The reader can easily check that if $A$ is mapped to $S$, then $\psint{A}
\semequiv{} \psint{S}$.

\begin{marginfigure}
  $$
  \R[\intro{e}]
    {\Gamma, (\Delta, A \seq B) \seq C}
    {\Gamma, (\Delta \seq A \limp B) \seq C}
  $$
  \caption{\kl{Left introduction rule} for $\limp$ in \kl{JN}}
  \labfig{jn-rule-e}
\end{marginfigure}

This seems to be a fundamental limitation of \kl{system~B}, caused by its
symmetric treatment of implication and subtraction. For instance in the \kl{nested
sequent} calculus \intro{JN} of Guenot for implicative logic \cite[Chapter
3]{guenot_nested_2013}, which is fully decomposable, \kl{nested sequents} that appear
in negative \kl{contexts} are interpreted as implications, as illustrated by the
\kl{left introduction rule} \kl{e} (\reffig{jn-rule-e}). But we cannot do this
in \kl{system~B}, because this would conflict with the subtractive reading of
\kl{negative} \kl{solutions}, i.e. $\nsint{A \seq B} = \nsint{A} \lsub \nsint{B}$. In
\refch{flowers}, the problem will also be solved through an asymmetric treatment
of \kl{nested sequents}, capturing only \kl{intuitionistic} logic instead of
\kl{bi-intuitionistic} logic. But this is a small price to pay, since
\kl{bi-intuitionistic} logic does not (currently) have any applications in the
realm of interactive theorem proving.

\end{scope}


% \section{Decomposable calculus}\labsec{decomposable-calculus}

% \begin{figure*}
%   \fontsize{10}{10.5}\selectfont
\begin{framed}
\renewcommand{\arraystretch}{2}
\begin{mathpar}
\begin{array}{c}
\identity \\[1em]

\R[\mathsf{i}{\da}]
    {\Gamma \piq{} \Delta}
    {\Gamma, A \J A, \Delta}
\end{array}
\\
\begin{array}{c@{\quad}c}
\multicolumn{2}{c}{\flow} \\[1em]

% \R[\mathsf{s{-}}]
%     {\piq{S \seq} \mix \Gamma \J \Delta}
%     {\Gamma, (\piq{S}) \J \Delta}
% &
% \R[\mathsf{s{+}}]
%     {\Gamma \J \Delta \mix \piq{\seq S}}
%     {\Gamma \J (\piq{S}), \Delta}
% \\

\R[\mathsf{f{-}{\da}}]
    {\Gamma, I \piq{\Gamma', I \JB \Delta' \sep \cS} \Delta}
    {\Gamma, I \piq{\Gamma' \JB \Delta' \sep \cS} \Delta}
&
\R[\mathsf{f{+}{\da}}]
    {\Gamma \piq{\cS \sep \Gamma' \JB I, \Delta'} I, \Delta}
    {\Gamma \piq{\cS \sep \Gamma' \JB \Delta'} I, \Delta}
\\
\R[\mathsf{f{-}{+}}{\da}]
    {\Gamma, I \J (\Gamma', I \JB \Delta'), \Delta}
    {\Gamma, I \J (\Gamma' \JB \Delta'), \Delta}
&
\R[\mathsf{f{+}{-}}{\da}]
    {\Gamma, (\Gamma' \JB I, \Delta') \J I, \Delta}
    {\Gamma, (\Gamma' \JB \Delta') \J I, \Delta}
\\
\R[\mathsf{f{-}{-}{\ua}}]
    {\Gamma, I, (\Gamma', I \JB \Delta') \J \Delta}
    {\Gamma, (\Gamma', I \JB \Delta') \J \Delta}
&
\R[\mathsf{f{+}{+}{\ua}}]
    {\Gamma \J (\Gamma' \JB I, \Delta'), I, \Delta}
    {\Gamma \J (\Gamma' \JB I, \Delta'), \Delta}
\\
\R[\mathsf{f{-}{+}}{\ua}]
    {\Gamma, I \J (\Gamma', I \JB \Delta'), \Delta}
    {\Gamma \J (\Gamma', I \JB \Delta'), \Delta}
&
\R[\mathsf{f{+}{-}}{\ua}]
    {\Gamma, (\Gamma' \JB I, \Delta') \J I, \Delta}
    {\Gamma, (\Gamma' \JB I, \Delta') \J \Delta}
\\
\R[\mathsf{f{-}{-}{\da}}]
    {\Gamma, I, (\Gamma', I \JB \Delta') \J \Delta}
    {\Gamma, I, (\Gamma' \JB \Delta') \J \Delta}
&
\R[\mathsf{f{+}{+}{\da}}]
    {\Gamma \J (\Gamma' \JB I, \Delta'), I, \Delta}
    {\Gamma \J (\Gamma' \JB \Delta'), I, \Delta}
\end{array}
\and
\begin{array}{cc}
\multicolumn{2}{c}{\membrane} \\[1em]

\multicolumn{2}{c}{
\R[\mathsf{p}]
    {\Gamma \piq{\cS} \Delta}
    {\Gamma \piq{\cS \sep \piq{}} \Delta}
} \\
\R[\mathsf{p{-}}]
    {\Gamma \piq{} \Delta}
    {\Gamma, (\piq{}) \J \Delta}
&
\R[\mathsf{p{+}}]
    {\Gamma \piq{} \Delta}
    {\Gamma \J (\piq{}), \Delta}
\\

\multicolumn{2}{c}{
\R[\mathsf{a}]
    {\Gamma \piq{S} \Delta}
    {\Gamma \piq{\piq{S}} \Delta}
} \\
\R[\mathsf{a{-}}]
    {\Gamma, S \J \Delta}
    {\Gamma, (\piq{S}) \J \Delta}
&
\R[\mathsf{a{+}}]
    {\Gamma \J S, \Delta}
    {\Gamma \J (\piq{S}), \Delta}
\\
\end{array}
\\
\begin{array}{c@{\quad}c}
\multicolumn{2}{c}{\heating} \\[1em]

\R[\top{-}]
    {\Gamma \J \Delta}
    {\Gamma, \top \J \Delta}
&
\R[\top{+}]
    {\Gamma  \piq{} \Delta}
    {\Gamma \J \top, \Delta}
\\
\R[\bot{-}]
    {\Gamma \piq{} \Delta}
    {\Gamma, \bot \J \Delta}
&
\R[\bot{+}]
    {\Gamma \J \Delta}
    {\Gamma \J \bot, \Delta}
\\
\R[\land{-}]
    {\Gamma, A, B \J \Delta}
    {\Gamma, A \land B \J \Delta}
&
\R[\land{+}]
    {\Gamma \J (\piq{\seq A \sep \seq B {}}), \Delta}
    {\Gamma \J A \land B, \Delta}
\\
\R[\lor{-}]
    {\Gamma \J (\piq{A \seq \Delta \sep B \seq \Delta})}
    {\Gamma, A \lor B \J \Delta}
&
\R[\lor{+}]
    {\Gamma \J A, B, \Delta}
    {\Gamma \J A \lor B, \Delta}
\\
\R[{\limp}{-}]
    {\Gamma \J (\piq{\seq A \sep B \seq \Delta}), \Delta}
    {\Gamma, A \limp B \J \Delta}
&
\R[{\limp}{+}]
    {\Gamma \J (A \seq B), \Delta}
    {\Gamma \J A \limp B, \Delta}
\\
\R[{\lsub}{-}]
    {\Gamma, (A \seq B) \J \Delta}
    {\Gamma, A \lsub B \J \Delta}
&
\R[{\lsub}{+}]
    {\Gamma, (\piq{\Gamma \seq A \sep B\seq}) \J \Delta}
    {\Gamma \J A \lsub B, \Delta}
\\
\R[\forall{-}]
    {\Gamma, \subst{A}{t}{x} \J \Delta}
    {\Gamma, \forall x. A \J \Delta}
&
\R[\forall{+}]
    {\Gamma \J A, \Delta}
    {\Gamma \J \forall x. A, \Delta}
\\
\R[\exists{-}]
    {\Gamma, A \J \Delta}
    {\Gamma, \exists x. A \J \Delta}
&
\R[\exists{+}]
    {\Gamma \J \subst{A}{t}{x}, \Delta}
    {\Gamma \J \exists x. A, \Delta}
\end{array}
\end{mathpar}

In the {\rnm{\forall{+}}} and {\rnm{\exists{-}}} rules, $x$ is not free in
$\Gamma$, $\Delta$ and $\J$.
\end{framed}

%   \caption{Rules for the decomposable bubble calculus \kl{B_{dec}}}
%   \labfig{sequent-B-dec}
% \end{figure*}

% \todo{Talk about factorizability}

% \todo{ Tradeoff between perfectly local rules, where many are restricted to
%   open solutions, and factorizing rules that are uniformly applicable to any
%   kind of solution, but are neither local nor linear (because they rely on
%   duplicating the abstract proof in all subgoals) }

% \todo{IDEA: add (duplicating variants of) the {\rnm{pl_2}} and {\rnm{pr_2}}
% rules of \cite{clouston-annotation-free-2013} into the calculus. Indeed, they
% allow taking red items outside of red bubbles: thus if the proof can be made
% outside with a smaller context, it is more general and immediately solves all
% subgoals, improving \emph{factorizability}.}


\end{scope}