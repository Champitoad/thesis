\setchapterpreamble[u]{\margintoc}
\chapter{Introduction}
\labch{intro}

\epigraph{The ultimate meaning of logic is this ability to manipulate.}
{\textbf{Jean-Yves Girard}, \textit{The blind spot}, 2011}

% \paragraph{Formal manipulations}

\intro{Proof assistants} --- also called \reintro{interactive theorem provers}
(\reintro{ITPs}) --- are software systems that allow to both create and check
the correctness of mathematical proofs. They are based on the idea that
mathematical knowledge can be represented unambiguously inside \intro{proof
formalisms} --- also called \reintro{proof systems}, where the truth of a
statement can be reduced to the mechanical application of symbolic manipulation
rules. For instance, consider the equation
$$4x + 6x = (12 - 2)x$$ 
While any mathematician would immediately recognize it as true, a middle school
student learning algebra would have to carry manually some computations to
convince herself (and her teacher) of its validity. A first step might consist
in applying the distributivity of multiplication over addition on the left-hand
side of the equation, yielding the new equation
$$(4 + 6)x = (12 - 2)x$$
Then, computing the sum on the left-hand side and the difference on the
right-hand side gives the final equation
$$10x = 10x$$
which is trivially true. This is a very simple example, but it already shows the
two main aspects of \kl{proof formalisms}: on the one hand, they allow to represent
mathematical statements in a formal language, here that of equations between
linear univariate polynomials; on the other hand, they allow to manipulate
this representation in order to prove the statements, here through term
rewriting rules that transform a valid equation into another valid equation.

Algebra lends itself particularly well to formalization, as it is arguably the
very study of the rules governing symbolic manipulations in mathematics. It also
heavily relies on computations, which explains why it was the target of the
first, and to this day most popular application of computers to mathematics:
computer algebra systems.

However in this thesis, we are interested in improving the usability of \kl{proof
assistants}, which have a much broader scope than computer algebra systems: their
ambition is to enable the formalization on computers of virtually \emph{any}
kind of mathematics. Ultimately, the dream is to provide a platform that helps
humans in creating \emph{new} mathematics: both novel solutions and proofs to
existing problems, and brand new theories involving new types of mathematical
objects. This seemingly disproportionate ambition is not entirely utopic: it is
based on the great discoveries of 19\textsuperscript{th} and
20\textsuperscript{th} century mathematicians and logicians, in the broad
research area now known as \emph{mathematical logic}.

\section{Proof theory}

\subsection{Mathematical logic}

\paragraph{Universal language}

At the dawn of the 20\textsuperscript{th} century, some mathematicians started
to realize that it might be possible to formalize not only specific branches of
mathematics like algebra with their own language, but the \emph{whole} of
mathematics in a single, universal language. This idea was first intuited in the
17\textsuperscript{th} century by Leibniz with his dream of a
\textit{characteristica universalis}, an ideal language in which all
propositions --- mathematical propositions, but also scientific propositions
about the real world, and even metaphysical propositions --- could be expressed
and understood unambiguously by every human. Also, Leibniz introduced the
concept of a \textit{calculus ratiocinator}, a systematic method for determining
the truth of any proposition expressed in the \textit{characteristica
universalis}, providing a definitive and objective way to settle any argument
through simple calculations\sidenote{Leibniz himself might have been inspired by
his predecessor Galileo, who famously declared that ``the universe [...] is
written in the language of mathematics'' \cite{assayer}.}.

\paragraph{Predicate logic and set theory}

The possibility of a universal language for mathematics became credible at the
dusk of the 19\textsuperscript{th} century, thanks to the works of logicians
like Boole, Frege and Peirce on one hand
\sidecite{Boole1854-BOOTLO-4,frege79,peirce_algebra_1885}, and mathematicians
like Cantor and Dedekind on the other hand
\sidecite{07557982-f50c-352d-bd73-3e2bc6403d4f}. The first laid the groundwork
for a formal account of deduction that greatly improved on Aristotle's
syllogistic, by inventing notations and rules that can express reasoning about
not only \emph{properties} of individuals, but also \emph{relations} between
them. The second invented \emph{set theory}, which provided the first setting
where a general notion of \emph{function} or mapping could be rigorously
defined, a notion that became increasingly central in modern mathematics.

\paragraph{Foundations}

This formed the basis for a unification of many branches of mathematics on the
same \emph{foundation}: it was realized that with enough effort, every
mathematical structure could be encoded with the sets of Cantor, and all the
laws governing sets could be expressed with a finite number of \emph{axioms}
expressed in \intro{predicate logic}, i.e. the language and calculus of relations
devised by 19\textsuperscript{th} century logicians. This crystallized into two
famous axiomatic systems for set theory: the \textit{Principia Mathematica} of
Russell and Whitehead \sidecite{russell25}; and Zermelo-Fraenkel set theory
(\kl{ZF}, or \sys{ZFC} with the axiom of choice), which is the most popular
foundation nowadays because of its greater simplicity.

\paragraph{Truth and proofs}

An axiomatic system specifies the formal language in which statements about
mathematical objects are expressed, as well as a collection of such statements
--- the axioms --- that are taken to be true from the outset, without further
justification. One does not even need to speak about \emph{truth} to define the
system: although it can be a guiding intuition when designing the system, the
fact that axioms denote true properties of abstract objects in some
``mathematical universe'' is a particular philosophical stance (platonism),
which has nothing to do with concrete reasoning on the formal representation.

Traditionally, the branch of mathematical logic that tries to model the
``semantic'' content of axioms through the notion of truth is called \emph{model
theory}. In this thesis, we are concerned with the construction of formal proofs
that derive the consequences of axioms by pure ``syntactic'' manipulation,
through the application of so-called \intro{inference rules}. Accordingly, the
branch of mathematical logic studying this activity is called \emph{proof
theory}. We will still do a bit of model theory in a few places
(\refsec{bubbles-soundness}, \refsec{Completeness}), but only as a means to
justify the properties of our syntax. Thus to avoid any unnecessary
philosophical commitment, we will only consider axioms of a given system as
ordinary \emph{assumptions} that can be used in the course of reasoning, without
according any particular status to their truth. This is very much in line with
the \emph{formalist} school of thought in philosophy of mathematics, represented
by the great mathematician and main instigator of proof theory David Hilbert.

\begin{emphpar}
The real focus throughout this thesis is on the \emph{\kl{inference rules}} used to
build (correct) proofs from axioms/assumptions. Those form the theoretical basis
for both the \emph{interactive creation}, and the \emph{automatic checking} of
formal proofs in \kl{proof assistants}. The branch of proof theory concerned with the
study of \kl{inference rules} is called \emph{structural proof theory}.
\end{emphpar}

\paragraph{Axiomatic systems} 

In the very beginnings of proof theory in the 1920s, under the influence of
Hilbert, the axiomatic method was predominant, and thus \kl{proof systems} of this
era --- now called \intro{Hilbert systems} --- featured very few \kl{inference
rules}. Almost all logical reasoning principles were encoded as \emph{axiom
schemas} involving generic \emph{propositional variables}. For instance, the
famous \emph{law of excluded middle}, that states that every proposition is
either true or false, is expressed formally by the schema
$$A \lor \neg A$$
Here, $\lor$ and $\neg$ are symbols denoting the logical connectives of
\emph{disjunction} (``or'') and \emph{negation} (``not''), and $A$ is a
propositional variable that can be substituted with any concrete \emph{formula}
built from \emph{atomic propositions} and logical connectives. An atomic
proposition is typically a property of a mathematical object, that does not
involve any logical connective. An example of \emph{instance} of this schema
would be the proposition ``$n$ is prime or $n$ is not prime'' with $n$ some
natural number, which can be written formally as
$$\mathrm{prime}(n) \lor \neg\mathrm{prime}(n)$$
Another related principle is the \emph{law of non-contradiction}, which states
that no proposition can be both true and false at the same time. It is expressed
by the schema
$$\neg (A \land \neg A)$$
where $\land$ is the symbol denoting \emph{conjunction} (``and'').

\paragraph{Intuitionistic logic}

One motivating factor in the development of a new foundation for mathematics was
the discovery of strange theorems that defy intuition, like the existence of the
Weierstra{\ss} function which is continuous everywhere but differentiable
nowhere \sidecite{weirstrass_function}, or the Banach-Tarski paradox which
asserts that a ball can be decomposed and reassembled into two exact copies of
itself \sidecite{banach_sur_1924}. Some mathematicians like Brouwer and Weyl
rejected the truth of such theorems, on the basis that their proofs rely on
reasoning principles that are not \emph{constructive}\sidenote{This is true for
the Banach-Tarski paradox, which relies crucially on non-constructive
definitions of the concepts of partitions and equivalence classes \cite{175699}.
But for the Weierstra{\ss} function, it is possible to give it a constructive
definition with some efforts \cite{439047}.}. In particular, these principles
allow to prove the existence of objects satisfying certain properties without
ever providing a \emph{witness}, i.e. a concrete object that satisfies the
properties in question. This marked the birth of \emph{constructivism} in
philosophy of mathematics, whose most famous incarnation is Brouwer's
\emph{intuitionism}.

The original intuitionism of Brouwer was strongly opposed to any attempt at
formalizing mathematics, standing against both Frege and Russell's logicism that
saw mathematics as a mere branch of logic, and Hilbert's formalism that reduced
mathematics to a game of symbol manipulation. However, this did not prevent
Heyting, one of Brouwer's students, from developing an axiomatic system in the
style of Hilbert and Frege, in an attempt to capture formally the objections of
Brouwer towards \intro{classical} logic --- i.e. the logic developed by
19\textsuperscript{th} century logicians that was at the heart of the new
set-theoretical foundations. Heyting's system captures what is now called
\emph{\intro{intuitionistic} logic}, which can be succinctly summarized as being
exactly \kl{classical} logic, but \emph{without} the law of excluded middle.
Thus \kl{intuitionistic} logic is a generalization of \kl{classical} logic,
where propositions cannot be assigned a truth value \emph{a priori}: they are
only considered true if they can be proved with \emph{direct}, constructive
evidence.

To this day, there is no consensus among mathematicians as to which logic ---
\kl{intuitionistic} or \kl{classical} --- is the right one to found mathematics upon.
Since \kl{intuitionistic} logic is more restrictive than \kl{classical} logic, some
fundamental theorems of \kl{classical} mathematics do not hold anymore, requiring in
the worst cases to recreate entire branches of mathematics from scratch, like in
\emph{constructive analysis}. This explains why a large majority of
mathematicians still work in \kl{classical} logic, and are often even unaware of the
existence of constructive mathematics.

\begin{emphpar}
To account for this diversity, in this thesis we design \kl{proof systems} that
support \emph{both} \kl{classical} and \kl{intuitionistic} reasoning. Because every
theorem of \kl{intuitionistic} logic is also a theorem of \kl{classical} logic (but not
the converse), we will often focus first on the \kl{intuitionistic} ``kernel'' of our
systems, designing the \kl{classical} part as an extension of the former.
\end{emphpar}

\subsection{Structural proof theory}

\paragraph{Inference rules}

In \kl{Hilbert systems}, the only \kl{inference rule} is that of \emph{modus ponens},
which is expressed formally with the following figure:
$$\R[\intro{mp}]{A}{A \limp B}{B}$$
Like axioms, it is a \emph{schema} that involves generic propositional variables
$A$ and $B$, which may be \emph{instantiated} with arbitrary formulas. It can be
read from top to bottom as follows: for any propositions $A$ and $B$, if we have
a proof of $A$ and a proof of $A \limp B$, i.e. a proof that $A$ implies $B$,
then we can immediately derive a proof of $B$ by virtue of the rule, here
designated by the abbreviated name \kl{mp}.This reading of the rule
corresponds to a form of \intro{forward} reasoning: starting from the known
\emph{premises} that $A$ and $A \limp B$ are true, it \emph{necessarily} follows
that the \emph{conclusion} $B$ is true.

Conversely, one can also have a bottom-up reading of the rule: to build a proof
of any proposition $B$, one way to proceed is to come up with another
proposition $A$ such that both $A$ and $A \limp B$ are provable. This reading
corresponds to a form of \intro{backward} reasoning: we start from the conclusion
$B$ that we want to reach, also called the \intro{goal}, and try to find
\intro{subgoals} $A$ and $A \limp B$ that are provable, and hopefully simpler to
prove; then the rule guarantees that proving these subgoals is \emph{sufficient}
to ensure the truth of the original goal.

\kl{Forward} reasoning is typically how mathematicians write (informal) proofs on
paper, for the \emph{presentation} of their proofs to other mathematicians.
Indeed, it is more natural for humans to follow an argument by starting from its
premises, because the latter will always contain all the information required to
deduce the conclusion, the argument only serving as a means to explicate how
this information is combined. On the other hand, \kl{backward} reasoning is more
natural during the \emph{construction} phase of a proof, because the information
required to reach the conclusion (e.g. the proposition $A$ in the \rsf{mp} rule)
is not yet known.

\paragraph{Natural deduction}

\kl[Hilbert system]{Axiomatic systems} can be relatively concise, in that many
logics can be expressed in them with a small number of axioms. In return, they
produce very long and verbose formal proofs that are hard for humans to follow,
and almost impossible to come up with in most cases. In a series of seminars
started in 1926, the Polish logician Łukasiewicz became one of the first to
advocate for a more \emph{natural} approach in proof theory, that models more
closely the way mathematicians actually reason
\sidecite{Jaskowski1934-JAKOTR-4}. A few years later, in a dissertation
delivered to the faculty of mathematical sciences of the University of Göttingen
\sidecite{gentzen_untersuchungen_1935}, the German logician Gerhard Gentzen
proposed independently his famous calculus of \intro{natural deduction}.

This formalism follows the opposite approach to \kl{Hilbert systems}: it
features as few axioms as possible, favoring the use of \emph{\kl{inference
rules}} to model the forms of reasoning found in mathematical practice. Those
are divided into two categories: \emph{introduction} rules \emph{define} the
meaning of logical connectives, by prescribing how to prove complex formulas
from proofs of their components. Dually, \emph{elimination rules} explain how to
\emph{use} complex formulas, by giving a canonical way to derive new conclusions
from them. \reffig{calculi-NJ} shows the complete set of \kl{natural deduction}
rules for all connectives and quantifiers in \kl{intuitionistic} logic, that was
introduced by Gentzen under the name \intro{NJ}\sidenote{Gentzen simultaneously
introduced a \kl{natural deduction} calculus named \intro{NK} for \kl{classical}
logic, which is just \kl{NJ} with an additional rule modelling the principle of
\emph{indirect proof} --- i.e. the possibility to prove any proposition $A$ by
deriving a contradiction from its negation $\neg A$ (\reffig{NK-ip}). Indeed,
this principle can be shown to be strictly equivalent to the law of excluded
middle, in the sense that the former is (\kl{intuitionistically}) provable if
and only if the latter is.} \cite{gentzen_untersuchungen_1935}. The most simple
example can be found in the rules for the conjunction connective $\land$: the
introduction rule \rsf{\land i} allows to build a proof of $A \land B$ by
combining a proof of $A$ and a proof of $B$; while the elimination rules
\rsf{\land e_1} and \rsf{\land e_2} allow to derive proofs of $A$ and $B$ from a
proof of $A \land B$.

\begin{figure*}
    \begin{framed}
  \begin{mathpar}
  \R[\rsf{\bot e}]
  {\bot}
  {A}
  \and
  \prftree[r][l]{\rsf{{\limp}i}}{\small[\prfref<HA>]}
  {\summ
    {\Gamma, \prfboundedassumption<HA>{A}}
    {B}}
  {A \limp B}    
  \and
  \R[\rsf{{\limp}e}]
  {A}
  {A \limp B}
  {B}
  \\
  \R[\rsf{\land i}]
  {A}{B}
  {A \land B}
  \and
  \R[\rsf{\land e_1}]
  {A \land B}
  {A}
  \and
  \R[\rsf{\land e_2}]
  {A \land B}
  {B}
  \\
  \R[\rsf{\lor i_1}]
  {A}
  {A \lor B}
  \and
  \R[\rsf{\lor i_2}]
  {B}
  {A \lor B}
  \and
  \prftree[r][l]{\rsf{\lor e}}{\small[\prfref<HA>,\prfref<HB>]}
  {A \lor B}
  {\summ
    {\Gamma, \prfboundedassumption<HA>{A}}
    {C}}
  {\summ
    {\Delta, \prfboundedassumption<HB>{B}}
    {C}}
  {C}
  \\
  \R[\rsf{\forall i}]
  {\summ{\Gamma}{A}}
  {\forall x. A}
  \and
  \R[\rsf{\forall e}]
  {\forall x. A}
  {\subst{A}{t}{x}}
  \and
  \R[\rsf{\exists i}]
  {\subst{A}{t}{x}}
  {\exists x. A}
  \and
  \R[\rsf{\exists e}]
  {\exists x. A}
  {\summ{\Gamma, A}{C}}
  {C}
  \vspace{1em}
  \end{mathpar}
  In the rules \rsf{\forall i} and \rsf{\exists e}, $x$ must not occur free in
  $\Gamma$ and $C$.
  \end{framed}

  \caption{\kl{Natural deduction} calculus \kl{NJ} for \kl{intuitionistic} logic}
  \labfig{calculi-NJ}
\end{figure*}

\begin{marginfigure}
  $$
  \prftree[r][l]{\rsf{ip}}{\small[\prfref<H1>]}
  {\summ
    {\Gamma, \prfboundedassumption<H1>{\neg A}}
    {\bot}}
  {A}    
  $$
  \caption{Rule of indirect proof in \kl{natural deduction}}
  \labfig{NK-ip}
\end{marginfigure}

\begin{remark}
  Note that the rules for negation $\neg$ are not present in
  \reffig{calculi-NJ}: indeed, it is customary in \kl{intuitionistic} logic to define
  negation by $\neg A \defeq A \limp \bot$, identifying the negation of any
  proposition $A$ with its implying of a contradiction. Thus the rules for
  negation are subsumed by those for implication $\limp$ and absurdity $\bot$.
\end{remark}

\begin{marginfigure}
  $$
  \prftree[r][l]{\rsf{{\limp}i}}{\small[\prfref<H1>]}
    {\R[\rsf{{\limp}e}]
      {\R[\rsf{\land e_1}]
        {\prfboundedassumption<H1>{A \land \neg A}}
        {A}}
      {\R[\rsf{\land e_2}]
        {\prfboundedassumption<H1>{A \land \neg A}}
        {\neg A}}
      {\bot}}
    {\neg (A \land \neg A)}
  $$
  \caption{Proof of the principle of non-contradiction in \kl{natural deduction}}
  \labfig{PNC-NJ}
\end{marginfigure}

All logical reasoning principles that were \emph{axiomatized} in \kl{Hilbert
systems} can be \emph{derived} in \kl{natural deduction}. For example,
\reffig{PNC-NJ} shows a proof of the principle of non-contradiction, built by
\emph{composing} instances of rules from \reffig{calculi-NJ}. The composition of
a rule instance \rsf{r1} with another rule instance \rsf{r2} simply consists in
using the conclusion of \rsf{r1} as one of the premisses of \rsf{r2}.

\begin{figure*}
    \begin{framed}
  \textsc{Identity}
  \begin{mathpar}
  \R[\rsf{ax}]
  {A \seq A}
  \and
  \R[\rsf{cut}]
  {\Gamma \seq A}
  {\Delta, A \seq C}
  {\Gamma, \Delta \seq C}
  \vspace{1em}
  \end{mathpar}
  \textsc{Structural}
  \begin{mathpar}
  \R[\rsf{ex}]
  {\Gamma, A, B, \Delta \seq C}
  {\Gamma, B, A, \Delta \seq C}
  \and
  \R[\rsf{ctr}]
  {\Gamma, A, A \seq C}
  {\Gamma, A \seq C}
  \and
  \R[\rsf{wkn}]
  {\Gamma \seq C}
  {\Gamma, A \seq C}
  \vspace{1em}
  \end{mathpar}
  \textsc{Logical}
  \begin{mathpar}
  \R[\rsf{\bot L}]
  {\Gamma, \bot \seq C}
  \and
  \R[\rsf{{\limp}L}]
  {\Gamma \seq A}
  {\Delta, B \seq C}
  {\Gamma, \Delta, A \limp B \seq C}
  \and
  \R[\rsf{{\limp}R}]
  {\Gamma, A \seq B}
  {\Gamma \seq A \limp B}
  \\
  \R[\rsf{\land L_1}]
  {\Gamma, A \seq C}
  {\Gamma, A \land B \seq C}
  \and
  \R[\rsf{\land L_2}]
  {\Gamma, B \seq C}
  {\Gamma, A \land B \seq C}
  \and
  \R[\rsf{\land R}]
  {\Gamma \seq A}
  {\Delta \seq B}
  {\Gamma, \Delta \seq A \land B}
  \\
  \R[\rsf{\lor L}]
  {\Gamma, A \seq C}
  {\Delta, B \seq C}
  {\Gamma, \Delta, A \lor B \seq C}
  \and
  \R[\rsf{\lor R_1}]
  {\Gamma \seq A}
  {\Gamma \seq A \lor B}
  \and
  \R[\rsf{\lor R_2}]
  {\Gamma \seq B}
  {\Gamma \seq A \lor B}
  \\
  \R[\rsf{\forall L}]
  {\Gamma, \subst{A}{t}{x} \seq C}
  {\Gamma, \forall x. A \seq C}
  \and
  \R[\rsf{\forall R}]
  {\Gamma \seq A}
  {\Gamma \seq \forall x. A}
  \and
  \R[\rsf{\exists L}]
  {\Gamma, A \seq C}
  {\Gamma, \exists x. A \seq C}
  \and
  \R[\rsf{\exists R}]
  {\Gamma \seq \subst{A}{t}{x}}
  {\Gamma \seq \exists x. A}
  \vspace{1em}
  \end{mathpar}
  In the rules \rsf{\forall R} and \rsf{\exists L}, $x$ must not occur free in
  $\Gamma$ and $C$.
  \end{framed}

  \caption{\kl{Sequent calculus} \kl{LJ} for \kl{intuitionistic} logic}
  \labfig{calculi-LJ}
\end{figure*}

\paragraph{Sequent calculus}

In addition to the constructivists' objections, some doubts were raised by the
discovery of fatal flaws in early attempts at defining foundational axiomatic
systems, the most famous one being the \emph{antinomy} of Russell's paradox
caused by the unrestricted axiom of comprehension in naive set theory. In order
to restore absolute trust in the foundations of (\kl{classical}) mathematics, Hilbert
proposed in the early 1920s to prove \emph{mathematically} the consistency of
the axiomatic system for arithmetic introduced in 1889 by Peano\sidenote{A more
involved axiomatic system was proposed one year earlier by Dedekind
\cite{dedekind_nat}. Less known is that Peirce had already published in 1881 an
equivalent axiomatization of natural numbers \cite{peirce_logic_1881}.}, i.e.
that no contradiction can be derived from Peano's axioms. Indeed, he believed
that every mathematical truth could be derived from the principles of
arithmetic, thus reducing the problem of the consistency of mathematics to that
of arithmetic. Moreover, Hilbert's program was to be carried by \emph{finitist}
means, without resorting to any reasoning principle involving infinite
collections --- which were at the heart of the controversy started by
constructivists. This was the initial impulse for developing proof theory, since
it provided a mathematical definition of ``mathematical proofs'' as
\emph{finite} sequences of symbols satisfying certain properties.

Gentzen's work on \kl{natural deduction} was an integral part of this program, as an
attempt to render the \emph{metamathematics} of proof theory more structured and
elegant. However, he could not devise any argument for consistency in this
framework. He thus set out to devise a new formalism that would be a
reformulation of \kl{natural deduction} with better mathematical properties, such as
\emph{symmetries}. This gave us \intro{sequent calculus}, which is widely
regarded as the cornerstone for most developments in proof theory to this day.
Gentzen introduced simultaneously \emph{two} \kl{sequent calculi}: one for \kl{classical}
logic called \kl{LK}, and one for \kl{intuitionistic} logic called \sys{LJ}. Here we
focus on the \kl{intuitionistic} system \kl{LJ}, whose rules are shown in
\reffig{calculi-LJ}.

\kl{Sequent calculus} is based on the observation that some rules in \kl{natural
deduction} depend crucially on the use of \emph{hypotheses} that appear
``higher'' or earlier in the proof. The prototypical example is the introduction
rule \rsf{{\limp}i} for implication: to prove $A \limp B$, it suffices to prove
$B$ under the assumption that $A$ is true. Then the hypothesis $A$ is
\emph{discharged} by the rule (bracket notation in \reffig{calculi-NJ}), meaning
that the conclusion $A \limp B$ holds \emph{unconditionally}, without the
assumption. In \kl{sequent calculus}, this relation of provability of a
conclusion $C$ under a collection/\intro(sequent){context} of
assumptions/hypotheses $\Gamma$ is captured by the expression $\Gamma
\intro*\seq C$, called a \intro{sequent}. The introduction rule \rsf{{\limp}i}
is then expressed by the so-called \emph{right introduction} rule
\rsf{{\limp}R}, which keeps track of the full \emph{context} of hypotheses by
having sequents as premiss and conclusion, instead of just formulas. Right
introduction rules for other connectives are also obtained straightforwardly
from the corresponding introduction rules in \kl{natural deduction}, by simply
making the contexts of hypotheses $\Gamma$ and $\Delta$ always explicit.

Following the original presentation of Gentzen, contexts are taken to be
\emph{lists} of formulas (``sequenz'' in German), i.e. ordered collections where
repetitions are allowed. Still, we really want to see them as \emph{sets} of
formulas, since it is implicit in mathematical practice that:
\begin{enumerate}
\item the \emph{order} in which hypotheses are listed does not matter;
\item hypotheses may be used \emph{more than once} in a proof (as in
\reffig{PNC-NJ}).
\end{enumerate}
These two conventions are respectively captured by the \emph{structural rules}
\rsf{ex} of \emph{exchange} and \rsf{ctr} of \emph{contraction} in
\reffig{calculi-LJ}. A third structural rule, \rsf{wkn} for \emph{weakening},
accounts for the presence of unused assumptions in some proofs, by allowing the
introduction of new hypotheses at will (with a top-down reading of rules).

\begin{marginfigure}
  $$
  \R[\rsf{{\limp} R}]
  {\R[\rsf{ctr}]
  {\R[\rsf{\land L_1}]
  {\R[\rsf{\land L_2}]
  {\R[\rsf{{\limp} L}]
    {\R[\rsf{ax}]
    {A \seq A}}
    {\R[\rsf{ax}]
    {\bot \seq \bot}}
    {A, \neg A \seq \bot}}
  {A, A \land \neg A \seq \bot}}
  {A \land \neg A, A \land \neg A \seq \bot}}
  {A \land \neg A \seq \bot}}
  {\seq \neg (A \land \neg A)}
  $$
  \caption{Proof of the law of non-contradiction in \kl{sequent calculus}}
  \labfig{PNC-LJ}
\end{marginfigure}

The main difference between \kl{sequent calculus} and \kl{natural deduction} lies in its
splitting of elimination rules into two parts: \emph{left introduction} rules,
and the \emph{cut} rule. As their name indicates, left introduction rules serve
a purpose symmetric to right introduction rules: while the latter define how to
introduce a connective in the conclusion of a sequent, the former define how to
introduce a connective in one of its hypotheses. Then, the only way to
\emph{use} such an hypothesis $A$ is through the \rsf{cut} rule, which erases
$A$ from the context in the conclusion of the rule, by justifying it with the
proof of $A$ given as premiss. The \rsf{cut} rule can also be seen as a
generalization of the \textit{modus ponens} rule of \kl{Hilbert systems},
replacing the logical connective $\limp$ of implication by the ``structural
connective'' $\seq$ of sequents. The remaining \emph{axiom} rule \rsf{ax} is the
only axiom of \kl{sequent calculus}, and is in a sense dual to the \rsf{cut} rule:
while the latter allows justifying a hypothesis by an identical conclusion, the
\rsf{ax} rule allows justifying a conclusion by an identical hypothesis.
\reffig{PNC-LJ} shows a proof of the principle of non-contradiction in \kl{LJ}.

Both the \rsf{cut} and \rsf{ax} rules seem completely trivial. Yet surprisingly,
Gentzen managed to prove a powerful result called alternatively
\textit{Haupstatz}, \emph{fundamental theorem} (of proof theory), or
\emph{cut-elimination}: every provable formula in \kl{sequent calculus} has a
\emph{cut-free} proof, i.e. a proof that does not make use of the \rsf{cut}
rule. Intuitively, it can be understood as a formal justification for the
possibility to \emph{inline} proofs of lemmas, by seeing an instance of
\rsf{cut} on $A$ as a way to invoke the lemma $A$ without duplicating its proof.
Moreover, Gentzen's proof of the Haupstatz is itself constructive: it describes
an \emph{algorithm} for transforming every \kl{sequent calculus} proof into a
cut-free one. Thus the \rsf{cut} rule is said to be \emph{admissible}, since any
provable sequent can be proved without it.

An important consequence of cut-elimination, which was the original motivation
of Gentzen, is the consistency of the logic (\kl{intuitionistic} \kl{predicate logic} in
the case of \kl{LJ}). This stems from the fact that all rules apart from the
\rsf{cut} rule satisfy the \emph{subformula property}: every formula $A$
appearing in the premisses is a \emph{subformula} of some formula $B$ in the
conclusion, i.e. $A$ already appears inside $B$. Thus there cannot exist a proof
of the absurd sequent $\seq \bot$, since the only formula that is a subformula
of $\bot$ is $\bot$ itself, and there is no rule instance with $\seq \bot$ as
conclusion that only contains $\bot$ in its premisses. The subformula property
is the first occurrence of the concept of \emph{analyticity} in proof theory,
and can be seen as a technical realization of the philosophical notion of
analyticity first applied to propositions by Kant, and later to proofs in
mathematics by Bolzano \sidecite{bolzano}.

Unfortunately, the proof of cut-elimination for the \kl{sequent calculus}
incorporating Peano's axioms, found by Gentzen a few years after proving
cut-elimination for \kl{LJ} \sidecite{gentzen_widerspruchsfreiheit_1936}, is
not finitist: it makes use of a transfinite induction up to the ordinal
$\epsilon_0$. But the very ideas of cut-elimination and analyticity will have
far-reaching applications in proof theory and beyond, including many of the
results presented in this thesis.

\paragraph{Deep inference}

Many years after Gentzen's seminal work, at the advent of the
21\textsuperscript{th} century, Alessio Guglielmi introduced a new methodology
for designing \kl{proof formalisms} called \intro{deep inference}
\sidecite{Guglielmi1999ACO}. The idea was to overcome some limitations of
Gentzen formalisms while preserving their good properties, by allowing
\kl{inference rules} to be applied \emph{anywhere} inside formulas, instead of
only at the top-level of sequents\sidenote{In fact, Schütte had already proposed
a \kl{deep inference} system as early as 1977 \cite{schutte_proof_1977}, as did
Peirce one century before him with his \emph{entitative graphs} (as we will see
in \refch{eg}). But the idea did not generate much interest at the time.}. The
first \kl{deep inference} system was the \intro{calculus of structures}
(\reintro{CoS}), which can be succinctly described as a \emph{rewriting system}
on formulas. For instance, the following \emph{switch} rule, when read bottom-up
(i.e. in \kl{backward} mode), indicates that the formula $A \lor (B \land C)$
may be rewritten into $(A \lor B) \land C$:
$$\R[\rsf{s}]{S\select{(A \lor B) \land C}}{S\select{A \lor (B \land C)}}$$
Importantly, the rule can be applied in any \emph{context} $S\hole$. A context
$S\hole$ is simply a formula containing a single occurrence of a special
subformula $\hole$ called its \emph{hole}, which can be \emph{filled} (i.e.
substituted) with any formula $A$ to give a new formula $S\select{A}$. This
notion of context serves two purposes:
\begin{itemize}
  \item it formalizes the ability of rewriting rules to be applied at an
  arbitrary \emph{depth} inside expressions, while retaining all the information
  available in the surrounding context;
  \item it generalizes the contexts $\Gamma, \Delta$ of \kl{sequent calculus}, by
  giving them the full structure of formulas instead of just flat lists of
  formulas. Indeed, a sequent $\Gamma \seq C$ can be interpreted as the formula
  $\bigwedge \Gamma \limp C$, where $\bigwedge \Gamma$ denotes the conjunction
  of all the formulas in $\Gamma$.
\end{itemize}
Then, a proof of a formula $A$ in the \kl{calculus of structures} is not a
\emph{tree} of rule instances as in \kl{natural deduction} and \kl{sequent calculus}, but
a \emph{sequence} of rewritings $A \steps{} \top$ that reduces $A$ to the
trivially true goal $\top$.

The \kl{calculus of structures}, as a rewriting system, is closer to the equational
reasoning that mathematicians are accustomed to in algebra. The main difference
is that most rules (including the switch rule \rsf{s}) can only be applied in a
\emph{single} direction, because the premiss and conclusion are not
\emph{equivalent}\sidenote{A better analogy would be with the rewriting rules
that are sometimes used to solve \emph{inequations} between numbers.}. When the
premiss and conclusion of a rule are equivalent, we say that the rule is
\intro{invertible}.

\begin{emphpar}
\emph{All} the \kl{proof formalisms} designed in this thesis are \kl{deep inference}
rewriting systems in the style of the \kl{calculus of structures}.
\end{emphpar}

\section{Proof assistants}

\kl{Proof assistants} are a direct application of proof theory, exploiting the
ability of programming languages to represent and manipulate arbitrary data
structures to give a concrete implementation of \kl{proof formalisms}. Crucially,
they open the possibility to \emph{automate} the construction and verification
of formal proofs, by acting on two fronts:
\begin{itemize}
  \item on the \textbf{human} side, the design of \emph{high-level interfaces}
  for representing and manipulating statements and proofs can bridge the gap
  between the low-level and very detailed proofs of formal logic, and the
  informal proofs of mathematicians;
  \item on the \textbf{machine} side, the design of \emph{algorithms} that both
  find proofs of given statements and ensure their correctness can --- to some
  extent\sidenote{For instance, it is well-known that the problem of provability
  in \kl{predicate logic} is \emph{undecidable}.} --- relieve mathematicians from the
  burdens of proof-writing and proof-checking.
\end{itemize}
Thus the advent of computers gave a new purpose to proof theory, going beyond
its foundational role with the hope to support and change the everyday practice
of mathematicians.

\begin{emphpar}
In this thesis, we are concerned mostly with the \emph{human} side of the
equation: we aim to provide smoother means for the user to communicate her
intent to the \kl{proof assistant}, and conversely for the \kl{proof assistant}
to communicate its results and suggestions on how to solve problems.
\end{emphpar}

\subsection{Logical frameworks}

\paragraph{Type theory}

The ancestor of all \kl{proof assistants} was the Automath project, initiated by
Nicolaas Govert de Bruijn as soon as 1967. Citing Geuvers
\sidecite{geuvers_proof_2009}:
\begin{quote}
[One] aim of the project was to develop a mathematical language in which all of
mathematics can be expressed accurately, in the sense that linguistic
correctness implies mathematical correctness. This language should be computer
checkable and it should be helpful in improving the reliability of mathematical
results.
\end{quote}
Thus the design of Automath was focused on the automatic \emph{verification} of
proofs through \emph{linguistic} means. It introduced many fundamental ideas
that are still at work in modern \kl{proof assistants}, the most prominent being the
use of a \emph{type theory} to encode formal proofs.

Contrary to \kl{predicate logic} in traditional proof theory, type theories break the
syntactic hierarchy imposed upon mathematical objects, propositions and proofs,
by giving them a uniform representation as so-called \emph{terms} that can be
assigned a \emph{type}. The assertion that a term $t$ has type $T$ is usually
written with the expression $t : T$, which has come to be called a \emph{typing
judgment} after Martin-Löf. For instance, the judgment $3 : \nats$ states that
the term $3$ has the type $\nats$ of natural numbers, and $1 + 1 = 2 : \Prop$
states that the term $1 + 1 = 2$ has the type $\Prop$ of propositions.

\paragraph{First-order vs. higher-order}

Almost all type theories are \emph{higher-order}: they give a first-class status
to functions and predicates, by allowing them to take other functions and
predicates as arguments. This is because they are based on the
\emph{$\lambda$-calculus} of Alonzo Church
\sidecite{15897363-af72-3dac-82e6-fde144ad66c0}, an intensional theory of
higher-order functions that is now considered to be the first functional
programming language in history. For instance in \emph{simply-typed}
$\lambda$-calculus \sidecite{e4d9a073-5bba-3a5b-90d3-81832cd433e5}, one may type
the \emph{sum operator} over sequences of natural numbers with the judgment
$\lambda u. \lambda n. \sum_{i = 1}^{n}{u~i} : (\nats \to \nats) \to \nats \to
\nats$, where $\lambda u. \lambda n. \sum_{i = 1}^{n}{u~i}$ is a
\emph{$\lambda$-term} encoding the higher-order function that takes a sequence
represented as a function $u : \nats \to \nats$ and a bound $n : \nats$, and
returns the sum of each of $u$'s values at index $1 \leq i \leq n$ encoded as
the \emph{function application} $u~i$.

By contrast, the \kl{predicate logic} developed in the 19\textsuperscript{th} century
and studied in traditional proof theory is \emph{first-order}: functions and
predicates can only take so-called \emph{first-order individuals} as arguments,
which usually model ``non-functional'' mathematical objects like numbers and
sets.

\begin{emphpar}
In this thesis, we exclusively study \kl{proof formalisms} for \emph{first-order}
\kl{predicate logic} (\intro{FOL} hereafter).
\end{emphpar}

We identified a few reasons for working in \kl{FOL}:
\begin{itemize}
  \item it is a standard and well-understood setting that has received a lot of
  attention, allowing us to exploit various existing works from the structural
  proof theory literature, and even from some overlooked theories of
  19\textsuperscript{th} century logicians;
  \item by contrast, type theory is a quite recent subject\sidenote{Almost 100
  years younger than \kl{FOL} if we ignore Russell's theory of types (1902), that was
  based on a set theory encoded in \kl{FOL}.}, which explains why there is still a
  great diversity of type theories that differ in subtle and often incompatible
  ways;
  \item \kl{FOL} is a common kernel of virtually every type theory, making our work
  directly applicable to all present and future \kl{proof assistants};
  \item it is also a simpler setting, that is powerful enough to study the
  essential features of logical reasoning, without the idiosyncracies of type
  theories aimed at capturing the full complexity of mathematics.
\end{itemize}

\paragraph{Curry-Howard correspondence}

De Bruijn came up with the revolutionary idea that propositions could themselves
be seen as types, by having judgments such as $t : 1 + 1 = 2$ where $t$ is a
\emph{proof term} representing a proof of the proposition $1 + 1 = 2$. This
\emph{propositions-as-types} principle was rediscovered independently by Howard
in 1978 \sidecite{Howard1980-HOWTFN-2}, and developed further into a
\emph{proofs-as-programs} correspondence --- also called Curry-Howard
correspondence or isomorphism\sidenote{In fact, Curry had already noticed in
1958 a similar connection between the types of combinators in his
\emph{combinatory logic}, and the axioms of \kl{Hilbert systems} for implication
\cite{Curry1959-CURCLV} --- hence the mention of Curry.} --- where
$\lambda$-terms in the simply-typed $\lambda$-calculus are put in one-to-one
correspondence with proofs in the implicational fragment of the \kl{natural
deduction} system \kl{NJ} of Gentzen (\reffig{calculi-NJ}).

Thus the core of type theory is \emph{intuitionistic} in nature, and the
Curry-Howard correspondence has fostered many fruitful interactions between
computer science, logic and constructive mathematics, with \kl{proof assistants}
acting as a crucial tool and source of investigations. One influential
development in this direction has been the \intro{intuitionistic type theory} of
Martin-Löf, which formed the basis for the implementation of many \kl{proof
assistants} like NuPrl, Alf, and most recently Agda \cite{geuvers_proof_2009}. A
system closely related to \kl{intuitionistic type theory}, the \emph{Calculus of
Inductive Constructions} (\kl{CoIC}), is also implemented in two leading \kl{proof
assistants}: Coq \sidecite{the_coq_development_team_2022_7313584} and Lean
\sidecite{10.1007/978-3-030-79876-5_37}. Following the proofs-as-programs
correspondence, these systems support the creation of both proofs and programs
that manipulate and compute mathematical objects, by compiling everything down
to typed terms.

\subsection{Interfaces}

\paragraph{Elaboration}

Type theories are the logical foundation for the \emph{kernel} of \kl{proof
assistants}, i.e. the part of the system that is responsible for checking formal
proofs expressed in a \emph{terse}, \emph{machine-oriented} format. But it
quickly became clear that this was not enough to make \kl{proof assistants} a viable
alternative to paper proofs: de Bruijn estimates that it takes a time factor of
20 to translate a paper proof into a formalized proof in Automath
\sidecite{7a66996d195e4eba8c3fe5dc5bab8fc5}. This factor has been estimated to
be shrinkable to 4 in the Mizar \kl{proof assistant} \sidecite{debruijn_factor},
thanks to the design of high-level \emph{languages} for representing
mathematical statements and proofs, that sit on top of the core logical
theory\sidenote{In the case of Mizar, a typed set theory based on \kl{FOL}.}. The
process of compiling a high-level proof text into a low-level proof term is
called \emph{elaboration}.

\paragraph{Statement languages}

Any \kl{proof assistant} must provide the two following features in its statement
language:

\begin{itemize}
  \item[\textbf{Logical primitives}] Naturally, one needs a way to write
  propositions formed with logical connectives and quantifiers. This is usually
  done in symbolic form, either with a custom ASCII notation or with Unicode
  characters corresponding to the standard symbols in more modern \kl{proof
  assistants}. 
  \item[\textbf{Mathematical notations}] In addition to the logical primitives,
  one needs to be able to express mathematical objects and operations in the
  domain of interest, e.g. numbers and arithmetic operators. Contrary to the
  logical language that can be hardcoded once and for all in the \kl{proof
  assistant}, the mathematical language needs to be \emph{extensible} by the
  user, so that custom notations can be defined for new mathematical objects.
\end{itemize}

\begin{emphpar}
  In this thesis, we focus exclusively on the \emph{logical primitives}, because
  they are found universally in all types of mathematical reasoning. 
\end{emphpar}

We leave aside the question of providing domain-specific languages for
particular branches of mathematics, which is nonetheless as much important. It
has been tackled extensively in Ayers' thesis \sidecite{ayers_thesis}, and more
specifically in his framework ProofWidgets for user-extensible, interactive
graphical notations in the Lean \kl{proof assistant}\sidenote{For a similar approach
to ProofWidgets in the context of functional programming, see
\cite{omar-filling-2021}.} \sidecite{ayers_graphical_2021}. De Moura and Ullrich
have also designed a powerful macro system for Lean 4, that supports the
elaboration of abstract notations into terms of the underlying type theory
\sidecite{ullrich_beyond_2022}.

\paragraph{Proof languages}

Once one disposes of a convenient way to state mathematical propositions, comes
the question of how to efficiently write \emph{proofs} of these propositions.
There have been broadly two approaches in the design of high-level proof
languages:

\begin{description}[labelsep=0pt]
  \item[Imperative~]proof languages, like imperative programming languages,
  offer a set of \emph{commands} or instructions than can be given to the
  computer to modify some state stored in memory. The latter is called the
  \emph{proof state}, and corresponds to the \emph{partial} proof that is built
  by the system incrementally through the execution of commands. In the dominant
  paradigm, these commands are provided by the user in text form; since Robin
  Milner and the LCF theorem prover~\sidecite{doi:10.1098/rsta.1984.0067}, they
  are called \emph{tactics}. Proof files are literally \emph{proof scripts},
  that is the sequence of tactics typed in by the user.
  
  Contrary to imperative programming languages, the main execution paradigm for
  proof scripts is \emph{interactive}: the user triggers commands one at a time,
  so that she can visualize the intermediate proof states and determine the next
  steps to take. In most tactics-based \kl{ITPs},
  % \sidenote{One notable exception is the Agda proof assistant, where the proof
  % state is assimilated with the whole file being edited, which is itself a
  % partial \emph{program} in a dependently-typed programming language,
  % following the Curry-Howard correspondence.}
  only the \emph{statement} part of the proof state is shown, in a so-called
  \emph{proof view} or \emph{goal view}. This corresponds to the goals that the
  user needs to prove, and each goal is presented in the form of a sequent
  $\Gamma \seq C$, where $C$ is the conclusion that must be reached under the
  assumptions in $\Gamma$. Tactics generally apply to one goal at a time, and
  the user can choose which goal to \emph{focus} at any point during the
  interaction. When the set of goals becomes empty, we say that the initial goal
  or conjecture has been \emph{solved}.

  The transformations performed by tactics can be more or less sophisticated.
  But, fundamentally, one finds elementary commands that correspond roughly to
  the logical rules, generally of \kl{natural deduction} or \kl{sequent calculus}. For
  instance, a goal $\Gamma\seq A\lor B$ (resp. $\Gamma\seq A\land B$) can be
  turned into either a goal $\Gamma\seq A$ or a goal $\Gamma\seq B$ (resp. into
  two goals $\Gamma\seq A$ and $\Gamma\seq B$), corresponding to the rules
  \rsf{\lor R_1} and \rsf{\lor R_2} (resp. \rsf{\land R}) of \kl{LJ} in their
  \kl{backward} reading (\reffig{calculi-LJ}).

  Coq and Lean are examples of state-of-the-art \kl{proof assistants} in the
  imperative paradigm.
  
  \item[Declarative~]proof languages follow a different approach, by
  aiming to provide an \emph{explicit}, \emph{self-contained} description of the
  proof. Although a goal view is still available to guide the user during the
  proof construction process, the finished proof must be readable
  \emph{statically} by a human, without relying on the \kl{ITP} to compute and
  display intermediate proof states. Consequently, declarative proofs are more
  verbose and take longer to type than their imperative homologues. But since
  they contain more information, they have the advantage of being more
  \emph{robust} to slight changes in definitions or to the statement of the
  theorem being proved, and are generally easier to \emph{debug}. They can also
  be put in correspondence with some proof-theoretical formalisms, usually
  Fitch-style \kl{natural deduction} \cite{geuvers_proof_2009}.
  
  Agda, Mizar and Isabelle (with its Isar proof language \sidecite{isar}) are
  examples of state-of-the-art \kl{proof assistants} in the declarative paradigm.
\end{description}

\begin{emphpar}
  In this thesis, we focus mostly on exploring new modalities of interaction in
  the \emph{imperative} paradigm. Only in \refsubsec{flowers-unified} do we
  sketch a possible escape from the imperative/declarative dichotomy.
\end{emphpar}

\begin{remark}
In some rare cases, an additional \emph{natural language} layer is added on top
of the proof language, to be as close as possible to informal proofs. With the
recent development of large language models like GPT, such natural language
translations are becoming increasingly convincing (see e.g. Patrick Massot's
work in Lean \cite{LeanIPAM}).
\end{remark}


\section{This thesis}

\subsection{Research goals}

\paragraph{Interoperability}

Existing \kl{proof assistants} can (almost) never \emph{interoperate}: a proof or
theorem written in one system cannot be imported into a different system. This
is unsatisfactory for many reasons, including fragmentation in the formal
methods community, and duplication of effort instead of reuse. There have been
many attempts in recent years to address this issue in the \emph{backend}:
various pairs of systems are made to exchange formal results through
translations between their input (high-level) and output (low-level) languages,
or in some cases by translations to and from more universal languages (see e.g.
\sidecite{proofcert,dedukti}).

\paragraph{Universal user interface}

The interoperability problem can also potentially be approached from the
\emph{front-end}. Many kinds of logical manipulations such as discharging
assumptions, instantiating quantifiers, and composing lemmas, are conceptually
universal; yet, a user wishing to carry out such manipulations in a particular
\kl{proof assistant} must express the wish in terms of the specific proof language of
the system. A \emph{universal user interface} would instead allow performing
such manipulations directly on the \emph{goal} itself, using physical,
reversible, and incremental actions with immediate feedback. A sequence of user
actions should then be representable in terms of any particular proof language.

\paragraph{Direct manipulation}

This approach to user interfaces has been termed \emph{direct manipulation} by
Shneiderman in the 1980s \sidecite{shneiderman_direct_1983}. It is now at the
heart of virtually every modern user interface and is arguably the major factor
in the \emph{personal computing revolution}. Indeed, it has opened the use of
computing devices to a much wider audience, by allowing users to interact with
them in an intuitive way that resembles interactions with physical objects in
the real world. According to Shneiderman, this contrasts with the
\emph{command-line} paradigm, where users need to memorize a complex command
language that often varies from one software to another within the \emph{same}
domain. This induces an unnecessary cognitive burden for newcomers, especially
those unfamiliar with textual interfaces, which constitute the majority of users
of computing devices nowadays; to the point that the hurdle is too great to
overcome for most potential users.

Unfortunately, the user interfaces of state-of-the-art \kl{proof assistants} are
mostly stuck in the pre-80s era of command-line interfaces, making them reserved
to an audience of highly motivated, computer-savvy individuals. One could argue
that like programming languages, this is due to their inherent
\emph{abstraction} capabilities, that can only be captured through the symbolic
power of language; hence that this state of fact is unavoidable, and can only be
solved through the addition (or improvement) of computer science curricula in
primary and secondary education. We do not agree with this conception: we
believe that the current state of user interfaces in \kl{ITPs} is one of the major
obstacles to their wider adoption in the mathematical community, both by
professional researchers, teachers, and novice students alike.

\begin{emphpar}
  Our first main working hypothesis is that the direct manipulation paradigm is
  not only possible, but also \emph{necessary} for building formal proofs in
  \kl{ITPs}, if they are to become a viable alternative to paper proofs in
  mathematics.
\end{emphpar}

In fact, following the proofs-as-programs correspondence, we also believe that
this applies (to some extent) to \emph{programming}, and that it is only a
matter of time before direct manipulation becomes viable for building
(general-purpose) programs, in the spirit of \emph{visual programming
languages}. We do not explore this direction in this thesis, but it is one of
our hopes that some of the techniques we develop will apply to programming as
well; and one of the reasons why we chose to focus as much on
\emph{intuitionistic} logic.

\paragraph{Graphical deep inference}

A first attempt to design direct manipulation principles for interactive proof
building was made in the 90s by the team of Gilles Kahn at Inria, where they
coined the ``Proof-by-Pointing'' (hereafter ``PbP'') paradigm \sidecite{PbP}.
The idea was to synthesize complex tactics from the simple act of
\emph{pointing} at specific locations inside expressions occurring in the goal,
typically with a mouse cursor. More recently \sidecite{Chaudhuri2013}, Kaustuv
Chaudhuri proposed a variation on this idea termed ``subformula linking'' or
``Proof-by-Linking'' (``SFL'' and ``PbL'' for short), where instead of selecting
expressions in isolation, one can \emph{link} two of them together to make them
interact.

In both cases, the expressions considered were logical formulas and the
associated actions chains of inferences in \kl{FOL}. Importantly, both paradigms rely
on the use of \emph{\kl{deep inference}}, since the user can point at subformulas
that occur at an arbitrary depth inside the goal. This is in contrast with the
basic commands found in the proof languages of \kl{ITPs}, where the user can only
designate formulas appearing at the top level of sequents\sidenote{A notable
exception is the \texttt{rewrite} tactic of imperative proof languages, where
the user can specify \emph{patterns} to designate particular occurrences of a
term $t$ that are to be rewritten into an equal term $u$, usually thanks to an
assumed equation $t = u$. But coming up with such patterns is a lot slower than
directly \emph{pointing} at the locations of interest onscreen.}. While the
semantics of PbP actions is still based on the \emph{shallow} \kl{inference rules} of
\kl{sequent calculus}, PbL fully embraces the \kl{deep inference} paradigm by relying on
\kl{CoS}-style rewriting rules.

\begin{emphpar}
  All the works presented in this thesis can be seen as a direct continuation of
  the research programme initiated by PbP and PbL. The aim is to \emph{replace}
  textual proof languages with \emph{gestural} actions performed directly upon
  \emph{goals} in a \emph{graphical} user interface (GUI). To be as general as
  possible, we call such a paradigm ``\intro{Proof-by-Action}'' (\reintro{PbA}).
\end{emphpar}

\paragraph{Proof exploration}

Note that we believe such a replacement to be useful mostly during the
\emph{construction} or \emph{writing} phase of proofs. Quoting Shneiderman
\cite{shneiderman_direct_1983}:
\begin{quote}
  The pleasure in using these systems stems from the capacity to manipulate the
object of interest directly and to generate multiple alternatives rapidly.
\end{quote}
Thus in the context of \kl{ITPs}, the major advantage of a (well-designed) GUI in the
\kl{PbA} paradigm would be to enable the \emph{rapid exploration} of multiple paths
towards the construction of a complete proof. But once a proof has been found,
the \emph{dynamic} sequence of actions that led to it could be ``compiled'' into
a \emph{static}, textual representation of the proof in the favorite proof
language of the user, to facilitate the \emph{reading} phase.

As for the \emph{modification} phase of proofs, the \kl{PbA} paradigm requires a way
to navigate and edit directly a recorded sequence of actions, and possibly a
mechanism for mapping parts of a static proof text to the actions that generated
them.

\begin{emphpar}
  In this thesis, we leave the question of \emph{proof evolution} in \kl{PbA} ---
  i.e. the design of interfaces for the reading and modification phases, that
  support a smooth interaction with the writing phase --- for future work. A
  more detailed discussion can be found in \refsubsec{proof-evolution}.
\end{emphpar}

\paragraph{Iconicity}

% The dominant paradigm for proof formalisms is \emph{linguistic} in nature:
% mathematical objects and assertions about them are represented in a formal
% language of propositions, also called \emph{formulas}, understood as a (usually
% infinite) set of strings of symbols.

Contrary to a common misconception among logicians, Leibniz did not conceive of
his \textit{characteristica universalis} as a symbolic language, but rather as
an \emph{iconic} one \sidecite[][Chpt.~3]{logique_leibniz}:
\begin{quote}
    The true ``real characteristic'' [...] would express the composition of
    concepts by the combination of signs representing their simple elements,
    such that the correspondence between composite ideas and their symbols would
    be natural and no longer conventional. [...] This shows that the real
    characteristic was for him an ideography, that is, a system of signs that
    directly represent things (or, rather, ideas) and not words.
\end{quote}
This is to be compared to Frege's \textit{Begriffsschrift}, a graphical,
two-dimensional language and calculus of ``pure thought'', whose name has
repeatedly been translated as \emph{ideography}
\sidecite{Frege1952-FRETFT,Frege1999-FRELUL}.

Following the seminal work of Charles Sanders Peirce in
\emph{semiotics}\sidenote{Semiotics is the systematic study of sign processes
and the communication of meaning.}, we define an \intro{iconic} language as one
whose signs are mainly \reintro{icons}, i.e. signs that \emph{resemble} or share
qualities with the objects they denote. This is to be contrasted with
\intro{symbolic} languages where most signs are just \reintro{symbols}, i.e.
signs that \emph{conventionally} denote their objects. In his systematic usage
of \emph{triads} of concepts, Peirce identified a third kind of sign,
\emph{indexes} \sidecite{atkin_peirces_2023}:
\begin{quote}
  if the constraints of successful signification require that the sign reflect
  qualitative features of the object, then the sign is an icon. If the
  constraints of successful signification require that the sign utilize some
  existential or physical connection between it and its object, then the sign is
  an index. And finally, if successful signification of the object requires that
  the sign utilize some convention, habit, or social rule or law that connects
  it with its object, then the sign is a symbol.
\end{quote}
He even went further by analyzing icons into another
trichotomy\sidenote{Actually he only applied this trichotomy to pure icons
devoid of any indexical elements, that he called \emph{hypo-icons}.}:
\emph{images} that depend on simple quality; \emph{diagrams}, who share
\emph{structural} relations among their constituents that are analogous to that
of their object; and \emph{metaphors}, that denote features of their object by
relating them to features of another object \sidecite{legg_problem_2008}.

Interestingly, Peirce held that mathematics relies mostly on \emph{diagrammatic}
thinking --- observation of, and experimentation on, diagrams
\sidecite[][Chpt.~6]{Hookway1985-HOOP-2}. This agrees with the contemporary
practice of mathematics: indeed, there is an increasing number of areas in
mathematics --- the most prominent one being category theory --- where the heart
of a proof lies in the dynamical construction of a \emph{diagram} capturing the
structure of interest in given mathematical objects. The natural language proof
text is often just a means to explicit the meaning or intuition behind the
diagrammatic manipulations, or simply a retranscription of the commentary that
the mathematician would give when unfolding the construction on a blackboard.

If one views logic as one particular type of mathematical reasoning --- albeit
one that is omnipresent in all branches of mathematics, then it is only natural
to expect that some diagrammatic system should exist for it, that can express in
the most natural way most (if not all) logical arguments. This is the
\emph{iconicity} thesis of Peirce, which motivated his inquiry into what is
arguably the first diagrammatic \kl{proof system} in history: the \intro{existential
graphs} (\reintro{EGs}). This will be the subject of \refch{eg}, and the basis
for the development of a \emph{metaphorical} \kl{proof system} in \refch{flowers}.

\begin{emphpar}
  Our second main working hypothesis exploited in the second part of this
  thesis, is that iconic representations of logical \emph{statements}, and the
  proofs that result from their manipulation, can play a crucial role in the
  design of intuitive proof building interfaces.
\end{emphpar}

\subsection{Contributions}

This thesis proposes several contributions toward the research goals highlighted
above.

\paragraph{Symbolic manipulations}
  
In the first part of this thesis, we substantiate the \kl{PbA} paradigm in the
context of traditional representations of goals, by presenting a number of
techniques based on the direct manipulation of \emph{symbolic} formulas in
sequents.

We start in \refch{pba} with an introduction to PbP and PbL, by describing how
to reason with logical connectives, quantifiers and equality through
\emph{click} and \emph{drag-and-drop} (\intro{DnD}) actions in a prototype of
GUI called Actema. In particular, \kl{DnD} actions can be seen as a graphical
generalization of both the \texttt{apply} and \texttt{rewrite} tactics of
imperative proof languages.

In \refch{sfl}, we ground the semantics of \kl{DnD} actions in \kl{deep
inference} proof theory, by designing a variant of the subformula linking
\kl{CoS} of Chaudhuri introduced in \cite{Chaudhuri2013} for \kl{intuitionistic}
\kl{FOL} with equality. Our approach differs mainly from Chaudhuri's through our
notion of \emph{\kl(dnd){valid} \kl{linkage}}, which filters out
\emph{non-productive} \kl{DnD} actions by restricting them to
\emph{\kl{unifiable}} subformulas.

In \refch{advanced}, we present more advanced techniques in the \kl{PbA} paradigm,
that handle practical kinds of reasoning found in mathematical practice such as
the use of \emph{definitions}, reasoning by \emph{induction}, and the
\emph{simplification} of subterms through automatic computation. This is
illustrated through a few case studies of basic logical and mathematical
problems.

In \refch{sfl-classical}, we investigate an extension of \kl{PbA} to sequents with
\emph{multiple} alternative conclusions, as opposed to the
\emph{single}-conclusion sequents found in the interface of almost every \kl{ITP}. We
argue that the use of direct manipulation greatly facilitates the handling of
multiple conclusions, and introduce a so-called \emph{parallel} linking operator
to model reasoning in \kl{classical} logic that involves the interaction of two
conclusions.

Lastly in \refch{plugin}, we present the architecture, interaction protocols and
elaboration/compilation strategy implemented in \texttt{coq-actema}, a plugin
that integrates the Actema web app as an interactive proof view in Coq. We also
discuss its current shortcomings and future directions for improvement, in
particular concerning the question of \emph{proof evolution}.

\paragraph{Iconic manipulations}
  
In the second part of this thesis, we explore a series of \kl{deep inference} \kl{proof
systems} that give more structure to the notion of \emph{logical goal}. These
systems share the ability to represent goals in two alternative ways: either
\emph{textually} through a standard inductive syntax, or \emph{graphically}
through a metaphorical notation well-suited to direct manipulation. The first
can be used as a machine representation in the backend of an \kl{ITP}, and the latter
as the substrate for GUIs in the frontend.

\begin{description}
  \item[Bubble calculi] In the first two chapters, we introduce a family of
  systems called \emph{bubble calculi}. They are an extension of the theory of
  \emph{nested sequents} first introduced by Brünnler
  \sidecite{brunnler_deep_2009}, that we reframe as local rewriting systems with
  a graphical and topological interpretation. Bubble calculi enable an efficient
  sharing of contexts between subgoals, making them well-suited to the
  factorization of both \kl{forward} and \kl{backward} reasoning steps in proofs.

  \refch{bubbles} presents the \emph{asymmetric} bubble calculus \kl{BJ} for
  \kl{intuitionistic} logic, modelled after the \emph{asymmetric} sequents of the
  \kl{intuitionistic} \kl{sequent calculus} \kl{LJ} of Gentzen (\reffig{calculi-LJ}). It
  introduces the metaphor of \emph{bubbles} as a way to iconically represent the
  separation and sharing of contexts between different subgoals.

  \refch{bubbles-symm} refines \kl{BJ} into a more general and symmetric
  calculus for \kl{classical} logic called ``\kl{system B}'', where bubbles can be
  \emph{polarized} in addition to formulas. \kl{Intuitionistic}, \kl{dual-intuitionistic}
  and \kl{bi-intuitionistic} logic can be recovered as fragments of \kl{system B},
  by forbidding certain \kl{inference rules} that characterize the \emph{porosity} of
  bubbles. We also devise a fully invertible variant of \kl{system B}, that we
  conjecture to be complete.
  
  \item[Existential graphs] In the last two chapters, we study two systems based
  on the \kl{existential graphs} of Peirce, that allow us to achieve \emph{full
  iconicity}: every logical construction has an associated icon, and thus there
  is no use anymore for the connectives and quantifiers of symbolic formulas.
  Hopefully, this shall remove a first barrier in the learning of formal logic,
  which lies in the \emph{arbitrary} correspondence between symbols and their
  meaning.

  In \refch{eg}, we give a complete review of the original \kl{EGs} systems of
  Peirce for propositional and first-order \kl{classical} logic, which have been
  consistently neglected in the proof theory literature\sidenote{One reason
  might be that \kl{EGs} have been invented at the end of the
  19\textsuperscript{th} century, \emph{before} the birth of proof theory as a
  discipline in the 1920s under the impulse of Hilbert.}. We propose in
  particular a novel inductive characterization of the syntax of \kl{EGs}, as
  well as the first identification of an \emph{analytic} fragment of the system
  for propositional logic that is complete for provability.

  Finally, we introduce in \refch{flowers} the \emph{flower calculus}, an
  \kl{intuitionistic} variant of \kl{EGs} where statements are represented
  metaphorically as \emph{flowers}. We partition the system into a
  \emph{natural} fragment where every rule is both analytic and invertible, and
  a \emph{cultural} fragment where every rule is non-invertible. We prove that
  the cultural fragment is admissible thanks to a completeness proof for the
  natural fragment with respect to Kripke semantics. We exploit these
  meta-theoretical results to design the Flower Prover, a prototype of GUI in
  the \kl{PbA} paradigm that aims to unify the concepts of \emph{goal} and
  \emph{theory} in a \emph{modal interface}: goals correspond to flowers
  manipulated with natural rules in \Proof mode; while theories correspond to
  the same flowers manipulated with cultural rules in \Edit mode. The Flower
  Prover is also the first \emph{mobile-friendly} interface for \kl{ITPs} that we
  know of.
\end{description}

\begin{kaonote}
Most of the content of \refch{pba} and \refch{sfl} has been previously published
in \cite{10.1145/3497775.3503692}, and the \texttt{coq-actema} system described
in \refch{plugin} is under active development by us and Benjamin Werner
\cite{coq-actema}. A shortened version of \refch{eg} and \refch{flowers} has
been submitted for publication at FSCD 2024 (TODO: cite arXiv). All other
chapters present completely original and personal work.
\end{kaonote}

\subsection{How to read}

\paragraph{Reading order}

The ordering of chapters in this thesis is mostly \emph{chronological},
reflecting the order in which the ideas were developed. For readers interested
in all of the contributions, we thus advise reading all chapters in order.

Still, the investigations into iconic manipulations in the second part started
as an offshoot of those on symbolic manipulations in the first part, and were
carried mostly in parallel. Although we sometimes reference ideas from chapters
in the first part, the second part can thus be read mostly independently from
the first one.

\begin{emphpar}
  In all cases, the reader should start with \refch{pba}, which gives a taste of
  the \kl{PbA} paradigm explored in all other chapters.
\end{emphpar}

\reffig{chapter-deps} shows the precise graph of dependencies between chapters.
Four independent paths can be followed:
\begin{itemize}
  \item[\textbf{The applied road (\ding{175} $\to$ \ding{177})}] If
  you want to see to what extent the \kl{PbA} paradigm can currently be applied for
  practical theorem proving in real \kl{proof assistants}, this is the right path for
  you.

  \item[\textbf{Proof theory of SFL (\ding{174} $\to$ \ding{176})}]
  This path is for readers only interested in the proof theory of subformula
  linking, which is the foundation for the semantics of \kl{DnD} actions on symbolic
  formulas.

  \item[\textbf{Bubble calculi (\ding{178} $\to$ \ding{179})}]
  This path is for readers only interested in the proof theory and potential
  applications of bubble calculi.
  
  \item[\textbf{Flower calculus (\ding{180} $\to$ \ding{181})}] This path is for
  readers only interested in the proof theory of the flower calculus, and its
  applications to automated and interactive theorem proving.
\end{itemize}

A last option is to read only \refch{eg}, skipping even \refch{pba}. This might
be of interest to people looking for an introduction to the existential graphs
of Peirce.

\begin{figure*}
  \stkfig{1}{chapter-deps}
  \caption{Dependency graph between chapters}
  \labfig{chapter-deps}
\end{figure*}

\begin{digression}
  We will sometimes develop ideas loosely related to the main text in
  \emph{digression} boxes such as this one: at least on first reading, they can
  be safely ignored. We distinguish them from normal side notes, which are
  usually shorter and more relevant to the matter at hand.
\end{digression}

\paragraph{Color}

Some parts of this document make a heavy, \emph{semantic} use of colors.
Although all important concepts still have a textual, color-independent
presentation, it is recommended to print this document with a decent amount of
color levels.

\paragraph{Hyperlinks}
  
We tend to cross-reference many ideas from different chapters with the help of
\emph{hyperlinks}. In particular, we use the \kl(package){knowledge} package
from Thomas Colcombet to hyperlink occurrences of concepts to the place where
they are introduced. We thus recommend the usage of a PDF reader that supports
at least hyperlink \emph{jumping}, and if possible hyperlink \emph{preview} for
a more comfortable reading experience.


\section{Related works}\labsec{intro-rw}

\paragraph{Window inference}

Other researchers have stressed the importance of being able to reason
\emph{deep} inside formulas to provide intuitive proof steps. The first and
biggest line of research supporting this idea is probably that of \emph{window
inference}, which started in 1993 with the seminal article of P.J. Robinson and
J. Staples \sidecite{robinson-formalizing-1993}, and slowly became out of
fashion during the 2000s. This is well expressed in the following quote from one
of its main contributors, Serge Autexier \sidecite[][p.~184--187]{autexier_phd}:

\begin{quote}
We believe it is an essential feature of a calculus for intuitive reasoning to
support the transformation of parts of a formula without actually being forced
to decompose the formula. In that respect the \kl{inference rules} of Schütte's proof
theory are a clear contribution. [...] One motivation for the development of the
CORE proof theory was to overcome the need for formula decomposition as enforced
by sequent and \kl{natural deduction} calculi in order to support an intuitive
reasoning style.
\end{quote}

Thus we are not the first to attempt to design new \kl{proof systems} based on \kl{deep
inference} principles, with the explicit objective of improving the usability of
\kl{ITPs}.
\begin{digression}
  Note that during most of the period when window inference was developed, the
terminology of ``deep inference'' had not been introduced yet. Indeed, the first
article on the subject appeared in 1999 \cite{Guglielmi1999ACO}, with very
different motivations in mind: namely, the development of a proof-theoretical
approach unifying concurrent and sequential computation, resulting in the
\kl{calculus of structures} for the logic \kl{BV}. However, some \kl{proof systems} based
on \kl{deep inference} principles already existed and inspired researchers in window
inference, as witnessed by the reference to Schütte's proof theory in the above
quote.
\end{digression}
However, we believe our approach is unique in that it emphasizes two aspects:
\begin{itemize}
  \item the use of \emph{direct manipulation} on goals to perform proof steps
  (although some pointing interactions were already at work in window
  inference-based systems);
  \item in the second part of this thesis, the use of \emph{iconic}
  representations for the proof state, that stray away from traditional symbolic
  formulas.
\end{itemize}

\paragraph{Ayers' thesis}

More recently, Ayers described in his thesis a new tool for producing verifiable
and explainable (formal) proofs, including both theoretical discussions of novel
concepts and designs for components of \kl{proof assistants}, and practical
implementations of software evaluated through user studies
\sidecite{ayers_thesis}. Notable contributions from our point of view are:
\begin{itemize}
  \item his \texttt{Box} development calculus, which introduces a unified
\texttt{Box} data structure representing at the same time goals and partial
proofs, with the aim to offer more ``human-like'' interfaces for both the
\emph{construction} and the \emph{presentation} of proofs;
  \item and his \texttt{ProofWidgets} framework, that allows to extend the Lean
\kl{proof assistant} with new interactive and domain-specific notations for
mathematical objects, thus offering a form of \emph{end-user} programming.
\end{itemize}
The \texttt{Box} data structure easily lends itself to visualization in a
two-dimensional graphical notation, while \texttt{ProofWidgets} promises great
capabilities for proofs by both direct and iconic manipulation.

However, the work of Ayers focuses mainly on designing a general framework that
can integrate modern interfaces for proofs in the Lean \kl{proof assistant}, while we
focus on exploring various proof calculi that provide the foundations for such
interfaces at the purely logical level, mostly independently of any particular
\kl{proof assistant}. Thus we believe that our work is quite complementary to Ayers':
it emphasizes different aspects while sharing a common vision for the future of
\kl{proof assistants}, where modern graphical interfaces play a crucial role in
improving the interaction between the user and the computer.