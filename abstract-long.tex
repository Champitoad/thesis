Proof assistants are software systems that allow for the precise checking of
mathematical reasoning. They can be general purpose (like Coq, Lean,
Isabelle...) or more specialized like EasyCrypt. They enable a level of accuracy
which certifies that no error can occur, but remain difficult to use.

We propose a new paradigm of formal proof construction through actions performed
in a graphical user interface, to enable a more comfortable and intuitive use.
Our paradigm builds upon direct manipulation principles, combining both old
(Proof-by-Pointing) and new (Proof-by-Linking) interaction techniques that
exploit recent advances in deep inference proof theory. We implement this
paradigm in a prototype of graphical user interface called Actema, using modern
web-based technologies. We also design a generic protocol for plugging Actema on
any proof system that supports first-order intuitionistic logic. This protocol
is deployed inside the Coq proof assistant through the coq-actema plugin,
offering users the ability to integrate graphical proofs into existing textual
developments.

Then, driven by the will to improve the various interaction techniques of Actema
in a unified formalism, we explore a series of deep inference proof systems that
give more structure to the notion of logical goal. These systems share the
ability to represent goals in two alternative ways: either textually through a
standard inductive syntax, or graphically through a metaphorical notation
well-suited to direct manipulation.

The first family of systems, called bubble calculi, is an extension of the
theory of nested sequents, that we reframe as local rewriting systems with a
graphical and topological interpretation. Bubble calculi enable an efficient
sharing of contexts between subgoals, making them well-suited to the
factorization of both forward and backward reasoning steps in proofs. The second
system, called flower calculus, is an intuitionistic refinement of C.S. Peirce's
theory of existential graphs, understood as a system for interactive,
goal-directed proof building. It provides more iconic and economical means of
reasoning than bubble calculi, by exposing a small number of expressive rules
that apply to the goals themselves, removing the need for logical connectives.
Both types of systems are shown to be analytic and fully invertible, making them
amenable to proof automation techniques.

We finally go back to practical experimentation by designing and implementing
the Flower Prover, another web-based prototype of GUI for interactive proof
building based on the flower calculus. An innovative feature of the Flower
Prover is that it works well on modern mobile devices, thanks to its responsive
layout and first-class support for touch interactions.