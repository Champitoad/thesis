Les assistants de preuve sont des outils avancés utilisés en mathématiques et en informatique pour garantir la précision du raisonnement. Cependant ils sont difficiles à utiliser, ce qui les réserve à un public d'expert.e.s.

Nous proposons une nouvelle façon de construire des preuves avec une interface graphique conviviale appelée Actema. Actema simplifie le processus en permettant aux utilisat.eur.rice.s d'interagir directement avec les énoncés mathématiques grâce à des actions de clic et de glisser-déposer, évitant ainsi l'utilisation d'un langage de programmation complexe. Nous avons intégré Actema avec Coq, un assistant de preuve populaire, afin que les utilisat.eur.rice.s puissent intégrer facilement des preuves graphiques dans leurs projets.

Pour améliorer la convivialité d'Actema, nous explorons de nouvelles façons de représenter les énoncés mathématiques. Ces représentations, appelées calculs de bulles et calcul des fleurs, offrent des moyens efficaces de gérer et de construire des preuves de manière interactive. Les calculs de bulles se concentrent sur le partage d'informations entre différentes parties d'une preuve, tandis que le calcul des fleurs simplifie le raisonnement en utilisant des diagrammes intuitifs qui ressemblent à des fleurs. Les deux approches garantissent que le raisonnement est logique et précis.

Nous avons également développé le Flower Prover, un prototype d'interface utilisateur graphique basé sur le calcul de fleurs. Il est conçu pour fonctionner efficacement sur les appareils mobiles modernes, le rendant accessible à un public plus large. Notre objectif est de rendre la construction de preuves plus intuitive et accessible à toutes et à tous.