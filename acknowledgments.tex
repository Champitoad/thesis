% \section*{Foreword}

This thesis is the result of a long, challenging and deeply personal journey. It has been equally demanding on both my intellectual and emotional resources, pushing their limits to unforeseen heights. But in return it rewarded me with a great wealth of experiences and insights, nourishing my mind and my soul on an unprecedented scale. It would be no exaggeration to state that it forged me into a fully fledged adult, with my PhD defense acting as a coming of age ceremony\footnote{I have my mother to thank for this remarkable analogy.}.

Thus it is only natural that I write these acknowledgments in the first-person, singular ``I''. While in the rest of this document my individuality is dissolved into the plural, objective and spatio-temporally spread ``we'' of science, here I want to remind the reader of the situatedness and subjectivity involved in the production of knowledge, of its irreducibly human nature. Knowledge only exists because it is \emph{meaningful} to humans, and because humans are meaningful to eachother.

Due to my fleeting memory, I am afraid I can only mention \emph{some} of the people who were instrumental in making my PhD a meaningful endeavor; whether by supporting my work, or inspiring my life (and \textit{vice versa}); either because they believed in the science, or cared for the person that I am (and the other way around).

\section*{The road behind}

People often say they ``fall in love'' with a given subject: I tend to believe they mean it quite literally. At least this is what happened to me 8 years ago, when I discovered formal logic. I instantly knew that it would occupy much of my thoughts, filling my days with intense questions and reflections, and my nights with feverish dreams full of weird symbols and pictures. But I could never imagine having the opportunity to dedicate myself to this ``relationship'' for such an extended time span.

As with many things in life, it is a series of encounters that put me on this peculiar path, the first of which being Mathieu Jaume's logic course in the ``licence d'informatique'' of Université Pierre et Marie Curie (UPMC). The clarity of his teaching instilled precision and rigor in me, two essential qualities for an aspiring logician. I am also grateful to one of his teaching assistant at the time (a PhD student whose name I unfortunately cannot remember), who nurtured my newfound passion for logic through passionate discussions and specially-tailored bonus homeworks. It is also in Mathieu's lab sessions that I discovered \kl{Edukera} (mentioned in \refsec{edukera} and \refsec{flowers-search}), which might have ingrained at a very early stage the idea at the heart of my thesis, that logic is intimately related to (direct) \emph{manipulation}.

The next fateful encounter happened with Pierre Talbot, another teaching assistant at UPMC and a PhD student at IRCAM (Institut de Recherche et de Coordination Acoustique/Musique). I went and talked to him after a lab session for our Compilation course, because I was interested in his research at the intersection of logic programming and music theory. After mentioning my deep interest in logic, he recommended that I take a look into this theory called ``linear logic'', and in particular into a book titled ``Le Point Aveugle'' written by Jean-Yves Girard.

Saying that this tremendously impacted my vision of logic would be an understatement. I found in Girard's works a school of thought, an unexpected echo to my own philosophical reflections about language and reality. It provided a revolutionary framework in which I could deconstruct the notion of ``truth'', something that traditional logic only hinted at in incomplete and unsatisfactory ways. It has really guided and shaped much of my studies and research in the following years, helping me quench my thirst for foundational answers along the way. In particular, it led me to study philosophy for one year at Université Paris 1 Panthéon-Sorbonne, writing my memoir on Girard's ludics under the supervision of Alberto Naibo. The year after I got my first research internship at Université Paris Diderot supervised by Alexis Saurin, working again on a ludics-related subject.

\section*{The road proper}

My PhD journey truly started in 2020 with my 6 months research internship at École Polytechnique under the supervision of Benjamin Werner and Pierre-Yves Strub. When I got wind of their project to create better GUIs for proof assistants, I immediately sent my application. Although it did not involve any linear logic, it was a perfect mix of theoretical and applied research that aligned nicely with one of my broader goals: making programming more correct, more interactive, and overall more accessible to non-specialists. After some very encouraging results exploiting Chaudhuri's \kl{subformula linking} technique \cite{Chaudhuri2013}, Benjamin and Pierre-Yves proposed that I keep working on the subject of ``Graphical Interfaces for Proofs'' for an entire PhD thesis. I cannot thank them enough for giving me this wonderful opportunity.

My deepest gratitude goes to Benjamin, for unwaveringly supporting every idea I came up with during these past 4 years. The freedom of thought he gave me has been invaluable and liberating, allowing me to express my creativity while carving my own path in the scientific landscape. Few advisors take this risk, preferring to use their students as cheap and convenient workforce for their own agenda, or simply being afraid of throwing them in at the deep end. This was not at the price of his supervision however: he always gave me advice and direction when I needed them, and always made an effort to understand my ideas even when they strayed away from his area of expertise. Last but not least, he agreed to extend my PhD scholarship by 4 months so that I could finish the writing of this manuscript. The last months of writing and the few weeks preceding the defense were especially tough mentally, but he was still there, sharing words of encouragement and bits of wisdom. Thank you Benjamin for putting up with my stubbornness, my anxiety and my tears in this final rush. I hope the overall experience has been as much a pleasure for you as it was for me.

I also wish to thank all the members of my fantastic jury for agreeing to read my thesis and to come all the way to Saclay to attend my defense. Special thanks go to my reviewers Nicolas Magaud and Anupam Das for their thorough and thoughtful reports. Receiving these final tokens of recognition filled me with great joy, relief and pride. 