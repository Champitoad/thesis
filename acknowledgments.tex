% \section*{Foreword}

This thesis is the result of a long, challenging and deeply personal journey. It has been equally demanding on both my intellectual and emotional resources, pushing their limits to unforeseen heights. But in return it rewarded me with a great wealth of experiences and insights, nourishing my mind and my soul on an unprecedented scale. It would be no exaggeration to state that it forged me into a fully fledged adult, with my PhD defense acting as a coming of age ceremony\footnote{I have my mother to thank for this remarkable analogy.}.

Thus it is only natural that I write these acknowledgments in the first-person, singular ``I''. While in the rest of this document my individuality is dissolved into the plural, objective and spatio-temporally spread ``we'' of science, here I want to remind the reader of the situatedness and subjectivity involved in the production of knowledge, of its irreducibly human nature. Knowledge only exists because it is \emph{meaningful} to humans, and because humans are meaningful to eachother.

Due to my fleeting memory, I am afraid I can only mention \emph{some} of the people who were instrumental in making my PhD a meaningful endeavor; whether by supporting my work, or inspiring my life (and \textit{vice versa}); either because they believed in the science, or cared for the person that I am (and the other way around).

\section*{The road behind}

People often say that they ``fall in love'' with a given subject: I tend to believe they mean it quite literally. At least this is what happened to me 8 years ago, when I discovered formal logic. I instantly knew that it would occupy much of my thoughts, filling my days with intense questions and reflections, and my nights with feverish dreams full of weird symbols and pictures. But I could never imagine having the opportunity to dedicate myself to this ``relationship'' for such an extended time span.

As with many things in life, it is a series of encounters that put me on this peculiar path, the first of which being Mathieu Jaume's logic course in the ``licence d'informatique'' of Université Pierre et Marie Curie (UPMC). The clarity of his teaching instilled precision and rigor in me, two essential qualities for an aspiring logician. I am also grateful to one of his teaching assistant at the time (a PhD student whose name I unfortunately cannot remember), who nurtured my newfound passion for logic through passionate discussions and specially-tailored bonus homeworks. It is also in Mathieu's lab sessions that I discovered \kl{Edukera} (mentioned in \refsec{edukera} and \refsec{flowers-search}), which might have ingrained at a very early stage the idea at the heart of my thesis, that logic is intimately related to (direct) \emph{manipulation}.

The next fateful encounter happened with Pierre Talbot, another teaching assistant at UPMC and a PhD student at IRCAM (Institut de Recherche et de Coordination Acoustique/Musique). I went and talked to him after a lab session for our Compilation course, because I was interested in his research at the intersection of logic programming and music theory. After mentioning my deep interest in logic, he recommended that I take a look into this theory called ``linear logic'', and in particular into a book titled ``Le Point Aveugle'' written by Jean-Yves Girard.

Saying that this tremendously impacted my vision of logic would be an understatement. I found in Girard's works a school of thought, an unexpected echo to my own philosophical reflections about language and reality. It provided a revolutionary framework in which I could deconstruct the notion of ``truth'', something that traditional logic only hinted at in incomplete and unsatisfactory ways. It has really guided and shaped much of my studies and research in the following years, helping me quench my thirst for foundational answers along the way. In particular, it led me to study philosophy for one year at Université Paris 1 Panthéon-Sorbonne, writing my memoir on Girard's ludics under the supervision of Alberto Naibo. The year after I got my first research internship at Université Paris Diderot supervised by Alexis Saurin, working again on a ludics-related subject.

\section*{The road proper}

My PhD journey truly started in 2020 with my 6 months research internship at École Polytechnique under the supervision of Benjamin Werner and Pierre-Yves Strub. When I got wind of their project to create better GUIs for proof assistants, I immediately sent my application. Although it did not involve any linear logic, it was a perfect mix of theoretical and applied research that aligned nicely with one of my broader goals: making programming more correct, more interactive, and overall more accessible to non-specialists. After some very encouraging results exploiting Chaudhuri's \kl{subformula linking} technique \cite{Chaudhuri2013}, Benjamin and Pierre-Yves proposed that I keep exploring the subject of ``Graphical Interfaces for Proofs'' for an entire PhD thesis. I cannot thank them enough for giving me this life-changing opportunity.

My deepest gratitude goes to Benjamin, for unwaveringly supporting every idea I came up with during these past 4 years. The freedom of thought he gave me has been invaluable and liberating, allowing me to express my creativity while carving my own path in the scientific landscape. Few advisors take this risk, preferring to use their students as cheap and convenient workforce towards their own agenda, or simply being afraid of throwing them in at the deep end. This was not at the price of his supervision however: he always gave me advice and direction when I needed them, and always made an effort to understand my ideas even when they strayed away from his area of expertise. He was also a great collaborator, sharing ideas and making himself available for pair programming sessions on a regular basis. The \kl{coq-actema} plugin would never have seen the light of day without his mastery of the arcane art of \emph{small scale reflection}. Last but not least, he agreed to extend my PhD scholarship by 4 months so that I could finish the writing of this manuscript. The last months of writing and the few weeks preceding the defense were especially tough mentally, but he was still there, sharing words of encouragement and bits of wisdom. Thank you Benjamin for putting up with my stubbornness, my anxiety and my tears in the final rush. I hope the overall experience has been as enjoyable for you as it has been for me.

I also wish to thank all the members of my fantastic jury for agreeing to read my thesis and to come all the way to Saclay to attend my defense. Special thanks go to my reviewers Nicolas Magaud and Anupam Das for their thorough and thoughtful reports. Receiving these final tokens of recognition filled me with great joy, relief and pride.

Now, the astute reader might have noticed that my PhD began exactly when the Covid pandemic broke out. This made for two very lonely first years, unfortunately encouraging many people in our lab (including me) to work remotely most of the time, even after the pandemic. I wish I could have had richer and more frequent interactions with my fellow teammates in PARTOUT, following the merge of our 4-people team Typical in 2022. The work of the permanent researchers there had a decisive influence on the choice of (\kl{deep inference}) proof theory as the unifying methodology underpinning my entire thesis. I want to give special thanks to Dale Miller for his inspiring work relating proof theory and logic programming\footnote{Knowledge that he actually taught me in a course about linear logic at the MPRI (Master Parisien de Recherche en Informatique) in 2019-2020.}; Lutz Straßburger for his deep expertise in deep inference, in part available in his course notes \cite{tubella:hal-02390267}; and of course Kaustuv Chaudhuri for his wonderful invention of the \kl{subformula linking} technique, as well as many insightful discussions about the application of direct manipulation to interactive theorem proving. I also want to thank all the PhD students I had the chance to meet there: Antoine Séré, Giti Omidvar, Marianela Evelyn Morales Elena, Wendlasida Ouedraogo, Farah Al Wardani, Jui-Hsuan Wu, Matteo Manighetti. More generally, I am grateful for all the occasions we had to spend time together exchanging ideas and stories, whether at seminars or dinners.

\section*{The road for the trees}

I will now gradually shift away from professional relationships, entering the realm of personal ones. I shall start with a community of like-minded interlocutors that continuously and consistently pulled me out of my intellectual loneliness: the ReFL, or ``groupe de Réflexions sur les Fondements de la Logique''. Beginning as a small group of PhD students interested in Girard's latest works, it rapidly grew in number, scope and structure after our first IRL meeting during the "Linear Logic Winter School 2022" at the CIRM in Marseille\footnote{It is also where I met Nathan Haydon for the first time, the only fellow Peirce scholar with whom I have had the opportunity to exchange on the subject of existential graphs.}. Regular members with whom I have had the chance to interact in the past 3 years include Davide Barbarossa, Valentin Maestracci, Sidney Congard, Jérémy Hervé, Ambroise Lafont (now in PARTOUT), Luc Pommeret, Paul Séjourné, Vincent Moreau, Tito, Xavier Denis. Among those, Sidney and Jérémy had the hospitality to accomodate me for several days on two occasions. I have had the pleasure myself to host many seminars and informal gatherings in my living room, where our passionate philosophical and mathematical discussions often resulted in my blackboard being filled with inscrutable and mystical scribbles. A special shout-out goes to Boris Eng, main founder and organizer of the ReFL, with whom I entertained an intense, near to epistolary correspondence on a great variety of subjects, which somehow always found its way back to \kl{transcendental syntax} and the ``4 cases du tableau''.
% \footnote{I do not want to sound cabalistic, but the number `4' has now occurred 4 times in these acknowledgments. Just sayin'.}.

I want to thank my friends from the MPRI, Xavier Denis and Ada Vienot. They believed in my ability to do research when I did not, and I probably would not have pursued my studies in theoretical computer science without their words of encouragement. Their companionship at various stages of our PhDs allowed me to share some of my hopes and frustrations, and also helped me get a better understanding of the complex world of academia. In particular, Xavier's rants about the future of formal methods will always keep resonating in my mind, challenging my assumptions and orienting my purpose in this field.

Although they were far removed from my professional life (at least as much as I could make them), the moments I shared with my close friends have been vital to my personal balance. I want to thank the ``Vitryottes'' club for all the fun parties, movie nights, vacations, and of course ``soirées crêpes'' together: Luc, Adé, Thibaut, Théophile, Paul-Nicolas, Béranger, Lucie, Anthyme, Clém. Without my (ex-)gym buddy Adé, I would never have set foot in a gym, which became an essential part of my routine to get this manuscript to completion. As the saying goes: ``a healthy mind in a healthy body''\footnote{The contrapositive being: ``no physical activity $\limp$ going crazy''.}. Alex, Lucas, Thomas, your rivalry at Super Smash Bros has been a constant challenge that does not pale in comparison to this thesis (well maybe it does). Antoine, our deep conversations since ancient times have always kept me thinking, questioning myself and the things I value in a refreshing way. Alex, our sustained friendship over the past 8 years has brought me many moments of happiness, whether sharing our difficulties in relationships over spicy noodles at La Table du Lamen, playing indie video games and jamming at your place, or going to concerts together.

Indeed regarding this last point: music is not only a very important outlet for my creativity and emotions, but also a great source of inspiration for my mind. Thus I am grateful to all the people who accepted my numerous invitations to attend (way too) many concerts with me. A special thanks goes to every wonderful jazz musician out there in the Parisian scene, for creating and performing such beautiful art that soothed my soul almost every week. I also wish to thank Alexandre Gaspar and my bandmates in PAPYMOLUX (Pierre, Morgane, Lucas) for all the music we played together, whether jamming in the studio or performing on-stage. Last but not least, I want to acknowledge the immense strength I gained from listening to my favorite band: Leprous. I simply cannot imagine surviving in a world deprived of Einar's mesmerizing voice and profound songwriting. It brought meaning to every situation in my life, especially when I was at a loss. It gave me purpose by the simple act of existing and sending chills down my spine. Thank you Leprous, for being a handle to grasp onto.

I have already mentioned \emph{love} earlier, in analogy with the dedication one can have for a scientific subject. I will not dwell too long on this fundamental aspect of our human lives, as I believe I might have already shared too much personal information in these acknowledgments. I just want to emphasize how important, driving and challenging love has been for me, at least on the same scale as this PhD thesis. From the bottom of my heart, I am grateful to every person who loved me and let me love them back. In particular, I wish to thank you Ninon, for bearing with me during these last months despite all the difficulties life can put on our way; I truly enjoyed every moment we spent together. I am obviously in eternal debt towards my family, who raised me to make the person I am today. They have always supported my decisions and aspirations, accepting my feelings and who I am unconditionally. Thank you Mom, thank you Dad, thank you Cypria.

My final thanks go to my roommate, personal chef, day-to-day colleague, and long-date friend: Luc Chabassier. Moving in together 3 years ago was certainly the luckiest decision I made to improve my overall quality of life and guarantee the success of my PhD. Not only did you introduce me to the Vitryottes, renewing significantly my social circles; cook delicious meals every week, leaving me time to focus on my thesis; laugh with me while watching weird anime, moving my focus away from my thesis; torture me by bringing me occasionally to parkour and natfit, which somehow had a positive impact on my physical health. You also contributed significant ideas to my thesis, through our endless discussions about computer science, logic, maths, and metaphysics; you even gave me the opportunity for the post-doc that will define the next chapter of my career. You have been my main interlocutor for the past 3 years, shaping significantly both my personal and professional life. For these contributions and presence, I thank you.